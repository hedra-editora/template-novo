
\chapter{She walks in beauty}


{i}&
She walks in beauty, like the night&
Of cloudless climes and starry skies;&
And all that's best of dark and bright&
Meet in her aspect and her eyes:&
Thus mellowed to that tender light&
Which heaven to gaudy day denies.\&


{ii}&
One shade the more, one ray the less,&
Had half impaired the nameless grace&
Which waves in every raven tress,&
Or softly lightens o'er her face;&
Where thoughts serenely sweet express,&
How pure, how dear their dwelling-place.\&


{iii}&
And on that cheek, and o'er that brow,&
So soft, so calm, yet eloquent,&
The smiles that win, the tints that glow,&
But tell of days in goodness spent,&
A mind at peace with all below,&
A heart whose love is innocent!\&


\pagebreak


\chapter{Ela caminha em formosura\footnote[*]{ Pensa-se que este poema foi inspirado
							pela visão de Lady Wilmot Horton num baile, trajada de cores 
							de luto, mas com numerosas lantejoulas no vestido (D.~Dunn). É o primeiro 
							poema de \textit{Hebrew Melodies} (1815), que
							foram traduzidas em nosso Romantismo, por exemplo, por Antônio Franco 
							da Costa Meireles, 1869 (cf.~Onédia Célia, pp.~223 e
							ss.) ou J.~A.~de Oliveira Silva (1875).}

{i}&
Ela caminha em formosura, noite que anda&
Num céu sem nuvens e de estrelas palpitante,&
E o que é melhor em treva ou resplendor&
Se encontra em seu olhar e em seu semblante;&
Ela amadureceu à luz tão branda&
Que o Céu denega ao dia em seu fulgor.\&


{ii}&
Uma sombra de mais, um raio que faltasse,&
Teriam diminuído a graça indefinível&
Que em suas tranças cor de corvo ondeia& 
Ou meigamente lhe ilumina a face;&
Nessas feições revela, qualquer doce ideia,& 
Como é puro seu lar, como é aprazível.\&


{iii}&
Na fronte e rosto cheio de serenidade,& 
Tão suave, porém tão eloquente,&
O sorriso que vence e a tez que se enrubesce& 
Dizem apenas de um passado de bondade:&
De uma alma cuja paz com todos transparece,& 
De um coração de amor sempre inocente.\&

\pagebreak


\chapter*{The Isles of Greece, the Isles of Greece!}



{i}&
The isles of Greece, the isles of Greece!&
Where burning Sappho loved and sung,&
Where grew the arts of war and peace,&
Where Delos rose, and Phœbus sprung!&
Eternal summer gilds them yet,&
But all, except their sun, is set.\&


{ii}&
The Scian and the Teian muse,&
The hero's harp, the lover's lute,& 
Have found the fame your shores refuse;&
Their place of birth alone is mute& 
To sounds which echo further west& 
Than your sires' 'Islands of the Blest.'\&

{iii}&
The mountains look on Marathon ---&
And Marathon looks on the sea;& 
And musing there an hour alone,&
I dreamed that Greece might still be free;& 
For standing on the Persians' grave,& 
I could not deem myself a slave.\&


{iv}&
A king sate on the rocky brow&
Which looks o'er sea-born Salamis;&
And ships, by thousands, lay below,&
And men in nations; --- all were his!& 
He counted them at break of day ---&
And when the sun set where were they?\&


{v}&
And where are they? and where art thou,&
My country? On thy voiceless shore& 
The heroic lay is tuneless now ---&
The heroic bosom beats no more!& 
And must thy lyre, so long divine,& 
Degenerate into hands like mine?\&


{vi}&
'Tis something, in the dearth of fame,&
Though linked among a fettered race,& 
To feel at least a patriot's shame,&
Even as I sing, suffuse my face;& 
For what is left the poet here?& 
For Greeks a blush --- for Greece a tear.\&

{vii}&
Must \textit{we }but weep o'er days more blest?&
Must \textit{we }but blush? --- Our fathers bled.& 
Earth! render back from out thy breast&
A remnant of our Spartan dead!& 
Of the three hundred grant but three,& 
To make a new Thermopylae!\&

{viii}&
What, silent still? and silent all?&
Ah! no; --- the voices of the dead&
Sound like a distant torrent's fall,&
And answer, 'Let one living head,&
But one arise, --- we come, we come!'&
'Tis but the living who are dumb.\&


{ix}&
In vain --- in vain: strike other chords;&
Fill high the cup with Samian wine!&
Leave battles to the Turkish hordes,&
And shed the blood of Scio's vine!&
Hark! rising to the ignoble call ---& 
How answers each bold Bacchanal!\&


{x}&
You have the Pyrrhic dance as yet;&
Where is the Pyrrhic phalanx gone?&
Of two such lessons, why forget&
The nobler and the manlier one?&
You have the letters Cadmus gave ---& 
Think ye he meant them for a slave?\&

{xi}&
Fill high the bowl with Samian wine!&
We will not think of themes like these!&
It made Anacreon's song divine:&
He served --- but served Polycrates&
A tyrant; but our masters then&
Were still, at least, our countrymen.\&


{xii}&
The tyrant of the Chersonese&
Was freedom's best and bravest friend;&
\textit{That} tyrant was Miltiades!&
Oh! that the present hour would lend&
Another despot of the kind!&
Such chains as his were sure to bind.\&



{xiii}&
Fill high the bowl with Samian wine!&
On Suli's rock, and Parga's shore,&
Exists the remnant of a line&
Such as the Doric mothers bore;&
And there, perhaps, some seed is sown.&
There Heracleidan blood might own.\&


{xiv}&
Trust not for freedom to the Franks ---& 
They have a king who buys and sells:&
In native swords, and native ranks,&
The only hope of courage dwells:&
But Turkish force, and Latin fraud,&
Would break your shield, however broad.\&

{xv}&
Fill high the bowl with Samian wine!&
Our virgins dance beneath the shade ---& 
I see their glorious black eyes shine;&
But gazing on each glowing maid,&
My own the burning tear-drop laves,&
To think such breasts must suckle slaves.\&

{xvi}&
Place me on Sunium's marbled steep,&
Where nothing, save the waves and I,&
May hear our mutual murmurs sweep;&
There, swan-like, let me sing and die:&
A land of slaves shall ne'er be mine ---&
Dash down yon cup of Samian wine!\&

\pagebreak

\chapter{As ilhas da Grécia, as ilhas da Grécia!\footnote[*]{ (\textit{Don Juan}, canto \textsc{iii}, 
						\textsc{lxxxvi}.) Consta de \textit{Don Juan}, \textsc{iii}, depois de \textsc{lxxxvi}. Este hino, 							escreve Byron, devia ou podia ser cantado assim pelo grego moderno, isto é, de antes da
						independência, em ``tolerável verso'' (\textsc{lxxxvii}).}


{i}&
As ilhas da Grécia, as ilhas da Grécia!&
Onde a ardente Safo amou e cantou,&
Onde a arte da guerra e a da paz cresceram,&
{E Delos surgiu, que a Apolo abrigou!}%
							\footnote { Em Delos nasceram Apolo e Ártemis: a ilha acolhera
							 a erran­te e perseguida Leto, que lá
							os deu à luz no monte Cinto.}& 
Um eterno verão as doura ainda,&
Mas tudo, exceto o sol, já descambou.\&


{ii}&
{Nelas, a musa de Quios e de Teos}%
							\footnote { Em Quios supunha-se ter nascido Homero; Teos,
							cidade marítima da Jônia, era a pátria de Anacreonte.}%
					,& 
A harpa do herói, o alaúde do amante,&
A fama acharam que não dão agora;&
A pátria deles muda está, perante&
{Sons que passam as ``Ilhas dos Felizes''}%
							\footnote { Assinala Byron que as ``nesoi macáron'' 
                                                        (Ilhas dos Bem-aventurados) dos poetas gregos supõe-se que tenham sido 
                                                        as ilhas de Cabo Verde ou Canárias.}%
						,&
Para ecoar no oeste mais distante.\&

{iii}&
As montanhas contemplam Maratona& 
E Maratona olha para o mar;&
E sonhei, uma hora lá sozinho,&
Que livre a Grécia poderia estar;&
Pois de pé sobre o túmulo dos persas,& 
Escravo eu não podia me julgar.\&


{iv}&
Um rei sentou-se na rochosa borda&
{Que encara Salamina, dom do mar}%
							\footnote { Maratona e Salamina, locais onde os
							persas foram vencidos por Milcíades (490 a.~C.)
							e em batalha naval (480 a.~C.).}%
					;&
Naus, aos milhares, viam-se lá embaixo,&
E nações, que eram dele, iam lutar!&
Ele contou-as ao nascer do dia,&
{E ao pôr-do-sol quem as iria achar?}%
							\footnote { Byron cita um trecho de \textit{Os persas}, Ésquilo, em que Xerxes, 
							vendo a derrota, ordena a retirada.}\&


{v}&
E onde estão eles? Minha pátria, onde&
Estás? Em tuas praias já sem voz&
O canto heroico não ressoa agora&
E já não bate o peito dos heróis!&
Tanto tempo divina, deve a lira&
Em mãos como estas decair após?\&


{vi}&
Na carência da fama, é alguma coisa,&
Preso embora entre raça agrilhoada,&
Sentir que uma vergonha patriótica,&
Mesmo se eu canto, em minha face brada;&
Pois que deixou o poeta aqui? Aos gregos&
Rubor, à Grécia a lágrima sagrada.\&


{vii}&
Só devemos chorar os belos dias?&
Chorar? --- Deram seu sangue os nossos pais.& 
Terra! devolve de teu seio uns poucos&
Dos espartanos mortos --- uns, não mais!&
{Para novas Termópilas fazermos}%
							\footnote { Desfiladeiro em que Leônidas, rei de Esparta, 	
							deteve os per­sas, à custa de sua própria
							vida e da de seus soldados.}%
					,& 
Dá-nos três dos trezentos imortais!\&

{viii}&
Que? Calados ainda? Todos, todos? &
Não! As vozes dos mortos eis a soar&
Como queda longínqua de torrente,&
E respondem: ``Estamos a chegar!&
Que uma só fronte viva se subleve!''& 
Só os vivos não se atrevem a falar.\&


{ix}&
Em vão, em vão: feri vós outras cordas;&
Com vinho sâmio coroai a taça!&
Deixai as pugnas para as hostes turcas,&
De Quios vertei o sangue, o da vinhaça!&
Ouvi! erguendo-se ao chamado ignóbil,&
Como responde a bacanal que grassa!\&


{x}&
{A dança pírrica tendes como antes}%
							\footnote { \textit{Dança pírrica}, dança guerreira 
							dos antigos gregos; \textit{falange pírrica,} de Pirro, rei do 	
							Epiro, que combateu os romanos e sobre eles obteve
							vitórias desgastantes, que o levaram ao desastre posterior.}%
					 ,&
Por que a falange pírrica está ausente?&
Por que, dos dois legados, esquecer&
O que é mais nobre e másculo somente?&
{As letras com que Cadmo vos brindou}%
							\footnote { Cadmo, fenício irmão
							de Europa, o lendário introdutor do alfabeto na Grécia.}%
						,&
Queria-as ele para escrava gente?\&


{xi}&
Com vinho sâmio coroai a taça!&
Não nos empolga assunto miliciano!& 
O vinho fez cantar Anacreonte,&
Que serviu a Polícrates, tirano;&
Mas então cada amo nosso ainda& 
Era patrício --- não um otomano.\&


{xii}&
{O tirano do Quersoneso, o amigo}
							\footnote { Milcíades, que seria
							o vencedor de Maratona, foi tirano do Quersoneso ao tempo 
							da invasão persa da Trácia (512 a.~C.).}& 
Mais bravo era e o melhor da liberdade;& 
Era Milcíades esse tirano!&
Oh, que nos emprestasse, a nossa idade,& 
Um outro déspota da mesma espécie!& 
Os seus laços ligavam de verdade!\&


{xiii}&
Com vinho sâmio coroai a taça!&
Existe o resto de uma estirpe grada,&
De Suli a Parga no rochedo e praia,&
Como a que por mães dórias foi gerada;&
{Lá podia mandar sangue heraclida}%
							\footnote { \textit{Heraclida}, descendente de Hércules; 
							no caso, dório (do Peloponeso) ou melhor, restritamente 
							espartano.}%
					.& 
Lá, talvez, a semente está semeada.\&


{xiv}&
Para ser livres não confieis nos francos&
 --- Têm como rei alguém que compra e vende;& 
De nativas espadas e fileiras&
A só esperança de coragem pende:&
Com a força turca essa latina fraude&
Romper-vos-ia o escudo que defende.\&

{xv}&
Com vinho sâmio coroai a taça!&
Virgens dançam à sombra --- reluzentes&
Eu vejo os seus gloriosos olhos negros;&
Mas olhando essas moças resplendentes,&
Se penso que hão de amamentar escravos,&
Banham-me a vista lágrimas candentes.\&


{xvi}&
Ao promontório, a Súnio, conduzi-me,& 
Onde eu somente, mais a onda que passa,&
Possamos escutar nossos murmúrios:&
{Lá, cisne, eu cante à morte que me abraça}%
							\footnote { Como o cisne --- Byron alude a uns versos que
							cita, de Sófocles, \textit{Ájax}, \textsc{v}, 1217.}%
						 ;&
Terra de escravos nunca será minha:&
--- Do vinho sâmio jogai fora a taça!\&

\quebra

\chapter*{To Thomas Moore}



My boat is on the shore,&
And my bark is on the sea;&
But, before I go, Tom Moore,&
Here's a double health to thee!\&


Here's a sigh to those who love me,&
And a smile to those who hate;&
And, whatever sky's above me,&
Here's a heart for every fate.\&


Though the ocean roar around me,&
Yet it still shall bear me on;&
Though a desert should surround me,&
It hath springs that may be won.\&


Were't the last drop in the well,&
As I gasped upon the brink,&
Ere my fainting spirit fell,&
'Tis to thee that I would drink.\&


With that water, as this wine,&
The libation I would pour&
Should be --- peace with thine and mine,&
And a health to thee, Tom Moore.\&

\pagebreak


\chapter{A Thomas Moore\footnote[*]{ Este poema, que traduzi em versos de sete	sílabas (o original tem número de sílabas 
						correspondente, mas é silábico-acentual), foi escrito em 1817. Thomas Moore (1779--1852),
						conhecido poeta, foi amigo de Byron.}


Está na praia o meu bote,&
Meu navio está no mar:& 
Mas antes que eu vá, Tom Moore,& 
Quero em dobro te brindar!\&


Eis um suspiro aos que me amam,&
Aos que odeiam, um sorriso;&
Qualquer o céu que me cubra,&
Enfrento o que for preciso.\&


Ruja o oceano em torno a mim,& 
Em suas águas irei;& 
Um deserto me rodeie,& 
Nele a fontes chegarei.\&


Só uma gota no meu poço,& 
E eu nas bordas a ofegar:& 
Antes de ir-se o meu espírito,& 
A ti é que a vou tomar.\&


Com esta água e com este vinho,& 
A libação que eu verter& 
Será --- paz aos teus e aos meus& 
E a ti, Tom Moore, vou beber.\&
\quebra

\chapter*{The Destruction of Sennacherib}


The Assyrian came down like the wolf on the fold,& 
And his cohorts were gleaming in purple and gold;& 
And the sheen of their spears was like stars on the sea,& 
When the blue wave rolls nightly on deep Galilee.\&


Like the leaves of the forest when summer is green,& 
That host with their banners at sunset were seen;& 
Like the leaves of the forest when autumn hath blown,& 
That host on the morrow lay withered and strown.\&


For the Angel of Death spread his wings on the blast,&
And breathed in the face of the foe as he passed;&
And the eyes of the sleepers waxed deadly and chill,& 
\mbox{And their hearts but once heaved, and for ever grew still!}\&


And there lay the steed with his nostril all wide,& 
But through it there rolled not the breath of his pride;& 
And the foam of his gasping lay white on the turf,& 
And cold as the spray of the rock-beating surf.\&


And there lay the rider distorted and pale,&
With the dew on his brow, and the rust on his mail:&
And the tents were all silent, the banners alone,&
The lances unlifted, the trumpet unblown.\&


And the widows of Ashur are loud in their wail,& 
And the idols are broke in the temple of Baal;& 
And the might of the Gentile, unsmote by the sword,& 
Hath melted like snow in the glance of the Lord!\&

\pagebreak


\chapter{A destruição de Senaqueribe\footnote[*]{ Originalmente de
						\textit{Hebrew Melodies} (1815), o poema figura nas traduções
						dessas \textit{Melodias} por Costa Meireles (1869) e Oliveira
						Silva (1875). O ritmo ternário do texto inglês desmereceria,
						por martelante demais em português, a equivalência que fosse
						mantida.}

Tendo ouro e tendo púrpura a brilhar em suas cortes,&
Desceram os assírios --- lobo em busca do redil:&
Luziam suas lanças como estrelas pelo mar&
Quando na Galileia, à noite, rola a onda anil.\& 


Como as folhas no bosque quando o estio reverdeja,&
Com bandeiras, ao pôr-do-sol, o exército surgia;&
Como as folhas no bosque quando o outono se adianta,&
Um dia após, sem viço e esparso, o exército jazia.\& 


Pois os Anjos da morte as asas na rajada abriram;&
No rosto do inimigo, perpassando, eles sopraram;&
E os olhos dos dormidos se apagaram, regelados,&
E, após arfar uma só vez, os corações pararam.\&


Lá jazia o corcel, jazia, ventas dilatadas,&
Porém por elas não passava o alento de seu brio;&
Branca na grama a espuma via-se de seu ofego,&
Fria tal como a onda ao borrifar no penedio.\&


E lá jazia o cavaleiro contorcido e pálido,& 
Tendo orvalho na fronte, e com a armadura enferrujada;& 
E as tendas silenciosas, e as bandeiras já largadas,& 
As lanças não erguidas, a trombeta não tocada.\&


E os ídolos quebraram-se nos templos de Baal,&
E as viúvas de Assur em prantos erguem seu clamor,&
E o poder do infiel, sem que o tocasse a espada ao menos,& 
Qual neve derreteu-se ao pôr-lhe os olhos o Senhor!\&
\pagebreak


\chapter*{So, We’ll Go no More A-roving}

%Bruno: Cf. diagram. da estrofe.


{i}&
So, we'll go no more a-roving&
So late into the night,&
Though the heart be still as loving,&
And the moon be still as bright.\&


{ii}&
For the sword outwears its sheath,&
And the soul wears out the breast,&
And the heart must pause to breathe,&
And love itself have rest.\&


{iii}&
Though the night was made for loving,&
And the day returns too soon,&
Yet we'll go no more a-roving&
By the light of the moon.\&
\pagebreak


\chapter{Assim não mais iremos vaguear\footnote[*]{ Esta canção foi
						escrita em Veneza, 1817. Parece refletir passeios de gôndola, à
						noite.}



{i}&
Tarde assim dentro da noite& 
Não mais iremos vaguear,&
Embora o coração inda ame tanto&
E a lua continue a fulgurar.\&



{ii}&
Pois mais do que a bainha dura a espada& 
E a alma gasta o peito& 
E o coração faz pausa para respirar& 
E o amor para descansar.\&



{iii}&
Embora o dia volte muito cedo&
E a noite fosse feita para amar,&
Não mais iremos vaguear& 
Ao luar.\&
\pagebreak



\chapter*{Fare Thee Well}


Fare thee well! and if for ever,&
Still for ever, fare thee well:&
Even though unforgiving, never&
'Gainst thee shall my heart rebel.\&


Would that breast were bared before thee&
Where thy head so oft hath lain,&
While that placid sleep came o'er thee&
Which thou ne'er canst know again:\&


Would that breast, by thee glanced over,&
Every inmost thought could show!&
Then thou wouldst at last discover&
'Twas not well to spurn it so.\&


Though the world for this commend thee ---& 
Though it smile upon the blow,&
Even its praises must offend thee,&
Founded on another's woe:\&

Though my many faults defaced me,&
Could no other arm be found,&
Than the one which once embraced me,&
To inflict a cureless wound?\&


Yet, oh yet, thyself deceive not;&
Love may sink by slow decay,&
But by sudden wrench, believe not&
Hearts can thus be torn away:\&


Still thine own its life retaineth ---&  
Still must mine, though bleeding, beat;&
And the undying thought which paineth&
Is --- that we no more may meet.\&


These are words of deeper sorrow&
Than the wail above the dead;&
Both shall live, but every morrow&
Wake us from a widowed bed.\&


And when thou would solace gather,&
When our child's first accents flow,&
Wilt thou teach her to say 'Father!'&
Though his care she must forego?\&


When her little hands shall press thee,&
When her lip to thine is pressed,&
Think of him whose prayer shall bless thee,&
Think of him thy love \textit{had }blessed!\&


Should her lineaments resemble&
Those thou never more may'st see,&
Then thy heart will softly tremble&
With a pulse yet true to me.\&

All my faults perchance thou knowest,&
All my madness none can know;&
All my hopes, where'er thou goest,&
Wither, yet with \textit{thee }they go.\&


Every feeling hath been shaken;&
Pride, which not a world could bow,&
Bows to thee --- by thee forsaken,&
Even my soul forsakes me now:\&


But 'tis done --- all words are idle ---& 
Words from me are vainer still;&
But the thoughts we cannot bridle&
Force their way without the will.\&


Fare thee well! thus disunited,&
Torn from every nearer tie,&
Seared in heart, and lone, and blighted,&
More than this I scarce can die.\&

\pagebreak

\chapter{Adeus\footnote[*]{ O poema, datado de 17 de março de 1816,
foi escrito quando da separação de Lady Byron, e parece querer chamar a esposa
à reconciliação.  O original traz uma longa epígrafe de Coleridge,
``Christabel''.  Foi traduzido em 1857 no \textit{Diário do Rio de Janeiro} (15
de novembro) escrevendo \textsc{x.~y.} o seguinte: ``Há uma poesia escrita a
Lady Byron, da qual Mme.~de Staël disse estas palavras: --- Eu trocaria as
minhas glórias pela glória de ter inspirado uma semelhante poesia. ---
Entretanto a mulher que a inspirou foi inexorável até o fim, nunca perdoou.''
Ver Onédia Célia, p.~56 ss.} 



Adeus! e para sempre embora,& 
Que seja para nunca mais.& 
Sei teu rancor --- mas contra ti& 
Não me rebelarei jamais.\&


Visses nu meu peito, onde a fronte& 
Tu descansavas mansamente& 
E te tomava um calmo sono& 
Que perderás completamente:\&


Que cada fundo pensamento& 
No coração pudesses ver!& 
Que estava mal deixá-lo assim& 
Por fim virias a saber.\&


Louve-te o mundo por teu ato,& 
Sorria ele ante a ação feia:& 
Esse louvor deve ofender-te,& 
Pois funda-se na dor alheia.\&


Desfigurassem-me defeitos:& 
Mão não havia menos dura& 
Que a de quem antes me abraçava& 
Que me ferisse assim sem cura?\&


Não te iludas contudo: o amor& 
Pode afundar-se devagar;& 
Porém não pode corações& 
Um golpe súbito apartar.\&


O teu retém a sua vida,& 
E o meu, também, bata sangrando;& 
E a eterna ideia que me aflige& 
É que nos vermos não tem quando.\&


Digo palavras de tristeza& 
Maior que os mortos lastimar;& 
Hão de as manhãs, pois viveremos,& 
De um leito viúvo despertar.\&


E ao achares consolo, quando& 
A nossa filha balbuciar,& 
Ensiná-la-ás a dizer Pai!,& 
Se o meu desvelo vai faltar?\&


Quando as mãozinhas te apertarem& 
E ela teu lábio houver beijado,& 
Pensa em mim, que te bendirei!& 
Teu amor ter-me-ia abençoado.\&


Se parecerem os seus traços& 
Com os de quem podes não mais ver,& 
Teu coração pulsará suave,& 
E fiel a mim há de tremer.\&


Talvez conheças minhas faltas,& 
Minha loucura ninguém sabe;& 
Minha esperança, aonde tu vás,& 
Murcha, mas vai, que ela em ti cabe.\&


Abalou-se o que sinto; o orgulho,& 
Que o mundo não pôde curvar,& 
Curvou-se a ti: se a abandonaste,& 
Minha alma vejo-a a me deixar.\&


Tudo acabou --- é vão falar ---,& 
Mais vão ainda o que eu disser;& 
Mas forçam rumo os pensamentos&
Que não podemos empecer.\&


Adeus! assim de ti afastado,& 
Cada laço estreito a perder,& 
O coração só e murcho e seco,& 
Mais que isto mal posso morrer.\&
\pagebreak



\chapter*{Waterloo}


{xvii}&
Stop! --- for thy tread is on an Empire's dust!&
An Earthquake's spoil is sepulchred below!& 
Is the spot mark'd with no colossal bust?& 
Nor column trophied for triumphal show?& 
None; but the moral's truth tells simpler so,& 
As the ground was before, thus let it be; ---& 
How that red rain hath made the harvest grow!& 
And is this all the world has gain'd by thee,&
Thou first and last of fields! king-making Victory?\&


{xviii}&
And Harold stands upon this place of skulls, &
The grave of France, the deadly Waterloo!& 
How in an hour the power which gave annuls& 
Its gifts, transferring fame as fleeting too!& 
In `pride of place' here last the eagle flew,& 
Then tore with bloody talon the rent plain,& 
Pierced by the shaft of banded nations through;& 
Ambition's life and labours all were vain;&
He wears the shatter'd links of the world's broken chain.\&

{xix}&
Fit retribution! Gaul may champ the bit &
And foam in fetters; --- but is Earth more free?& 
Did nations combat to make \textit{One }submit;& 
Or league to teach all kings true sovereignty? &
What! shall reviving Thraldom again be &
The patch'd-up idol of enlighten'd days?& 
Shall we, who struck the Lion down, shall we& 
Pay the Wolf homage? proffering lowly gaze&
And servile knees to thrones? No; \textit{prove} before ye praise!\&


{xx}&
If not, o'er one fallen despot boast no more!&
In vain fair cheeks were furrow'd with hot tears&
For Europe's flowers long rooted up before&
The trampler of her vineyards; in vain years&
Of death, depopulation, bondage, fears,&
Have all been borne, and broken by the accord&
Of roused-up millions; all that most endears&
Glory, is when the myrtle wreathes a sword&
Such as Harmodius drew on Athens' tyrant lord.\&
\pagebreak


\chapter{Waterloo\footnote[*]{ (\textit{Childe Harold}, canto \textsc{iii}) A popularidade das obras de Byron na era
						vitoriana entre operários intelectualizados foi testemunhada por
						Friedrich Engels em 1845, e no termo do período leitores não
						sofisticados de todas as classes sociais ainda se lembravam de
						\textit{morceaux de bravoure} tais como o que descreve Waterloo em
						\textit{Childe Harold}, \textsc{iii} (John Jump). Apesar de seus versos contra
						Napoleão, sabe-se que Byron o prezava, sendo essa até uma
						das razões que o antipatizaram nos seus círculos ingleses. O poeta era
						contrário ao estado de coisas que sucedeu a queda do corso, como se vê
						nos vv.~20 ss.}

\vspace*{-\baselineskip}



{xvii}&
Detém na poeira de um Império essas passadas!& 
Ruínas de um terremoto aqui estão sepultadas! &
Não orna este lugar um busto colossal? &
Nem troféus nem colunas em visão triunfal?&
Não, contudo moral mais simples aqui aflora:& 
Como o solo era antes, seja assim agora! &
Como com a chuva rubra vicejou a messe! &
Fizeste que contigo o mundo isto obtivesse,& 
Primeiro campo de batalha e terminal!\&


{xviii}&
Haroldo está de pé --- que de ossos no local!---&
No sepulcro da França, Waterloo mortal!&
Como teus dons anulas, tu, poder que dás,&
Como transferes uma fama tão fugaz!&
{Aqui a águia em seu último auge foi notada}%
							\footnote { ``Pride of place'' é um termo de falcoaria e
							significa o mais alto alcance do voo. Ver \textit{Macbeth}:
							 ``An eagle towering in his pride of place'' etc. (Byron).}&
E feriu, garra em sangue, a terra lacerada, &
Pela flecha da união dos povos traspassada; &
Todo o trabalho da ambição foi infecundo: &
Leva os anéis partidos dos grilhões do mundo.\&

{xix}&
Pode a Gália morder --- é justo --- o freio a fundo&
E em ferros escumar: está mais livre o mundo? &
Lutaram as nações para vergar só\textit{ um}? &
Ou ensinar os reis a preexceler na ação?& 
Quê! a Escravidão ressuscitada será algum& 
Ídolo tosco de dias de ilustração?&
Devemos render preito ao Lobo, nós que o Leão& 
Derrubamos? Perante o trono olhos baixar, &
Dobrar os joelhos? Não, \textit{provai} para exaltar!\&


{xx}&
Senão, que um déspota caiu não clameis tanto!&
Em vão em belos rostos derramou-se pranto&
Por tanta flor da Europa, que viu arrancada&
O que pisava as vinhas; épocas em vão&
De fim, despovoamento, servidão e horror&
Foram sofridas, mas quebrou-as a união&
De milhões que se ergueram; o que faz amada&
A glória, é quando o mirto vem coroar a espada,&
{Como a que Harmódio opôs --- de Atenas ao senhor}%
							\footnote { Harmódio e Aristogito conspiraram contra os 
							Pisistrátidas --- Hípias e Hiparco. Só Hípias foi morto, sendo 
							Harmódio abatido pela guarda de Hiparco e Aristogito preso 
							e executado (514 a.~C.). Foram posteriormente
							reverenciados em Atenas como campeões da liberdade. Byron 
							cita o verso a que alude, ``With myrth my sword will I
							wreathe'', do canto grego traduzido por Denman.}%
							.\&
\quebra

\chapter*{To Inez}



{i}&
Nay, smile not at my sullen brow;&
Alas! I cannot smile again:&
Yet Heaven avert that ever thou&
Shouldst weep, and haply weep in vain.\&


{ii}&
And dost thou ask what secret woe&
I bear, corroding joy and youth?&
And wilt thou vainly seek to know&
A pang, ev'n thou must fail to soothe?\&


{iii}&
It is not love, it is not hate,&
Nor low Ambition's honours lost,&
That bids me loathe my present state,&
And fly from all I prized the most:\&


{iv}&
It is that weariness which springs&
From all I meet, or hear, or see:&
To me no pleasure Beauty brings;&
Thine eyes have scarce a charm for me.\&

{v}&
It is that settled, ceaseless gloom&
The fabled Hebrew wanderer bore;&
That will not look beyond the tomb,&
But cannot hope for rest before.\&


{vi}&
What Exile from himself can flee?&
To zones though more and more remote,&
Still, still pursues, where'er I be,&
The blight of life --- the demon Thought.\&


{vii}&
Yet others rapt in pleasure seem,&
And taste of all that I forsake;&
Oh! may they still of transport dream,&
And ne'er, at least like me, awake!\&


{viii}&
Through many a clime 'tis mine to go,&
With many a retrospection curst;&
And all my solace is to know,&
Whate'er betides, I've known the worst.\&


{ix}&
What is the worst? Nay, do not ask ---& 
In pity from the search forbear:&
Smile on --- nor venture to unmask&
Man's heart, and view the Hell that's there.\&

\pagebreak


\chapter{A Inês\footnote[*]{ Este poema, que consta do \textit{Childe Harold}, \textsc{i}, entre \textsc{lxxxiv} e \textsc{lxxxv}, foi um dos 								mais cultuados em nosso Romantismo. Onédia Célia consigna as traduções de Francisco 
							Otaviano, Francisco de Assis Vieira Bueno, João Júlio dos
							Santos, Sousândrade, V.~S.~(Luís Vieira da Silva), J.~A.~
							de Oliveira Silva, Barão de Paranapiacaba e Fagundes Varela.
							Onédia considera a tradução de Sousândrade não a mais fiel, 
							nem a mais infiel, mas certamente a mais bela das versões 
							brasileiras de ``To Inez''. Há de fato na tradução um 
							``anjo eterno'' que nada tem a ver com o original, mas,
							inesperado como é, parece-nos sugestivo.}

\vspace*{-2\baselineskip}



{i}&
Não me sorrias à sombria fronte,& 
Ai! sorrir eu não posso novamente:& 
Que o céu afaste o que tu chorarias& 
E em vão talvez chorasses, tão-somente.\&


{ii}&
E perguntas que dor trago secreta,& 
A roer minha alegria e juventude?& 
E em vão procuras conhecer-me a angústia& 
Que nem tu tornarias menos rude?\&


{iii}&
Não é o amor, não é nem mesmo o ódio,& 
Nem de baixa ambição honras perdidas,& 
Que me fazem opor-me ao meu estado& 
E evadir-me das coisas mais queridas.\&


{iv}&
De tudo o que eu encontro, escuto, ou vejo,& 
É esse tédio que deriva, e quanto! &
Não, a Beleza não me dá prazer, &
Teus olhos para mim mal têm encanto.\&


{v}&
Esta tristeza imóvel e sem fim &
É a do judeu errante e fabuloso &
Que não verá além da sepultura &
E em vida não terá nenhum repouso.\&


{vi}&
Que exilado --- de si pode fugir? &
Mesmo nas zonas mais e mais distantes,& 
Sempre me caça a praga da existência, &
O Pensamento, que é um demônio, antes.\&


{vii}&
Mas os outros parecem transportar-se &
De prazer e, o que eu deixo, apreciar; &
Possam sempre sonhar com esses arroubos &
E como acordo nunca despertar!\&


{viii}&
Por muitos climas o meu fado é ir-me,& 
Ir-me com um recordar amaldiçoado; &
Meu consolo é saber que ocorra embora &
O que ocorrer, o pior já me foi dado.\&


{ix}&
Qual foi esse pior? Não me perguntes, &
Não pesquises por que é que me consterno!& 
Sorri! não sofras risco em desvendar &
O coração de um homem: dentro é o Inferno.\&
\pagebreak


%Bruno: problemas com o grego. Cf.
\chapter*{Maid of Athens, Ere We Part}


\skipnumbering{\textit{Ζωή μου, σᾶς ἀγαπῶ.}\&

\vspace{1ex}

Maid of Athens, ere we part,& 
Give, oh, give back my heart!& 
Or, since that has left my breast,& 
Keep it now, and take the rest!& 
Hear my vow before I go,& 
\textit{Ζωή μου, σᾶς ἀγαπῶ.}\&


By those tresses unconfined,& 
Wooed by each Aegean wind;& 
By those lids whose jetty fringe& 
Kiss thy soft cheeks' blooming tinge;& 
By those wild eyes like the roe,& 
\textit{Ζωή μου, σᾶς ἀγαπῶ.}\&


By that lip I long to taste; &
By that zone-encircled waist;& 
By all the token-flowers that tell& 
What words can never speak so well;& 
By love's alternate joy and woe,& 
\textit{Ζωή μου, σᾶς ἀγαπῶ.}\&


Maid of Athens! I am gone: &
Think of me, sweet! when alone.& 
Though I fly to Istambol, &
Athens holds my heart and soul:& 
Can I cease to love thee? No!&
\textit{Ζωή μου, σᾶς ἀγαπῶ.}\&
\pagebreak


\chapter{Moça de Atenas, antes da separação\footnote[*]{ O refrão quer
							dizer, como o próprio Byron esclarece, ``Vida minha, eu te
							amo!''. O poema data de Atenas, 1810. Quanto à moça a
							quem foi endereçada, ver a Introdução.}

\vspace*{-2.5\baselineskip}



\skipnumbering{\textit{Ζωή μου, σᾶς ἀγαπῶ.}\&

\vspace{1ex}

Moça de Atenas, antes da separação& 
Dá-me de volta o coração! &
Ou, pois deixou meu peito, lesto, &
Conserva-o tu, e toma o resto! &
Antes que eu parta, a minha jura escuta-a só,& 
{\itshape Ζωή μου, σᾶς ἀγαπῶ.}\&



Por essas tranças desatadas,&
Do vento egeu tão cortejadas, &
Por essas pálpebras --- sua franja de negror& 
Beija-te as faces cor de flor ---,&
Por teus olhos ariscos como a cabra só,& 
{\itshape Ζωή μου, σᾶς ἀγαπῶ.}\&



Por esses lábios --- quero-os fruir ---,&
Por esse peito que uma faixa está a cingir,& 
Pelas flores que vêm falar&
O que a voz não pode igualar, &
Pela alegria e dor que o amor alterna, e só,& 
{\itshape Ζωή μου, σᾶς ἀγαπῶ.}\&



Moça de Atenas, já parti, ó minha vida!& 
Sozinha, pensa em mim, querida!& 
Embora eu vá para Istambul,& 
Atenas prende-me a alma ao sul.& 
Posso deixar de amar-te? Nem uma hora só! &
{\itshape Ζωή μου, σᾶς ἀγαπῶ.}\&
\pagebreak


\chapter*{Oh! Snatched Away in Beauty’s Bloom}


Oh! snatched away in beauty's bloom,&
On thee shall press no ponderous tomb;&
But on thy turf shall roses rear&
Their leaves, the earliest of the year;&
And the wild cypress wave in tender gloom:\&



And oft by yon blue gushing stream&
Shall Sorrow lean her drooping head,&
And feed deep thought with many a dream,&
And lingering pause and lightly tread;&
Fond wretch! as if her step disturbed the dead!\&



Away! we know that tears are vain,&
That death nor heeds nor hears distress:&
Will this unteach us to complain?&
Or make one mourner weep the less?&
And thou --- who tell'st me to forget,&
Thy looks are wan, thine eyes are wet.\&
\pagebreak


\chapter{Oh! Na flor da beleza arrebatada\footnote[*]{ Faz parte das \textit{Hebrew Melodies} 
						(1815). Traduzido por Antônio Franco da Costa Meireles em 
						1869; em prosa, por Manuel dos Reis em
						1872; por J.~A.~de Oliveira Silva em 1875, segundo o
						registro de Onédia Célia.}


Oh! na flor da beleza arrebatada,&
Não há de te oprimir tumba pesada;&
Em tua relva as rosas criarão&
Pétalas, as primeiras que virão,&
E oscilará o cipreste em branda escuridão.\&



E junto da água a fluir azul da fonte &
Inclinará a Tristeza a langue fronte &
E as cismas nutrirá de sonho ardente; &
Pausará lenta, e andará suavemente, &
Como se com seus passos, pobre ente!&
Os mortos perturbasse, mesmo levemente!\&


Basta! sabemos nós que o pranto é vão,&
Que a morte, à nossa dor, não dá atenção.&
Isso fará esquecer-nos de prantear?&
Ou que choremos menos fará então?&
E tu, que dizes para eu me olvidar,&
Teu rosto acha-se pálido, úmido esse olhar.\&
\pagebreak






























\chapter*{On This Day I Complete\break My Thirty-sixth Year}

\skipnumbering \hspace{\fill}\small{\textsc{jan.\,22nd, 1824}

\skipnumbering \hspace{\fill}\small{\textit{Missolonghi}



'Tis time this heart should be unmoved,&
Since others it hath ceased to move:&
Yet, though I cannot be beloved, &
Still let me love!\&



My days are in the yellow leaf;&
The flowers and fruits of love are gone;&
The worm, the canker, and the grief&
Are mine alone!\&



The fire that on my bosom preys&
Is lone as some volcanic isle;&
No torch is kindled at its blaze ---& 
A funeral pile.\&



The hope, the fear, the jealous care,&
The exalted portion of the pain&
And power of love, cannot share,& 
But wear the chain.\&


But 'tis not \textit{thus} --- and 'tis not \textit{here} ---& 
Such thoughts should shake my soul, nor \textit{now}&
Where glory decks the hero's bier,& 
Or binds his brow.\&



The sword, the banner, and the field,&
Glory and Greece, around me see!&
The Spartan, borne upon his shield,&  
Was not more free.\&



Awake! (not Greece --- she \textit{is} awake!)&
Awake, my spirit! Think through \textit{whom}&
Thy life-blood tracks its parent lake, &
And then strike home!\&


Tread those reviving passions down,&
 Unworthy manhood! --- unto thee&
Indifferent should the smile or frown &
 Of beauty be.\&


If thou regret'st thy youth, \textit{why live}?&
The land of honourable death&
Is here: --- up to the field, and give&
 Away thy breath!\&


Seek out --- less often sought than found ---& 
A soldier's grave, for thee the best;&
Then look around, and choose thy ground,& 
 And take thy rest.\&
\pagebreak


\chapter[Neste dia eu completo trinta e seis anos]{Neste dia eu completo\break trinta e seis anos\footnote[*]{ Assinala-se que foi o último poema escrito por Byron.}


\skipnumbering \hspace{\fill}\small{\textsc{22 de janeiro de 1824}

\skipnumbering \hspace{\fill}\small{\textit{Missolonghi}




Meu coração já é tempo de não comover-se, &
Pois outros já cessei de emocionar; &
Mas, embora eu não possa ser amado, &
Que possa ainda amar!\&



Do amor foram-se as flores, mais os frutos;&
De folha amarelada os dias meus estão;&
O verme, a praga e a dor&
Meus somente é que são!\&



A chama que em meu peito faz despojos &
Como ilha de vulcão é solitária; &
Nenhuma tocha é acesa no seu fogo &
--- De pira funerária!\&



Medo, e esperança, e inquietação do ciúme, a augusta&
Porção da mágoa e do poder do amor &
Não podem compartir, mas carregar &
Dos ferros o rigor.\&


Mas não é \textit{assim} --- e não é \textit{aqui}, nem mesmo \textit{agora}&
Que essas ideias deveriam me agitar,&
Quando a glória recobre o féretro do herói& 
Ou vem-lhe a fronte ornar.\&



Bandeira, e espada, e campo de batalha, &
A glória e a Grécia, vede-os em redor de mim!& 
O espartano, levado sobre o escudo,&
Não era livre assim.\&



Desperta! --- Não a Grécia, ei-la desperta! ---&
Mas meu espírito! Pensa através de \textit{quem}&
O teu sangue vital busca o paterno lago, &
E então golpeia bem!\&


Pisa aquelas paixões que ressuscitam,&
Indigna humanidade! --- para ti&
Iguais seriam o sorriso ou a carranca& 
Da beleza em si.\&



\textit{Por que viver, }se carpes tua mocidade? &
Da morte honrosa a terra aqui está &
--- Ergue-te para o campo de batalha,& 
E esvai o alento já!\&



Busca --- menos buscada do que achada&
E para ti a melhor --- a tumba do soldado;& 
Olha depois em torno, escolhe o solo teu, &
E dorme descansado.\&

\pagebreak


\chapter*{Sun of the Sleepless!}


Sun of the sleepless! melancholy star!& 
Whose tearful beam glows tremulously far,& 
That show'st the darkness thou canst not dispel,& 
How like art thou to joy remember'd well!\&


So gleams the past, the light of other days,& 
Which shines, but warms not with its powerless rays;& 
A night-beam Sorrow watcheth to behold, &
Distinct, but distant --- clear --- but, oh how cold!\&
\pagebreak


\chapter{Sol dos insones\footnote[*]{ De \textit{Hebrew Melodies}
					(1815). Além de figurar entre as traduções de Costa Meireles (1869) e
					Oliveira Silva (1875), teve uma tradução pirateada no \textit{Diário do
					Rio de Janeiro}, 19/4/1855. Ver Onédia Célia, p.~54.}



Sol dos insones! Ó astro de melancolia! &
Arde teu raio em pranto, longe a tremular,& 
E expões a treva que não podes dissipar: &
Que semelhante és à lembrança da alegria!\&


Assim raia o passado, a luz de tanto dia,& 
Que brilha sem com raios fracos aquecer;& 
Noturna, uma tristeza vela para ver, &
Distinta mas distante --- clara --- mas que fria!\&
\pagebreak


\chapter*{Written after Swimming\break from Sestos to Abydos}


If, in the month of dark December,&
Leander, who was nightly wont&
(What maid will not the tale remember?)&
To cross thy stream, broad Hellespont!\&


If, when the wintry tempest roared,&
He sped to Hero, nothing loth,&
And thus of old thy curent poured,&
Fair Venus! how I pity both!\&


For \textit{me}, degenerate modern wretch,&
Though in the genial month of May,&
My dripping limbs I faintly stretch,&
And think I've done a feat today.\&


But since he crossed the rapid tide,&
According to the doubtful story,&
To woo, --- and --- Lord knows what beside.&
And swam for Love, as I for Glory;\&


'Twere hard to say who fared the best:&
Sad mortals! thus the Gods still plague you!&
He lost his labour, I my jest;&
For he was drowned, and I've the ague.\&
\pagebreak


\chapter{Escritos após nadar de Sestos a Abidos\footnote[*]{ O poema data
						de 9 de maio de 1810. Byron partiu do lado euro­peu dos Dardanelos
						para a travessia. O feito permaneceu lembra­do localmente, pois,
						segundo Vulliamy, mostraram-lhe, na estrada entre Chanak e
						o promontório de Nagara, ``a casa na qual Lord Byron
						vivia ao tempo em que atravessou a nado os
						Dardanelos''.}


Se, pelo mês sombrio de dezembro,&
(Que moça não se lembrará desse reconto?)& 
Leandro, que toda noite desejava &
Tuas águas cruzar, largo Helesponto!\&


Se ao bramir da procela em pleno inverno&
Leandro buscava Hero, com contentamento, &
E assim em tuas vagas se apressava, &
Bela Vênus! como ambos eu lamento!\&


Pois eu, degênere e infeliz moderno,&
Embora seja maio, à amenidão propenso,& 
Movo débil meus membros encharcados: &
Hoje cumpri uma façanha, eu penso.\&


Já que ele atravessou a maré rápida,&
Do modo como o conta a duvidosa história,&
Para um namoro --- e sabe Deus que mais,&
E nadou por amor --- e eu pela glória,\&


Duro é dizer quem se saiu melhor:&
Assim os deuses, ó mortais, punindo vão!& 
Ele perdeu o esforço, eu a graçola: &
Ele afogou-se, ataca-me a sezão.\&
\pagebreak


\chapter*{Sonnet on Chillon}


Eternal Spirit of the chainless Mind!&
Brightest in dungeons, Liberty! thou art;&
For there thy habitation is the heart ---& 
The heart which love of thee alone can bind;\&


And when thy sons to fetters are consigned ---& 
To fetters, and the damp vault's dayless gloom,&
Their country conquers with their martyrdom,&
And Freedom's fame finds wings on every wind.\&


Chillon! thy prison is a holy place,&
And thy sad floor an altar --- for 'twas trod,&
Until his very steps have left a trace&
Worn, as if thy cold pavement were a sod,\&


By Bonnivard! --- May none those marks efface!&
For they appeal from tyranny to God.\&

\pagebreak


\chapter{Soneto sobre Chillon}


Eterno Espírito da insubjugável Mente! &
No calabouço brilhas mais, ó Liberdade, &
Pois lá no coração habitas de verdade, &
No coração que prende o teu amor somente.\&


Quando teus filhos são entregues à corrente&
E ao perpétuo negror de úmida cavidade,&
A pátria vence com o martírio da hombridade,&
E a fama de ser livre ao vento se ala, ardente!\&


Chillon! tua prisão é um lugar sagrado&
E altar teu triste chão, pois ele foi pisado&
--- Até gastar-se com o vestígio de seu passo,\&


Qual se fosse de terra o piso nesse espaço ---& 
{Por Bonnivard! Ninguém apague os rastros seus}%
						\footnote { François
						de Bonnivard (Seyssel, 1493-Genebra, 1570), historiador e
						patriota genebrino, esteve encarcerado por seis anos
						(1530--1536) no castelo de Chillon, junto ao lago Leman, por
						Carlos \textsc{iii}, duque de Saboia, sendo afinal libertado pelos
						bernenses. O soneto abre ``The Prisoner of Chillon'', que
						Byron escreveu em junho de 1816, na Suíça, quando não pôde, por causa
						do mau tempo, sair durante dois dias de um pequeno albergue na aldeia
						de Ouchy, perto de Lausanne.}%
							,& 
Pois apelam da tirania para Deus!\&

\quebra

\chapter*{And Thou Art Dead, as Young as Fair}


\textit{Heu, quanto minus est cum reliquis versari quam tui meminisse!}\&



\skipnumbering{And thou art dead, as young and fair,}& 
As aught of mortal birth;&
And form so soft, and charms so rare,& 
Too soon returned to Earth!&
Though Earth received them in her bed,&
And o'er the spot the crowd may tread&
In carelessness or mirth,&
There is an eye which could not brook&
A moment on that grave to look.\&


I will not ask where thou liest low, &
Nor gaze upon the spot;&
There flowers or weeds at will may grow,& 
So I behold them not:&
It is enough for me to prove&
That what I loved, and long must love,& 
Like common earth can rot;&
To me there needs no stone to tell,&
'Tis Nothing that I loved so well.\&

Yet did I love thee to the last&
As fervently as thou,&
Who didst not change through all the past,& 
And canst not alter now.&
The love where Death has set his seal,&
Nor age can chill, nor rival steal, &
Nor falsehood disavow:&
And, what were worse, thou canst not see&
Or wrong, or change, or fault in me.\&


The better days of life were ours;&
The worst can be but mine:&
The sun that cheers, the storm that lowers,& 
Shall never more be thine.&
The silence of that dreamless sleep&
I envy now too much to weep;&
Nor need I to repine&
That all those charms have passed away;&
I might have watched through long decay.\&


The flower in ripened bloom unmatched &
Must fall the earliest prey;&
Though by no hand untimely snatched,& 
The leaves must drop away:&
And yet it were a greater grief&
To watch it withering, leaf by leaf,& 
Than see it plucked today;&
Since earthly eye but ill can bear&
To trace the change to foul from fair.\&



I know not if I could have borne &
To see thy beauties fade;&
The night that followed such a morn& 
Had worn a deeper shade:&
Thy day without a cloud hath passed,&
And thou wert lovely to the last;&
Extinguished, not decayed;&
As stars that shoot along the sky&
Shine brightest as they fall from high.\&


As once I wept, if I could weep,&
My tears might well be shed,&
To think I was not near to keep& 
One vigil o'er thy bed;&
To gaze, how fondly! on thy face,&
To fold thee in a faint embrace,&
Uphold thy drooping head;&
And show that love, however vain,&
Nor thou nor I can feel again.\&


Yet how much less it were to gain,&
Though thou hast left me free,&
The loveliest things that still remain,&
Than thus remember thee!&
The all of thine that cannot die&
Through dark and dread Eternity&
Returns again to me,&
And more thy buried love endears&
Than aught, except its living years.\&
\pagebreak































\chapter{E morreste tão jovem e formosa\footnote[*]{ O poema é datado de fevereiro de 1812. A epígrafe significa: ``Ai, quão menos é viver com os que restaram do que recordar-te!''}


\textit{Heu, quanto minus est cum reliquis versari quam tui meminisse!}\&
\vspace{1em}



\skipnumbering{E morreste --- tão jovem e formosa ---} &
Tal como tudo que nasceu mortal;&
Tão suave em formas, e em primores tão preciosa& 
Cedo tornaste à terra maternal! &
Possa a terra guardar-te no seu leito& 
E a multidão nele pisar, de jeito &
Descuidoso ou jovial, &
Não poderei eu suportar&
Um momento sequer o teu sepulcro olhar.\&



Onde jazes não buscarei saber,&
Nem sobre o teu jazigo a vista baixarei;&
Nele flores ou ervas poderão crescer, &
Que não as olharei; &
Bastante para mim é perceber& 
Que a amada --- e longamente a devo amar ---& 
Como terra comum vai terminar:&
Não preciso de pedra que me persuada&
Que aquela que eu amava tanto não é nada.\&


Contudo amei-te, até tudo acabado,&
E como tu fervidamente;&
Jamais mudaste ao longo do passado&
E não podes mudar na hora presente.&
O amor no qual a morte põe seu selo&
Não pode esfriá-lo a idade, nem rival havê-lo,& 
Nem perfídia o desmente.&
E, o que seria pior, não podes ver assim& 
Erro, mudança ou falta em mim.\&



Foram nossos os dias bons da vida;& 
Podem os piores ser somente meus;&
O sol que alegra, a tempestade que intimida,& 
Nunca mais serão teus.&
O silêncio sem sonhos desse teu dormir,& 
Ora o invejo demais para carpir;&
Nem preciso chorar que tenham dito adeus& 
Os teus encantos florescentes:&
Melhor assim que vê-los sempre decadentes.\&



Sem par em sua plena formosura, a flor&
Deve tombar como primeira presa;&
As pétalas cairão em seu primor,&
Mesmo que mão nenhuma colha essa beleza,&
E embora fosse maior dor&
Ver cada pétala de flor fanada&
Que vê-la hoje, súbito, apanhada,&
Uma vez que suporta mal, o humano olhar,& 
Ver o belo no feio se mudar.\&



Eu não sei se teria suportado&
Ver a tua beleza se murchar:&
Mostrou um tom de treva mais fechado&
A noite que seguiu essa manhã sem par.&
O teu dia sem nuvem se passou assim & 
E foste encantadora até o fim: &
Extinta, sem porém se degradar &
Como a estrela, que o céu riscando,& 
Resplende mais quando tombando.\&



Pudesse ora eu chorar, como chorei outrora,&
Bem vertido seria o pranto meu,&
Pois não estava eu perto nessa hora&
Para manter vigília sobre o leito teu;& 
Para teu rosto ternamente olhar, &
Para num leve abraço te estreitar;&
Para a fronte inclinada te apoiar, sim, eu,&
Para mostrar-te amor, embora inutilmente, &
Pois não podemos tê-lo novamente.\&



Embora livre me hajas tu deixado, &
Como seria menos alcançar&
Tudo o que seja encantador de ser notado,& 
Bem menos do que assim te recordar!&
Tudo de ti que é imperecível&
Na Eternidade tão sombria, tão terrível,&
Eis para mim a retornar,&
E vale mais que tudo o teu sepulto amor,&
Tirando os anos que viveu em seu dulçor.\&
\pagebreak



\chapter*{Bright Be the Place of Thy Soul}


Bright be the place of thy soul!&
No lovelier spirit than thine&
E'er burst from its mortal control,&
In the orbs of the blessed to shine.\&


On earth thou wert all but divine,&
As thy soul shall immortally be;&
And our sorrow may cease to repine,&
When we know that thy God is with thee.\&


Light be the turf of thy tomb!&
May its verdure like emeralds be:&
There should not be the shadow of gloom&
In aught that reminds us of thee.\&


Young flowers and an evergreen tree&
May spring from the spot of thy rest:&
But nor cypress nor yew let us see;&
For why should we mourn for the blest?\&
\pagebreak






























\chapter{Brilhante seja o pouso de tua alma\footnote[*]{ O poema data de 1808.}


Brilhante seja o pouso de tua alma! &
Espírito nenhum mais atraente &
Rompeu jamais o invólucro mortal& 
Para esplender na esfera transcendente.\&


Em vida quase que eras tu divina, &
E tua alma imortalmente o deve ser;& 
Se, sabemos, teu Deus está contigo, &
Deixe nossa tristeza de sofrer.\&


Leve seja o gramado em teu sepulcro! &
Fulgure de esmeraldas a folhagem! &
Não caia sombra de melancolia &
Sobre o que nos evoque tua imagem.\&


Árvore sempre verde, mais botões, &
Desse teu chão vejamo-los brotados;& 
Mas não vejamos teixo nem cipreste: &
Por que chorar os bem-aventurados?\&
\pagebreak






























\chapter*{The ocean}



{clxxix}&
Roll on, thou deep and dark blue Ocean --- roll!& 
Ten thousand fleets sweep over thee in vain; &
Man marks the earth with ruin --- his control &
Stops with the shore; upon the watery plain &
The wrecks are all thy deed, nor doth remain &
A shadow of man's ravage, save his own, &
When, for a moment, like a drop of rain, &
He sinks into thy depths with bubbling groan,& 
Without a grave, unknell'd, uncoffin'd, and unknown.\&


{clxxx}&
His steps are not upon thy paths, --- thy fields &
Are not a spoil for him, --- thou dost arise &
And shake him from thee; the vile strength he wields& 
For earth's destruction thou dost all depise, &
Spurning him from thy bosom to the skies, &
And send'st him, shivering in thy playful spray& 
And howling, to his Gods, where haply lies &
His petty hope in some near port or bay, &
And dashest him again to earth: --- there let him lay.\&


{clxxxi}&
The armaments which thunderstrike the walls &
Of rock-built cities, bidding nations quake,& 
And monarchs tremble in their capitals, &
The oak leviathans, whose huge ribs make &
Their clay creator the vain title take &
Of lord of thee, and arbiter of war ---& 
These are thy toys, and, as the snowy flake,& 
They melt into thy yeast of waves, which mar &
Alike the Armada's pride or spoils of Trafalgar.\&


{clxxxii}&
Thy shores are empires, changed in all save thee ---& 
Assyria, Greece, Rome, Carthage, what are they? &
Thy waters wash'd them power while they were free,& 
And many a tyrant since; their shores obey &
The stranger, slave, or savage; their decay &
Has dried up realms to deserts: --- not so thou;&
Unchangeable, save to thy wild waves' play,& 
Time writes no wrinkle on thine azure brow:&
Such as creation's dawn beheld, thou rollest now.\&


{clxxxiii}&
Thou glorious mirror, where the Almighty's form &
Glasses itself in tempests; in all time, ---& 
Calm or convulsed, in breeze, or gale, or storm,& 
Icing the pole, or in the torrid clime &
Dark-heaving --- boundless, endless, and sublime,& 
The image of eternity, the throne &
Of the Invisible; even from out thy slime& 
The monsters of the deep are made; each zone& 
Obeys thee; thou goest forth, dread, fathomless, alone.\&



{clxxxiv}&
And I have loved thee, Ocean! and my joy &
Of youthful sports was on thy breast to be& 
Borne, like thy bubbles, onward: from a boy &
I wanton'd with thy breakers --- they to me& 
Were a delight; and if the freshening sea &
Made them a terror --- 'twas a pleasing fear,& 
For I was as it were a child of thee, &
And trusted to thy billows far and near,& 
And laid my hand upon thy mane --- as I do here.\&

\pagebreak


\chapter{O oceano\footnote[*]{ (\textit{Childe Harold's Pilgrimage}, Canto \textsc{iv}). Em seu ``O
						Mar'', de \textit{Vozes d'América}, poema aliás
						impetuoso e vivo, Fagundes Varela demonstra, num ou noutro verso,
						que conhecia este trecho do \textit{Childe Harold}. O tema do mar não
						era alheio ao nosso Romantismo: já Gonçalves Dias dele tratava num de
						seus hinos.}


{clxxix}&
Rola, Oceano profundo e azul sombrio, rola! &
Caminham dez mil frotas sobre ti, em vão; &
de ruínas o homem marca a terra, mas se evola& 
na praia o seu domínio. Na úmida extensão &
só tu causas naufrágios; não, da destruição& 
feita pelo homem sombra alguma se mantém, &
exceto se, gota de chuva, ele também &
se afunda a borbulhar com seu gemido, &
sem féretro, sem túmulo, desconhecido.\&


{clxxx}&
Do passo do homem não há traço em teus caminhos, &
nem são presa teus campos. Ergues-te e o sacodes& 
de ti; desprezas os poderes tão mesquinhos &
que usa para assolar a terra, já que podes &
de teu seio atirá-lo aos céus; assim o lanças& 
tremendo e uivando em teus borrifos escarninhos &
rumo a seus deuses --- nos quais firma as esperanças&
de achar um porto ou angra próxima, talvez ---&
e o devolves à terra: --- jaza aí, de vez.\&


{clxxxi}&
Os armamentos que fulminam as muralhas &
das cidades de pedra --- e tremem as nações &
ante eles, como os reis em suas capitais ---,& 
os leviatãs de roble, cujas proporções &
levam o seu criador de barro a se apontar& 
como Senhor do Oceano e árbitro das batalhas,& 
fundem-se todos nessas ondas tão fatais &
para a orgulhosa Armada ou para Trafalgar.\&


{clxxxii }&
Tuas bordas são reinos, mas o tempo os traga: &
Grécia, Roma, Cartago, Assíria, onde é que estão?& 
Quando outrora eram livres tu as devastavas, &
e tiranos copiaram-te, a partir de então; &
manda o estrangeiro em praias rudes ou escravas;& 
reinos secaram-se em desertos, nesse espaço, &
mas tu não mudas, salvo no florear da vaga; &
em tua fronte azul o tempo não põe traço; &
como és agora, viu-te a aurora da criação.\&


{clxxxiii}&
Tu, espelho glorioso, onde no temporal &
reflete sua imagem Deus onipotente; &
calmo ou convulso, quando há brisa ou vendaval, &
quer a gelar o polo, quer em clima ardente &
a ondear sombrio, --- tu és sublime e sem final,& 
cópia da eternidade, trono do Invisível; &
os monstros dos abismos nascem do teu lodo;& 
todas as zonas te obedecem: porque és todo &
insondável, sozinho avanças, és terrível.\&


{clxxxiv}&
Amei-te, Oceano! Em meus folguedos juvenis &
ir levado em teu peito, como tua espuma, &
era um prazer; desde meus tempos infantis &
divertir­-me com as ondas dava-me alegria;& 
quando, porém, ao refrescar-se o mar, alguma& 
de tuas vagas de causar pavor se erguia, &
sendo eu teu filho esse pavor me seduzia &
e era agradável: nessas ondas eu confiava &
e, como agora, a tua juba eu alisava.\&
\pagebreak


\chapter*{Stanzas to Augusta}


Though the day of my destiny's over,&
And the star of my fate hath declined,&
Thy soft heart refused to discover&
The faults which so many could find;&
Though thy soul with my grief was acquainted,&
It shrunk not to share it with me,&
And the love which my spirit hath painted&
It never hath found but in \textit{thee.}\&


Then when nature around me is smiling,&
The last smile which answers to mine,&
I do not believe it beguiling,&
Because it reminds me of thine;&
And when winds are at war with the ocean,&
As the breasts I believed in with me,&
If their billows excite an emotion,&
It is that they bear me from \textit{thee.}\&


Though the rock of my last hope is shivered,&
And its fragments are sunk in the wave,&
Though I feel that my soul is delivered&
To pain --- it shall not be its slave.&
There is many a pang to pursue me:&
They may crush, but they shall not contemn;&
They may torture, but shall not subdue me ---& 
'Tis of \textit{thee }that I think --- not of them.\&


Though human, thou didst not deceive me,&
Though woman, thou didst not forsake,&
Though loved, thou forborest to grieve me,&
Though slandered, thou never couldst shake;&
Though trusted, thou didst not disclaim me,&
Though parted, it was not to fly,&
Though watchful, 'twas not to defame me,&
Nor, mute, that the world might belie.\&



Yet I blame not the world, nor despise it,&
Nor the war of the many with one;&
If my soul was not fitted to prize it,&
'Twas folly not sooner to shun:&
And if dearly that error hath cost me,&
And more than I once could foresee,&
I have found that, whatever it lost me,&
It could not deprive me of \textit{thee.}\&


From the wreck of the past, which hath perished&
Thus much I at least may recall,&
It hath taught me that what I most cherished&
Deserved to be dearest of all:&
In the desert a fountain is springing,&
In the wide waste there still is a tree,&
And a bird in the solitude singing,&
Which speaks to my spirit of \textit{thee.}\&
\pagebreak


\chapter{Estâncias para Augusta}


Embora, concluído o dia de meu fado,&
A estrela desta sina tenha declinado,& 
Teu brando coração se recusou a achar &
As faltas que puderam, tantos, encontrar;& 
Embora tua alma conhecesse a minha dor, &
Não recusou comigo partilhá-la, e o amor& 
Que minha mente ideou, pintando-o para si,& 
Jamais ela o encontrou, nunca, a não ser em ti.\&


Quando sorri a natureza em torno a mim &
Esse último sorrir, que ao meu responde assim,& 
Não acredito que ele seja enganador, &
Porquanto me recorda o teu, em seu frescor; &
E quando em guerra os ventos se erguem contra o mar,& 
Tal como os peitos, em que eu cria, a me atacar, &
Se as vagas inda me despertam emoção,&
É que afastando-­me de ti elas estão.\&


Embora a rocha da esperança haja estalado& 
E os seus pedaços n'água tenham-se afundado,& 
Embora eu sinta entregue à dor meu coração, &
Ele não há de dar-se a ela em servidão; &
Muitas angústias há, tantas, a me acossar; &
Se podem me esmagar, não podem desprezar; &
Podem me torturar, não me subjugarão&
--- Eu penso em ti unicamente, nelas não.\&


Embora humana, em tempo algum tu me enganaste,&
Mulher embora, tu jamais me abandonaste, &
Embora amada, tu evitaste me afligir, &
Embora caluniada, firme em resistir, &
Embora eu cresse em ti, tu não me repudiaste;& 
Não foi para fugir que um dia te apartaste; &
Embora atenta, não para me denegrir; &
Nem silenciaste para o mundo então mentir.\&


Contudo eu não censuro nem desprezo a terra&
Nem, contra um, da multidão censuro a guerra;& 
Para apreciar tal coisa não fui eu formado, &
Foi loucura eu não ter mais cedo me afastado;& 
E se caro esse erro veio a me custar, &
Bem mais do que algum dia eu pude suspeitar,& 
Por mais que me fizesse --- achava eu --- perder,& 
De ti não poderia nunca me tolher.\&


Vi o passado perecer e naufragar &
E dele eu posso ao menos isto recordar:& 
Aquilo --- ensinou-me ele --- que mais eu queria,& 
Ser o mais caro, em meio a tudo, merecia: &
No deserto uma fonte --- eu vejo --- está brotando,& 
Uma árvore no ermo ainda frondejando, &
E um pássaro cantando em meio à solidão&
{Que me fala de ti à mente e ao coração}%
						\footnote { Estas estâncias datam de 24 de julho de 1816; 
						contemporâneas da ``Carta'', à qual 
						sobrelevam como lirismo, são também ambíguas e autobiográficas.}%
						.\&

\quebra

\chapter*{Stanzas}

Oh, talk not to me of a name great in story;& 
The days of your youth are the days of our glory;& 
And the myrtle and ivy of sweet two-and-twenty& 
Are worth all your laurels, though ever so plenty.\&


What are garlands and crowns to the brow that is wrinkled?& 
'Tis but as a dead-flower with May-dew besprinkled. &
Then away with all such from the head that is hoary! &
What care I for the wreaths that can \textit{only} give glory?\&


Oh Fame! --- if I e'er took delight in thy praises,& 
'Twas less for the sake of thy high-sounding phrases,& 
Than to see the bright eyes of the dear one discover &
She thought that I was not unworthy to love her.\&


\textit{There }chiefly I sought thee, \textit{there} only I found thee;& 
Her glance was the best of the rays that surround thee; &
When it sparkled o'er aught that was bright in my story,& 
I knew it was love, and I felt it was glory.\&
\pagebreak


\chapter{Estâncias\footnote[*]{ Escritas na estrada entre Florença e Pisa. O poema data de novembro de 1821.}



Não me faleis de grande nome ter na história;&
Nossos dias de moço os dias são de nossa glória; &
Hera e mirto dos vinte e dois anos viçosos &
Valem todos os louros, mesmo se copiosos.\&



Que são guirlandas para a testa já enrugada,&
Que é como extinta flor por maio rorejada? &
Longe com isso, pois, da fronte encanecida!&
Não me importam coroas que \textit{só} trazem glória à vida!\&



Fama! Se teus louvores me agradaram antes,&
Foi menos por tuas frases tão altissonantes&
Que para ver a amada expor em seu olhar&
Que ela não me julgava desmerecedor de a amar.\&



\textit{Lá} mais eu te busquei; \textit{lá} só, eu te encontrei;&
O melhor raio que te cerca, em seu olhar achei;&
Quando faiscava sobre um quê brilhante em minha história,&
Eu bem sabia que era o amor, sentia que era a glória.\&
\pagebreak


\chapter*{Stanzas for Music}


There be none of Beauty's daughters &
With a magic like thee;&
And like music on the waters&
Is thy sweet voice to me:&
When, as if its sound were causing&
The charmed ocean's pausing,&
The waves lie still and gleaming,&
And the lulled winds seem dreaming:\&


And the midnight moon is weaving&
Her bright chain o'er the deep;&
Whose breast is gently heaving, &
As an infant's asleep:&
So the spirit bows before thee,&
To listen and adore thee;&
With a full but soft emotion,&
Like the swell of Summer's ocean.\&
\pagebreak


\chapter{Estâncias para música\footnote[*]{ Segundo a tradição, este
					poema, escrito em março de 1816, dirigia-se a Claire
					Clairmont, um dos casos de Byron, que com ele teve a filha Allegra.}



Filha não há da Formosura &
Que te iguale em magia; &
E como sobre as águas música macia,& 
A tua voz tem para mim doçura, &
Quando, como se ao soar houvesse provocado& 
O enlevo do oceano encantado, &
As ondas põem-se quietas, quase sem brilhar,& 
E os ventos acalmados, como que a sonhar:\&



E à meia-noite a lua mostra-se a tecer& 
Sua corrente fulgurante sobre o mar, &
Cujo peito se põe suavemente a arfar, &
Como criança que se fez adormecer: &
Assim o espírito se inclina à tua frente &
Para te ouvir e te adorar unicamente, &
Com plena, mas suave emoção, &
Como a vaga do oceano de verão.\&
\pagebreak


\chapter*{Euthanasia}


When Time, or soon or late, shall bring&
The dreamless sleep that lulls the dead,&
Oblivion! may thy languid wing&
Wave gently o'er my dying bed!\&


No band of friends or heirs be there,&
To weep, or wish, the coming blow:&
No maiden, with dishevelled hair,&
To feel, or feign, decorous woe.\&


But silent let me sink to earth,&
With no officious mourners near:&
I would not mar one hour of mirth,&
Nor startle friendship with a tear.\&


Yet Love, if Love in such an hour&
Could nobly check its useless sighs,&
Might then exert its latest power&
In her who lives, and him who dies.\&


'Twere sweet, my Psyche! to the last&
Thy features still serene to see:&
Forgetful of its struggles past,&
E'en Pain itself should smile on thee.\&


But vain the wish --- for Beauty still&
Will shrink, as shrinks the ebbing breath;&
And women's tears, produced at will,&
Deceive in life, unman in death.\&


Then lonely be my latest hour,&
Without regret, without a groan;&
For thousands Death hath ceas'd to lower,&
And pain been transient or unknown.\&



`Ay, but to die, and go,' alas!&
Where all have gone, and all must go! &
To be the nothing that I was&
Ere born to life and living woe!\&


Count o'er the joys thine hours have seen,&
Count o'er thy days from anguish free, &
And know, whatever thou hast been,&
'Tis something better not to be.\&
\pagebreak


\chapter{Eutanásia\footnote[*]{ O texto deve ser de 1811. Em nosso
					Romantismo o poema foi parafraseado por Francisco Otaviano, com dicção
					ultra-romântica, em decassílabos brancos, e traduzido por
					Carlos de Meneses e Sousa Júnior, em quadras alexandrinas de rimas
					cruzadas. Onédia Célia cuida da paráfrase de Otaviano.}


 Quando o Tempo trouxer, ou cedo ou tarde, &
Esse sono sem sonhos para me embalar, &
Sobre meu leito de agonia possa, Olvido!& 
Tua asa langue levemente tremular.\&



Nem róis de amigos nem de herdeiros lá estarão,&
Para chorar, ou desejar, o que há de vir,&
Nem virgem de cabelo desgrenhado&
Para sentir dor decorosa, ou bem fingir.\&



Sem ter por perto carpidores oficiosos, &
Deixai que a terra me recubra silencioso:& 
 Que eu não tire à amizade uma só lágrima,& 
Que eu não estrague um só momento jubiloso.\&



Contudo o Amor, se o Amor em tal momento &
Seus inúteis soluços nobre contivesse, &
Poderia mostrar a sua força última &
Na que ficasse viva ou no que então morresse.\&



Seria doce até o fim, ó minha Psique!&
Ver as tuas feições ainda serenas; &
Para ti sorriria a própria Dor, &
Esquecida de suas idas penas.\&



Mas é vão o desejo --- dado que a Beleza &
Se retrairá, ao esgotar-se o último alento;& 
E o pranto da mulher, que rola a bel-prazer,& 
Engana em vida, abate no último momento.\&



Sozinho eu fique pois em minha hora final,&
 Sem que mostre pesar, sem um gemido;&
Para milhares já não franze o cenho a Morte,&
E passageira ou ignorada a dor tem sido.\&



``Ah! morrer todavia e ir-se para sempre!''& 
Aonde todos foram já ou devem ir!&
 Ser o nada que eu era, anteriormente &
A nascer para a vida e para a dor curtir!\&



As alegrias conta que tuas horas viram,&
Conta os teus dias sem nenhum sofrer:&
E sabe, não importa o que hajas sido, &
É bem melhor não ser.\&
\pagebreak



\chapter*{Darkness}


I had a dream, which was not all a dream. &
The bright sun was extinguish'd, and the stars &
Did wander darkling in the eternal space, &
Rayless, and pathless, and the icy earth &
Swung blind and blackening in the moonless air;& 
Morn came and went --- and came, and brought no day,& 
And men forgot their passions in the dread &
Of this their desolation; and all hearts &
Were chill'd into a selfish prayer for light:& 
And they did live by watchfires --- and the thrones, &
The palaces of crowned kings --- the huts, &
The habitations of all things which dwell,& 
Were burnt for beacons; cities were consumed,& 
And men were gather'd round their blazing homes& 
To look once more into each other's face; &
Happy were those who dwelt within the eye &
Of the volcanos, and their mountain-torch:& 
A fearful hope was all the world contain'd;& 
Forests were set on fire --- but hour by hour &
They fell and faded --- and the crackling trunks& 
Extinguish'd with a crash --- and all was black.& 
The brows of men by the despairing light \&


Wore an unearthly aspect, as by fits &
The flashes fell upon them; some lay down& 
And hid their eyes and wept; and some did rest& 
Their chins upon their clenched hands, and smiled;& 
And others hurried to and fro, and fed &
Their funeral piles with fuel, and look'd up & 
With mad disquietude on the dull sky, &
The pall of a past world; and then again& 
With curses cast them down upon the dust, &
And gnash'd their teeth and howl'd: the wild birds shriek'd &
And, terrified, did flutter on the ground, &
And flap their useless wings; the wildest brutes& 
Came tame and tremulous; and vipers crawl'd& 
And twined themselves among the multitude.&
Hissing, but stingless --- they were slain for food.&
And War, which for a moment was no more,&
Did glut himself again: --- a meal was bought&
With blood, and each sate sullenly apart&
Gorging himself in gloom: no love was left;&
All earth was but one thought --- and that was death&
Immediate and inglorious; and the pang&
Of famine fed upon all entrails --- men&
Died, and their bones were tombless as their flesh;&
The meagre by the meagre were devour'd,&
Even dogs assail'd their masters, all save one,&
And he was faithful to a corse, and kept&
The birds and beasts and famish'd men at bay,&
Till hunger clung them, or the dropping dead&
Lured their lank jaws; himself sought out no food,&
But with a piteous and perpetual mean,&
And a quick desolate cry, licking the hand&
Which answer'd not with a caress --- he died.&
The crowd was famish'd by degrees; but two&
Of an enormous city did survive,&
And they were enemies: they met beside\&


The dying embers of an altar-place&
Where had been heap'd a mass of holy things &
For an unholy usage; they raked up,&
And shivering scraped with their cold skeleton hands&
The feeble ashes, and their feeble breath&
Blew for a little life, and made a flame&
Which was a mockery; then they lifted up&
Their eyes as it grew lighter, and beheld&
Each other's aspects --- saw, and shriek'd, and died ---& 
Even of their mutual hideousness they died,&
Unknowing who he was upon whose brow&
Famine had written Fiend. The world was void, &
The populous and the powerful was a lump,&
Seasonless, herbless, treeless, manless, lifeless,&
A lump of death --- a chaos of hard clay.&
The rivers, lakes, and ocean all stood still,&
And nothing stirr'd within their silent depths;&
Ships sailorless lay rotting on the sea,&
And their masts fell down piecemeal: as they dropp'd& 
They slept on the abyss without a surge ---& 
The waves were dead; the tides were in their grave,& 
The moon, their mistress, had expired before; &
The winds were wither'd in the stagnant air,& 
And the clouds perish'd; Darkness had no need& 
Of aid from them --- She was the Universe.\&
\pagebreak



\chapter{Trevas\footnote[*]{O poema foi composto na Villa Diodati, em julho de 1816, e entre nós foi 
						traduzido por Castro Alves com bastante felicidade,
						como opina Onédia Célia, que considera o original um belo poema, e um 
						dos únicos de Byron que ``apresenta realmente alguns
						elementos que o podem colocar dentro da linha desse romantismo hórrido, 
						que acabou por se confundir no romantismo paulista com byronismo''.}


Eu tive um sonho que não era em tudo um sonho. &
O sol esplêndido extinguira-se, e as estrelas &
Vaguejavam escuras pelo espaço eterno, &
Sem raios nem roteiro, e a enregelada terra& 
Girava cega e negrejante no ar sem lua; &
Veio e foi-se a manhã --- veio e não trouxe o dia;&
E os homens esqueceram as paixões, no horror &
Dessa desolação; e os corações esfriaram &
Numa prece egoísta que implorava luz: &
E eles viviam ao redor do fogo; e os tronos,& 
Os palácios dos reis coroados, as cabanas, &
As moradas, enfim, do gênero que fosse, &
Em chamas davam luz; cidades consumiam-se& 
E os homens se juntavam junto às casas ígneas& 
Para ainda uma vez olhar o rosto um do outro; &
Felizes quantos residiam bem à vista &
Dos vulcões e de sua tocha montanhosa;& 
Expectativa apavorada era a do mundo; &
Queimavam-se as florestas --- mas de hora em hora& 
Tombavam, desfaziam-se --- e, estralando, os troncos& 
Findavam num estrondo --- e tudo era negror. &
À luz desesperante a fronte dos humanos \&



Tinha um aspecto não terreno, se espasmódicos& 
Neles batiam os clarões; alguns, por terra, &
Escondiam chorando os olhos; apoiavam &
Outros o queixo às mãos fechadas, e sorriam;& 
Muitos corriam para cá e para lá, &
Alimentando a pira, e a vista levantavam & 
Com doida inquietação para o trevoso céu,& 
A mortalha de um mundo extinto; e então de novo &
Com maldições olhavam para a poeira, e uivavam,& 
Rangendo os dentes; e aves bravas davam gritos &
E cheias de terror voejavam junto ao solo, &
Batendo asas inúteis; as mais rudes feras &
Chegavam mansas e a tremer; rojavam víboras,& 
E entrelaçavam-se por entre a multidão,&
Silvando, mas sem presas --- e eram devoradas.& 
E fartava-se a Guerra que cessara um tempo, &
E qualquer refeição comprava-se com sangue; &
E cada um sentava-se isolado e torvo, &
Empanturrando-se no escuro; o amor findara;& 
A terra era uma ideia só --- e era a de morte &
Imediata e inglória; e se cevava o mal &
Da fome em todas as entranhas; e morriam& 
Os homens, insepultos sua carne e ossos; &
Os magros pelos magros eram devorados, &
Os cães salteavam os seus donos, exceto um,& 
Que se mantinha fiel a um corpo, e conservava& 
Em guarda as bestas e aves e os famintos homens,& 
Até a fome os levar, ou os que caíam mortos &
Atraírem seus dentes; ele não comia, &
Mas com um gemido comovente e longo, e um grito& 
Rápido e desolado, e relambendo a mão &
Que já não o agradava em paga --- ele morreu.& 
Finou-se a multidão de fome, aos poucos; dois,& 
Porém, de uma cidade enorme resistiram, &
Dois inimigos, que vieram a encontrar-se\& 



Junto às brasas agonizantes de um altar &
Onde se haviam empilhado coisas santas &
Para um uso profano; eles as revolveram &
E trêmulos rasparam, com as mãos esqueléticas,& 
As débeis cinzas, e com um débil assoprar &
Para viver um nada, ergueram uma chama &
Que não passava de arremedo; então alçaram& 
Os olhos quando ela se fez mais viva, e espiaram& 
O rosto um do outro --- ao ver, gritaram e morreram & 
--- Morreram de sua própria e mútua hediondez, &
Sem um reconhecer o outro em cuja fronte &
Grafara a fome Diabo. O mundo se esvaziara,& 
O populoso e forte era uma informe massa, &
Sem estações nem árvore, erva, homem, vida,& 
Massa informe de morte --- um caos de argila dura.& 
Pararam lagos, rios, oceanos: nada &
Mexia em suas profundezas silenciosas;& 
Sem marujos, no mar as naus apodreciam,&
Caindo os mastros aos pedaços; e, ao caírem,&
Dormiam nos abismos sem fazer mareta, &
Mortas as ondas, e as marés na sepultura,& 
Que já findara sua lua senhoril. &
Os ventos feneceram no ar inerte, e as nuvens& 
Tiveram fim; a Escuridão não precisava &
De seu auxílio --- as Trevas eram o Universo.\&
\pagebreak



\chapter*{Stanzas for Music}

There's not a joy the world can give like that it takes away,&
When the glow of early thought declines in feeling's dull decay;&
'Tis not on youth's smooth cheek the blush alone, which fades so \qb{}fast,&
But the tender bloom of heart is gone, ere youth itself be past.\&


Then the few whose spirits float above the wreck of happiness&
Are driven o'er the shoals of guilt or ocean of excess:&
The magnet of their course is gone, or only points in vain&
The shore to which their shivered sail shall never stretch again.\&


Then the mortal coldness of the soul like death itself comes down;&
It cannot feel for others' woes, it dare not dream its own;&
That heavy chill has frozen o'er the fountain of our tears,&
And though the eye may sparkle still, 'tis where the ice appears.\&


Though wit may flash from fluent lips, and mirth distract  the \qb{}breast,&
Through midnight hours that yield no more their former hope \qb{}of rest;&
'Tis but as ivy-leaves around the ruined turret wreath,&
All green and wildly fresh without, but worn and grey beneath.\&


Oh, could I feel as I have felt, --- or be what I have been,&
Or weep as I could once have wept, o'er many a vanished scene;&
As springs in deserts found seem sweet, all brackish though they \qb{}be,&
So, midst the withered waste of life, those tears would flow to me.\&
\pagebreak



\chapter{Estâncias para música\footnote[*]{ Datam de março de 1815. Trazem
						no original quatro versos latinos dos \textit{Poemata}, de Gray, como epígrafe:
						cuidam de como é feliz quem sente jorrar no íntimo a fonte das lágrimas.}



Alegria não há que o mundo dê, como a que tira. &
Quando, do pensamento de antes, a paixão expira &
Na triste decadência do sentir;&
Não é na jovem face apenas o rubor& 
Que esmaia rápido, porém do pensamento a flor& 
Vai-se antes de que a própria juventude possa ir.\&




Alguns cuja alma boia no naufrágio da ventura &
Aos escolhos da culpa ou mar do excesso são levados;& 
O ímã da rota foi-se, ou só e em vão aponta a obscura& 
Praia que nunca atingirão os panos lacerados.\&


Então, o frio mortal da alma, como a noite desce; &
Não sente ela a dor de outrem, nem a sua ousa sonhar;& 
Toda a fonte do pranto, o frio a veio enregelar; &
Brilham ainda os olhos: é o gelo que aparece.\&


Dos lábios flua o espírito, e a alegria o peito invada,& 
Na meia-noite já sem esperança de repouso: &
É como a hera em torno de uma torre já arruinada,& 
Verde por fora, e fresca, mas por baixo cinza anoso.\&


Pudesse eu me sentir ou ser como em horas passadas, &
Ou como outrora sobre cenas idas chorar tanto; &
Parecem doces no deserto as fontes, se salgadas:& 
No ermo da vida assim seria para mim o pranto.\&
\pagebreak



\chapter*{Lines Inscribed upon a Cup\break Formed From a Skull}



Start not --- nor deem my spirit fled;&
In me behold the only skull,&
From which, unlike a living head,&
Whatever flows is never dull.\&


I lived, I loved, I quaff'd, like thee:&
I died: let earth my bones resign;&
Fill up --- thou canst not injure me;&
The worm hath fouler lips than thine.\&


Better to hold the sparkling grape,&
Than nurse the earth-worm's slimy brood;&
And circle in the goblet's shape&
The drink of gods, than reptile's food.\&


Where once my wit, perchance, hath shone,&
In aid of others' let me shine;&
And when, alas! our brains are gone,&
What nobler substitute than wine?\&


Quaff while thou canst: another race,&
When thou and thine, like me, are sped,&
May rescue thee from earth's embrace,&
And rhyme and revel with the dead.\&


Why not? since through life's little day&
Our heads such sad effects produce;&
Redeem'd from worms and wasting clay,&
This chance is theirs, to be of use.\&

\pagebreak


\chapter[Versos inscritos numa taça feita de um crânio]{Versos inscritos numa taça\break feita de um crânio\footnote[*]{ Este
						poema --- escrito na Abadia de Newstead, em 1808
						--- pareceu muito atraente a poetas nossos como Castro Alves,
						que o traduziu. O mesmo fez Luís Delfino, que se valeu, segundo Onédia
						Célia, de interposta tradução francesa. A taça realmente existiu e foi
						usada por Byron e amigos em festa em Newstead.}



Não, não te assustes; não fugiu o meu espírito;&
Vê em mim um crânio, o único que existe,&
Do qual, muito ao contrário de uma fronte viva, &
Tudo aquilo que flui jamais é triste.\&


Vivi, amei, bebi, tal como tu; morri: &
Que renuncie a terra aos ossos meus;&
Enche! Não podes injuriar-me; tem o verme&
Lábios mais repugnantes do que os teus.\&



Antes do que nutrir a geração dos vermes,& 
Melhor conter a uva espumejante;&
Melhor é como taça distribuir o néctar&
Dos deuses, que a ração da larva rastejante.\&



Onde outrora brilhou, talvez, minha razão,&
Para ajudar os outros brilhe agora eu;&
Substituto haverá mais nobre do que o vinho &
Se o nosso cérebro já se perdeu?\&



Bebe enquanto puderes; quando tu e os teus &
Já tiverdes partido, uma outra gente&
Possa te redimir da terra que abraçar-te,&
E festeje com o morto e a própria rima tente.\&



E por que não? Se as frontes geram tal tristeza&
Através da existência --- curto dia ---,&
Redimidas dos vermes e da argila&
Ao menos possam ter alguma serventia.\&
\pagebreak


\chapter*{The Vision of Belshazzar}


The King was on his throne,&
The Satraps thronged the hall:&
A thousand bright lamps shone&
O'er that high festival.&
A thousand cups of gold,&
In Judah deemed divine ---&  
Jehovah's vessels hold&
The godless Heathen's wine.\&


In that same hour and hall,&
The fingers of a hand&
Came forth against the wall,&
And wrote as if on sand:&
The fingers of a man; ---& 
A solitary hand&
Along the letters ran,&
And traced them like a wand.\&


The monarch saw, and shook,&
And bade no more rejoice;&
All bloodless waxed his look,&
And tremulous his voice.&
`Let the men of lore appear,&
The wisest of the earth,&
And expound the words of fear,&
Which mar our royal mirth.'\&

Chaldea's seers are good,&
But here they have no skill;&
And the unknown letters stood&
Untold and awful still.&
And Babel's men of age&
Are wise and deep in lore;&
But now they were not sage,&
They saw --- but knew no more.\&


A captive in the land,&
A stranger and a youth,&
He heard the king's command,&
He saw that writing's truth.&
The lamps around were bright,&
The prophecy in view;&
He read it on that night ---& 
The morrow proved it true.\&


'Belshazzar's grave is made.&
His kingdom passed away,&
He, in the balance weighed,&
Is light and worthless clay;&
The shroud, his robe of state,&
His canopy the stone;&
The Mede is at his gate!&
The Persian on his throne!'\&
\pagebreak



\chapter{A visão de Baltasar\footnote[*]{ Este poema, que faz
						parte das \textit{Hebrew Melodies} (1815), foi traduzido
						por Costa Meireles (1869) e Oliveira Silva (1875). O
						tema sugestionou os nossos românticos, como demonstram  						``Babilônia'' de Cardoso de Meneses e Sousa
						Júnior, e ``O Festim de Baltasar'', de
						Elzeário da Lapa Pinto.}

\vspace*{-\baselineskip}



O rei estava no trono,& 
Os sátrapas no salão; &
Mil lâmpadas, de clarão &
Enchiam o festival. &
Por mil vasos de Jeová,& 
Todos santos em Judá &
--- Taças de ouro sem igual ---& 
Fluía o vinho pagão.\&


Na mesma hora e salão &
Os dedos de estranha mão& 
Na parede se moveram, &
Como em areia escreveram:& 
Eram dedos de varão, &
Mas, isolada, essa mão& 
A parede percorreu: &
Letras, qual vara, escreveu.\&


Viu o monarca, e tremeu, &
Não mais mandou que festassem;& 
Pôs-se exangue o rosto seu, &
Sua voz estremeceu. &
Os homens de mais saber& 
Que venham e expliquem logo &
Essas palavras de fogo &
Que estragam nosso prazer.\&


Os caldeus são bons videntes, &
Mas nisso foram inscientes; &
E as letras, não traduzidas, &
Seguiram desconhecidas. &
E os anciãos de Babel, &
Que são de extremo saber,& 
Já não tinham conhecer, &
Viam, mas sem compreender.\&


Um cativo dessa gente,&
Um jovem que era estrangeiro,&
O mando ouviu, real e urgente,& 
E o escrito achou verdadeiro.&
Das lâmpadas ao clarão,&
Bem se via a predição.&
Nessa noite ele a explicou,&
O outro dia a comprovou:\&


Já vai jazer Baltasar.&
Seu reino já está passado; &
Ele é, em balança pesado, &
Leve argila desvaliosa;&
Lage, é seu dossel sem par, &
Mortalha, a roupa faustosa. &
Está às portas a hoste adversa;& 
No seu trono, vejo o persa!\&
						
\pagebreak}

\chapter*{The prophecy of Dante}

\vspace{1ex}
The poet
%Bruno: problemas aqui?! Falta a indicação do canto; cf. original.
\vspace{1ex}


Many are poets who have never penned&
Their inspiration, and perchance the best:&
They felt, and loved, and died, but would not lend\&

Their thoughts to meaner beings; they compressed&
The god within them, and rejoined the stars&
Unlaurelled upon earth, but far more blessed\&

Than these who are degraded by the jars&
Of passion, and their frailties linked to fame,&
Conquerors of high renown, but full of scars.\&

Many are poets but without the name,&
For what is poesy but to create&
From overfeeling good or ill; and aim\&

At an external life beyond our fate,&
And be the new Prometheus of new men,&
Bestowing fire from heaven, and, then, too late,\&

Finding the pleasure given repaid with pain,&
And vultures to the heart of the bestower,&
Who, having lavished his high gift in vain,\&

Lies chained to his lone rock by the sea-shore?\&
\pagebreak



\chapter{A profecia de Dante\footnote[*]{ (Canto \textsc{iv}, vv.~1--19.) O poema sobre Dante, em terza
						rima, tem quatro cantos, que Byron dava como possível prosseguir, se os
						versos fossem compreendidos e aprovados.}


\vspace{1ex}
O poeta
\vspace{1ex}


Muitos são poetas que jamais a inspiração &
Puseram por escrito --- e os melhores, talvez;& 
Sentiram e viveram, mas sem concessão \&

Dos pensamentos seus a nenhum ser mais soez;& 
Comprimiram o deus em seu interior &
E juntaram-se aos astros, sem lauréis na terra,\& 

Mais felizes porém que aqueles que o estridor &
Da paixão degenera, e cuja fama encerra &
Suas fragilidades, os conquistadores \&

De alto renome, mas cheios de cicatrizes.& 
Muitos são poetas, mas do nome não senhores,& 
Pois que é a poesia mais do que buscar raízes\& 

No bem ou mal ultra-emotivos e querer &
Uma vida exterior além de nosso fado? &
E novo Prometeu do novo homem ser, \&

Dando o fogo do céu e, tudo consumado,& 
Vendo o prazer da oferta pago, mas com dor,& 
E abutres roendo o coração do benfeitor, \&

Que, tendo dissipado dádiva sem par, &
Jaz encadeado num rochedo junto ao mar?\&
\pagebreak}


\chapter{\footnote[*]{One Struggle More, and I Am Free}}


One struggle more, and I am free&
From pangs that rend my heart in twain;&
One last long sigh to love and thee,&
Then back to busy life again.&
It suits me well to mingle now&
With things that never pleased before:&
Though every joy is fled below,&
What future grief can touch me more?\&


Then bring me wine, the banquet bring;&
Man was not formed to live alone:&
I'll be that light, unmeaning thing,&
That smiles with all, and weeps with none.&
It was not thus in days more dear,&
It never would have been, but thou&
Hast fled, and left me lonely here;&
Thou'rt nothing, --- all are nothing now.\&


In vain my lyre would lightly breathe!&
The smile that sorrow fain would wear&
But mocks the woe that lurks beneath,&
Like roses o'er a sepulchre.&
Though gay companions o'er the bowl&
Dispel awhile the sense of ill;&
Though pleasure fires the maddening soul,&
The heart --- the heart is lonely still!\&


On many a lone and lovely night&
It soothed to gaze upon the sky;&
For then I deemed the heavenly light&
Shone sweetly on thy pensive eye:&
And oft I thought at Cynthia's noon,&
When sailing o'er the Aegean wave,&
'Now Thyrza gazes on that moon' ---& 
Alas, it gleamed upon her grave!\&


When stretched on fever's sleepless bed,&
And sickness shrunk my throbbing veins.&
`'Tis comfort still,' I faintly said,&
'That Thyrza cannot know my pains:'&
Like freedom to the time-worn slave ---& 
A boon 'tis idle then to give ---&  
Relenting Nature vainly gave&
My life, when Thyrza ceased to live!\&


My Thyrza's pledge in better days,&
When love and life alike were new!&
How different now thou meet'st my gaze!&
How tinged by time with sorrow's hue!&
The heart that gave itself with thee&
Is silent --- ah, were mine as still!&
Though cold as e'en the dead can be,&
It feels, it sickens with the chill.\&


Thou bitter pledge! thou mournful token!&
Though painful, welcome to my breast!&
Still, still, preserve that love unbroken,&
Or break the heart to which thou'art pressed!&
Time tempers love, but not removes,&
More hallowed when its hope is fled:&
Oh! what are thousand living loves&
To that which cannot quit the dead?\&
\pagebreak

\chapter{Mais um esforço, e livre estou depois\footnote[*]{ Uma das ``Occasional Pieces'', 
						parece datar de fins de 1811. É um dos poemas a Tirza (supõe{}­se que
						Edleston, então morto). Na 4ª estrofe Byron fala que viajava pelo mar
						Egeu pensando que Tirza estivesse viva, e esta já havia morrido sem 
						que ele soubesse.}}}

\vspace*{-\baselineskip}



Mais um esforço, e livre estou depois&
Da angústia que me parte o coração em dois;&
Um último suspiro a ti e ao teu amor&
E à vida ativa retornar então: &
Serve-me agora misturar-me sem calor &
Com seres pelos quais jamais tive atração:& 
Já que toda alegria aqui eu vi fugir, &
Que dor futura ainda pode me atingir?\&


Venha a mim o banquete pois, trazei-me vinho; &
O homem, ninguém o fez para viver sozinho: &
Serei inexpressiva e frívola criatura &
Que com todos sorri, porém com ninguém chora.&
Não foi assim em dias de ventura,&
Nunca teria sido, mas tu agora &
Te foste, e tão sozinho me deixaste cá;&
Nada és --- tudo não é nada já.\&


Em vão alegre a minha lira soaria! &
O sorriso que dê a melancolia&
Apenas escarnece a dor nele emboscada &
--- É igual a rosas sobre sepultura.&
Embora companheiros, taça coroada, &
Afastem por um pouco a sensação da desventura, &
E o prazer incendeie a alma demente, &
Sozinho ainda o coração se sente!\&


Em muita noite solitária e fascinante&
Confortava-me olhar o céu faiscante;& 
Achava eu que a luz celestial, então, &
Brilhava doce nesse pensativo olhar &
E acreditava --- Cíntia no auge do clarão ---,& 
Quando na onda do Egeu a velejar:&
Agora Tirza olha aquela lua& 
--- Ai, ela fulgurava sobre a tumba tua!\&


No leito sem dormir da febre eu me estendia&
E as veias a pulsar a doença contraía:&
Inda é um conforto --- eu me dizia debilmente ---&
``Que não possa Tirza conhecer-me as dores.'' &
A liberdade será dada inutilmente &
Ao servo a quem cansou o tempo em seus rigores:& 
Assim a Natureza em vão deixou, condoída, &
Que eu continuasse vivo --- e Tirza já sem vida!\&


Penhor de minha Tirza em época melhor, &
Quando eram jovens nossa vida e nosso amor! &
Que diferente hoje te vê o meu olhar! &
Tingiu-te o tempo com as cores da tristura!&
Como te deste, o coração que vieste a dar&
Está silente --- ah! o meu tivesse tal quietura!&
Embora frio como os mortos podem ser,&
Bem que ele sente, e vem de frio a adoecer.\&


Amarga prenda, tu! lembrança dolorosa! &
És bem-vinda a meu peito, embora assim penosa!&
Conserva sem cessar aquele amor constante, &
Oh! rompe o coração ao qual és estreitada! &
O amor, modera-o o tempo, não o faz distante, &
Mais santo é ele já a esperança terminada: &
Oh! amor múltiplo às que vivem pouca importa,& 
Junto ao que não consegue abandonar a morta.\&
\pagebreak



\chapter*{Intensities of Blue}


{cviii}&
Oh! ye, who make the fortunes of all books!&
Benigh Ceruleans of the second sex!&
Who advertise new poems by your looks,&
Your 'imprimatur' will ye not annex?&
What! must I go to the oblivious cooks,&
Those Cornish plunderers of Parnassian wrecks?&
Ah! must I then the only minstrel be,&
Proscribed from tasting your Castalian tea?\&


{cix}&
What! can I prove 'a lion' then no more?&
A ball-room bard, a foolscap, hot-press darling?&
To bear the compliments of man a bore,&
And sigh, 'I can't get 
out,' like Yorick's starling;&
Why then I'll swear, as poet Wordy swore,&
(Because the world won't read him, always snarling)&
That taste is gone, that fame is but a lottery,&
Drawn by the blue-coat misses of a coterie.\&


{cx}&
Oh! 'darkly, deeply, beautifully blue,'&
As some one somewhere sings about the sky,&
And I, ye learned ladies, say of you;&
They say your stockings are so --- (Heaven knows why,&
I have examined few pair of that hue);&
Blue as the garters which serenely lie&
Round the Patrician left-legs, which adorn&
The festal midnight, and the levée morn.\&


{cxi}&
Yet some of you are most seraphic creatures ---& 
But times are altered since, a rhyming lover,& 
You read my stanzas, and I read your features:&
And --- but no matter, all those things are over;& 
Still I have no dislike to learned natures,&
For sometimes such a world of virtues cover;& 
I knew one woman of that purple school, &
The loveliest, chastest, best, but --- quite a fool.\&


{cxii}&
Humboldt, 'the first of travellers,' but not&
The last, if late accounts be accurate, &
Invented, by some name I have forgot,&
As well as the sublime discovery's date,&
An airy instrument, with which he sought&
To ascertain the atmospheric state, &
By measuring 'the \textit{intensity of blue}:'&
Oh, Lady Daphne! let me measure you!\&
\pagebreak





























\chapter{Os graus do azul\footnote[*]{\textit{Don Juan}, canto \textsc{iv}.}

\vspace*{-\baselineskip}



{cviii}&
Vós que a sorte dos livros todos resolveis,& 
{Vós, azuis do segundo sexo, tão cordiais!}%
					\footnote { \textit{Ceruleans}, cerúleas $=$ azuis. \textit{Blue}, além 
					do significado normal de azul, tem no poema, referindo-se a mulheres,
					o sentido de eruditas, pedantes (\textit{Oxford English Dictionary}). Esse dicionário remete a
					\textit{blue-stocking} e explica: em reuniões mantidas em
					Londres, por volta de 1750, na casa de Mrs.~Montague e outras, trocaram
					as damas o jogo de cartas por maneiras mais intelectuais de passar o
					tempo, inclusive conversação sobre assuntos literários, das quais
					muitos homens de letras às vezes tomavam parte. Muitas das
					participantes evitavam trajes formais: uma delas era Mrs.~Benjamin
					Stillingfleet, que usava habitualmente meias de lã cinza ou azul, em
					vez de seda preta. Daí um almirante Boscawen ter chamado a
					``coterie'' de ``the blue stocking Society''. As senhoras do grupo eram chamadas
					Blue Stockingers, Blue Stocking Ladies, mais tarde Blue Stockings,
					depois abreviado para \textit{blues}, no \textit{slang}. A graça do
					excerto de Byron é a ambiguidade de \textit{blues}, nos sentidos de
					mulheres de gosto literário e azul.}&
Que anunciais os nossos poemas com o olhar, &
Acrescentar vosso ``imprimatur'' não quereis? &
Quê! Devo os esquecidos cucas procurar, &
Os córnicos que pilham ruínas imortais? &
Devo ser eu o único menestrel, que já &
Proibistes de tomar vosso castálio chá?\&


{cix}&
Quê! não posso mostrar-me nunca mais um ``leão''? &
Um barrete de bobo, um bardo de salão, &
Para aguentar os rapapés de algum pateta,& 
Gemer como a ave de Yorick, ``não posso sair'', &
Ou então jurarei, Wordy assim fez, o poeta, &
(O mundo não vai lê-lo, está sempre a rosnir),& 
Já não há gosto, e a fama é dada em loteria &
Por moças de casaco azul em parceria.\&


{cx}&
Oh! ``Escura, forte, belamente azul'' --- é o que &
Nalgum lugar alguém cantou do firmamento, &
E de vós, doutas damas, coisa igual sustento;& 
Dizem que vossas meias são azuis (por quê, &
Sabe o céu, poucos pares vi eu dessa cor); &
Azuis tais como as jarreteiras, que ao dispor& 
Da perna esquerda dos senhores, ornarão &
A festa à noite ou a manhã de recepção.\&


{cxi}&
Muitas de vós, como criatura, é um serafim,& 
Mas foi-se o tempo em que, de rimas amador,& 
Líeis minhas estâncias, e eu o vosso rosto: &
Mas não importa, que isso tudo teve fim; &
Porém de sábias naturezas não desgosto,&
Que às vezes cobrem virtuosíssimo primor;& 
Conheci uma mulher da escola rebuscada, &
Linda, casta, a melhor --- mas tola rematada.\&


{cxii}&
Humboldt, esse ``primeiro dos viajantes'', não&
O último, se recente informe é sem senão,&
Inventou, e lhe deu um nome que olvidei&
Com a data dessa descoberta assim de lei,&
Um aéreo instrumento, para procurar&
Do estado da atmosfera se certificar,&
Medindo os graus do azul, tal como ele apareça.&
{Deixai portanto, ó Lady Daphne, que eu vos meça!}%
						\footnote { Este verso e o anterior deixam patentes os dois tipos de
						azul: o literário e a cor.}\&
%Bruno: Confusa a nota acima, mudei-a um pouco, mas não está satisfatória.
\quebra

\chapter*{Sonnet to George the Fourth}


To be the father of the fatherless,&
To stretch the hand from the throne's height, and raise&
\textit{His} offspring, who expired in other days\&


To make thy sire's sway by a kingdom less, ---& 
\textit{This} is to be a monarch, and repress&
Envy into unutterable praise.\&


Dismiss thy guard, and trust thee to such traits,&
For who would lift a hand, except to bless?&
Were it not easy, sir, and is't not sweet\&


To make thyself beloved? and to be&
Omnipotent by mercy's means? for thus&
Thy sovereignty would grow but more complete:\&


A despot thou, and yet thy people free,&
And by the heart, not hand, enslaving us.\&
\pagebreak


\chapter[Soneto a George \textsc{iv}]{Soneto a George IV\footnote[*]{ Ao Anular a Condenação de Lord Edward 
						Fitzgerald. Escrito em Bolonha, em 
						12 de agosto de 1819.}


Fazer-se o pai dos órfãos, estender a mão&
Do alto do trono e erguer a prole do que um dia&
Morreu para privar teu pai da monarquia&
--- \textit{Isto} é ser rei deveras, pôr em defecção\&


A inveja e substituí-la pela admiração.&
Despede a tua guarda, e em tais ações confia,&
Pois só para abençoar qualquer mão se ergueria.&
Senhor, seria fácil, não é doce então\&


Fazer-se amado pelos súditos? e ser &
Onipotente pela força da indulgência,&
Ser todo-poderoso, mas pela clemência?\&


E mais completo se faria o teu poder:&
Teria o povo um déspota, mas sem corrente,&
Escravizado pelo coração somente. \&
\pagebreak


\chapter*{Lines to Mr.\,Hodgson}


Huzza! Hodgson, we are going,&
Our embargo's off at last;&
Favourable breezes blowing&
Bend the canvas o'er the mast.&
From aloft the signal's streaming,&
Hark! the farewell gun is fired;&
Women screeching, tars blaspheming,&
Tell us that our time's expired. &
Here's a rascal &
Come to task all,&
Prying from the Custom-house;&
Trunks unpacking &
Cases cracking,&
Not a corner for a mouse&
'Scapes unsearched amid the racket,&
Ere we sail on board the Packet.\&


Now our boatmen quit their mooring,&
And all hands must ply the oar;&
Baggage from the quay is lowering,&
We're impatient, push from shore.&
'Have a care! that case holds liquor ---&
Stop the boat --- I'm sick --- oh Lord!' &
'Sick, ma'am, damme, you'll be sicker,&
Ere you've been an hour on board.' &
Thus are screaming &
Men and women,&
Gemmen, ladies, servants, Jacks;&
Here entangling,&
All are wrangling,&
Stuck together close as wax. ---& 
Such the general noise and racket,&
Ere we reach the Lisbon Packet.\&


Now we've reached her, lo! the Captain,&
Gallant Kidd, commands the crew;&
Passengers their berths are clapt in,&
Some to grumble, some to spew.&
'Hey day! call you that a cabin?&
Why 'tis hardly three feet square;&
Not enough to stow Queen Mab in ---& 
Who the deuce can harbour there?' &
'Who, sir? plenty ---& 
Nobles twenty&
Did at once my vessel fill.' ---& 
'Did they? Jesus, &
How you squeeze us!&
Would to God they did so still:&
Then I'd scape the heat and racket&
Of the good ship, Lisbon Packet.'\&


Fletcher! Murray! Bob! where are you?&
Stretched along the deck like logs ---& 
Bear a hand, you jolly tar, you!&
Here's a rope's end for the dogs.&
Hobhouse muttering fearful curses,&
As the hatchway down he rolls,&
Now his breakfast, now his verses,&
Vomits forth --- and damns our souls.& 
'Here's a stanza &On Braganza ---& 
Help!' --- A couplet?' --- 'No, a cup &
Of warm water --- '&'What's the matter?'&
'Zounds! my liver's coming up;&
I shall not survive the racket&
Of this brutal Lisbon Packet.'\&



Now at length we're off for Turkey,&
Lord knows when we shall come back!&
Breezes foul and tempests murky&
May unship us in a crack,&
But, since life at most a jest is,&
As philosophers allow,&
Still to laugh by far the best is,&
Then laugh on --- as I do now.&
Laugh at all things,&
Great and small things,&
Sick or well, at sea or shore; &
While we're quaffing, &
Let's have laughing ---& 
Who the devil cares for more? &
Some good wine! and who would lack it,&
Ev'n on board the Lisbon Packet?\&
\pagebreak



\chapter{Versos a Mr.\,Hodgson\footnote[*]{ Escritos a bordo do Paquete de Lisboa, em 30 de junho
					de 1809. Era o início da primeira viagem de Byron e têm um tom de
					realismo humorístico que é também uma das vertentes de nosso
					Romantismo. \textit{Rainha Mab}, na terceira estrofe: segundo
					Catherine Briggs, nos séculos \textsc{xvi} e \textsc{xvii} a maior parte dos poetas fez a
					Rainha Mab a rainha das Fadas e particularmente das \textit{fadas
					diminutas} da \textit{Nymphidia} de Drayton. Também diminuta é a de
					Shakespeare, \textit{Romeu e Julieta.}}


Viva! Hodgson, estamos indo,&
Nosso embargo afinal foi levantado;&
Um vento afla bem-vindo,&
No mastro o pano já está enfunado.&
Lá em cima o sinal vêmo-lo dado,&
Ouve! o canhão do adeus foi disparado.&
Mulheres a gritar, marujos blasfemando,&
Mostram que nosso tempo está esgotado.&
Eis senão quando&
Um maroto pela aduana enviado &
Vem para, inspecionando, &
Chamar todos à fala:&
Desfaz-se, é aberta muita mala;&
Nem mesmo o esconderijo de algum rato&
Escapa da revista a fio,&
Em meio a tal espalhafato,& 
Antes que velejemos no navio.\&



Nossos barqueiros saem da amarração,&
Todos a manejar o remo:&
As malas já do cais descendo estão;&
Partimos, impacientes ao extremo.&
Cuidado! que essa caixa tem bebida.&
--- Parem o bote! --- Oh Deus! --- Estou enjoada!&
Madame, enjoada? Estará mais nauseada&
Depois de uma hora a bordo decorrida!&
Assim estão fazendo assuada&
Mulheres e homens, cavalheiros,&
Senhoras, criados, marinheiros;&
Ali se confundindo,&
Todos vão discutindo,&
Grudados entre si, toda e qualquer pessoa.&
Rumor geral, esta é a algazarra que ressoa&
Até chegarmos ao Paquete de Lisboa.\&



Agora que chegamos, vede, o capitão:& 
Comanda o bravo Kidd nossa tripulação.&
Os passageiros põem-se nos seus leitos,& 
Ou para resmungar ou para vomitar.&
``Caramba! De cabine isto chamar?&
Mal tem três pés quadrados,&
Para a Rainha Mab já bem estreitos,&
Quem diabo pode ali dar com os costados?''&
``Ora, senhor! Pois muita gente!&
Vinte nobres, certamente,&
Encheram o navio alguma vez.''&
``Encheram? Por Jesus, que coisa linda!&
Só para nós o aperto é que se fez!&
Quisera Deus eles o enchessem inda!&
Nem o calor nem a algazarra que ressoa &
Me veriam do bom Paquete de Lisboa.\&



Ó Fletcher! Murray! Bob! onde é que estais?&
Deitados como toras no convés!&
Ajuda aqui, marujo, dos joviais!&
Uma ponta de cabo para os lorpas vês.&
Hobhouse terríveis pragas vai rogando&
Ao rolar escotilha abaixo:&
Ou desjejum ou versos vomitando,&
Condena as nossas almas a demônio e tacho.&
Eis uma estança&
Sobre Bragança.&
Dai-me! Um dístico? Não, quero uma taça&
De água quente.&
``Que é que se passa?''&
Meu fígado está subindo, gente! &
Morro antes da algazarra que ressoa &
Neste brutal Paquete de Lisboa.\&



Enfim vamos no mar rumo à Turquia,&
Sabe Deus quando iremos regressar!&
Ventos maus e procela a mais sombria&
Podem fazer-nos num instante soçobrar.&
Mas a vida é no máximo um gracejo&
--- Entre os filósofos não há discordes ---&
Rir é o melhor, havendo o ensejo:&
Ride então qual eu rio, estai concordes.&
Ride de todas as coisas,&
Das grandes e pequenas coisas, &
Ou bons ou doentes, quer em terra, quer no mar;&
Enquanto estamos a entornar, &
Ponhamo-nos a rir. &
Quem diabo vai com mais se preocupar? &
--- Um vinho bom! Quem não o iria consumir, &
Mesmo se a bordo do Paquete de Lisboa?\&
\pagebreak}



\chapter*{Epistle to Augusta}



\quad{i}&
My sister! my sweet sister! if a name &
Dearer and purer were, it should be thine.& 
Mountains and seas divide us, but I claim &
No tears, but tenderness to answer mine: &
Go where I will, to me thou art the same ---& 
A loved regret which I would not resign. &
There yet are two things in my destiny, ---& 
A world to roam through, and a home with thee.\&


{ii}&
The first were nothing --- had I still the last, &
It were the haven of my happiness; &
But other claims and other ties thou hast,& 
And mine is not the wish to make them less. &
A strange doom is thy father's son's, and past &
Recalling, as it lies beyond redress; &
Reversed for him our grandsire's fate of yore, ---& 
He had no rest at sea, nor I on shore.\&

{iii}&
If my inheritance of storms hath been &
In other elements, and on the rocks &
Of perils, overlooked or unforeseen, &
I have sustained my share of worldly shocks,& 
The fault was mine; nor do I seek to screen &
My errors with defensive paradox; &
I have been cunning in mine overthrow,&
The careful pilot of my proper woe.\&


{iv}&
Mine were my faults, and mine be their reward.& 
My whole life was a contest, since the day &
That gave me being, gave me that which marred& 
The gift, --- a fate, or will, that walked astray;& 
And I at times have found the struggle hard, &
And thought of shaking off my bonds of clay: &
But now I fain would for a time survive,&
If but to see what next can well arrive.\&


{v}&
Kingdoms and empires in my little day &
I have outlived, and yet I am not old; &
And when I look on this, the petty spray& 
Of my own years of trouble, which have rolled& 
Like a wild bay of breakers, melts away: &
Something --- I know not what --- does still uphold& 
A spirit of slight patience; --- not in vain,&
Even for its own sake, do we purchase pain.\&

{vi}&
Perhaps the workings of defiance stir &
Within me --- or perhaps a cold despair, &
Brought on when ills habitually recur, ---& 
Perhaps a kinder clime, or purer air, &
(For even to this may change of soul refer,& 
And with light armour we may learn to bear,) &
Have taught me a strange quiet, which was not&
The chief companion of a calmer lot.\&


{vii}&
I feel almost at times as I have felt &
In happy childhood; trees, and flowers, and brooks, &
Which do remember me of where I dwelt &
Ere my young mind was sacrificed to books,& 
Come as of yore upon me, and can melt &
My heart with recognition of their looks;& 
And even at moments I could think I see&
Some living things to love --- but none like thee.\&


{viii}&
Here are the Alpine landscapes which create &
A fund for contemplation; --- to admire &
Is a brief feeling of a trivial date; &
But something worthier do such scenes inspire:& 
Here to be lonely is not desolate, &
For much I view which I could most desire, &
And, above all, a lake I can behold&
Lovelier, not dearer, than our own of old.\&


{ix}&
Oh that thou wert but with me! --- but I grow &
The fool of my own wishes, and forget &
The solitude which I have vaunted so &
Has lost its praise in this but one regret;& 
There may be others which I less may show; ---& 
I am not of the plaintive mood, and yet &
I feel an ebb in my philosophy,&
And the tide rising in my altered eye.\&


{x}&
I did remind thee of our own dear Lake, &
By the old Hall which may be mine no more.& 
Leman's is fair; but think not I forsake& 
The sweet remembrance of a dearer shore: &
Sad havoc Time must with my memory make, &
Ere \textit{that} or \textit{thou} can fade these eyes before;& 
Though, like all things which I have loved, they are &
Resigned for ever, or divided far.\&


{xi}&
The world is all before me; I but ask& 
Of Nature that with which she will comply ---& 
It is but in her summer's sun to bask,& 
To mingle with the quiet of her sky, &
To see her gentle face without a mask,& 
And never gaze on it with apathy. &
She was my early friend, and now shall be& 
My sister --- till I look again on thee.\&


{xii}&
I can reduce all feelings but this one;& 
And that I would not; --- for at length I see& 
Such scenes as those wherein my life begun. &
The earliest --- even the only paths for me ---& 
Had I but sooner learnt the crowd to shun,& 
I had been better than I now can be; &
The passions which have torn me would have slept;& 
I had not suffered, and \textit{thou }hadst not wept.\&


{xiii}&
With false Ambition what had I to do? &
Little with Love, and least of all with Fame;& 
And yet they came unsought, and with me grew, &
And made me all which they can make --- a name. &
Yet this was not the end I did pursue; &
Surely I once beheld a nobler aim. &
But all is over --- I am one the more&
To baffled millions which have gone before.\&


{xiv}&
And for the future, this world's future may& 
From me demand but little of my rare; &
I have outlived myself by many a day; &
Having survived so many things that were;& 
My years have been no slumber, but the prey& 
Of seaseless vigils; for I had the share &
Of life which might have filled a century,&
Before its fourth in time had passed me by.\&


{xv}&
And for the remnant which may be to come &
I am content; and for the past I feel &
Not thankless, --- for within the crowded sum& 
Of struggles, happiness at times would steal,& 
And for the present, I would not benumb &
My feelings farther. --- Nor shall I conceal& 
That with all this I still can look around,&
And worshin Nature with a thought profound.\&


{xvi}&
For thee, my own sweet sister, in thy heart &
I know myself secure, as thou in mine; &
We were and are --- I am, even as thou art ---& 
Beings who ne'er each other can resign;& 
It is the same, together or apart, &
From life's commencement to its slow decline& 
We are entwined --- let death come slow or fast, &
The tie which bound the first endures the last!\&
\pagebreak



\chapter{Carta a Augusta\footnote[*]{ Escrita na Villa Diodati em 1816.
					Augusta Leigh permitiu --- informa Douglas Dunn --- que fosse
					publicada em 1830. O poema é autobiográfico, e Augusta, ambiguamente,
					parece ser encarada mais do que como irmã.}



{i}&
Minha irmã! Minha amada irmã! Se um nome tanto &
Mais caro e puro houvesse, ele seria o teu. &
Serras e mares nos separam, porém pranto &
Não quero, e sim afeto que responda ao meu:& 
Aonde eu vá, és a mesma para mim, portanto &
Uma saudade amada, a qual não deixo eu.&
Em meu destino, há duas coisas que realçar:& 
--- Um mundo para percorrer, contigo um lar.\&


{ii}&
Nada seria o mundo, se eu tivesse o lar, &
Se ele me fosse o porto da felicidade; &
Mas outros laços tens, e o que reivindicar,& 
E para mudar isso falha-me a vontade. &
Do filho de teu pai a sina é singular, &
E irrevocável, porque sem tranquilidade.&
Oposto é ao do avô o que meu fado encerra:&
{No mar sossego o avô não teve, e eu peno em terra}%
						\footnote { Alusão a Jack ``Mau-tempo'',
						o avô almirante do poeta.}%
							   .\&



{iii}&
Minha herança de tempestades, se é enfrentada&
Em outros elementos; se nos penhascais&
Cheios de ameaça repentina ou desprezada&
Minha parte aguentei de choques terrenais,& 
Foi minha a falta; nem tento esconder em nada& 
Os erros meus com os paradoxos mais parciais; &
Eu tenho sido esperto em minha destruição, &
Piloto cuidadoso de última aflição.\&


{iv}&
Minhas as faltas, seja minha a punição.&
A vida para mim foi uma luta, desde o dia&
Que me deu ser, e trouxe junto a frustração&
Do dom --- a sina ou o querer de errada via;&
Às vezes achei árdua a luta, e sem perdão,&
E meus laços de argila, eu quase que os rompia;&
Difícil me seria agora inda viver,&
Não fosse ter em vista o que pode ocorrer.\&


{v}&
A reinos como a impérios, em meu dia breve &
Sobrevivi, e não sou velho no entretanto;&
Se penso nisso, funde-se o borrifo leve&
De meus anos difíceis, a rolarem tanto&
Como ondas bravas, que a enseada às pedras leve;&
Algo --- não sei o que --- sustém-me ainda, entanto,&
Uma ligeira paciência; --- a aquisição&
Da dor, mesmo que só por ela, nunca é em vão.\&


{vi}&
Talvez dentro de mim se mova o desafio, &
Ou, vindo quando os males vivem a voltar,& 
Esteja a se agitar um desespero frio&
--- Talvez mais doce clima, ou mais límpido ar &
(Das voltas da alma pode aí estar o fio, &
Com leve arnês é de aprendê-lo a suportar),& 
Me ensinaram quietude estranha, que não era &
De um fado mais tranquilo a companheira vera.\&


{vii}&
Às vezes quase sinto como eu me sentia&
Na infância; árvores, flores, riachos a correr,&
Que me relembram do lugar onde eu vivia&
Antes de em livros minha mente se perder,&
Afluem sobre mim: recordo tal magia,&
A qual meu coração bem pode comover;&
Pus-me a pensar às vezes que diviso aqui&
Criaturas para amar --- nenhuma igual a ti.\&


{viii}&
As paisagens alpinas, ei-las, a criar &
Um fundo para contemplar-se: a admiração& 
É um rápido sentir, de duração vulgar;&
Mas são de algo melhor, tais cenas, ocasião:&
Aqui, estar só não é desconsolado estar;&
Muito de desejável colhe-me a visão,&
{E sobretudo um lago, o qual, para se ver}%
					\footnote { O lago Leman e o lago junto à Abadia de Newstead.}%
						,&
Mais lindo, caro não, que o nosso vem a ser.\&


{ix}&
Se estivesses comigo! --- eu torno-me no entanto& 
O ludíbrio de meus desejos, a me esqueço &
De que essa própria solidão que louvei tanto& 
Neste único pesar perdeu todo o seu preço, &
Pode haver --- mostro-o menos --- qualquer outro pranto;&
--- Não sou dos de ânimo queixoso, mas pareço &
Minha filosofia na vazante aluir &
E ter no olho alterado a maré, a subir.\&


{x}&
Eu recordei-te nosso lago tão querido &
Junto à velha mansão que poderei perder.&
Belo é o Leman; não cuides haja eu preterido& 
Na lembrança outras margens, ou no bem-querer.& 
Minha memória, o tempo já a terá destruído &
Antes de \textit{aquele}, ou \textit{tu}, à vista me esvaecer,& 
Embora, como todos já por mim amados,&
Estejais longe, ou para sempre renunciados.\&


{xi}&
O mundo inteiro está ante mim; à natureza &
Só peço aquilo com que irá condescender: &
--- Dar-me um sol de verão que poupe da frieza,& 
Misturar-me com a calma de seu céu, e ver&
Sem máscara o seu rosto cheio de brandeza,&
Nunca fitá-lo com apatia. Vinha a ser&
Ela a minha primeira amiga e será agora&
A minha irmã --- até eu ver-te, alguma hora.\&


{xii}&
Os sentimentos posso frear, mas esse não;&
Nem eu o quereria; --- vejo finalmente&
Cenas como as da vida em sua iniciação.&
As mais antigas --- minhas trilhas, tão-somente ---&
Mais cedo houvesse eu evitado a multidão,&
Melhor teria eu sido do que no presente;&
Lacerantes paixões teriam sossegado;&
Sofrido não teria \textit{eu}, nem \textit{tu} chorado.\&


{xiii}&
Com a falsa ambição que tinha eu a fazer?&
Pouco com o amor, menos de tudo com o renome;&
Mas vieram sem procura, e entraram de crescer&
E me fizeram quanto podem --- dar um nome.&
Mas tal não era a meta para eu escolher;&
De alvo mais nobre outrora eu tive a nobre fome.&
Mas tudo terminou: --- sou um a mais, somente,&
Entre os milhões frustrados que partiram à frente.\&


{xiv}&
Quanto ao futuro, o deste mundo poderia &
Demandar-me somente mínima atenção; &
A mim próprio sobrevivi por muito dia; &
Perduro, e tantas coisas, tantas, já não são;&
Meu tempo não foi sono, sob a tirania&
De incessantes vigílias: tive eu a porção&
De vida que podia a um séc'lo ter enchido&
Antes que a quarta parte houvesse me servido.\&


{xv}&
Quanto ao restante que me pode ainda vir,&
Estou contente; e sinto eu, quanto ao passado,&
Alguma gratidão; pois no denso montante&
De lutas, a ventura tem de leve andado;&
Quanto ao presente, eu não queria mais um instante&
Dormir meus sentimentos. --- Mostro de bom grado&
Que eu posso, com tudo isso, ainda em torno olhar,&
E com um pensar profundo a natureza amar.\&


{xvi}&
Minha querida irmã, sinto-me resguardado,& 
Tal como tu no meu, dentro em teu coração; &
Fomos e somos --- sou, e assim és por teu lado,& 
Seres que não desistem um do outro, oh não; &
É o mesmo, juntos ou um do outro separado: &
Desde o início da vida à sua descensão, &
Unidos --- venha a morte, em breve ou devagar,& 
Nosso primeiro liame é o último a findar!\&}


\chapter*{Sonnet to Lake Leman}


Rousseau --- Voltaire --- our Gibbon --- and De Staël&
Leman! these names are worthy of thy shore,&
Thy shore of names like these! wert thou no more,&
Their memory thy remembrance would recall:&
To them thy banks were lovely as to all,\&


But they have made them lovelier, for the lore&
Of mighty minds doth hallow in the core&
Of human hearts the ruin of a wall&
Where dwelt the wise and wondrous; but by \textit{thee,}\&


How much more, Lake of Beauty! do we feel,&
In sweetly gliding o'er thy crystal sea,&
The wild glow of that not ungentle zeal,\&


Which of the heirs of immortality &
Is proud, and makes the breath of glory real!\&
\pagebreak}


\chapter{Soneto ao Lago Leman\footnote[*]{ Byron, na Suíça, morou certo
						tempo junto a esse lago, famoso por sua beleza. Escrito em julho de
						1816, na Villa Diodati.}


Rousseau, Voltaire, o nosso Gibbon, e De Staël: &
Leman! tais nomes, dignos são dessa ribeira, &
E esta de nomes tais! Passando tu, afinal, &
Lembrá-los nos traria tua memória inteira;\&


Tua borda os encantava, e aos homens em geral,& 
Mas eles a tornaram inda mais fagueira, &
Porquanto santifica, a mente magistral, &
Nos corações a queda da parede, em poeira,\&


Da morada dos sábios e maravilhosos; &
Mas junto a ti, lago formoso entre os formosos!& 
Sentimos a planar por sobre o teu cristal\&


O veemente esplendor de um entusiasmo real,& 
Que, ufano dos herdeiros da imortalidade, &
Faz da respiração da glória uma verdade!\&
\pagebreak}



\chapter*{Lines on Hearing that\break Lady Byron Was Ill}


And thou wert sad --- yet I was not with thee;&
And thou wert sick, and yet I was not near;&
Methought that joy and health alone could be&
Where I was \textit{not} --- and pain and sorrow here!&
And is it thus? --- it is as I foretold,&
And shall be more so; for the mind recoils&
Upon itself, and the wrecked heart lies cold,&
While heaviness collects the shattered spoils.&
It is not in the storm nor in the strife&
We feel benumbed, and wish to be no more,\&


But in the after-silence on the shore,&
When all is lost, except a little life.\&


I am too well avenged! --- but 'twas my right;&
Whate'er my sins might be, \textit{thou} wert not sent&
To be the Nemesis who should requite ---& 
Nor did Heaven choose so near an instrument.&
Mercy is for the merciful! --- if thou&
Hast been of such, 'twill be accorded now.&
Thy nights are banished from the realms of sleep. ---& 
Yes! they may flatter thee, but thou shalt feel\&


A hollow agony which will not heal,&
For thou art pillowed on a course too deep;&
Thou hast sown in my sorrow, and must reap\&



The bitter harvest in a woe as real!&
I have had many foes, but none like thee;&
For 'gainst the rest myself I could defend,&
And be avenged, or turn them into friend;&
But thou in safe implacability&
Hadst nought to dread --- in thy own weakness shielded,&
And in my love, which hath but too much yielded,&
And spared, for thy sake, some I should not spare;&
And thus upon the world --- trust in thy truth,&
And the wild fame of my ungoverned youth ---& 
On things that were not, and on things that are ---& 
Even upon such a basis hast thou built\&


A monument, whose cement hath been guilt!&
The moral Clytemnestra of thy lord,&
And hewed down, with an unsuspected sword,&
Fame, peace, and hope --- and all the better life&
Which, but for this cold treason of thy heart,&
Might still have risen from out the grave of strife,&
And found a nobler duty than to part. &
But of thy virtues didst thou make a vice,&
Trafficking with them in a purpose cold,\&


For present anger, and for future gold ---& 
And buying other's grief at any price.& 
And thus once entered into crooked ways, &
The early truth, which was thy proper praise,& 
Did not still walk beside thee --- but at times,& 
And with a breast unknowing its own crimes, &
Deceit, averments incompatible, \&


Equivocations, and the thoughts which dwell&
In Janus-spirits --- the significant eye &
Which learns to lie with silence --- the pretext& 
Of prudence, with advantages annexed --- \&



The acquiescence in all things which tend,& 
No matter how, to the desired end ---& 
All found a place in thy philosophy.& 
The means were worthy, and the end is won ---& 
I would not do by thee as thou hast done!\&
\pagebreak}



\chapter[Versos escritos ao ouvir que Lady Byron estava doente]{Versos escritos ao ouvir que\break Lady Byron estava doente\footnote[*]{``Este amargo poema foi escrito em setembro 
						de 1816 (isto é, mais ou menos ao tempo em que o foi a 
						`Carta a Augusta'), mas só foi 
						impresso em 1832. Sua ira e paixão foram provavelmente despertados ao 
						ouvir, de Shelley, os rumores distintos quanto à natureza das relações
						entre Byron e Augusta, e Byron e sua mulher; também, e pouco antes de o 
						poema ter sido escrito, a tentativa de Mme.~de Staël de reconciliar 	
						Lord e Lady Byron foi frustrada pelo fato de Lady Byron nem querer 
						ouvir falar no assunto'' (D.~Dunn). Aparecem no
						poema Nêmesis, deusa vingadora ($=$ víndice) dos crimes, e Clitemnestra,
						a esposa de Agamêmnon, que o matou. Lady Byron é caracterizada como uma
						``Clitemnestra moral''.}


Estavas triste --- mas contigo eu não me achava; &
E estavas doente, mas eu não estava aí; &
Só havia júbilo e saúde --- eu o pensava ---&
Longe de mim --- tristeza e dor somente aqui!& 
E é assim? --- Como eu predisse, e será sempre pior:& 
A alma recua sobre si, e o coração &
Jaz naufragado e frio, enquanto a lentidão& 
Os despojos partidos reúne sem calor. &
Não sentimos torpor em luta ou tempestade,& 
E, a vontade de ser, nelas não é perdida,\&

Porém no pós-silêncio que já a praia invade& 
Quando se perdeu tudo, exceto a simples vida.\&



Estou vingado! Mas isso era meu direito;&
Se cometi pecados, tu não foste enviada&
Para ser Nêmesis que me punisse a jeito&
--- Nem o céu escolheu víndice tão chegada!&
Para os piedosos a piedade! Fosses tal,&
Piedade agora te seria concedida.&
Do reinado do sono, a noite ei-la banida.&
Sim, podem bajular-te: sentirás um mal,\&

Uma cava agonia que será sem cura,&
Pois a praga mais funda tens por travesseiro;\&



Semeaste em minha dor, e hás de colher inteiro&
O mais triste colher em tão real agrura!&
Nenhum igual a ti, em tantos inimigos:&
Pois contra os mais eu me podia defender,&
Ou me vingar ou transformá-los em amigos;&
Mas, implacável, nada tinhas a temer,&
Pois a tua fraqueza qual broquel se erguia&
E erguia-se este amor, que muito concedeu,&
Poupando, só por ti, alguns que não devia;&
E sobre o mundo assim, que em tua verdade creu&
E no rumor de eu ser um moço desregrado&
--- Em coisas que não eram, e outras que são reais ---,&
Sobre essa própria base construíste a mais\&

Um monumento, por tua culpa cimentado!&
Clitemnestra moral contra o marido erguida,&
Derrubaste, com gládio insuspeitado, a calma,&
Fama e esperança --- e aquela mais amena vida &
Que, não fosse essa fria traição de tua alma, &
Tirar do túmulo da briga eu poderia &
E achar dever mais nobre do que o de partir. &
Porém tua virtude em vício se fazia, &
Traficando com ela, com a intenção mais fria, \&



Pela ira do presente e o ouro do porvir  &
--- E a um preço qualquer comprando a alheia dor. &
E assim por sendas tortuosas caminhando &
Tua antiga franqueza, digna de louvor, &
Não mais andou contigo --- mas de quando em quando, &
De teus delitos ignorante o coração, &
A falsidade, as asserções sem pé nem par, &
As evasivas e as ideias de plantão &
Em espíritos bifrontes --- o expressivo olhar &
Que mente com o silêncio --- e ainda a pretextada &
Prudência, com as vantagens a ela reunidas, \&


A aquiescência nas coisas todas dirigidas &
E como, não importa, à meta desejada  &
--- Tudo isso achou lugar em tua filosofia. &
Hábeis os meios, a intenção foi consumada:& 
--- Tu me fizeste o que contigo eu não faria!\&


\chapter*{Stanzas} %\protect\FootnoteZ{   }{\ \break\ \break\ }



When a man hath no freedom to fight for at home,&
Let him combat for that of his neighbours;&
Let him think of the glories of Greece and of Rome,&
And get knocked on the head for his labours.\&



To do good to mankind is the chivalrous plan,&
And is always as nobly requited;&
Then battle for freedom wherever you can,&
And, if not shot or hanged, you'll get knighted.\&

\pagebreak



\chapter{Estâncias\footnote[*]{ Composto em novembro de 1820.}


Liberdade não há por que lutar na pátria &
Que pela dos vizinhos se combata;&
Que se pense na glória da Hélade ou de Roma,&
E se receba a morte em troca da bravata.\&



Fazer o bem à humanidade é o fim cavalheiresco,&
E sempre nobremente ele é recompensado;&
Lutai portanto pela liberdade onde puderdes:&
Sofreis fuzil ou forca, ou sois condecorado.\&

\pagebreak


\chapter*{Ode to Napoleon Buonaparte}


'Tis done --- but yesterday a King!&
And armed with kings to strive ---& 
And now thou art a nameless thing:&
So abject --- yet alive!&
Is this the man of thousand thrones,&
Who strewed our earth with hostile bones,&
And can he thus survive?&
Since he, miscalled the Morning Star,&
Nor man nor fiend hath fallen so far.\&


Ill-minded man! why scourge thy kind&
Who bowed so low the knee?&
By gazing on thyself grown blind,&
Thou taughtest the rest to see.&
With might unquestioned --- power to save ---& 
Thine only gift hath been the grave,&
To those that worshipped thee;&
Nor till thy fall could mortals guess&
Ambition's less than littleness!\&

Thanks for that lesson --- it will teach&
To after-warriors more,&
Than high philosophy can preach,&
And vainly preached before.&
That spell upon the minds of men&
Breaks never to unite again,&
That led them to adore&
Those Pagod things of sabre-sway,&
With fronts of brass, and feet of clay.\&


The triumph, and the vanity,&
The rapture of the strife ---& 
The earthquake voice of victory,&
To thee the breath of life;&
The sword, the sceptre, and that sway&
Which man seemed made but to obey,&
Wherewith renown was rife ---& 
All quelled! --- Dark Spirit! what must be&
The madness of thy memory!\&


The desolator desolate!&
The victor overthrown!&
The arbiter of others' fate&
A suppliant for his own!&
Is it some yet imperial hope,&
That with such change can calmly cope?&
Or dread of death alone?&
To die a prince --- or live a slave ---& 
Thy choice is most ignobly brave!\&


He who of old would rend the oak,&
Dreamed not of the rebound;&
Chained by the trunk he vainly broke ---& 
Alone --- how looked he round?&
Thou, in the sternness of thy strength,&
An equal deed hast done at length,&
And darker fate hast found:&
He fell, the forest prowlers' prey;&
But thou must eat thy heart away!\&


The Roman, when his burning heart&
Was slaked with blood of Rome,&
Threw down the dagger --- dared depart,&
In savage grandeur, home.&
He dared depart in utter scorn&
Of men that such a yoke had borne,&
Yet left him such a doom!&
His only glory was that hour&
Of self-upheld abandoned power.\&


The Spaniard, when the lust of sway&
Had lost its quickening spell,&
Cast crowns for rosaries away,&
An empire for a cell;&
A strict accountant of his beads,&
A subtle disputant on creeds,&
His dotage trifled well:&
Yet better had he neither known&
A bigot's shrine, nor despot's throne.\&


But thou --- from thy reluctant hand&
The thunderbolt is wrung ---& 
Too late thou leavest the high command& 
To which thy weakness clung;&
All evil spirit as thou art,&
It is enough to grieve the heart&
To see thine own unstrung;&
To think that God's fair world hath been&
The footstool of a thing so mean;\&


And earth hath spilt her blood for him,& 
Who thus can hoard his own!&
And monarchs bowed the trembling limb,& 
And thanked him for a throne!&
Fair freedom! we may hold thee dear,&
When thus thy mightiest foes their fear& 
In humblest guise have shown.&
Oh! ne'er may tyrant leave behind&
A bright name to lure mankind!\&


Thine evil deeds are writ in gore,&
Nor written thus in vain ---& 
Thy triumphs tell of fame no more,& 
Or deepen every stain:&
If thou hadst died as honour dies,&
Some new Napoleon might arise,&
To shame the world again ---& 
But who would soar the solar height,&
To set in such a starless night?\&



Weighed in the balance, hero dust& 
Is vile as vulgar clay;&
Thy scales, mortality! are just& 
To all that pass away:&
But yet methought the living great&
Some higher sparks should animate, &
To dazzle and dismay:&
Nor deemed contempt could thus make mirth&
Of these, the conquerors of the earth.\&


And she, proud Austria's mournful flower,&
Thy still imperial bride;&
How bears her breast the torturing hour?&
Still clings she to thy side?&
Must she too bend, must she too share&
Thy late repentance, long despair,&
Thou throneless homicide?&
If still she loves thee, hoard that gem;&
'Tis worth thy vanished diadem!\&


Then haste thee to thy sullen isle,&
And gaze upon the sea;&
That element may meet thy smile ---& 
It ne'er was ruled by thee!&
Or trace with thine all idle hand,&
In loitering mood upon the sand,&
That earth is now as free!&
That Corinth's pedagogue hath now&
Transferred his by-word to thy brow.\&


Thou Timour! in his captive's cage&
What thoughts will there be thine,&
While brooding in thy prisoned rage?&
But one --- `The world \textit{was} mine!'&
Unless, like he of Babylon,&
All sense is with thy sceptre gone,&
Life will not long confine&
That spirit poured so widely forth ---& 
So long obeyed --- so little worth!\&


Or, like the thief of fire from heaven,&
Wilt thou withstand the shock?&
And share with him, the unforgiven,&
His vulture and his rock!&
Foredoomed by God --- by man accurst,&
And that last act, though not thy worst,&
The very Fiend's arch mock;&
He in his fall preserved his pride,&
And, if a mortal, had as proudly died!\&


There was a day --- there was an hour,&
While earth was Gaul's --- Gaul thine ---& 
When that immeasurable power&
Unsated to resign&
Had been an act of purer fame&
Than gathers round Marengo's name&
And gilded thy decline,&
Through the long twilight of all time,&
Despite some passing clouds of crime.\&



But thou forsooth must be a king&
And don the purple vest,&
As if that foolish robe could wring&
Remembrance from thy breast.&
Where is that faded garment? where&
The gewgaws thou wert fond to wear,&
The star, the string, the crest?&
Vain froward child of empire! say,&
Are all thy playthings snatched away?\&


Where may the wearied eye repose,&
When gazing on the great;&
Where neither guilty glory glows,&
Nor despicable state?&
Yes --- One --- the first --- the last --- the best ---& 
The Cincinnatus of the West,&
Whom envy dared not hate,&
Bequeath the name of Washington,&
To make man blush there was but One!\&


\pagebreak}


\chapter{Ode a Napoleão Bonaparte\footnote[*]{ Byron era favorável a
						Napoleão e só uma vez, lembra-o Bertrand Russell, se voltou
						contra o seu herói: nesta ode, que damos como curiosidade, o suicídio
						de Napoleão seria melhor, julgava Byron em 1814, do que a renúncia e a
						ilha de Elba. Mas, assinala ainda Russell, durante os Cem Dias Byron
						proclamou seu desejo de que Napoleão triunfasse e, ao receber a notícia
						de Waterloo, disse: ``Lamento-o
						terrivelmente''. O mito de Napoleão destruidor de reis
						e, portanto, mais próximo do povo, foi explorado em nosso Romantismo. O
						``Napoleão em Waterloo'', de Gonçalves de 
 						Magalhães, é um documento exponencial da supra-­humanização do herói.}

\vspace*{-\baselineskip}



Tudo acabado: Ontem um rei, porém!&
E armado para com outros reis lutar,&
És hoje um ser que já nem nome tem!&
Estar tão degradado --- e vivo estar!&
Este é o homem de mil tronos, então,&
Que de ossos de inimigos recobriu o chão&
E pode em tal vileza perdurar?&
{Como o anjo Estrela da Manhã, tão mal chamado}%
							\footnote { Lúcifer.}%
							,&
Homem nem demo, após, tão baixo foi lançado.\&



Por que açoitar, homem mal tencionado!&
Tua espécie que soube o joelho flexionar?&
Transfeito em cego, por te olhares demasiado,&
Tu ensinaste os outros a enxergar!&
Com o poder de salvar --- inquestionado e forte ---&
Tua dádiva foi tão-só a morte&
{Para quem viveu sempre a te adorar}%
						\footnote { Mílon de Crotona tentou abrir com as mãos 
						um tronco já fendido de carvalho. Ficou preso e foi devorado pelas
						feras. Esse atleta do século \textsc{vi} a.~C., várias vezes vencedor dos jogos 
						olímpicos, tentou o feito que o levou à morte ao ficar velho.}%
						:&
Até caíres nem sonharam os mortais&
Que é mais a pequenez do que a ambição, tão mais!\&



Graças por tal lição, que há de ensinar,&
Depois de ti, bem mais ao combatente&
Do que a filosofia pode predicar&
E foi em vão pregado anteriormente.&
Rompe-se para não se restaurar jamais&
Aquele encanto sobre a mente dos mortais&
Que os levou a cultuar fervidamente&
Tais ídolos --- que do poder da espada advêm&
E têm fronte de bronze e pés de barro têm!\&



O triunfo, e a vanglória,&
{O arroubo da luta renhida}%
						\footnote { Anota Byron que a expressão
						``the rapture of the strife'' corresponde a
						``certaminis gaudia'' --- expressão
						de Átila em fala ao seu exército antes da batalha de Chalons, expressão
						essa registrada por Cassiodoro.}%
					,&
A voz de terremoto da vitória,&
Para ti o sopro da vida;&
A espada, o cetro, e aquele teu poder&
Que os homens pareciam ter de obedecer&
E de que a fama estava tão provida&
--- Tudo caiu! --- Sombrio espírito! que história&
Mais louca não será tua memória!\&



O vitorioso, derrotado;&
Desolado, o desolador:&
O árbitro do alheio fado&
Feito, do seu, suplicador!&
Existe de imperar inda esperança&
Que possa competir com essa mudança?&
Ou só da morte o terror?&
Morrer príncipe --- ou viver escravo, à toa,&
Só na vileza a tua escolha é boa!\&



O que queria abrir o roble antigamente&
Não pensava no risco do retorno:&
Preso no tronco que forçara inutilmente&
--- Sozinho --- como procurava em torno?&
Tu, na rudez de teu vigor&
Cumpriste cabalmente um feito de igual cor,&
E teve, o fado teu, negríssimo contorno:&
Se a fera da floresta devorou então,&
Deves a pouco e pouco roer o coração!\&




{Quando o romano teve o coração ardente}%
						\footnote { Cincinato, célebre pela simplicidade de seus costumes,
						após ter sido cônsul (460 a.~C.) e duas vezes ditador, retornou às 
						lides do campo e à charrua.}&
Com o sangue de Roma saciado,&
Largou a adaga, e ousou rumar consciente,&
E com rude grandeza, para o lar amado.&
Ousou partir com extrema derrisão&
De homens que haviam suportado tal grilhão,&
Mas lhe deixavam esse fado!&
Sua glória foi só essa hora alada&
De força, que alcançara só, abandonada.\&



{O espanhol, quando o mando voluptuário}%
						\footnote { O imperador Carlos \textsc{v} de Espanha, que renunciou 
						ao trono e recolheu-se ao convento.}&
Perdeu o encanto estimulante,&
As coroas trocou pelo rosário,&
O império pela cela sufocante!&
Quer das camândulas estrito contador,&
Quer dos credos sutil discutidor,&
Seu dote ele esbanjou bem e bastante:&
Melhor não conhecesse, no seu dia humano,&
O santuário do beato e o trono do tirano.\&


Mas tu --- de tua mão que estava relutando&
O raio trovejante foi tirado ---&
Deixas bem tarde o alto comando&
Ao qual tua fraqueza tinha-se agarrado;&
Embora em tudo espírito malsão,&
Basta para oprimir o nosso coração&
Ver o teu entibiado;&
E esse mundo de Deus tem sido --- é crível? ---&
O escabelo de um ser tão desprezível.\&



Se para ele a terra derramou seu sangue,&
O sangue próprio pode ele resguardar!&
E os reis curvaram, a tremer, o joelho langue,&
Gratos por ele o trono lhes deixar!&
Liberdade! Podemos ver o teu valor,&
Se os teus mais fortes adversários seu pavor&
Humildemente vieram a mostrar.&
Oh, que não possa o déspota legar jamais&
Um nome rútilo que engode a nós, mortais!\&



Cada maldade tua em sangue está escrita,&
Porém não foi escrita em vão:&
A fama, cada triunfo teu já não a incita,&
Ou já de cada mancha agrava a abjeção.&
Se como a honra morre houvesses tu morrido,&
Poderia surgir novo tirano ardido&
Para vexar o mundo em mais uma versão.&
--- Mas quem tão alto como o sol voaria&
Para criar noite sem astros, tão sombria?\&



O pó do herói, pesado na balança,&
É vil, tal como o barro mais vulgar;&
Teu par de pratos, ó mortalidade, alcança&
Quem quer que esteja destinado a se findar;&
Porém os vivos, quando grandes, eu o cria,&
Chispa mais alta é que os animaria,&
Para virem a deslumbrar e apavorar!&
Nem julguei que o desdém zombar assim pudesse&
De tanto vencedor do mundo, que os padece.\&



E ela, a tristonha flor da Áustria orgulhosa,&
A tua esposa inda imperial,&
Como aguenta seu peito a hora angustiosa? &
Inda a teu lado está ela, afinal?&
Ela também deve inclinar-se e partilhar&
Tua tarda compunção, desperançado ar,&
Tu, homicida que perdeste a insígnia real?&
Se ela te ama ainda, guarda bem tal gema, &
Que vale o teu perdido diadema!\&



{Vai então para tua ilha sombria}%
						\footnote { A ilha de Elba.}&
E as águas põe-te a olhar;&
O teu sorriso, a onda o desafia,&
--- Nunca a vieste a governar!&
Ou traça com tua mão de todo ociosa,&
Na areia, com uma têmpera morosa,&
Que a terra agora está tão livre como o mar!&
{Que o pedagogo de Corinto a sua sentença}%
						\footnote { Segundo B.~Laroche, ``Dionísio, o jovem, 
						que passa por ter sido
						tirano ainda maior que o pai dele. Banido duas vezes de Siracusa,
						retirou-se para Corinto, onde se fez
						mestre-escola para ganhar a vida''.
						Ocorre-nos, todavia, um dos sete sábios da Grécia, também
						tirano, Periandro de Corinto, que nos deixou a máxima não seguida por
						Napoleão: ``Previdência em todas as coisas''.}&
Agora transferiu para tua fronte pensa.\&



{Tu, ó Timur! em sua jaula de cativo}%
						\footnote { ``A jaula de Bajazet, por ordem de 
						Tamerlão'' (Byron). O primeiro foi sultão otomano sempre vencedor (1347--1403),
						mas acabou capturado por Tamerlão em Ancira (1402).}&
Qual há de ser o pensamento teu,&
Ao veres que está preso o teu furor tão vivo!&
Apenas um: ``O mundo já me \textit{pertenceu}!''&
A menos, como o babilônio decaído,&
Que com teu cetro todo o senso hajas perdido,&
Por muito a vida não terá no cárcer seu&
O teu espírito que fluiu extensamente,&
Tanto tempo acatado, mas tão indigente!\&



{Do fogo celestial houve o que foi ladrão:}%
						\footnote { Prometeu.}&
Quererás aguentar-lhe o dissabor?&
E compartir com ele, o que não viu perdão, &
O seu rochedo e o seu açor!&
Por Deus fadado, pelo homem amaldiçoado,&
Esse último ato teu --- não o mais depravado&
A abóboda do diabo leva em derrisão;&
Ele manteve o orgulho, em sua descaída, &
E, se mortal, com orgulho igual perdera a vida!\&



Um dia, uma hora vieram a ocorrer,&
Quando a terra era a Gália, a Gália, teu lugar, &
Quando aquele imensíssimo poder, &
Não saciado para resignar,&
Teria feito ação de mais puro renome&
Que a que vive em Marengo a lhe rodear o nome&
E redourou teu descambar&
Através do crepúsculo da era inteira,&
Apesar de algum crime --- nuvem passageira.\&


Mas tu deves ser rei, rei inconteste,&
E o traje púrpura envergar,&
Como se conseguisse, tão simplória veste,&
De teu peito as lembranças arrancar.&
Onde está esse manto que perdeu a cor?&
Onde as nonadas que eram teu fervor,&
A estrela, a fita, o cocar?&
Dize, filho do império, irracional e vão!&
Teus brinquedos, tiraram-nos de tua mão?\&



{Onde terá repouso a vista, já cansada}%
						\footnote { Depois de ter sido duas vezes presidente dos Estados 
						Unidos, de que foi libertador, Washington voltou, como Cincinato, para
						os trabalhos agrícolas.}%
						,&
Quando se volte para os grandes a visão;&
Onde não brilha glória não culpada,&
Nem desprezível condição?&
Sim, o primeiro, o último, o excelente,& 
O Cincinato do Ocidente, &
Em quem a inveja não ousou pôr o ferrão:&
Lega o nome de Washington, tão-só, &
Para que os homens corem de ter sido um só!\&

\quebra

