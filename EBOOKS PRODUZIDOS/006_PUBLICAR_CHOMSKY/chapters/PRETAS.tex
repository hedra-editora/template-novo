\textbf{Noam Chomsky} (Filadélfia, 1928) é analista político e professor de Linguísticano Massachussetts Institute of Technology (\textsc{mit}). Além do trabalho na área de linguística, Chomsky é reconhecido internacionalmente como um dos maiores intelectuais vivos da esquerda, tendo publicado centenas de artigos e livros que abordam temas como mídia, movimentos sociais, política e economia global. Muito cedo, já aos dez anos de idade, escreveu um texto sobre a Revolução Espanhola, que lhe abriu as portas para um contato mais aproximado com o anarquismo, o qual, já nos anos seguintes, o influenciaria significativamente,fazendo com que se assumisse um socialista libertário. 
Iniciou seus estudos em linguística e filosofia em 1945 na Universidade da Pensilvânia e chegou a viver algum tempo em um kibbutz, em 1953. Nos anos 1950, iniciou o desenvolvimento de sua teoria sobre a “gramática gerativa”, a qual teve um profundo impacto no campo dos estudos linguísticos, fundamentalmente por meio da obra \textit{Estruturas sintáticas} (Edições 70, 1980), de 1957. Também formulou a chamada “Hierarquia de Chomsky”, uma classificação das linguagens formais a partir de seu poder gerativo. Ingressando no \textit{mit} em 1955, tornou-se professor titular em 1961, posição que ocupa até os dias de hoje. Adquiriu grande importância e notoriedade a partir da década de 1960 com o artigo “A responsabilidade dos intelectuais”, publicado em 1969 no livro \textit{O poder americano e os novos mandarins} (Record, 2006) --- uma compilação de artigos críticos à política externa dos Estados Unidos, particularmente levada a cabo na Guerra do Vietnã ---, que destaca Chomsky entre os intelectuais dissidentes da esquerda norte-americana. Escreveu, também, sobre o papel propagandista da mídia, publicando, com Edward S. Herman, em 1988, \textit{Manufacturing Consent [A manipulação do público: política e poder econômico no uso da mídia]} (Futura, 2003). 

\pagebreak
\thispagestyle{empty}

Ainda que constantemente ameaçado de morte por razão de seus escritos políticos, Chomsky segue escrevendo e publicando permanentemente no mundo todo. Dentre seus livros publicados no Brasil, estão: \textit{11 de setembro} (Bertrand Brasil, 2003), \textit{Contendo a democracia} (Record, 2003), \textit{O império americano} (Campus, 2004), \textit{Para entender o poder} (Bertrand Brasil, 2005), \textit{O lucro ou as pessoas} (Bertrand Brasil, 2006), \textit{O governo do futuro} (Record, 2007) e \textit{Razões de Estado} (Record, 2008).

\textbf{Notas sobre o anarquismo} é a maior compilação de Noam Chomsky já publicada sobre o assunto. Contando com oito entrevistas e dois artigos, o livro expõe pontos de vista acerca das bases ideológicas que fundamentam sua análise e sua proposta estratégica de transformação social. Partindo de clássicos como Mikhail Bakunin, Piotr Kropotkin e Rudolf Rocker, Chomsky defende suasposições sobre o anarquismo, que teria forjado as bases fundamentais do socialismo libertário, e afirma uma concepção significativamente eclética e antidogmática, cuja filiação ideológica seria proveniente de uma união entre o socialismo e o liberalismo. O socialismo, forjado na liberdade, concilia seus aspectos individuais e coletivos, e qualquer opressão, qualquer autoridade, quando ilegítima, deve ser denunciada e combatida. Chomsky acredita que esse é o princípio fundamental do anarquismo: luta e combate às estruturas autoritárias de poder, que são responsáveis pela dominação em todos os níveis. Por isso, critica severamente o socialismo de Estado, levado a cabo pelo marxismo de inspiração leninista, que restringiu severamente os espaços de liberdade, reforçando instituições como o Estado e os partidos. Discutindo estratégias de lutas populares --- as quais, segundo acredita, devem conciliar as lutas por reformas, e portanto de curto prazo, com a busca de um horizonte revolucionário de longo prazo ---, Chomsky sustenta posições pragmáticas de ganhos em relação às empresas e ao Estado. Ainda assim, para ele, o Estado precisaria ser algumas vezes reforçado, visando impedir \textit{tiranias ainda piores}, estabelecidas pelos poderes privados das corporações capitalistas.

\textbf{Alexandre Samis} é doutor em história pela Universidade Federal Fluminense (\textsc{uff}) e professor do Colégio Pedro \textsc{ii}. É autor dos livros \textit{Clevelândia: anarquismo, sindicalismo e repressão política no Brasil} (Imaginário/\,Achiamé, 2002), e \textit{Minha pátria é o mundo inteiro: Neno Vasco, o anarquismo e o sindicalismo revolucionário em dois mundos} (Letra Livre, 2009).

\textbf{Anarc} reúne obras escritas pelos expoentes da corrente libertária do socialismo, em sua maioria inéditas em língua portuguesa. Importante base teórica para a interpretação das grandes lutas sociais travadas desde a segunda metade do século \textsc{xix}, explicitam a evolução da ideia e da experimentação libertárias nos campos político, social e econômico, à luz dos princípios federalista e autogestionário.


