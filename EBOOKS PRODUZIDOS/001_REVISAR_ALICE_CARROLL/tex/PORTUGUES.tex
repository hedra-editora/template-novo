\hyphenation{con-tra-ria-do}
\hyphenation{brra-zo}
\hyphenation{extraor-di-ná-rio}
\hyphenation{voan-do}

%Capítulo I
\chapter{Na toca do coelho}
\pagestyle{baruch}

Alice estava começando a ficar muito cansada de ficar sentada ao lado da
irmã na margem do rio e de não ter nada para fazer: uma ou duas vezes,
ela espiara no livro que a irmã estava lendo, mas não tinha
nenhuma figura nem diálogos, ``e para que serve um livro'', pensou
Alice, ``sem figuras nem diálogos?'' \label{ref1}

Assim, ela estava considerando em seus pensamentos (da melhor forma que
podia, pois o dia quente a deixava muito sonolenta e estúpida) se o
prazer de fazer uma guirlanda de margaridas valeria o trabalho de
levantar e colher margaridas, quando, de repente, um Coelho Branco, de
olhos rosados, passou correndo perto dela.

Não havia nada de \emph{muito} marcante nisso; nem Alice achou
\emph{tão} estranho o fato de ela ter ouvido o Coelho dizer consigo
mesmo, ``Oh, céus! Oh, céus! Estou muito atrasado!''. (Quando ela pensou
sobre isso mais tarde, ocorreu-lhe que devia ter achado estranho,
mas na hora lhe pareceu bastante natural); mas quando o Coelho \label{ref2}
realmente \emph{tirou um relógio do bolso do colete} e olhou para as
horas, e então se apressou, Alice se pôs de pé, pois lhe passou pela
cabeça que nunca tinha visto nem coelho de colete, nem relógio guardado
no bolso, e, ardendo de curiosidade, ela correu pelo campo atrás dele e
chegou a tempo de vê-lo pular dentro de uma grande toca embaixo da
cerca viva.

No momento seguinte, Alice desceu atrás dele, sem jamais pensar em como
faria para sair dali depois.

A toca do coelho seguia reta como um túnel até parte do caminho, e de
repente descia vertiginosamente, tão de repente que Alice não teve um
momento sequer para pensar em parar, até se ver caindo pelo que parecia ser um
poço muito profundo.

Ou o poço era muito fundo, ou ela caiu muito devagar, pois ela teve
tempo suficiente na descida para olhar ao seu redor e se perguntar o
que lhe aconteceria em seguida. Primeiro, ela tentou olhar para baixo e
descobrir aonde estava caindo, mas era escuro demais para enxergar;
então ela olhou para os lados do poço e reparou que as paredes estavam
cobertas de armários de louça e estantes de livros: aqui e ali, ela viu
mapas e quadros pendurados em pregos. Ela tirou um pote de vidro de uma
das prateleiras ao passar; o rótulo dizia ``\textsc{geleia de laranja}'', mas
para sua frustração estava vazio; ela não soltou o pote de vidro com
medo de matar alguém lá embaixo, então conseguiu devolvê-lo em um dos
armários de louça pelos quais passou em sua queda.

``Bem!'', pensou Alice consigo mesma. ``Depois de uma queda dessas, não
vou achar nada demais em rolar escada abaixo! Como me acharão corajosa
lá em casa! Ora, eu não contaria nada a ninguém, nem que eu caísse do
telhado de casa!'' (O que provavelmente era verdade.)

Caindo, caindo, caindo. Aquela queda não terminava \emph{nunca}?
``Quantos quilômetros eu já tinha caído àquela altura?'', ela disse em
voz alta. ``Já devo estar chegando ao centro da terra. Vejamos: o centro
fica a mais de seis mil e quatrocentos quilômetros. Acho\ldots{}'' (pois,
como se vê, Alice havia aprendido várias coisas desse tipo na escola, e,
embora aquela não fosse uma oportunidade \emph{muito} boa para exibir
seus conhecimentos, pois não havia ninguém ali para ouvi-la, ainda assim
era um bom costume repetir o que se aprendeu)'' --- sim, a distância é
mais ou menos essa --- mas eu me pergunto: em que latitude ou longitude
devo ter chegado?'' (Alice não fazia ideia do que era latitude ou
longitude, mas achava que eram belas palavras grandiosas de serem
ditas.)

Então ela começou de novo. ``Será que vou cair \emph{através} da terra?
Seria engraçado sair lá do outro lado, entre as pessoas que andam de
ponta-cabeça! Os antipáticos, acho que chamam\ldots{}'' (ela ficou contente
por \emph{não} haver ninguém ouvindo dessa vez, pois não parecia mesmo
ser essa a palavra certa)'' --- ``mas vou precisar perguntar o nome do
país, sabe como é\ldots{} Por favor, senhora, aqui é a Nova Zelândia ou a
Austrália?'' (e ela tentou fazer uma mesura enquanto falava --- imagine
fazer uma \emph{mesura} enquanto se está caindo em plano ar! Você acha
que conseguiria?) ``E me achariam uma garotinha ignorante! Não, não vou
perguntar nada: talvez esteja escrito em algum lugar.''

Caindo, caindo, caindo. Não havia nada a fazer, então Alice começou a
falar de novo. ``A Dinah vai sentir minha falta hoje à noite, isso
sim!'' (Dinah era a gata.) ``Espero que alguém se lembre de dar o pires de
leite para ela na hora do chá. Dinah, querida, queria que você estivesse
aqui embaixo comigo! Não há camundongos no ar, infelizmente, mas você
poderia tentar caçar um morcego, que é muito parecido com um camundongo,
você sabe. Mas será que gata come morcego, é o que eu me pergunto\ldots{}'' E
aqui Alice começou a ficar com sono e seguiu falando sozinha, de um
modo um tanto sonhador, ``Gata come morcego? Gata come morcego?'' e, por
vezes, ``Morcego come gata?'', pois, você sabe, como ela não tinha
resposta a essas perguntas, não importava a ordem dos termos. Ela sentiu
como se estivesse cochilando, e havia começado a sonhar que estava de
mãos dadas com Dinah, e dizia muito seriamente: ``Ora, Dinah, diga a
verdade: você já comeu morcego?'', quando, de repente, tum! tum!, ela
caiu em um monte de gravetos e folhas secas, e a queda terminou.

Alice não se machucou nem um pouquinho e pulou para o chão no momento
seguinte: ela olhou para cima, mas estava tudo escuro sobre sua cabeça;
diante dela, havia outro longo corredor, e o Coelho Branco ainda estava
à vista, descendo esse corredor apressadamente. Não havia tempo a
perder: lá se foi Alice, feito o vento, chegando a tempo de ouvir,
quando o coelho virou uma esquina: ``Oh, minhas orelhas, minhas suíças,
como está ficando tarde!''. Ela estava logo atrás dele, quando virou a
esquina, mas o Coelho não estava mais à vista: ela se viu em um grande
salão, comprido, de teto baixo, iluminado por uma fileira de lustres que
pendiam do teto.

Esse salão era cercado de portas, mas estavam todas trancadas; e, depois
que Alice percorreu todo o recinto, de um lado para o outro,
experimentando todas, ela caminhou tristonha até o meio do salão,
perguntando-se se algum dia conseguiria sair dali.

De repente, ela se deparou com uma mesinha de três pernas, toda feita de
vidro maciço; não havia nada na mesa além de uma minúscula chave
dourada, e a primeira ideia de Alice foi que talvez a chave pertencesse
a uma das portas do salão; mas, infelizmente, ou as fechaduras eram
muito grandes, ou a chave era muito pequena, mas de qualquer forma não
abrira nenhuma delas. Contudo, na segunda tentativa, ela se deparou com
uma cortina baixa em que não havia reparado antes, e detrás da cortina
havia uma portinha de menos de quarenta centímetros de altura: ela
experimentou a chavezinha dourada na fechadura, e, para sua grande
alegria, a chave servia!

Alice abriu a portinha e descobriu que dava em um pequeno corredor, não
muito maior do que um buraco de rato: ela se ajoelhou e olhou através do
corredor e do outro lado avistou o jardim mais adorável que alguém já
viu na vida.\,Como ela quis sair daquele salão escuro e passear entre
aqueles canteiros de flores coloridas e fontes de água fresca! Mas ela
não conseguia passar nem mesmo a cabeça pela portinha. ``Mesmo que a
minha cabeça passe'', pensou a pobre Alice, ``ela seria de pouca
utilidade sem meus ombros. Oh, como eu queria me fechar feito um
telescópio! Acho que eu conseguiria, se ao menos soubesse como
começar.'' Pois, você sabe, tantas coisas estranhas haviam acontecido
ultimamente que Alice havia começado a pensar que pouquíssimas coisas
de fato deviam ser realmente impossíveis.

Aparentemente era inútil esperar diante da portinha, de modo que ela
voltou até a mesa, com certa esperança de encontrar ali outra chave, ou
ao menos um livro de regras para fechar pessoas como telescópios: dessa
vez ela encontrou sobre a mesa uma garrafinha (``que seguramente não
estava ali antes'', disse Alice) e amarrado ao gargalo da garrafa havia
uma etiqueta de papel onde se lia ``\textsc{beba-me}'' lindamente impresso em letras
garrafais.

Não era um problema estar escrito ``Beba-me'', mas a pequena sabichona
Alice não faria \emph{isso} com pressa. ``Não, antes vou olhar'', ela
disse, ``e ver se está escrito `veneno' ou não'', pois ela havia lido
várias historinhas agradáveis sobre crianças que eram queimadas ou
devoradas por animais selvagens, e outras coisas desagradáveis, tudo
porque elas \emph{não} se lembraram das regras mais simples que os pais
ensinavam: por exemplo, saber que um atiçador incandescente
acabará queimando se você segurá-lo por muito tempo; e que, se você
fizer um corte \emph{muito} fundo com uma faca, geralmente vai sair
sangue; e ela jamais esqueceria que, se você beber muito de uma garrafa
escrito ``veneno'', quase certamente passará mal, cedo ou tarde.

No entanto, nessa garrafa \emph{não} estava escrito ``veneno'', de modo
que Alice arriscou provar e, achando muito gostoso (o líquido, na
verdade, tinha um sabor misturado de torta de cereja, creme, abacaxi,
peru assado, café e torrada com manteiga), logo ela bebeu todo o
conteúdo.

``Que sensação engraçada!'', disse Alice. ``Devo estar me fechando como
um telescópio.''

E, de fato, ela estava agora com apenas vinte e cinco
centímetros de altura, e seu rosto se iluminou com a ideia de que agora
ela tinha o tamanho certo para passar pela portinha e ir até aquele
jardim adorável. Primeiro, contudo, ela esperou alguns minutos para
ver se ainda diminuiria mais: ela ficou um pouco nervosa quanto a isso,
``pois pode acontecer, você sabe'', disse Alice consigo mesma, ``de eu
sumir totalmente, como uma vela. Como será que eu seria então?'' E ela
tentou imaginar como seria a chama de uma vela depois que a vela se
apaga, pois ela não se lembrava de ter visto algo assim acontecer.

Após algum tempo, vendo que nada mais acontecia, ela resolveu ir de uma
vez até o jardim; mas, pobre Alice!, quando chegou na portinha, percebeu
que havia esquecido a chavezinha dourada, e, quando voltou à mesa para
buscá-la, notou que era impossível alcançá-la: ela podia vê-la
claramente através do vidro da mesa, e fez o melhor que podia para
escalar por um dos pés da mesa, mas o vidro era muito escorregadio; e,
quando a pobrezinha se cansou de tentar, sentou-se no chão e começou a
chorar.

``Ora, vamos, não adianta nada ficar assim chorando!'', disse Alice
consigo mesma, um tanto ríspida. ``Aconselho você a sair daqui agora!''\,Ela geralmente se dava muito bons conselhos (embora raramente os
seguisse), e algumas vezes ralhava consigo mesma tão severamente que lhe
vinham lágrimas aos olhos; e, uma vez, ela se lembrava de ter tentado
guardar as lágrimas em~uma caixa, por ter trapaceado a si mesma em uma
partida de \textit{croquet} que estava jogando sozinha, pois essa curiosa criança
gostava muito de fingir ser duas pessoas. ``Mas agora não adianta
nada'', pensou a pobre Alice, ``fingir ser duas pessoas! Pois restou tão
pouco de mim que mal daria \emph{uma} pessoa de respeito!''

Logo seus olhos depararam com uma caixinha de vidro que estava embaixo \label{ref3}
da mesa: ela abriu e encontrou um bolinho muito pequeno, no qual estava
escrito, lindamente composto com uvas-passas: ``coma-me''. ``Bem coma'',
disse Alice, ``e se isso me fizer crescer, posso alcançar a chave; e, se
me fizer diminuir, poderei me arrastar por baixo da porta; de modo que
seja como for chegarei ao jardim, e não me importa o que vai
acontecer!''

Ela deu uma mordidinha e disse aflita consigo mesma: ``Será que estou
crescendo? Será que estou diminuindo?'', pondo a mão no topo da cabeça
para sentir o sentido do crescimento, e ficou muito surpresa ao notar
que permanecia do mesmo tamanho; sem dúvida, é o que geralmente acontece
quando comemos bolo, mas Alice já estava tão acostumada a esperar que
coisas inesperadas acontecessem que lhe pareceu uma tolice, sem graça,
que a vida continuasse do mesmo jeito.

Então ela resolveu agir e, em pouquíssimo tempo, comeu o bolo inteiro.

%Capítulo II

\quebra\chapter{Piscina de lágrimas}

``Que bizarrismo!'', gritou Alice (ela ficou tão surpresa que por um
momento se esqueceu das regras da própria língua; ``agora estou me
abrindo como o maior telescópio do mundo! Adeus, meus pés!'' (pois,
quando ela olhou para baixo, seus pés pareciam quase fora do alcance da
visão de tão distantes que estavam). ``Oh, meus pobres pezinhos! Quem
vai colocar sapato e meia em vocês agora, meus queridos? Sem dúvida, eu
que não vou poder! Vou estar muito longe para me ocupar de vocês: vocês
terão que se virar sozinhos da melhor forma que puderem\ldots{} mas devo ser
boazinha com eles'', pensou Alice, ``porque talvez eles resolvam não me
obedecer depois! Vejamos: eu lhes darei botas novas todo Natal.''

E ela prosseguiu planejando consigo mesma como faria aquilo. ``As botas
devem vir pelo correio'', ela pensou, ``e que engraçado seria mandar
presentes para os próprios pés!\,E como ficaria absurdo no endereço!

\begin{quote}
Sr. Pé Direito de Alice.\\
Tapete da Lareira,\\
próximo ao Guarda-Fogo,\\
(com Amor, da Alice).
\end{quote}

Oh, céus, quanta bobagem estou falando!''

Nesse instante, sua cabeça bateu no teto do salão: na verdade, ela
estava agora com quase três metros de altura e imediatamente pegou
a chavezinha dourada e correu para a porta do jardim.

Pobre Alice! Foi o máximo que ela conseguiu fazer, deitar-se de lado,
para olhar para o jardim com um olho só; mas atravessar agora era ainda
mais impossível do que nunca: ela se sentou no chão e começou a chorar
de novo.

``Você devia ter vergonha de si mesma'', disse Alice, ``uma meninona
grande como você'' (nisso ela tinha razão), ``ficar assim chorando desse
jeito! Pare já com isso, estou lhe dizendo!'' Mas ela continuou da mesma
forma, despejando galões e galões de lágrimas, até que se formou uma
piscina de uns dez centímetros de profundidade, ocupando metade do
salão.

Após algum tempo, ela ouviu o som de passos ao longe e rapidamente
enxugou os olhos para ver o que estava acontecendo. Era o Coelho Branco
que voltava, esplendidamente vestido, com luvas brancas em uma mão e um
grande leque na outra: ele vinha trotando, muito apressado, resmungando
consigo mesmo no caminho, ``Oh! A Duquesa, a Duquesa! Oh, ela vai ficar
furiosa com a minha demora!''. Alice se sentia tão desesperada que
estava disposta a pedir ajuda a qualquer um; então, quando o Coelho se
aproximou, ela começou, em voz baixa e tímida: ``Por favor, senhor\ldots{}''
O Coelho se assustou brutalmente, deixou cair as luvas brancas e o
leque e fugiu correndo para a escuridão, o mais depressa que podia.

Alice recolheu o leque e as luvas e, como o salão estava muito abafado,
ficou se abanando com o leque todo o tempo em que ficou falando! ``Ora, ora!
Tudo está tão estranho hoje! Ontem tudo parecia normal. Será que eu
mudei durante a noite? Deixe-me pensar: será que eu \emph{era} a mesma
pessoa quando acordei esta manhã? Acho que quase lembro de ter me
sentido um pouquinho diferente. Mas se eu não sou mais a mesma, a
pergunta seguinte é: quem afinal eu sou? Ah, \emph{esse} é o grande
enigma!'' E ela começou a recapitular todas as crianças que tinham a
mesma idade que ela para ver se ela podia ter sido trocada por alguma
delas.

``Com certeza, eu não sou a Ada'', ela disse, ``porque o cabelo dela é
comprido e encaracolado, e o meu não faz nenhum cacho; e tenho certeza
de que não posso ser a Mabel, porque eu sei todo tipo de coisa, e ela, oh!,
ela sabe muito pouco! Além do mais, \emph{ela} é ela, e \emph{eu sou}
eu, e\ldots{} oh, céus, como isso tudo é enigmático! Deixe-me ver se eu ainda
sei todas as coisas que eu sabia. Deixe-me ver: quatro vezes cinco,
doze, e quatro vezes seis, treze, e quatro vezes sete é\ldots{} oh, céus!
Assim nunca vou chegar no vinte! Mas Tabuada não faz sentido: vamos ver
Geografia. Londres é a capital de Paris, e Paris é a capital de Roma, e
Roma\ldots{} não, \emph{isso} está tudo errado, tenho certeza! Devo ter sido
trocada pela Mabel! Vou tentar recitar `\emph{Como o pequeno\ldots{}}'\,'', e
ela cruzou as mãos no colo, como se estivesse lendo uma lição, e começou
a repetir o poema, mas sua voz saiu rouca e estranha, e as palavras não
eram as palavras de costume:\ldots{}

\begin{quote}
Como o pequeno crocodilo\\
Levanta a luzidia cauda,\\
Espalham águas do Nilo\\
As escamas douradas!
\medskip
Como alegre gargalha,\\
Como abre suas garras,\\
E recebe bem peixinhos\\
Na risonha bocarra!
\end{quote}

``Tenho certeza de que essas palavras estão erradas'', disse a pobre Alice,
e seus olhos se encheram de lágrimas de novo, conforme ela prosseguiu.
``Devo ser a Mabel mesmo, e vou ter que ir morar naquela casinha
minúscula, e não ter quase nenhum brinquedo, e oh! E ainda tanta lição
para aprender! Não, quanto a isso está decidido: se eu for a Mabel, eu
vou ficar aqui embaixo. Não vai adiantar nem enfiarem a cabeça aqui e
dizerem `Volte, querida!'. Eu vou olhar lá para cima e falar, `Quem eu
sou afinal? Primeiro, digam e, depois, se eu gostar da pessoa, eu volto:
se não, fico aqui até que eu seja outra pessoa'\ldots{} mas, oh, céus!'',
gritou Alice com um súbito acesso de choro: ``Quem dera eles enfiassem a
cabeça aqui embaixo! Estou \emph{muito} cansada de ficar aqui
sozinha!''.

Ao dizer isso, ela olhou para suas mãos e ficou surpresa ao ver que
havia posto uma das luvinhas brancas do Coelho enquanto falava. ``Como
eu \emph{consegui} vestir isto?'', ela pensou. ``Devo estar diminuindo
de novo.'' Ela se levantou e foi até a mesa para se medir em comparação,
e descobriu que, aproximadamente, devia estar agora com pouco mais de
meio metro de altura, e continuava diminuindo rapidamente: ela logo
descobriu que a causa daquilo era o leque que estava segurando e o
soltou bruscamente, a tempo de evitar uma diminuição total.

``Essa foi por pouco!'', disse Alice, um bocado assustada com a súbita
mudança, mas muito contente de se encontrar ainda existente; ``e agora
direto para o jardim!'', e ela correu a toda velocidade de volta até a
portinha: mas ai!, a portinha estava outra vez fechada, e a chavezinha
de ouro estava sobre a mesa de vidro como antes, ``e as coisas estão
piores do que nunca'', pensou a pobre criança, ``pois nunca fui tão
pequena assim antes, nunca! E posso afirmar que é muito ruim, isso
sim!''

Ao dizer essas palavras, seu pé escorregou e, no momento seguinte,
splash!, ela estava enfiada até o pescoço em água salgada. Sua primeira
ideia foi que de alguma forma havia caído no mar, ``e, nesse caso,
poderei voltar de trem'', ela disse consigo. (Alice tinha ido à praia
uma única vez na vida, e chegara à conclusão geral de que, aonde quer
que você fosse no litoral inglês, encontraria uma série de guarda-sóis à
beira-mar, algumas crianças cavando na areia com pazinhas de madeira e
uma fileira de chalés de aluguel, e atrás deles uma estação de trem.) No
entanto, ela logo se deu conta de que estava na piscina de suas próprias
lágrimas choradas quando estava com quase três metros de altura.

``Quem dera eu não tivesse chorado tanto!'', disse Alice, enquanto
nadava, tentando encontrar uma saída. ``Serei castigada por isso agora,
imagino, sendo afogada nas minhas próprias lágrimas! Isso sim seria uma
coisa estranha, com certeza! Mas tudo está estranho hoje.''

Nesse instante, ela ouviu algo se mexendo na água um pouco adiante e
nadou para perto para ver o que era: a princípio, ela pensou se tratar
de uma morsa ou de um hipopótamo, mas então se lembrou de quão minúscula
estava agora e logo percebeu que era um mero camundongo que também
havia escorregado como ela.\looseness=-1

``Será que adiantaria alguma coisa agora'', pensou Alice, ``falar com
esse camundongo? Tudo tem sido tão inesperado aqui embaixo que deveria
achar provável que esse camundongo falasse: seja como for, não faz mal
nenhum tentar.'' Então ela começou: ``Ó Camundongo, você sabe como se
sai desta piscina? Estou muito cansada de nadar aqui, ó Camundongo!''
(Alice achou que esse devia ser o modo apropriado de se dirigir a um
camundongo; ela nunca tinha feito nada parecido antes, mas se lembrou de
ter visto no livro de latim do irmão ``O camundongo --- de um camundongo
--- para um camundongo --- um camundongo --- Ó camundongo!''.) O Camundongo
olhou para ela um tanto inquisitivamente, e pareceu-lhe piscar com um de
seus olhinhos, mas não disse nada.

``Talvez ele não fale a minha língua'', pensou Alice. ``Talvez seja um
camundongo francês, que veio com Guilherme, o Conquistador.'' (Pois,
apesar de todo seu conhecimento de história, Alice não tinha uma ideia
muito clara de quanto tempo atrás as coisas tinham acontecido.) Então
ela começou de novo: ``\emph{Où est ma chatte?}'' {[}Onde está minha
gata?{]}, que era a primeira frase de seu livro de francês. O Camundongo
saltou subitamente para fora da água e parecia trêmulo de pavor. ``Oh,
me desculpe!'', gritou Alice apressadamente, com medo de ter magoado os
sentimentos do pobre animal. ``Eu esqueci que você não gosta de gato.''

``Não gosta de gato?!'', exclamou o Camundongo, com voz aguda,
passional. ``Você \emph{gostaria} de gato se fosse eu?''

``Bem, talvez não'', disse Alice em tom suave. ``Não se irrite. Mas você
deveria conhecer a nossa gata, a Dinah: acho que você passaria a gostar
de gatos se a visse. Ela é uma criatura tão querida e sossegada'', Alice
prosseguiu, em parte consigo mesma, como se estivesse nadando
preguiçosamente na piscina, ``e ela fica ronronando bem bonitinha diante
da lareira, lambendo as patinhas e limpando o rosto --- e ela é fofinha e
fica bem no colo --- e é ótima para apanhar camund\ldots{} oh, me desculpe!'',
exclamou Alice outra vez, pois agora o Camundongo estava todo eriçado, e
ela percebeu que certamente ele estava realmente ofendido. ``Não vamos
mais falar nela se você não quiser.''

``De fato, não vamos!'', exclamou o Camundongo, que estava tremendo até
a ponta da cauda. ``Como se eu fosse falar sobre isso!\,Nossa família
sempre odiou gato: criaturas cruéis, vis, vulgares! Não me obrigue a
ouvir esse nome outra vez!''

``Não, não obrigarei!'', disse Alice, querendo depressa mudar de
assunto. ``Você\ldots{} você gosta\ldots{} de\ldots{} cachorro?'' O Camundongo não
respondeu, de modo que Alice prosseguiu avidamente: ``Há um cachorrinho
tão bonitinho perto da nossa casa, eu adoraria lhe mostrar! Um terrier
de olho claro, sabe? Oh, ele tem o pelo comprido, encaracolado, marrom!
E ele busca coisas que a gente joga, e ele fica parado esperando a
comida, e todo esse tipo de coisa --- nem posso lembrar metade delas --- e
o dono é um fazendeiro, sabe? E ele diz que é um cachorro muito útil,
custou cem libras! Disse que mata todos os ratos e ca\ldots{} oh, céus!'',
exclamou Alice em tom tristonho. ``Receio que o tenha ofendido outra
vez!'' Pois o Camundongo estava indo embora nadando o mais depressa que
podia e causando grande comoção na piscina em seu trajeto.

Então ela o chamou suavemente: ``Camundongo querido! Volte aqui e não
vamos mais falar de gato ou de cachorro, já que você não gosta!''.

Quando o Camundongo ouviu isso, ele se virou e nadou de volta lentamente
para ela. Seu rosto estava muito pálido (de paixão, pensou Alice), e
disse em voz baixa e trêmula: ``Vamos para a margem, e então vou lhe
contar minha história, e você entenderá por que eu odeio gato e
cachorro''.

\textls[-10]{Estava na hora de ir, pois a piscina estava ficando muito cheia de
pássaros e animais que haviam caído ali dentro: havia um Pato e Dodo, um
Loro e uma Aguiazinha, e diversas outras criaturas curiosas. Alice foi
na frente, e o grupo inteiro nadou para a margem.}\looseness=-1

%Capítulo III
\quebra\chapter[Corrida eleitoral e uma história sinuosa]{Corrida eleitoral e uma\break história sinuosa}
\markboth{corrida eleitoral}{}

Eles eram mesmo um grupo de aparência esquisita que se reuniu na margem
do rio --- as aves com as penas molhadas, os animais com o pelo
escorrido, e todos ensopados, pingando, contrariados e incomodados.

A primeira questão, evidentemente, era como se secar: eles se
consultaram a respeito, e alguns minutos depois pareceu perfeitamente
natural para Alice se ver conversando com familiaridade com eles, como
se os conhecesse desde sempre.\,Na verdade, ela teve uma longa discussão
com o Louro, que ficou enfim carrancudo, e só dizia ``Sou mais velho
que você, eu é que sei'', e isso Alice não engoliu sem antes saber
quantos anos ele tinha, e, como o Louro se recusou terminantemente a
dizer sua idade, nada mais foi dito.

Por fim, o Camundongo, que parecia ter alguma autoridade entre eles,
exclamou: ``Sentem-se, todos vocês, e me escutem! Deixem comigo, vocês
rapidamente ficarão secos''. Todos eles se sentaram ao mesmo tempo, em
um grande círculo, com o Camundongo no meio. Alice ficou olhando fixa e
ansiosamente para ele, pois tinha certeza de que pegaria um resfriado se
não se secasse logo. \label{ref4}

``Aham!'', disse o Camundongo com um ar importante. ``Vocês estão
prontos? Essa é a coisa mais árida que eu conheço. Silêncio, todos
vocês, por favor! `Guilherme, o Conquistador, cuja causa é favorecida
pelo papa, logo submetido pelos ingleses, a quem faltavam líderes, e
ultimamente se acostumaram à usurpação e à conquista. Edwin e Morcar,
condes de Mercia e Northumbria\ldots{}'\,''

``Ugh!'', disse o Louro, com um tremor.

``Com sua licença!'', disse o Camundongo, franzindo a testa, mas muito
polidamente. ``Você disse alguma coisa?''

``Eu não!'', disse o Louro prontamente.

``Achei que você tivesse dito'', disse o Camundongo. ``\ldots{}~Prossigo: `Edwin e Morcar, condes de Mercia e Northumbria, declararam-no: e até
mesmo Stigand, o patriótico Arcebispo de Canterbury, achou isso
aconselhável\ldots{}''

``Achou o quê?'', disse o Pato.

``Achou \emph{isso}'', o Camundongo respondeu um tanto contrariado.
``Evidentemente você sabe o que significa \emph{isso}.''

``Sei muito bem o que significa \emph{isso}, quando \emph{acho}'', disse
o Pato; ``geralmente é uma rã ou uma minhoca. A questão é o que o
arcebispo achou.''

O Camundongo não deu ouvidos a essa pergunta, mas prosseguiu
apressadamente, ```achou aconselhável ir com Edgar Atheling ao encontro
de William e oferecer-lhe a coroa. A conduta de William a princípio foi
moderada. Mas a insolência de seus normandos\ldots{}' Como você está agora,
minha querida?'', ele continuou, virando-se para Alice ao falar.

``Molhada como nunca'', disse Alice em tom melancólico; ``aparentemente
isso não está me secando nada.''

``Nesse caso'', disse o Dodo solenemente, ficando de pé, ``proponho que
a sessão entre em recesso para a adoção imediata de remédios mais
efetivos\ldots{}''

``Fale a nossa língua!'', disse a Aguiazinha. ``Não entendo metade
dessas palavras difíceis e, mais do que isso, não creio nem que você
entenda!'' E a Aguiazinha inclinou a cabeça para esconder um sorriso:
algumas das outras aves deram risadinhas.

``O que eu ia dizer'', disse o Dodo em tom ofendido, ``era que a melhor
coisa para nos secar seria uma Corrida eleitoral.''

``O que é Corrida eleitoral?'', disse Alice; não que ela quisesse muito
saber, mas o Dodo fez uma pausa, como se achasse que alguém fosse falar,
e mais ninguém se sentiu tentado a dizer nada.

``Ora'', disse o Dodo, ``a melhor forma de explicar é fazendo.'' (E,
como você pode querer tentar fazer isso, em um dia de inverno, vou lhe
dizer o que fez o Dodo.)

Primeiro ele traçou uma pista de corrida, em uma espécie de círculo (``a
forma exata não tem importância'', ele disse), e então o grupo inteiro
foi posicionado na largada, aqui e ali. Não houve nenhum ``um, dois,
três e já'', mas eles começaram a correr, quando bem entenderam, e só
pararam quando quiseram, de modo que não foi fácil saber quando a
corrida acabou. No entanto, depois de correrem meia hora, mais ou menos,
quando se viram secos de novo, o Dodo subitamente anunciou ``Corrida
encerrada!'' e todos se reuniram em volta dele, ofegantes, perguntando
``Mas quem ganhou?''.

A essa questão o Dodo não conseguiu responder sem antes pensar bastante, e
ele ficou muito tempo com um dedo apontado na testa (posição em que
geralmente vemos Shakespeare, nos retratos dele), enquanto o resto
esperou em silêncio. Por fim, o Dodo disse ``Todo mundo ganhou e todos
devem receber prêmios''.

``Mas quem vai dar os prêmios?'', um belo coral de vozes perguntou.

``Ora, ela, é claro'', disse o Dodo, apontando para Alice com um dedo; e
o grupo inteiro ao mesmo tempo se aproximou em volta dela, exclamando de
modo confuso: ``Prêmios! Prêmios!''.

Alice não fazia ideia do que devia fazer e desesperadamente pôs a mão
no bolso, tirando uma caixa de doces (que, por sorte, a água salgada não
havia penetrado), e os distribuiu a todos como prêmios. Havia exatamente
um doce para cada um.

``Mas ela também deve receber um prêmio, vocês sabem\ldots{}'', disse o
Camundongo.

``É claro'', o Dodo respondeu muito gravemente.

``O que mais você tem no bolso?'', ele prosseguiu, virando-se para Alice.

``Só um dedal'', disse Alice, tristonha.

``Passe para cá'', disse o Dodo.

Todos se aproximaram dela outra vez, enquanto o Dodo solenemente
apresentou o dedal, dizendo: ``Esperamos que você aceite este elegante
dedal'', e, quando ele terminou o breve discurso, todos deram vivas.

Alice achou tudo aquilo muito absurdo, mas todos pareciam tão graves que
ela não ousou dar risada; e, como não lhe ocorreu mais nada para dizer,
ela simplesmente fez uma mesura e aceitou o dedal, aparentando a máxima
solenidade que conseguiu.

A próxima coisa a fazer era comer os doces, o que causou certo barulho e
confusão, pois as aves maiores se queixaram por não conseguirem nem
sentir o gosto dos seus, e as menores engasgaram e precisaram levar
tapinhas nas costas. No entanto, por fim, acabou, e todos sentaram de
novo em círculo e pediram ao Camundongo que lhes contasse mais alguma
coisa.

``Você prometeu que ia contar a sua história, você sabe\ldots{}'', disse
Alice, ``e o motivo de você odiar\ldots{} G e C'', ela acrescentou
sussurrante, com um certo medo de que ele se ofendesse outra vez.\looseness=-1

``Minha história é como o meu rabo: sinuosa e triste'', disse o
Camundongo, virando-se para Alice e suspirando.

``Sinuoso, de fato'', disse Alice olhando para o rabo do Camundongo.
``Mas por que triste?'' E ela continuou intrigada enquanto o Camundongo
falava, de modo que a ideia que lhe ficou da história foi algo mais ou
menos assim:

\begin{quote}
Furioso, cachorrão,\\
disse,\\
ao camundongo,\\
Que\\
encontrou na\\
casa,\\`Vamos pro\\
tribunal. Vou te pro-\\
cessar\ldots{}\\
Ora,\\
vamos, não\\
aceito não:\\
Faça-\\
mos logo\\
julga-\\
mento;\\
``De fato\\
hoje cedo\\
estou\\
livre''\\
Disse o\\
camundongo\\
ao cão,\\
``Esse pro-\\
cesso todo,\\
meu senhor,\\
Sem júri ou juiz\\
Seria uma\\
perda de tempo''.\\
``Serei teu\\
juiz,\\
serei teu\\
júri'',\\
disse\\
astuto\\
o velho\\
Furi-\\
oso:\\
``Pro-\\
cesso\\
teu\\
caso\\
e con-\\
deno-te\\
à\\
mor-\\
te.''
\end{quote}

``Você não está prestando atenção'', disse o Camundongo a Alice
severamente. ``No que você está pensando?''

``Desculpe'', disse Alice humildemente. ``Será que você não está se
desviando demais?''

``Não, está bem amarrado!'', exclamou o Camundongo, irritadiço.

``Amarrado!'', disse Alice, sempre prestativa, e olhando aflita para os
lados. ``Oh, deixe-me ajudar a desamarrar!''

``Não deixo coisa alguma'', disse o Camundongo, levantando-se e
retirando-se. ``Você me insulta com esses absurdos!''

``Não foi por mal!'', suplicou a pobre Alice. ``Mas você se ofende por
qualquer coisa, você sabe\ldots{}''

O Camundongo simplesmente grunhiu em resposta.

``Por favor, volte para terminar a sua história!'', Alice pediu a ele. E
os outros todos se juntaram em coro, ``Sim, por favor!'', mas o
Camundongo balançou a cabeça com impaciência e continuou a andar um
pouco mais depressa.

``Que pena que ele não ficou'', suspirou o Louro, quando já não se
avistava mais ele; e uma velha Carangueja aproveitou a oportunidade para
dizer à filha: ``Ah, meu bem! Que isso sirva de lição para você nunca
perder a paciência!''.

``Controle-se, mamãe!'', disse a jovem Caranguejinha, um tanto
impertinente. ``Você é demais até para a paciência de uma ostra!''

``Quem dera Dinah estivesse aqui, como eu queria!'', disse Alice em voz
alta, para ninguém em particular. ``Ela logo o buscaria!''

``E quem é Dinah, se você me permite perguntar?'', disse o Louro.

Alice respondeu avidamente, pois estava sempre disposta a falar de sua
gata: ``A Dinah é a nossa gata. E ela é ótima caçadora de camundongos,
vocês não imaginam! E, oh, vocês precisavam vê-la correndo atrás das
aves! Ora, ela devora todo pássaro que encontra!''

Esse discurso causou notável comoção no grupo. Algumas aves foram embora
imediatamente; uma velha Pêga começou a se arrumar com parcimônia,
comentando: ``Eu realmente preciso ir para casa; o ar frio da noite não
faz bem à minha garganta!'', e um Canário chamou com voz trêmula os
filhos: ``Vamos, meus queridos! Já passou da hora de vocês estarem na
cama!''. Sob vários pretextos, todos foram embora e logo Alice ficou
sozinha.

``Quem dera eu não tivesse mencionado a Dinah!'', ela disse consigo
mesma em tom melancólico. ``Pelo visto, ninguém gosta dela aqui embaixo,
e eu não tenho dúvida de que ela é a melhor gata do mundo! Oh, minha
querida Dinah! Será que algum dia voltarei a vê-la\ldots{}''

E aqui a pobre Alice começou a chorar de novo, pois se sentia sozinha e
desanimada. Dali a pouco, no entanto, outra vez, ela ouviu um som de passos
ao longe e ergueu os olhos avidamente, com certa esperança de que o
Camundongo tivesse mudado de ideia e estivesse voltando para terminar
sua história.

%Capítulo IV
\quebra\chapter[O Coelho manda um lagarto]{O Coelho manda\break um lagarto}

Era o Coelho Branco, voltando lentamente em seu trote e olhando aflito
para os lados, no caminho, como se tivesse perdido alguma coisa; e ela
o ouviu resmungar consigo mesmo: ``A Duquesa! A Duquesa! Oh, minhas
patas! Oh, minha pele e minhas suíças! Ela vai mandar me matar, tão
certo como furões são furões! Onde posso ter deixado cair?''. Alice
concluiu instantaneamente que ele estava procurando o leque e as luvas
brancas, e ela, muito generosamente, começou a procurá-los, mas não
encontrou em lugar nenhum --- tudo parecia ter se transformado desde que
ela nadara na piscina, e o grande salão, com a mesa de vidro e a
portinha, havia desaparecido completamente.

Rapidamente o Coelho notou a presença de Alice, conforme ela começou a
procurar, e chamou em tom irritadiço: ``Ora, Mary Ann, o que você está
fazendo aqui fora? Vá correndo já para casa e traga-me um par de luvas
e um leque! Depressa, agora!''. E Alice ficou tão assustada que saiu
correndo imediatamente na direção que ele apontou, sem tentar explicar o
equívoco que ele havia cometido.

``Ele acha que sou sua empregada'', ela disse consigo mesma enquanto
corria. ``Que surpresa ele terá ao descobrir quem sou eu! Mas é melhor
eu levar para ele o leque e as luvas --- isto é, se eu conseguir
encontrar.'' Ao dizer isso, ela deparou com uma bela casinha, em cuja
porta havia uma placa de latão polido escrito ``\textsc{coelho b}.'' em letras
gravadas. Ela entrou sem bater, com muito medo de encontrar a verdadeira
Mary Ann e de ser expulsa da casa antes de encontrar leque e luvas.

``Que esquisito'', Alice disse consigo mesma, ``cumprir tarefas para um
coelho! Imagino que Dinah também vá querer que eu faça favores para
ela!'' E ela começou a imaginar o tipo de coisa que aconteceria: ``
`Senhorita Alice! Venha já aqui e arrume-se para sair!' `Só um
minutinho, babá, já vou! Mas tenho antes que vigiar esse buraco de
camundongo enquanto a Dinah não volta, para ver se o camundongo não
sai.' Só que eu não acho'', Alice prosseguiu, ``que vão deixar a Dinah
continuar em casa se ela começar a dar ordens às pessoas desse jeito!''

A essa altura, ela já havia encontrado uma salinha arrumada com uma mesa
junto à janela, e em cima da mesa (como ela esperava que fosse) um leque
e dois ou três pares minúsculos de luvinhas brancas: ela pegou o leque e
um par de luvas e estava prestes a sair da salinha, quando seus olhos
depararam com uma garrafinha que estava perto de um espelho. Dessa vez,
não havia rótulo escrito ``\textsc{beba-me}'', mas mesmo assim ela tirou a rolha
e levou o gargalo aos lábios. ``Sei que alguma coisa interessante
certamente vai acontecer'', ela disse consigo mesma, ``sempre que eu
comer ou beber alguma coisa; então vou só esperar para ver o que essa
garrafinha faz. Espero que me faça crescer de novo, pois estou realmente
cansada de ser assim minúscula!''

De fato, fez, e muito antes do que ela esperava: antes de beber metade
da garrafa, sua cabeça já estava encostando no teto, e ela precisou
inclinar a cabeça para a frente para não quebrar o pescoço. Ela depôs a
garrafa depressa, dizendo consigo mesma: ``Já é o bastante\ldots{} espero que
eu pare de crescer agora\ldots{} Desse jeito, não vou conseguir sair pela
porta\ldots{} Quem dera eu não tivesse bebido tanto!''.

Pena! Era tarde demais para desejar isso! Ela continuou crescendo, e
crescendo, e logo precisou se ajoelhar no chão: no minuto seguinte, não
havia espaço nem para se ajoelhar, e ela tentou deitar no chão com um
cotovelo encostado na porta, e o outro braço abraçando a cabeça. Mesmo
assim ela continuou crescendo, e, como um último recurso, ela passou um
braço pela janela, e um pé pela chaminé, e disse consigo mesma: ``Agora
não dá mais, aconteça o que acontecer. O que será de mim?''.

Para sorte de Alice, a garrafinha mágica atingiu o efeito máximo, e ela
não cresceu mais: ainda assim estava desconfortável, e, como não parecia
haver nenhuma possibilidade de sair daquela salinha, não foi estranho
que ela se sentisse infeliz.

``Em casa era muito melhor'', pensou a pobre Alice, ``a gente não ficava
sempre crescendo e diminuindo sem parar, e recebendo ordens de
camundongos e coelhos. Quase desejo não ter entrado naquela toca de
coelho\ldots{} e no entanto\ldots{} e no entanto\ldots{} é bem curioso, sabe como é\ldots{}
esse tipo de vida! Eu me pergunto o que pode ter acontecido comigo!
Quando eu lia contos de fadas, eu achava que esse tipo de coisa não
acontecesse nunca, e agora aqui estou eu no meio de um desses contos!
Alguém devia escrever um livro sobre mim, isso sim! E, quando eu crescer,
vou escrever um\ldots{} bem, agora já estou bem crescidinha'', ela agregou em
tom tristonho; ``mas\ldots{} pelo menos \emph{aqui} não tem mais espaço para
crescer''.

``Mas será'', pensou Alice, ``que eu nunca vou ficar mais velha do que
sou agora? Isso seria um consolo, de certo modo\ldots{} nunca virar uma
velhinha\ldots{} mas\ldots{} ter que fazer lição de casa para sempre! Oh, eu não
ia gostar nada disso!''

``Oh, Alice, sua boba'', ela mesma se respondeu. ``Como você vai fazer
lição aqui? Ora, mal tem espaço para você, que dirá para os livros!''

E assim ela prosseguiu, pensando primeiro por um lado, depois por outro,
e fazendo disso uma verdadeira conversa consigo mesma; mas, depois de
alguns minutos, ela escutou uma voz lá fora e parou para ouvir.

``Mary Ann! Mary Ann!'', dizia a voz. ``Traga minhas luvas já!'' Então
ela ouviu o som de passos na escada. Alice sabia que era o Coelho vindo
procurá-la, e ela estremeceu tanto que fez tremer a casa, esquecendo-se
de que agora era mil vezes maior que o Coelho e não tinha motivos para
ter medo dele.

Então o Coelho chegou perto da porta e tentou abri-la; mas, como a
porta abria para dentro, e o cotovelo de Alice estava apoiado contra a
porta, essa tentativa fracassou. Alice o ouviu dizer de si para si:
``Sendo assim, vou dar a volta e entrar pela janela''.

``Isso você não vai conseguir'', pensou Alice, e, depois de esperar até
ouvir que o Coelho chegara embaixo da janela, ela de repente esticou a
mão e agarrou o ar. Ela não agarrou nada, mas ouviu um gritinho e o som
de algo caindo, e de vidro quebrando, ao que concluiu que era possível
que ele tivesse caído em uma estufa de pepinos ou algo do tipo.

Em seguida, veio uma voz furiosa --- a voz do Coelho --- ``Pat! Pat! Onde
está você?'' E então uma voz que ela nunca tinha ouvido antes: ``Estou
aqui, onde mais? Colhendo maçãs, excelência!''.

``Colhendo maçãs, ora!'', disse o Coelho, irritado. ``Aqui! Venha me
ajudar com isso!'' (Mais sons de vidro quebrado.)

``Agora me diga uma coisa, Pat, o que você está vendo na janela?''

``Sem dúvida, é um braço, excelência.'' (Ele pronunciava ``brrazo''.)

``Um braço, seu tonto! Onde já se viu um braço desse tamanho? Ora, ocupa
a janela inteira!''

``Sem dúvida, ocupa, excelência\ldots{} Mas mesmo assim é um braço.''

``Bem, isso não tem nada que fazer aí, seja como for: tire já isso
daí!''

Houve um longo silêncio depois disso, e Alice só ouviu sussurros aqui e
ali; como: ``Sem dúvida, não estou gostando, excelência, nem um pouco,
nem um pouco!'', ``Faça o que estou dizendo, seu covarde!'', e por fim ela
esticou a mão de novo e tentou novamente agarrar o ar. Dessa vez, houve
dois gritinhos, e mais sons de vidro se quebrando. ``Quantas estufas de
pepinos deve haver aí!'', pensou Alice. ``Imagino o que eles farão agora!
Quanto a me tirar pela janela, quem dera eles conseguissem! Com certeza
\emph{eu} não quero ficar mais aqui!''

Ela esperou algum tempo até não escutar mais nada: por fim, veio um
barulho de rodas de carroça, e o som de muitas vozes falando ao mesmo
tempo: ela conseguiu entender as palavras: ``Cadê a outra escada?''
``Ora, eu não trouxe outra, porque o Bill tinha uma\ldots{} Bill! Traga aqui
essa escada, rapaz!\ldots{} Aqui, ponha nesse canto\ldots{} Não, amarre as duas
primeiro\ldots{} ainda não chega na metade da altura\ldots{} Oh! Bem, é o que
temos; não seja tão implicante\ldots{} Aqui, Bill! Pegue essa corda\ldots{} Será
que o telhado vai suportar? Cuidado com essa telha solta\ldots{} Oh, caiu!
Cuidado com a cabeça aí embaixo!'' (um barulho estrondoso)\ldots{} --- ``Ora,
o que foi isso?'' ``Acho que foi o Bill\ldots{} Quem vai descer pela chaminé?
Não, eu que não vou! Vá você então! O Bill desce\ldots{} Aqui, Bill!, o
patrão mandou você descer pela chaminé!''

``Oh! Quer dizer que o Bill vai descer pela chaminé?'', disse Alice
consigo mesma. ``Ora, pelo visto, eles mandam esse Bill fazer tudo! Eu
não queria estar no lugar desse Bill por nada neste mundo: essa lareira
é estreita, sem dúvida; mas acho que consigo dar um chute!''

Ela esticou o pé até a chaminé, o máximo que conseguiu, e esperou até
ouvir o pequeno animal se aproximar (ela não fazia ideia do tipo de
animal), arrastando-se e esforçando-se para descer a chaminé acima dela:
então, dizendo consigo mesma, ``Deve ser o Bill'', ela deu um chute
firme e esperou para ver o que ia acontecer em seguida.

A primeira coisa que ela ouviu foi um coro geral de ``Lá se vai o
Bill'', então a voz do Coelho apenas: ``Não o deixe cair, você aí
perto da cerca!'', depois o silêncio, e em seguida outra confusão de
vozes --- ``Levante a cabeça dele\ldots{} Conhaque, já! Não deixe engasgar\ldots{}
Como você está, meu velho? O que aconteceu? Queremos saber tudo!''

Por fim, ouviu-se uma vozinha fraca, esganiçada (``Deve ser o Bill'',
pensou Alice), ``Bem, eu não sei exatamente\ldots{} Já chega, obrigado; já
estou melhor\ldots{} mas ainda estou confuso demais para explicar\ldots{} só sei
que alguma coisa me atingiu, como um boneco de mola numa caixa, e, quando
vi, eu estava subindo feito um foguete!''

``Você subiu mesmo, meu velho!'', disseram os outros.

``Vamos incendiar a casa!'', disse a voz do Coelho. E Alice exclamou o
mais alto que podia: ``Se vocês fizerem isso, mando a Dinah atrás de
vocês!''.

Por um momento, fez-se um silêncio mortal e Alice pensou consigo mesma:
``O que será que eles vão fazer agora? Se eles tivessem um pingo de bom
senso, arrancariam esse telhado''. Depois de um minuto ou dois, eles
voltaram a se mexer, e Alice ouviu o Coelho dizer: ``Um carrinho cheio
será o suficiente''.

``Um carrinho cheio do quê?'', pensou Alice. Mas sua dúvida não durou
muito, pois no momento seguinte uma chuva de pedrinhas se chocou contra
a janela, e algumas acertaram seu rosto. ``Vou dar um jeito nisso agora
mesmo'', ela disse consigo mesma, e gritou: ``Melhor vocês não fazerem
mais isso!'', o que produziu outro silêncio de morte.

Alice reparou com certa surpresa que as pedrinhas estavam todas se
transformando em bolinhos conforme caíam no chão, e uma ideia brilhante
lhe ocorreu. ``Se eu comer um desses bolinhos'', ela pensou,
``certamente alguma mudança de tamanho vou sofrer; e imagino que, como é
impossível crescer, vou acabar diminuindo.''

Então ela comeu um bolinho e ficou contente ao ver que começou a
diminuir instantaneamente. Assim que ficou pequena o suficiente para
passar pela porta, ela saiu correndo da casa e encontrou uma verdadeira
multidão de animaizinhos e aves esperando do lado de fora. O pobrezinho
do Lagarto, Bill, estava no meio, apoiado por dois porquinhos-da-índia
que lhe davam uma garrafa para beber. Todos correram na direção
de Alice no momento em que ela apareceu; mas ela correu o máximo que
podia e logo se viu a salvo no meio de uma densa floresta.

``A primeira coisa que devo fazer'', disse Alice para si mesma, enquanto
perambulava pela floresta, ``é voltar ao meu tamanho original; e a
segunda coisa é encontrar o caminho para aquele jardim adorável. Acho
que esse é o melhor plano.''

Parecia um plano excelente, sem dúvida, e planejado de modo muito exato
e simples; a única dificuldade era que ela não sabia por onde
começar; e, enquanto espiava aflita entre as árvores, ouviu um latidinho
agudo acima de sua cabeça que lhe fez olhar para cima rapidamente.

Um cachorrinho enorme olhava para baixo com grandes olhos redondos
voltados para ela e, mansamente esticando uma pata, tentava tocá-la.
``Pobrezinho!'', disse Alice, em tom de adulação, e tentou assobiar para
ele; mas o tempo todo ela estava terrivelmente assustada pensando que
talvez ele estivesse com fome, e nesse caso era muito provável que ele a
devorasse apesar de toda sua adulação.

Mal sabendo o que fazia, ela pegou um pedaço de um graveto, e mostrou
para o filhote; ao que o cachorrinho saltou no ar com as quatro patas,
com um ganido de prazer, e foi correndo atrás do graveto, e fingiu
rosnar para o graveto; então Alice se escondeu atrás de um cardo enorme
para evitar ser pisoteada; e, no momento em que reapareceu do outro
lado, o cachorrinho correu de novo atrás do graveto, tropeçou e deu
uma cambalhota na afobação de buscar o graveto; então Alice, pensando
que era como brincar entre as patas de um cavalo, e esperando a qualquer
momento ser pisoteada, tornou a se esconder atrás do cardo enorme;
então o cachorrinho começou uma série de ataques breves contra o
graveto, correndo um pouquinho para a frente e depois correndo mais para
trás, e latindo roucamente o tempo todo, até que por fim o cachorrinho
se sentou, ofegante, com a língua de fora, e os grandes olhos
entreabertos.

Parecia uma boa oportunidade para Alice conseguir escapar; então ela
logo partiu e correu até se cansar e ficar sem fôlego, e até que os
latidos do cachorrinho soaram fraquinhos à distância.

``E, no entanto, que cachorrinho fofo era esse!'', disse Alice,
apoiando-se a um ranúnculo para descansar, abanando-se com uma das
folhas. ``Eu teria adorado ensinar alguns outros truques para ele, se\ldots{}
se ao menos eu tivesse o tamanho certo para isso! Oh, céus! Quase me
esqueci de que preciso crescer de novo! Vejamos\ldots{} como isso poderia ser
resolvido? Acho que o melhor seria comer ou beber uma coisa ou outra;
mas a grande questão é o quê?''.

A grande questão certamente era: o quê? Alice olhou à sua volta, para as
flores e as folhas de relva, mas não conseguiu encontrar nada que
parecesse a coisa certa para comer ou beber naquelas circunstâncias.
Havia um grande cogumelo perto dela, quase da sua altura; e, depois que
observou por baixo, e pelos dois lados, e por trás do cogumelo,
ocorreu-lhe que podia também dar uma olhada por cima.

Ela se esticou na ponta dos pés e espiou pela borda do cogumelo, e seus
olhos imediatamente deram com os olhos de uma grande lagarta azul
sentada em cima do cogumelo, de braços cruzados, silenciosamente fumando
um narguilé, e sem dar a menor atenção a ela ou a qualquer outra coisa.

%Capítulo V
\quebra\chapter{Conselho de uma Lagarta}
%Paulo: manter os nomes referentes à Lagarta no masculino? No original, está "sir".

\vspace*{-2\baselineskip}

A Lagarta e Alice se entreolharam por algum tempo em silêncio: por fim a
Lagarta tirou o narguilé da boca e se dirigiu a ela, com uma voz
lânguida, sonolenta.

``Quem é você?'', disse a Lagarta.

Esse não foi um começo de conversa muito encorajador. Alice respondeu,
um tanto timidamente: ``Senhor, eu não sei bem, no momento\ldots{} pelo menos
eu sei quem eu era quando acordei hoje cedo, mas acho que devo ter
mudado várias vezes de lá para cá.''

``Como assim?'', disse seriamente a Lagarta. ``Explique-se!''

``Não sei me explicar, infelizmente, senhor'', disse Alice, ``porque não
sou mais eu mesma, sabe\ldots{}''

``Não sei'', disse a Lagarta.

``Infelizmente não consigo explicar com mais clareza'', disse Alice
muito educadamente, ``pois eu mesma não entendo, antes de mais nada, e
depois de ter vários tamanhos diferentes no mesmo dia, é muito
confuso.''

``Não é nada confuso'', disse a Lagarta.

``Bem, talvez o senhor não ache'', disse Alice, ``mas, quando você tiver
que virar crisálida\ldots{} você sabe, algum dia isso vai acontecer\ldots{} e
depois virar borboleta, acho que o senhor também vai se sentir um pouco
esquisito, não é mesmo?''

``Nem um pouco'', disse a Lagarta.

``Bem, talvez os seus sentimentos sejam diferentes dos meus'', disse
Alice; ``só sei que, se fosse comigo, eu acharia muito esquisito.''

``Você!'', disse a Lagarta com desdém. ``Quem é você?''

O que os trouxe de volta, mais uma vez, ao começo da conversa. Alice
sentiu-se um pouco irritada com a Lagarta fazendo comentários
\emph{muito} sucintos e se levantou e disse, muito gravemente: ``Acho
que o senhor deveria me dizer quem é você primeiro''.

``Por quê?'', disse a Lagarta.

Eis outra questão enigmática; e como Alice não conseguiu pensar em
nenhum bom motivo, e como a Lagarta parecia em um estado de espírito
\emph{muito} desagradável, ela se virou para ir embora.

``Volte aqui!'', a Lagarta chamou. ``Tenho algo importante a dizer!''

Isso pareceu promissor, sem dúvida. Alice se virou e voltou mais uma vez.

``Não perca a paciência'', disse a Lagarta.

``Era só isso?'', disse Alice, engolindo sua raiva o melhor que conseguia.

``Não'', disse a Lagarta.

Alice achou por bem esperar, já que não tinha mais nada para fazer, e
talvez afinal a Lagarta dissesse algo que valesse a pena ouvir. Por alguns
minutos, a Lagarta ficou soltando fumaça sem falar, mas por fim
descruzou os braços, tirou o narguilé da boca de novo e disse: ``Quer
dizer que você se transformou?''.

``Infelizmente, sim, senhor'', disse Alice. ``Não consigo me lembrar das
coisas que eu sabia\ldots{} e parece que não consigo manter o mesmo tamanho
por mais de dez minutos!''

``Não se lembra de quais coisas?'', disse a Lagarta.

``Bem, eu tentei recitar \emph{Como a pequena abelha}, mas saiu tudo
diferente!'', Alice respondeu com voz melancólica.

``Recite `Você está velho, pai William'\,'', disse a Lagarta.

Alice juntou as mãos e começou:

\begin{verse}
\footnotesize
``Você está velho, pai William'', disse o moço,\\
``E seu cabelo ficou branco;\\
No entanto, você planta bananeira ---\\
Na sua idade, não é um desplante?''

``No meu tempo'', pai William respondeu ao filho,\\
``Achava que fazia mal aos miolos;\\
Mas hoje, que sei que miolos não tenho,\\
Ora, planto bananeira a todo instante.''

``Você está velho'', disse o moço, ``como eu disse,\\
E ficou gordo demais desse jeito;\\
Ainda assim dá cambalhota para trás ---\\
Ora, qual é o seu intuito?''

``No meu tempo'', disse o sábio, das cacheadas cãs,\\
``Mantive braços e pernas flexíveis\\
Usando essa pomada --- um xelim a lata ---\\
Se quiser comprar, tenho algumas!''

``Você está velho'', disse o moço, ``desdentado,\\
Não morde nada mais duro que banha;\\
Ainda assim devorou o ganso, com ossos e bico ---\\
Ora, como consegue tal proeza?''

``No meu tempo'', disse o pai, ``fui tribuno,\\
Discutia meus casos com a esposa;\\
E a musculatura que isso me deu à mandíbula\\
Durou pelo resto da vida.''

``Você está velho'', disse o moço, ``ninguém diria\\
Que os seus olhos já tiveram boa mira;\\
Mas você equilibra uma enguia no nariz ---\\
Como ficou assim sagaz?''

``Já respondi três perguntas, já basta'',\\
Disse o pai, ``não fique se achando!\\
Você acha que tenho o dia inteiro?\\
Se não parar com isso, te chuto lá para baixo!''
\end{verse}

``Você recitou errado'', disse a Lagarta.

``Não saiu perfeito, infelizmente'', disse Alice, timidamente; ``algumas
palavras saíram alteradas.''

``Está errado do começo ao fim'', disse a Lagarta, decidida, e houve um
silêncio de alguns minutos.

A Lagarta falou primeiro.

``Qual tamanho você gostaria de ter?'', a Lagarta perguntou.

``Oh, eu não sou detalhista para tamanho'', Alice respondeu
apressadamente; ``só não gosto de ficar mudando tanto, sabe\ldots{}''

``Não sei'', disse a Lagarta.

Alice não falou nada: nunca ninguém a contradissera tanto na vida, e ela
sentiu que ia perder a paciência.

``Você está contente agora?'', disse a Lagarta.

``Bem, eu preferiria ser um pouquinho maior, senhor, se o senhor não se
importa'', disse Alice. ``Sete centímetros é uma altura horrível de se
ter.''

``Na verdade, é uma altura muito boa'', disse irritadamente a Lagarta,
erguendo-se, muito empertigado, enquanto falava (ele tinha justamente
sete centímetros de altura).

``Mas é que eu não estou acostumada!'', implorou a pobre Alice em tom
piedoso. E ela pensou consigo mesma: ``Quem dera as criaturas não se
ofendessem tão facilmente assim!''.

``Com o tempo, você vai se acostumar'', disse a Lagarta; e pôs o
narguilé na boca e começou a fumar outra vez.

Dessa vez Alice esperou com paciência até que a Lagarta quisesse falar
de novo. Dali um ou dois minutos, a Lagarta tirou o narguilé da boca e
bocejou uma ou duas vezes, e se sacudiu. Então desceu do cogumelo e
rastejou pela relva, simplesmente comentando no caminho: ``Um lado fará
você ficar mais alta, e o outro lado fará você ficar mais baixa''.

``Um lado do quê? O outro lado do quê?'', pensou Alice consigo mesma.

``Do cogumelo'', disse a Lagarta, como se Alice tivesse falado em voz
alta; e, no momento seguinte, a Lagarta sumiu.

Alice continuou olhando pensativamente para o cogumelo por um minuto,
tentando entender quais eram os dois lados do cogumelo; e, como o
cogumelo era perfeitamente redondo, ela achou essa uma questão muito
difícil. No entanto, por fim, ela esticou os braços, abertos, ao máximo,
e arrancou um pedaço da borda com cada mão.

``E agora: qual será o quê?'', disse consigo mesma, e mordiscou um
pedacinho do da mão direita para ver o efeito: no momento seguinte,
sentiu uma pancada violenta no queixo: seu queixo havia batido em seu
pé!

Ela ficou um bocado apavorada com essa mudança tão súbita, mas percebeu
que não havia tempo a perder, pois estava encolhendo rapidamente; então
resolveu logo comer um pedaço do outro. Seu queixo estava tão
pressionado contra seus pés que quase não tinha espaço para ela abrir a
boca; mas ela abriu enfim, e conseguiu engolir um pedaço do pedaço da
mão esquerda.

``Ora, finalmente minha cabeça está livre!'', disse Alice em tom
satisfeito, mas que se tornou um tom preocupado no momento seguinte,
quando ela descobriu que não conseguia ver os próprios ombros: a única
coisa que ela conseguiu ver, quando olhou para baixo, foi uma imensa
extensão de pescoço, que parecia brotar como um talo em meio a um mar de
folhas verdes, ao longe, muito lá embaixo.

``O que será toda essa coisa verde?'', disse Alice. ``E por onde andarão
meus ombros? E, oh, minhas pobres mãos, por que não consigo vê-las?''
Ela estava movimentando as mãos enquanto falava, mas aparentemente não
havia nenhum efeito, exceto um ligeiro tremor em meio às remotas
folhagens lá embaixo.

Como não parecia haver nenhuma possibilidade de levar as mãos até a
cabeça, ela tentou levar a cabeça até as mãos e ficou contente ao
descobrir que seu pescoço se dobrava facilmente em qualquer direção,
como uma serpente. Ela havia acabado de conseguir dobrá-lo em um
gracioso zigue-zague, e estava mergulhando entre as folhagens, que
descobriu serem simplesmente as copas das árvores sob as quais ela
estivera perambulando, quando um zunido agudo a fez recuar apressada:
uma pomba grande veio até seu rosto e bateu violentamente com as asas.

``Serpente!'', gritou a Pomba.

``Não sou serpente!'', disse Alice, indignada. ``Deixe-me em paz!''

``Serpente, eu repito!'', repetiu a Pomba, mas em tom mais ameno, e
agregou com uma espécie de soluço: ``Já tentei de tudo, nada parece bom
o suficiente para vocês!''.

``Não faço a menor ideia do que você está falando'', disse Alice.

\textls[-10]{``Tentei em raízes de árvores, tentei em ribanceiras de rios, e tentei
em cercas vivas'', a Pomba prosseguiu, sem dar atenção a ela; ``mas
vocês, serpentes! Nada nunca é bom o suficiente para vocês!''}

Alice estava cada vez mais intrigada, mas achou que não adiantaria nada
dizer qualquer coisa enquanto a Pomba não terminasse de falar.

``Como se já não bastasse o trabalho de chocar os ovos'', disse a Pomba;
``eu ainda tenho que me preocupar com vocês, serpentes, noite e dia!
Ora, eu não preguei o olho um minuto essas últimas três semanas!''

``Lamento ter incomodado'', disse Alice, que estava começando a entender
o sentido daquilo.

\textls[-10]{``E justamente quando eu escolho a árvore mais alta da floresta'',
continuou a Pomba, erguendo a voz em um grito, ``e justamente quando eu
achava que estava livre de vocês, serpentes, vocês agora também vêm do
céu se retorcendo atrás da gente! Ugh, Serpente!''}

``Mas eu não sou serpente, estou dizendo'', disse Alice. ``Eu sou\ldots{} Eu
sou\ldots{}''

``Bem! O que você é afinal?'', disse a Pomba. ``Pelo visto você está
tentando inventar uma mentira!''

\textls[-25]{``Eu sou\ldots{} menina'', disse Alice, um tanto desconfiada, pois se
lembrava das transformações que havia sofrido ao longo daquele dia.}\looseness=-1

``Até parece!'', disse a Pomba em tom do mais profundo desdém. ``Eu já
vi muitas meninas ao longo da vida, mas nunca vi uma menina com um
pescoço desse! Não, não! Você é uma serpente, não adianta negar. Daqui a
pouco você vai me dizer que nunca comeu ovo!''

``Já comi ovo, certamente'', disse Alice, que era uma criança muito
sincera; ``mas meninas também comem ovos, tanto quanto serpentes, você
sabe.''

``Eu não acredito'', disse a Pomba; ``mas, se as meninas comem ovos
tanto quanto as serpentes, ora, então elas são uma espécie de serpente,
eu diria.''

Essa foi uma ideia tão nova para Alice que ela ficou calada por um ou
dois minutos, o que deu à Pomba a oportunidade de acrescentar: ``Você
estava procurando ovos, isso eu sei muito bem; e que diferença faz para
mim se você é uma menina ou uma serpente?''.

``Para mim, faz muita diferença'', disse logo Alice; ``mas eu não estou
procurando ovo nenhum, na verdade; e, se eu estivesse, eu não iria comer
os seus. Não gosto de ovo cru.''

``Então vá embora!'', disse a Pomba, em tom contrariado, enquanto se
instalava de volta em seu ninho. Alice se abaixou entre as árvores o
melhor que conseguiu, pois seu pescoço continuava se enroscando nos
galhos, e de quando em quando ela precisava parar para desenroscar.
Algum tempo depois ela se lembrou de que ainda estava com os pedaços de
cogumelo nas mãos, e agiu cautelosamente, mordiscando primeiro um e
depois o outro, e ficando ora maior, ora menor, até conseguir voltar à
sua altura de sempre.

Fazia tanto tempo que ela não ficava do tamanho certo que a princípio
se sentiu muito estranha; mas se acostumou em poucos minutos e começou
a falar sozinha, como sempre. ``Ora, ora, cumpri metade do meu plano!
Como essas transformações são enigmáticas! Nunca sei o que vou ser de um
minuto para o outro! No entanto, voltei para o meu tamanho certo: a
próxima coisa agora é voltar para aquele lindo jardim\ldots{} como fazer
isso? É o que eu me pergunto agora\ldots{}'' Enquanto ela dizia isso, chegou
de repente a um lugar aberto, com uma casinha de menos de um metro e
meio de altura. ``Quem quer que seja o morador dessa casinha'', pensou
Alice, ``não poderei chegar com esse tamanho: ora, o morador
enlouqueceria de susto!'' De modo que ela começou a mordiscar o pedaço
da mão direita, e só ousou se aproximar da casa quando estava com uns
vinte centímetros de altura.

%Capítulo VI
\quebra\chapter{Porco e pimenta}

Por um minuto ou dois, ela ficou olhando para a casa e se perguntando o
que faria em seguida, quando de repente um criado de libré saiu correndo
da floresta --- (ela considerou que fosse um criado porque estava usando
libré: do contrário, a julgar apenas pelo rosto dele, ela o teria
chamado de peixe) --- e bateu com estardalhaço na porta com os nós dos
dedos. A porta foi aberta por outro criado de libré, de rosto redondo e
grandes olhos de rã; e ambos os criados, reparou Alice, usavam perucas
empoadas e cacheadas nas cabeças. Ela ficou muito curiosa para saber o
que significava tudo aquilo e se esgueirou um pouco para fora da mata
para escutar.

O criado Peixe começou tirando de debaixo do braço uma longa carta,
quase tão comprida quanto ele, e entregou essa carta ao outro, dizendo,
em tom solene: ``Para a Duquesa. Um convite da Rainha para jogar
\textit{croquet}''. O criado Rã repetiu, no mesmo tom solene, apenas mudando um
pouco a ordem das palavras: ``Da Rainha. Um convite para a Duquesa para
jogar \textit{croquet}''.

Então ambos fizeram mesuras, e seus cachos empoados se enroscaram.

Alice riu tanto nessa hora que precisou voltar correndo para a floresta
com medo de que a ouvissem; e, quando ela tornou a espiar, o criado
Peixe tinha ido embora, e o outro estava sentado no chão perto da porta,
contemplando o céu estupidamente.

Alice caminhou timidamente até a porta e bateu.

``Não adianta bater'', disse o Criado, ``e isso por dois motivos.
Primeiro, estou do mesmo lado da porta que você; segundo,
estão fazendo tanto barulho lá dentro que ninguém vai conseguir
ouvir.'' E certamente havia um barulho extraordinário lá dentro ---
berros e espirros incessantes, e de quando em quando estrondos
altíssimos, como de um prato ou de um bule estilhaçados.

``Por favor, então'', disse Alice, ``como posso entrar?''

``Talvez fizesse sentido bater'', o Criado prosseguiu sem olhar para
ela, ``se a porta estivesse entre nós. Por exemplo, se você estivesse lá
dentro, você poderia bater, e eu poderia deixar você sair, você
sabe\ldots{}'' Ele estava olhando para o céu todo o tempo em que falava, e
isso Alice achou definitivamente indelicado. ``Mas talvez ele não faça
por mal'', ela disse consigo mesma. ``Os olhos dele ficam muito juntos
no alto da cabeça. Mas mesmo assim ele podia responder. Como posso
entrar?'', ela repetiu em voz alta.

``Vou ficar aqui sentado'', comentou o Criado, ``até amanhã\ldots{}''

Nesse momento, a porta da casa abriu e um prato grande voou para fora,
na direção da cabeça do Criado: o prato apenas resvalou em seu nariz e
se espatifou contra as árvores atrás dele.

``\ldots{} ou talvez até depois de amanhã'', o Criado continuou no mesmo tom,
justamente como se nada tivesse acontecido.

``Como posso entrar?'', perguntou Alice outra vez em voz mais alta.

``Você tem mesmo que entrar?'', disse o Criado. ``Essa é a primeira
questão, você sabe\ldots{}''

Sem dúvida, era mesmo: só que Alice não gostou de ouvir. ``É realmente
infernal'', ela resmungou consigo mesma, ``o modo como todas as
criaturas questionam tudo aqui. É enlouquecedor!''\looseness=-1

O Criado pareceu considerar essa uma boa oportunidade para repetir seu
comentário, com variações. ``Vou ficar aqui sentado'', ele disse, ``sem
me mexer, durante dias e dias.''

``Mas o que eu vou fazer?'', disse Alice.

``O que você quiser'', disse o Criado, e começou a assobiar.

``Oh, não adianta falar com esse aí'', disse Alice desesperadamente. ``É
um perfeito idiota!'' E ela abriu a porta e entrou.

A porta dava diretamente em uma grande cozinha, toda
enfumaçada de ponta a ponta: a Duquesa estava sentada em um banco de
três pernas no meio, com um bebê no colo; a cozinheira estava inclinada
sobre o fogão, mexendo um grande caldeirão que parecia cheio de sopa.

``Sem dúvida tem muita pimenta nessa sopa!'', Alice disse consigo mesma,
da melhor forma que pôde, espirrando.

Sem dúvida, havia muita pimenta no ar. Até a Duquesa espirrava de
quando em quando; e o bebê ora dava espirro, ora dava uivos,
incessantemente. As únicas criaturas na cozinha que não estavam
espirrando sem parar eram a cozinheira e um gato grande, sentado perto
do fogo, sorrindo de orelha a orelha.

``Por favor, você poderia me dizer'', falou Alice um tanto timidamente,
pois ela não tinha certeza se era educado falar primeiro, ``por que o
seu gato sorri assim?''

``É um gato de Cheshire'', disse a Duquesa, ``é por isso. Porco!''

Ela disse a última palavra com violência tão súbita que Alice teve um
sobressalto; mas ela viu no momento seguinte que fora dirigida ao bebê,
e não a ela, de modo que ela tomou coragem e tornou a falar:

``Eu não sabia que os gatos de Cheshire sempre sorriam assim; na
verdade, eu não sabia que gato ria.''

``Todos os gatos sabem rir'', disse a Duquesa; ``e a maioria ri.''

``Eu não sabia disso, nunca vi um gato rindo'', Alice disse muito
educadamente, sentindo-se muito aliviada por conversar com alguém.

``Você não sabe muita coisa'', disse a Duquesa; ``e isso é um fato.''

Alice não gostou do tom desse comentário e achou que seria bom mudar o
assunto da conversa. Enquanto tentava pensar em algum, o cozinheiro
tirou o caldeirão de sopa do fogo e logo começou a atirar tudo o que
tinha à mão em cima da Duquesa e do bebê --- os atiçadores foram
primeiro; em seguida veio uma chuva de caçarolas, bandejas e pratos. A
Duquesa nem se abalou, nem mesmo quando foi atingida; e o bebê já estava
se esgoelando tanto que era praticamente impossível saber se o haviam
acertado ou não.

``Oh, por favor, cuidado!'', exclamou Alice, pulando e se abaixando em
uma agonia de terror. ``Oh, lá se vai o narizinho lindo dele''; pois uma
caçarola absurdamente grande passou raspando e quase arrancou o
narizinho do bebê.

``Se as pessoas cuidassem da própria vida'', a Duquesa disse com um
rosnado rouco, ``o mundo giraria muito mais depressa.''

``Mas isso não seria uma vantagem'', disse Alice, que ficou muito
contente pela oportunidade de exibir um pouco seus conhecimentos.
``Imagine o efeito disso sobre o dia e a noite! Você sabe, a Terra leva
vinte e quatro horas para dar uma volta em seu eixo\ldots{}''

``Por falar em efeito'', disse a Duquesa, ``corte a cabeça dela.''

Alice olhou aflita para a cozinheira para ver se ela obedeceria; mas a
cozinheira estava ocupada mexendo a sopa e não parecia estar escutando,
de modo que ela arriscou prosseguir: ``Vinte e quatro horas, eu acho, ou
será que são doze? Eu\ldots{}''

``Oh, não me aborreça'', disse a Duquesa; ``Nunca fui boa com números!''
E com isso ela voltou a embalar o filho, entoando uma espécie de canção de
ninar, e sacudindo violentamente o bebê ao final de cada verso:

\begin{quote}
Fale brava com seu filhinho,\\
Bata quando ele espirrar;\\
Ele só faz isso para me irritar,\\
Porque sabe que me irrita.\\
\medskip
\noindent\textsc{coro}\\
(Ao qual se juntam a cozinheira e o bebê):\\
\textit{Uou! uou! uou!}
\end{quote}

Enquanto cantava a segunda estrofe da canção, a Duquesa ficou
sacudindo o bebê tão violentamente, e o pobrezinho se esgoelou tanto,
que Alice mal conseguiu entender a letra:

\begin{quote}
Falo brava com meu filho,\\
Bato quando ele espirra;\\
Pois ele sabe se aproveitar\\
Da pimenta para espirrar!\\
\medskip
\noindent\textsc{coro}\\
\textit{Uou! uou! uou!}
\end{quote}

``Tome! Embale um pouco se quiser!'', a Duquesa disse a Alice, atirando
o bebê ao falar. ``Preciso ir me arrumar para jogar \textit{croquet} com a
Rainha'', e ela saiu correndo. A cozinheira atirou uma frigideira na Duquesa,
mas não a acertou.

Alice segurou o bebê com alguma dificuldade, pois era uma criaturazinha
de formato esquisito e esticava braços e pernas em todas as direções,
``igualzinho a uma estrela-do-mar'', pensou Alice. A pobre criaturazinha
estava espirrando como um motor a vapor quando ela o pegou no colo, e
ficou se contorcendo e se esticando sem parar, tanto que, por um minuto
ou dois, ela fez tudo o que pôde para tentar segurá-lo no colo.

Assim que ela entendeu a melhor forma de embalar o bebê (que era dar um
nó nele e depois segurar firme a orelha direita e o pé esquerdo, para
impedir que ele se desamarrasse), ela o levou para o ar livre. ``Se eu
não levar esse bebê comigo'', pensou Alice, ``sem dúvida ele vai acabar
morrendo daqui um ou dois dias: não seria um assassinato deixar o bebê
aqui?'' Ela disse essas últimas palavras em voz alta, e a criaturazinha
grunhiu em resposta (ele havia parado de espirrar a essa altura). ``Não
grunha'', disse Alice; ``não é um modo apropriado de se expressar.''

O bebê grunhiu de novo, e Alice olhou aflita para ver o que havia de
errado com ele. Não havia nenhuma dúvida de que o bebê tinha um nariz
bastante arrebitado, muito mais parecido com um focinho do que com um
nariz de verdade; além disso, seus olhos estavam ficando extremamente
pequenos para um bebê: de modo geral, Alice não gostou nada da aparência
daquilo. ``Talvez ele estivesse só soluçando'', ela pensou, e olhou
outra vez para os olhos dele para ver se havia lágrimas.

Não, não havia nenhuma lágrima. ``Se você se transformar em um porco,
meu querido'', disse Alice, seriamente, ``não vou mais querer ficar com
você. Pense bem!'' A pobre criatura soluçou de novo (ou grunhiu, era
impossível dizer exatamente), e ficaram em silêncio por algum tempo.

Alice estava começando a pensar, ``Ora, o que eu vou fazer com essa
criatura quando voltar para casa?'', quando ele grunhiu outra vez, tão
violentamente, que ela olhou preocupada para seu rosto. Dessa vez, não
havia mais dúvida: ele era nada mais, nada menos que um porco, e ela
achou muito absurdo continuar levando-o no colo.

Então ela depôs a criaturazinha no chão e se sentiu bastante aliviada
ao vê-lo trotar sossegadamente para dentro da floresta. ``Se tivesse
crescido'', ela disse consigo mesma, ``teria sido uma criança
horrivelmente feia: mas acho que vai acabar sendo um porco até bonito.''
E ela começou a pensar em outras crianças que conhecia, que dariam belos
porcos, e estava dizendo a si mesma, ``se a gente soubesse o jeito certo
de transformar\ldots{}'', quando teve um breve sobressalto ao ver o Gato de
Cheshire sentado em um galho de uma árvore alguns metros adiante.

O Gato simplesmente sorriu ao ver Alice. Ele parecia bondoso, ela
pensou: embora tivesse garras muito compridas e muitos dentes, de modo
que ela sentiu que ele devia ser tratado com respeito.

``Gatinho de Cheshire'', ela começou, um tanto timidamente, pois não
sabia se ele iria gostar do apelido: no entanto, ele apenas sorriu um
pouco mais. ``Ora, até agora, ele parece satisfeito'', pensou Alice, e
prosseguiu: ``Por favor, você poderia me dizer que caminho devo
seguir?''.

``Isso depende muito de saber aonde você quer chegar'', disse o Gato.

``Não faz muita diferença para mim\ldots{}'', disse Alice.

``Então não importa qual caminho seguir'', disse o Gato.

``\ldots{} desde que eu chegue em algum lugar'', Alice explicou.

``Oh, sem dúvida você chegará'', disse o Gato, ``basta andar bastante.''

Alice sentiu que isso era inegável, de modo que tentou outra pergunta.
``Que tipo de gente vive por aqui?''

``Naquela direção'', o Gato disse, acenando com a pata direita, ``mora
um Chapeleiro; e naquela direção'', apontando com a outra pata, ``mora
uma Lebre de Março. Qualquer um que você visite: todos os dois são
loucos.''

``Mas eu não quero visitar gente louca'', Alice comentou.

``Oh, isso você não pode evitar'', disse o Gato. ``Aqui somos todos
loucos. Eu sou louco. Você é louca.''

``Como você sabe que eu sou louca?'', disse Alice.

``Deve ser'', disse o Gato, ``ou não teria vindo aqui.''

Alice não achava que isso provava nada; mas prosseguiu. ``E como você
sabe que você é louco?''

``Antes de mais nada'', disse o Gato, ``cachorros não são loucos. Você
concorda?''

``Acho que sim'', disse Alice.

``Pois bem'', o Gato prosseguiu, ``você sabe, o cachorro rosna quando
está bravo e balança o rabo quando está contente. Ora, eu rosno quando
estou contente e balanço o rabo quando estou bravo. Portanto eu sou
louco.''

``Eu chamo de ronronar, não de rosnar'', disse Alice.

``Chame como quiser'', disse o Gato. ``Você vai jogar \textit{croquet} com a
Rainha hoje?''

``Eu adoraria'', disse Alice, ``mas ainda não fui convidada.''

``Você me verá por lá'', disse o Gato e sumiu.

Alice não ficou muito surpresa com isso, de tão acostumada que já estava
com coisas esquisitas acontecendo. Enquanto olhava para o lugar onde ele
estivera, subitamente ele apareceu de novo.

``Por falar nisso, o que aconteceu com o bebê?'', disse o Gato. ``Ia me
esquecendo de perguntar.''

``Ele virou um porco'', Alice disse sossegadamente, como se fosse algo
muito natural.

``Como eu pensei'', disse o Gato, e tornou a desaparecer.

Alice aguardou um pouquinho, com certa esperança de vê-lo novamente, mas
ele não apareceu, e, após um ou dois minutos, ela seguiu caminhando na
direção em que, segundo lhe dissera o Gato, vivia a Lebre de Março. ``Já
conheci outros chapeleiros'', ela disse consigo mesma; ``a Lebre de
Março será muito mais interessante, e talvez, como estamos em maio, a
lebre não esteja tão louca\ldots{} pelo menos não tanto quanto em março.'' Ao
dizer isso, ela olhou para cima, e lá estava o Gato outra vez, sentado
no galho de uma árvore.

``Você disse porco ou pouco?'', disse o Gato.

``Eu disse porco'', respondeu Alice, ``e eu gostaria que você não
ficasse aparecendo e desaparecendo de repente: você deixa a gente um
pouco tonta.''

``Está bem'', disse o Gato, e dessa vez foi sumindo bem devagar,
começando pela ponta do rabo, e terminando no sorriso, que permaneceu
por algum tempo depois que todo o resto havia desaparecido.

``Ora! Já vi muito gato sem sorriso'', pensou Alice; ``mas um sorriso
sem gato! É a coisa mais interessante que eu já vi em toda a minha
vida.''

Ela não chegou a caminhar muito quando avistou a casa da Lebre de Março:
ela achou que devia ser a casa certa, porque as chaminés eram em forma
de orelhas e o telhado era peludo. Era uma casa tão grande que ela não
quis se aproximar sem antes mordiscar mais um pouco do cogumelo da mão
esquerda e crescer até ficar com uns sessenta centímetros. Mesmo assim,
ela caminhou até a casa timidamente, dizendo consigo: ``E se a lebre
estiver muito louca?! Estou quase arrependida de não ter ido ao
Chapeleiro!''.

%Capítulo VII
\quebra\chapter{Chá de loucos}

Havia uma mesa posta embaixo de uma árvore na frente da casa, e a Lebre
de Março e o Chapeleiro estavam tomando chá em volta: um Arganaz estava
sentado entre eles, dormindo pesado, e os outros dois o usavam como
almofada para os cotovelos e conversavam por cima de sua cabeça. ``Isso
deve ser muito desconfortável para o Arganaz'', pensou Alice; ``mas,
como ele está dormindo, pelo visto, não se importa.''

A mesa era grande, mas os três estavam apinhados em um canto. ``Não tem
mais lugar! Não tem mais lugar!'', ele gritaram quando viram Alice
chegar.

``Estou vendo vários lugares!'', disse Alice, indignada, sentando-se
em uma grande poltrona em uma ponta da mesa.

``Beba vinho'', disse a Lebre de Março em tom encorajador.

Alice olhou para a mesa, mas só havia chá. ``Não estou vendo nenhum
vinho'', ela comentou.

``Não tem mesmo'', disse a Lebre de Março.

``Então não é muito educado oferecer'', disse Alice, irritada.

``Não é muito educado da sua parte sentar sem ser convidada'', disse a
Lebre de Março.

``Eu não sabia que a mesa era sua'', disse Alice. ``A mesa está posta
para muito mais do que três.''

``Você está precisando cortar esse cabelo'', disse o Chapeleiro. Ele
estivera observando Alice havia algum tempo com grande curiosidade, e
foi a primeira coisa que ele disse.

``Você devia aprender a não fazer comentários pessoais'', Alice disse
com certa severidade; ``é muito rude.''

O Chapeleiro arregalou bem os olhos ao ouvir isso, mas a única coisa que
ele disse foi: ``Qual é a semelhança entre o corvo e a escrivaninha?''.

``Ora, agora vamos nos divertir um pouco!'', pensou Alice. ``Que bom
que começaram a brincar de enigmas. --- Acho que posso adivinhar'', ela
acrescentou em voz alta.

``Você está dizendo que acha que pode descobrir a resposta?'', disse a
Lebre de Março.

``Exatamente'', disse Alice.

``Pois então diga'', a Lebre de Março prosseguiu.

``Quero dizer'', Alice respondeu rispidamente; ``pelo menos\ldots{} pelo
menos é o que eu acho\ldots{} o que dá na mesma, você sabe\ldots{}''

``Não dá nada na mesma!'', disse o Chapeleiro. ``Ora, você acha que é a
mesma coisa dizer que `eu vejo o que eu como' e `eu como o que eu
vejo'?''

``Ou ainda dizer'', agregou a Lebre de Março, ``que `eu gosto do que eu
tenho' é o mesmo que `eu tenho o que eu gosto'?''

``Daqui a pouco você vai dizer'', agregou o Arganaz, que parecia falar
enquanto dormia, ``que `eu respiro quando durmo' é a mesma coisa que `eu
durmo quando respiro'!''

``Só se der na mesma para você'', disse o Chapeleiro, e aqui a conversa
acabou, e o grupo se sentou em silêncio por um minuto, enquanto Alice
recapitulava tudo o que conseguia se lembrar sobre corvos e
escrivaninhas, o que não era muito.

O Chapeleiro foi o primeiro a romper o silêncio. ``Em que dia do mês
estamos?'', ele disse, virando-se para Alice. Ele havia tirado o relógio
do bolso e estava olhando contrariado para o relógio, sacudindo-o de
quando em quando e aproximando-o do ouvido.

Alice pensou um pouco e então disse: ``Dia quatro''.

``Dois dias atrasado!'', suspirou o Chapeleiro. ``Eu falei que manteiga
na engrenagem não ia resolver!'', ele acrescentou, olhando irritado para
a Lebre de Março.

``Era uma manteiga da \emph{melhor} qualidade'', a Lebre de Março
mansamente respondeu.

``Sim, mas devem ter caído também algumas migalhas na engrenagem'',
resmungou o Chapeleiro. ``Você não devia ter usado a faca de pão.''

A Lebre de Março tirou o relógio do bolso e olhou tristonho para os
ponteiros: depois mergulhou o relógio em sua xícara de chá e tornou a
olhar para o relógio: mas não conseguiu pensar em nada melhor para dizer
do que seu primeiro comentário. ``Era uma manteiga da \emph{melhor}
qualidade, você sabe\ldots{}''

Alice ficara observando por cima do ombro dele com certa curiosidade.
``Que relógio engraçado!'', ela comentou. ``Dá o dia do mês, mas não diz
que horas são!''

``Por que diria a hora?'', resmungou o Chapeleiro. ``O seu relógio diz
em que ano estamos?''

``Claro que não'', Alice respondeu muito prontamente. ``Mas isso é
porque continua sendo o mesmo ano por muito tempo.''

``Pois bem, o meu relógio diz'', disse o Chapeleiro.

Alice ficou terrivelmente intrigada. O comentário do Chapeleiro parecia
não fazer sentido, e no entanto ele falava sem dúvida a mesma língua.
``Não sei se entendi'', ela disse, o mais educadamente que conseguiu.

``O Arganaz dormiu de novo'', disse o Chapeleiro, e serviu chá quente em
seu nariz.

O Arganaz sacudiu a cabeça impaciente e disse, sem abrir os olhos:
``Claro, claro, era jsutamente o que eu ia dizer''.

``Você já adivinhou o enigma?'', o Chapeleiro disse, virando-se de novo
para Alice.

``Não, eu desisto'', Alice respondeu. ``Qual é a resposta?''

``Não faço a menor ideia'', disse o Chapeleiro.

``Nem eu'', disse a Lebre de Março.

Alice soltou um suspiro de desgosto. ``O tempo é uma coisa preciosa'',
ela disse, ``para desperdiçar com enigmas sem resposta.''

``Se você conhecesse o Tempo como eu conheço'', disse o Chapeleiro,
``você não diria que é uma coisa. Ele é uma pessoa.''

``Não entendi o que você disse'', disse Alice.

``Claro que não entendeu!'', o Chapeleiro disse, jogando a cabeça para
trás com desdém. ``Eu diria que você nunca conversou com o Tempo!''

``Talvez não'', Alice respondeu com cautela. ``Mas eu aprendi na aula de
música a marcar o tempo, batendo assim\ldots{}''

``Ah! Então está explicado'', disse o Chapeleiro. ``Ele não gosta que
batam. Agora, se você se dá bem com ele, ele faz praticamente tudo o que
você quiser com o relógio. Por exemplo, suponhamos que agora fossem nove
da manhã, hora de começar a estudar: basta você cochichar para o Tempo,
e a hora passa em uma piscar de olhos! Quando der uma e meia, pula para
a hora do jantar!''

(``Quem dera já fosse'', Lebre de Março sussurrou para si mesmo.)

``Sem dúvida, isso seria ótimo'', disse Alice, pensativamente: ``mas\ldots{}
eu ainda não estaria com fome, você sabe\ldots{}''

``A princípio, talvez não'', disse o Chapeleiro. ``Mas você poderia
fazer com que continuasse sendo uma e meia pelo tempo que quisesse.''

``É assim que você faz?'', Alice perguntou.

O Chapeleiro balançou a cabeça pesarosamente. ``Eu não!'', ele
respondeu. ``Nós discutimos isso março passado --- pouco antes de esse aí
enlouquecer, você sabe\ldots{}'' (apontando com a colher de chá para a Lebre
de Março), ``era o grande concerto oferecido pela Rainha de Copas, e eu
teria que cantar

\begin{quote}
Pisca, pisca, morceguinho!\\
Me pergunto onde é seu ninho!
\end{quote}

``Talvez você conheça essa canção.''

``Já ouvi algo parecido'', disse Alice.

``A canção continua, você sabe\ldots{}'', o Chapeleiro continuou, ``mais ou
menos assim\ldots{}''

\begin{quote}
Lá no céu sempre a voar,\\
Como bandeja de chá.\\
Pisca, pisca\ldots{}
\end{quote}

Aqui o Arganaz se sacudiu e começou a cantar enquanto dormia: ``Pisca,
pisca, pisca, pisca\ldots{}'', e continuou nisso por tanto tempo que
precisaram beliscá-lo para que ele parasse.

``Bem, eu mal havia terminado a primeira estrofe'', disse o Chapeleiro,
``quando a Rainha deu um pulo e berrou: `Ele está matando o Tempo!
Cortem-lhe a cabeça!'\,''.

``Mas que selvageria!'', exclamou Alice.

``E desde então'', o Chapeleiro prosseguiu em tom de lamento, ``o Tempo
não faz nada que eu peço! Agora são sempre seis horas.''

Uma ideia brilhante ocorreu a Alice. ``É por isso que há tantos
utensílios do chá da tarde aqui?'', ela perguntou.

``Sim, é por isso'', disse o Chapeleiro com um suspiro. ``É sempre hora
do chá, e não temos tempo nem de lavar a louça.''

``Quer dizer que vocês só ficam o tempo todo mudando de lugar de lá para
cá?'', perguntou Alice.

``Exatamente'', disse o Chapeleiro. ``Enquanto as coisas vão sendo
consumidas.''

``Mas o que acontece quando vocês voltam ao princípio de novo?'', Alice
arriscou perguntar.

``E se nós mudarmos de assunto'', a Lebre de Março interrompeu,
bocejando. ``Estou ficando cansado disso. Voto para que a mocinha nos
conte uma história.''

``Infelizmente não sei nenhuma'', disse Alice, um tanto preocupada com a
proposta.

``Então o Arganaz contará!'', ambos gritaram. ``Acorde, Arganaz!'' E
eles beliscaram o Arganaz pelos dois lados ao mesmo tempo.

O Arganaz abriu lentamente os olhos. ``Eu não estava dormindo'', ele
disse com uma voz rouca, fraca. ``Escutei cada palavra do que vocês
disseram.''

``Conte-nos uma história!'', disse a Lebre de Março.

``Sim, por favor, conte!'', implorou Alice.

``E depressa'', acrescentou o Chapeleiro, ``para não pegar no sono outra
vez antes de terminar.''

``Era uma vez três irmãzinhas'', o Arganaz começou a contar muito
depressa; ``que se chamavam Elsie, Lacie e Tillie; e viviam no fundo de
um poço\ldots{}''

``Viviam do quê?'', disse Alice, sempre muito interessada em questões de
comida e bebida.

``Elas comiam melaço'', disse o Arganaz, depois de pensar um ou dois
minutos.

``Isso seria impossível, você sabe\ldots{}'', Alice gentilmente observou;
``elas ficariam doentes.''

``E elas ficaram'', disse o Arganaz; ``\emph{muito} doentes.''

Alice tentou por um momento imaginar que modo de vida extraordinário
seria só comer melaço, mas ficou intrigada demais, de modo que
prosseguiu: ``Mas por que elas moravam no fundo de um poço?''.

``Beba mais chá'', a Lebre de Março disse a Alice, muito seriamente.

``Eu ainda não bebi \emph{nada}'', Alice respondeu em tom ofendido, ``de
modo que seria impossível beber \emph{mais}.''

``Você deve estar querendo dizer que é impossível beber \emph{menos}'',
disse o Chapeleiro; ``pois é muito fácil beber \emph{mais} do que
nada.''

``Ninguém perguntou a sua opinião'', disse Alice.

``Quem está fazendo comentários pessoais agora?'', o Chapeleiro
perguntou, triunfante.

Alice ficou sem saber o que dizer diante disso, então se serviu de
chá e pão com manteiga, e então se virou para o Arganaz e repetiu a
pergunta: ``Por que elas viviam no fundo de um poço?''.

O Arganaz novamente levou um ou dois minutos para pensar e então disse:
``Era um poço de melaço''.

``Isso não existe!'', Alice estava começando a dizer, muito irritada,
mas o Chapeleiro e a Lebre de Março disseram ``Psiu!'', e o Arganaz,
consternado, comentou: ``Se você não sabe se comportar, termine a
história sozinha''.

``Não, por favor, continue!'', Alice disse muito humildemente. ``Não vou
mais interromper. Vamos dizer que existiu um poço de melaço.''

``Sim, um único, esse'', disse o Arganaz, indignado. Enfim, ele consentiu
em prosseguir. ``E então essas três irmãzinhas\ldots{} elas estavam
aprendendo a fazer\ldots{}''

``Fazer o quê?'', disse Alice, esquecendo a promessa.

``Melaço'', disse o Arganaz, sem cerimônia dessa vez.

``Preciso de uma xícara limpa'', interrompeu o Chapeleiro. ``Vamos todos
passar para o lugar ao lado, nesse sentido.''

Ele mudou de lugar enquanto falava, e o Arganaz o acompanhou: a Lebre de
Março foi para o lugar do Arganaz, e Alice, um tanto a contragosto, foi
para o lugar da Lebre de Março. O Chapeleiro foi o único a se beneficiar
com a mudança: e Alice ficou bem pior do que antes, pois a Lebre de
Março havia derrubado a leiteira no prato.

Alice não queria mais ofender o Arganaz, então começou muito
cautelosamente: ``Mas eu não entendo. De onde vinha o melaço?''.

``É possível tirar água de um poço de água'', disse o Chapeleiro; ``de
modo que eu acho que seria possível obter melaço em um poço de melaço\ldots{}
bem\ldots{} é muita estupidez!''

``Mas elas estavam \emph{dentro} do poço'', Alice disse ao Arganaz,
preferindo não dar atenção a esse último comentário.

``Claro que estavam'', disse o Arganaz. ``\ldots{} Bem no fundo do poço.''

Essa resposta confundiu a pobre Alice, tanto que ela deixou o Arganaz
prosseguir por algum tempo sem interrompê-lo.

``As três irmãzinhas estavam um dia fazendo desenhos'', o Arganaz
prosseguiu, bocejando e esfregando os olhinhos, pois estava ficando com
muito sono; ``e elas desenharam todo tipo de coisas\ldots{} e todas começavam
com \textsc{m}\ldots{}''

``Por que com \textsc{m}?'', disse Alice.

``Por que não?'', disse a Lebre de Março.

Alice ficou quieta.

A essa altura, o Arganaz havia fechado os olhinhos e mergulhava em uma
soneca; mas, sendo beliscado pelo Chapeleiro, acordou de novo com um
gritinho e prosseguiu: ``\ldots{} Começavam com \textsc{m}, como malandros, meninas,
memórias e muitíssimos múltiplos mais --- ou, como se diz, `mesmíssimos'
--- vocês já viram `muitos mesmos' desenhados?''

``Olha, agora você me pegou'', disse Alice, muito confusa, ``mas acho
que não\ldots{}''

``Então você não deveria falar'', disse o Chapeleiro.

Essa grosseria foi maior do que Alice poderia suportar: ela se levantou
com grande desprezo e foi embora; o Arganaz adormeceu instantaneamente,
e nenhum dos outros dois sequer reparou que ela saíra, embora ela tivesse
olhado para trás uma ou duas vezes, com certa esperança de que eles
fossem chamá-la: na última vez que ela olhou, eles estavam tentando pôr
o Arganaz no bule.

``Seja como for, nunca mais volto para lá!'', disse Alice, seguindo um
caminho através da floresta. ``Nunca fui a um chá da tarde com gente tão
estúpida na minha vida!''

Assim que ela disse isso, reparou que uma das árvores tinha um porta bem
no meio. ``Que coisa curiosa!'', ela pensou. ``Mas hoje tudo é curioso.
Acho melhor eu entrar de uma vez.'' E ela entrou.

Mais uma vez ela se viu no grande salão, perto da mesinha de vidro.
``Agora vou me sair melhor'', ela disse a si mesma, e começou pegando a
chavinha dourada e abrindo a porta que dava para o jardim. Depois, ela
mordiscou o cogumelo (ela guardara um pedaço no bolso) até atingir
trinta centímetros de altura: então, ela foi andando pelo corredorzinho:
e \emph{aí}\ldots{} ela se viu, enfim, no lindo jardim, entre canteiros de
flores coloridas e fontes de água fresca.

%Capítulo VIII
\quebra\chapter[O campo de «croquet» da Rainha]{O campo de «croquet»\break da Rainha}

Havia uma grande roseira na entrada do jardim: as rosas que ali cresciam
eram brancas, mas havia três jardineiros pintando-as de vermelho. Alice
achou isso muito curioso e se aproximou para observá-los, e, assim que
chegou perto, escutou um deles dizer: ``Cuidado agora, Cinco! Não me suje
de tinta desse jeito!''.

``Não foi por querer'', disse o Cinco, em tom rabugento. ``O Sete
cutucou meu cotovelo.''

Ao que o Sete olhou para cima e falou: ``Isso, Cinco! Sempre pondo a
culpa nos outros!''.

``É bom você não falar nada!'', disse o Cinco. ``Ouvi dizer que a Rainha
falou ontem que você merecia ser decapitado!''

``Por quê?'', disse aquele que falara primeiro.

``Não interessa, Dois'', disse o Sete.

``Sim, interessa a ele, sim!'', disse o Cinco. ``E eu vou contar para
ele\ldots{} foi por ter trazido para a cozinheira bulbos de tulipas em vez de
cebolas!''

O Sete soltara seu pincel e havia começado: ``Bem, de todas as
injustiças\ldots{}'', quando seus olhos depararam com Alice, ali parada
olhando para eles, e ele parou de falar subitamente. Os outros também
olharam para os lados, e todos fizeram uma mesura reverente.

``Vocês poderiam me dizer'', disse Alice, um tanto timidamente, ``por
que estão pintando essas rosas?''

Cinco e Sete não disseram nada, mas olharam para o Dois. Dois começou em
voz baixa: ``Ora, o fato, sabe, senhorita, é que aqui deveria haver uma
roseira de rosas vermelhas, e nós plantamos uma branca por engano; e, se
a Rainha descobrir, cortarão nossas cabeças, sabe como é\ldots{} Então,
senhorita, sabe como é, estamos fazendo o nosso melhor, antes que ela
chegue, e\ldots{}'' Nesse momento, o Cinco, que ficara olhando aflito para o
outro lado do jardim, avisou ``A Rainha! A Rainha!'', e os três
jardineiros instantaneamente se curvaram com os rostos encostando no
chão. Houve um rumor de muitos passos, e Alice olhou de relance, ansiosa
para ver a Rainha.

Primeiro vieram dez soldados carregado paus; eram todos do mesmo formato
dos três jardineiros, finos e planos, com as mãos e os pés nos quatro
cantos do corpo: em seguida, dez cortesãos; esses eram todos
ornamentados com ouros e caminhavam de dois em dois, como os soldados.
Depois vieram as crianças da realeza; eram dez, e esses pequeninos
chegaram saltitando alegremente de mãos dadas, em pares; estavam
ornamentados com copas. A seguir, os convidados, a maioria Reis e
Rainhas, e entre eles Alice reconheceu o Coelho Branco: ele falava
depressa, nervoso, sorrindo para tudo que diziam, e passou sem olhar
para ela. Então o Valete de Copas, trazendo a coroa do Rei em uma
almofada de veludo carmesim; e, ao final de todo esse grandioso cortejo,
vieram o rei e a rainha de copas.

Alice ficou na dúvida se devia deitar com o rosto no chão como os três
jardineiros, mas não se lembrava de ter ouvido falar dessa regra em
cortejos; ``além do mais, para que serve um cortejo'', ela pensou, ``se
as pessoas deitam viradas para o chão e não conseguem ver nada?'' De
modo que ela ficou parada onde estava e esperou.

Quando o cortejo chegou bem na frente de Alice, todos pararam e olharam
para ela, e a Rainha disse severamente: ``Quem é essa?''. Ela disse ao
Valete de Copas, que apenas fez uma mesura e sorriu em resposta.

``Idiota!'', disse a Rainha, jogando a cabeça para trás com impaciência;
e, virando-se para Alice, prosseguiu: ``Qual é o seu nome, menina?''.

``Meu nome é Alice, à disposição de sua Majestade'', disse Alice muito
educadamente; mas ela acrescentou, para si mesma: ``Ora, são apenas
cartas de baralho, afinal. Não preciso ter medo deles!''.

``E quem são esses aí?'', disse a Rainha, apontando para os três
jardineiros que estavam curvados em volta da roseira; pois, você sabe,
como estavam com os rostos virados para o chão, e o padrão em suas
costas era o mesmo do resto do baralho, ela não sabia se eram
jardineiros, soldados, cortesãos ou três de seus próprios filhos.

``Como eu poderia saber?'', disse Alice, surpresa com a própria coragem.
``Não tenho nada com isso.''

A Rainha ficou vermelha de fúria e, depois de olhar fixamente para ela
por um momento como um animal selvagem, berrou: ``Cortem-lhe a cabeça!
Cortem\ldots{}''.

``Que absurdo!'', disse Alice, em uma voz muito alta e decidida, e a Rainha
ficou calada.

O Rei pôs a mão no braço da esposa e timidamente disse: ``Reconsidere,
querida: ela é apenas uma menina!''.

A Rainha irritada deu as costas ao marido e disse ao Valete:
``Vire-os!''.

O Valete virou os três, com muito cuidado, com a ponta do pé.

``Levantem-se!'', disse a Rainha, com uma voz estridente e volumosa, e
os três jardineiros prontamente se puseram de pé e começaram a fazer
mesuras para o Rei, a Rainha e as crianças da realeza, e para
todos os demais.

``Parem já com isso!'', berrou a Rainha. ``Vocês me deixam tonta.'' E
então, voltando-se para a roseira, ela prosseguiu: ``Ora, o que vocês
estavam fazendo aqui?''.

``Para agradar sua Majestade'', disse o Dois, em tom muito humilde,
apoiando-se em um joelho ao falar, ``estávamos tentando\ldots{}''

``Eu estou vendo!'', disse a Rainha, que nesse ínterim ficara examinando
as rosas. ``Cortem suas cabeças!'', e o cortejo prosseguiu, três
soldados ficando para trás para executar os pobres jardineiros, que
correram para Alice pedindo proteção.

``Vocês não serão decapitados!'', disse Alice, e escondeu-os em um
grande vaso de flor que havia ali perto. Os três soldados perambularam
por um ou dois minutos, procurando os jardineiros, e depois voltaram
marchando sossegadamente atrás dos outros.

``E as cabeças deles?'', berrou a Rainha.

``As cabeças sumiram, conforme o desejo de sua Majestade!'', os soldados
berraram em resposta.

``Muito bem!'', berrou a Rainha. ``Você joga \textit{croquet}?''

Os soldados ficaram calados e olharam para Alice, pois a pergunta
evidentemente se dirigia a ela.

``Sim!'', berrou Alice.

``Então vamos!'', rugiu a Rainha, e Alice se juntou ao cortejo,
imaginando intensamente o que aconteceria em seguida.

``Está\ldots{} um dia muito bonito!'', disse uma voz tímida ao seu lado. Ela
estava caminhando ao lado do Coelho Branco, que olhava ansiosamente para
ela.

``Muito'', disse Alice. ``\ldots{} Cadê a Duquesa?''

``Psiu!'', disse o Coelho em tom baixo e apressado. Ele olhava aflito
para os lados enquanto falava e então se levantou na ponta do pé, pôs a
boca perto do ouvido de Alice e sussurrou: ``Ela foi condenada à pena
de morte''.

``Mas por quê?'', disse Alice.

``Você disse `Mas que pena!'?'', o Coelho perguntou.

``Não, não disse'', disse Alice. ``Aliás, não acho que seja uma pena. Eu
disse `Mas por quê?'.''

``Ela bateu com as duas mãos nas orelhas da Rainha\ldots{}'', o Coelho
começou. Alice deixou escapar uma breve gargalhada. ``Oh, silêncio!'', o
Coelho sussurrou em tom assustado. ``A Rainha vai acabar escutando!
Sabe, ela chegou um tanto atrasada, e a Rainha falou\ldots{}''

``Aos seus lugares!'', berrou a Rainha com voz de trovão, e as pessoas
começaram a correr em todas as direções, tropeçando umas nas outras; até
que enfim todos se posicionaram, depois de um minuto ou dois, e o jogo
começou.

Alice nunca tinha visto um campo de \textit{croquet} tão curioso em toda sua
vida; era todo cheio de sulcos e plataformas; as bolas eram ouriços
vivos, os martelos flamingos vivos, e os soldados precisavam se dobrar,
com mãos e pés no chão, para formar os arcos.

A principal dificuldade que Alice encontrou a princípio foi lidar com
seu flamingo. Ela conseguiu colocar o corpo dele embaixo do braço, com algum
conforto, com as pernas frouxas pensas, mas, de modo geral, assim que
conseguiu fazê-lo esticar bem o pescoço, e estava prestes bater no
ouriço com a cabeça dele, o flamingo se virou e olhou para ela, com uma
expressão tão intrigada que ela não conseguiu evitar de explodir em uma
gargalhada. E quando ela o fez abaixar a cabeça, e estava prestes a
tentar de novo, foi muito aflitivo ver que o ouriço havia se desenrolado
e estava tentando fugir rastejando. Além de tudo isso, havia geralmente
um sulco e uma plataforma no caminho, aonde quer que ela mirasse o
ouriço, e, como os soldados dobrados estavam sempre se levantando e
andando para outras partes do campo, Alice logo concluiu que aquele era
realmente um jogo muito difícil.

Os jogadores jogavam todos ao mesmo tempo sem esperar a vez do outro,
brigando sem parar e disputando os mesmos ouriços; e dali a pouco a
Rainha foi tomada por uma paixão furiosa, bateu os pés e começou a
berrar ``Cortem a cabeça dele!'' ou ``Cortem a cabeça dela!'' pelo menos
uma vez por minuto.

Alice começou a se sentir muito incomodada: sem dúvida, ela não havia
entrado em nenhuma disputa com a Rainha, mas sabia que isso podia
acontecer a qualquer momento, ``e aí'', ela pensou, ``o que será de mim?
As pessoas por aqui adoram cortar cabeças: a grande questão é se vai
sobrar alguém vivo!''

Ela estava procurando uma forma de fugir, e se perguntava se conseguiria
fazê-lo sem ser notada, quando reparou em uma curiosa aparição em pleno
ar: a princípio, aquilo a deixou muito intrigada, mas, depois de
observar por um ou dois minutos, ela se deu conta de que era um sorriso,
e ela disse a si mesma: ``É o Gato de Cheshire: agora terei alguém com
quem conversar''.

``Como vai?'', disse o Gato, assim que se formou o suficiente da boca
para ele falar.

Alice esperou aparecerem os olhos e então meneou a cabeça. ``Não
adianta falar com ele'', ela pensou, ``enquanto não aparecerem as
orelhas, ou pelo menos uma delas''. No minuto seguinte, a cabeça inteira
surgiu, e então Alice depôs seu flamingo e começou a fazer um relato do
jogo, sentindo-se muito contente por ter alguém que a ouvisse. O Gato
parecia achar que já havia o bastante de si à vista e não apareceu mais.

``Não acho que eles joguem limpo'', Alice começou, em tom um tanto
queixoso, ``e todos brigam tão horrivelmente que a gente mal consegue se
ouvir\ldots{} e parece que não seguem nenhuma regra em particular; pelo
menos, se há alguma regra, ninguém cumpre\ldots{} e você não imagina como é
confuso o fato de todas as coisas estarem vivas; por exemplo, veja o arco
onde tenho que acertar, ele fica andando lá do outro lado do campo\ldots{} e
eu ia \textit{croquetear} o ouriço da Rainha agorinha mesmo, só que ele fugiu
quando viu o meu se aproximando!''

``Você gostou da Rainha?'', disse o Gato em voz baixa.

``Nem um pouco'', disse Alice. ``Ela é extremamente\ldots{}'' Nesse instante,
ela notou que a Rainha estava logo atrás escutando, de modo que
prosseguiu: ``\ldots{} invencível, de modo que nem vale a pena continuar
jogando''.

A Rainha sorriu e prosseguiu.

``Com quem você está falando?'', disse o Rei, aproximando-se de Alice e
olhando para a cabeça do Gato com grande curiosidade.

``Este é um amigo meu\ldots{} um Gato de Cheshire'', disse Alice. ``Permita
que eu o apresente.''

``Não gostei da aparência dele'', disse o Rei, ``mas ele pode beijar
minha mão se quiser.''

``Eu preferiria que não'', comentou o Gato.

``Não seja impertinente'', disse o Rei, ``e não me olhe assim!'' Ele se
escondeu atrás de Alice ao falar.

``O gato pode olhar para o rei'', disse Alice. ``Li isso em um livro,
mas não me lembro qual.''

``Bem, ele deve se retirar'', disse o Rei, muito convictamente, e chamou
a Rainha, que estava passando nesse momento. ``Querida! Quero que você
mande esse gato embora!''

A Rainha só tinha uma forma de resolver todas as dificuldades, grandes
ou pequenas. ``Cortem a cabeça dele!'', ela disse, sem sequer olhar para
o lado.

``Vou chamar o carrasco pessoalmente'', disse o Rei, avidamente, e saiu
correndo.

Alice achou que era melhor voltar e ver como estava indo o jogo, pois
ouvira a Rainha ao longe berrando apaixonadamente. Ela já tinha ouvido
a Rainha sentenciar três jogadores à pena de morte por terem perdido a
vez, e ela não estava gostando nada daquilo tudo, pois o jogo era tão
confuso que ela nunca sabia se era ou não sua vez de jogar. De modo que
ela foi procurar seu ouriço.

O ouriço estava brigando com outro ouriço, o que pareceu a Alice uma
excelente oportunidade de \textit{croquetear} um com o outro: a única dificuldade
foi que seu flamingo tinha ido para o outro lado do jardim, onde Alice
podia vê-lo tentando desajeitadamente subir voando em uma árvore.

Quando ela conseguiu apanhar o flamingo e o trouxe de volta ao campo, a
briga havia acabado e ambos os ouriços tinham desaparecido: ``mas nem tem tanta
importância'', pensou Alice, ``pois todos os arcos foram embora desse
lado do campo.'' De modo que ela pôs o flamingo debaixo do braço, para
ele não escapar de novo, e voltou para conversar mais um pouco com seu
amigo.

Quando ela voltou para o Gato de Cheshire, ficou surpresa ao encontrar
uma grande multidão reunida em torno dele: havia uma disputa ocorrendo
entre o carrasco, o Rei e a Rainha, que estavam todos falando ao mesmo
tempo, enquanto todos os outros esperavam calados e pareciam muito
incomodados.

No momento em que Alice apareceu, os três recorreram a ela para decidir
a questão e repetiram seus argumentos para ela, embora, como todos
falaram ao mesmo tempo, ela tenha achado muito difícil de fato entender
exatamente o que cada um disse.

O argumento do carrasco era que era impossível decapitar se não houvesse
um corpo do qual a cabeça deveria ser cortada: que ele nunca tivera que
fazer uma coisa dessas antes, e que não iria começar agora,
\emph{naquela altura de sua carreira}.

O argumento do Rei era que qualquer um que tivesse uma cabeça podia ser
decapitado, e que ele não devia falar aqueles absurdos.

O argumento da Rainha era que, se algo não fosse feito a respeito
imediatamente, ela mandaria executar todo mundo agora mesmo. (Este
último comentário foi o que deixou o grupo inteiro preocupado e aflito.)

Alice não conseguiu pensar em nada para dizer, exceto: ``O Gato é da
Duquesa: seria melhor perguntar para ela''.

``Ela está presa'', a Rainha disse ao carrasco; ``vá buscá-la.'' E o
carrasco partiu como uma flecha.

A cabeça do Gato começou a sumir no momento em que o carrasco partiu, e,
quando o carrasco voltou com a Duquesa, a cabeça do Gato já havia
desaparecido inteiramente, de modo que o Rei e o carrasco correram feito
loucos para cima e para baixo procurando a cabeça do Gato, enquanto o
resto do grupo voltou a jogar.

%Capítulo IX
\quebra\chapter[A história da falsa Tartaruga]{A história da falsa\break Tartaruga}

``Você não imagina como estou contente em revê-la, minha querida!'',
disse a Duquesa, dando afetuosamente o braço a Alice, e elas caminharam
juntas.

Alice ficou muito contente por encontrá-la com um temperamento tão
amável e pensou consigo mesma que talvez fosse a pimenta que a deixara
tão selvagem quando se conheceram na cozinha.

``Quado eu for Duquesa'', ela disse consigo mesma (mas não em tom muito
esperançoso), ``não usarei pimenta nenhuma na minha cozinha. Sopa não
precisa de pimenta\ldots{} Talvez seja a pimenta que deixa as pessoas
esquentadinhas'', ela prosseguiu, muito satisfeita por encontrar um novo
tipo de regra, ``e o vinagre que as deixa azedas\ldots{} e a camomila
amargas\ldots{} e bala de cevada e essas coisas deixam as crianças doces.
Quem dera as pessoas soubessem disso; aposto que não seriam tão
mesquinhas, sabe como é\ldots{}''

Ela havia praticamente se esquecido da Duquesa a essa altura e ficou um
pouco assustada ao ouvir a voz dela perto de seu ouvido. ``Você está
pensando alguma coisa, minha querida, e se esqueceu de falar. No
momento, não sei dizer a moral dessa história, mas daqui a pouco vou me
lembrar.''

``Talvez não haja nenhuma moral da história'', Alice ousou comentar.

``Nananina, menina!'', disse a Duquesa. ``Tudo tem uma moral da
história, se a gente souber procurar.'' E ela se espremeu ainda mais
perto de Alice ao falar.

Alice não gostou muito de ficar tão perto dela: primeiro, porque a
Duquesa era muito feia; e, segundo, porque a Duquesa tinha a altura
exata para apoiar o queixo no ombro de Alice, e seu queixo era muito
pontudo e incômodo. No entanto, ela não gostava de ser rude, de modo que
suportou aquilo o máximo que conseguiu. ``O jogo ficou melhor agora'',
ela disse, só para manter a conversa mais um pouco.

``De fato'', disse a Duquesa. ``E a moral da história é\ldots{} `Oh, é o
amor, o que faz o mundo girar!'\,''

``Alguém já disse'', Alice sussurrou, ``que isso aconteceria se cada um
cuidasse da própria vida!''

``Ah, bem! É praticamente a mesma coisa'', disse a Duquesa, cravando seu
queixinho pontudo no ombro de Alice e acrescentando: ``e a moral da
história é\ldots{} `Cuide do sentido, que os sons se resolvem sozinhos'.''

``Como ela gosta de encontrar moral nas coisas!'', Alice pensou consigo
mesma.

``Eu diria que você deve estar se perguntando por que eu não passo o
braço pela sua cintura'', a Duquesa disse após uma pausa. ``O motivo é
que tenho minhas dúvidas sobre o temperamento do seu flamingo. Será que
fazemos uma experiência?''

``Cuidado para não pegar na garganta dele'', Alice respondeu com
cautela, sentindo-se totalmente despreocupada com a experiência
proposta.

``É verdade'', disse a Duquesa, ``com flamingos e mostarda, é preciso
ter cuidado para não pegar na garganta. E a moral da história é\ldots{} `Aves
da mesma plumagem voam juntas'.''

``Bem, mostarda não é ave'', Alice comentou.

``Claro, como sempre'', disse a Duquesa, ``você usa as palavras de um
modo muito claro!''

``Mostarda é um minério, eu acho'', disse Alice.

``Claro que é'', disse a Duquesa, que parecia disposta a concordar com
qualquer coisa que Alice dissesse. ``Aqui perto há uma grande mina de
mostarda. E a moral dessa história é que\ldots{} `Minas, quanto mais minhas,
menos suas'.''

``Oh, lembrei!'', exclamou Alice, que não dera atenção a esse último
comentário. ``Mostarda é vegetal. Não parece, mas é.''

``Concordo totalmente'', disse a Duquesa; ``e a moral dessa história é
que\ldots{} `Seja o que aparentemente você seria'\ldots{} ou se você quiser dizer
de modo mais simples\ldots{} `Nunca pense que você é outra coisa senão o que
pode parecer para os outros que você é, porque, do contrário, o que você
é mesmo pareceria para os outros ser outra coisa'.''

``Acho que talvez eu entenda melhor'', Alice disse muito educadamente,
``se tivesse isso escrito, mas não consegui entender bem quando você
disse.''

``Não foi nada perto do que eu poderia dizer se quisesse'', a Duquesa
respondeu, em tom satisfeito.

``Não se incomode de dizer nada mais comprido do que isso'', disse
Alice.

``Oh, não me fale em trabalho!'', disse a Duquesa. ``Vou lhe dar de
presente tudo o que eu disse até agora por escrito.''

``Mas que presente de grego!'', pensou Alice. ``Ainda bem que ninguém dá
isso de aniversário!'' Mas ela não ousou dizer isso em voz alta.

``Pensando de novo?'', a Duquesa perguntou, fincando outra vez seu
queixinho ossudo.

``Tenho direito de pensar'', disse Alice rispidamente, pois ela estava
começando a se sentir um pouco preocupada.

``Tanto direito'', disse a Duquesa, ``quanto os porcos têm de voar, e a
mo\ldots{}''

Mas aqui, para grande surpresa de Alice, a voz da Duquesa começou a
diminuir, no meio de sua palavra favorita, ``moral'', e o braço que
estava abraçado ao seu começou a tremer. Alice ergueu os olhos, e lá
estava a Rainha diante delas, de braços cruzados, carrancuda como uma
tempestade.

``Que belo dia, Majestade!'', a Duquesa começou em voz baixa e fraca.

``Agora, vou lhe avisar uma coisa'', berrou a Rainha, batendo os pés no
chão ao falar, ``ou você vai embora, ou vai sua cabeça, e isso é para já!
Escolha!''

A Duquesa escolheu, e foi embora no mesmo instante.

``Vamos continuar a jogar'', a Rainha disse a Alice; e Alice estava
apavorada demais para dizer qualquer coisa, mas lentamente a acompanhou
de volta ao campo de \textit{croquet}.

Os outros convidados haviam aproveitado a ausência da Rainha e
descansavam à sombra. No entanto, no momento em que a avistaram,
voltaram correndo para o jogo, ao que a Rainha comentou que, se
demorassem mais um momento, isso lhes custaria suas cabeças.

Todo esse tempo em que ficaram jogando, a Rainha não parou de brigar com
os outros jogadores e de berrar ``Cortem a cabeça dele!'' ou ``Cortem
a cabeça dela!''. Aqueles que ela condenava eram levados presos pelos
soldados, que evidentemente precisavam deixar de ser arcos para
obedecer, de modo que ao cabo de meia hora, mais ou menos, já não havia
mais nenhum arco no campo, e todos os jogadores, exceto o Rei, a Rainha
e Alice, estavam presos e condenados à morte.

Então a Rainha se retirou, quase sem fôlego, e disse a Alice: ``Você já
viu a Falsa Tartaruga?''.

``Não'', disse Alice. ``Não sei nem o que é uma Falsa Tartaruga.''

``É aquilo que se usa para fazer Sopa de Falsa Tartaruga'', disse a
Rainha.

``Nunca vi, nem ouvi falar'', disse Alice.

``Então vamos'', disse a Rainha, ``e ela mesmo lhe contará sua
história.''

Enquanto caminhavam juntas, Alice escutou o Rei dizendo em voz baixa, ao
grupo em geral, ``Vocês estão todos perdoados''. ``Ora, isso é bom!'',
ela disse consigo mesma, pois ficara muito infeliz com o número de
execuções ordenadas pela Rainha.

Logo chegaram diante de um Grifo dormindo pesadamente deitado ao sol.
(Se você não sabe o que é um Grifo, procure uma figura.) ``Levante-se,
preguiçoso!'', disse a Rainha, ``e leve essa jovem dama para ver a Falsa
Tartaruga e escutar sua história. Preciso voltar e acompanhar algumas
execuções que ordenei'', e ela foi embora, deixando Alice sozinha com o
Grifo. Alice não gostou muito da aparência da criatura, mas em geral
achou que seria tão seguro ir com ele quanto continuar com aquela Rainha
selvagem, de modo que ela esperou.

O Grifo se levantou e esfregou os olhos, então ficou observando a
Rainha até que ela sumisse de seu campo de visão e gargalhou.
``Que engraçado!'', disse o Grifo, em parte para si mesmo, em parte para
Alice.

``Qual é a graça?'', disse Alice.

``Ora, \emph{ela}'', disse o Grifo. ``É tudo fantasia dela, isso aí:
eles nunca executam ninguém, você sabe. Vamos!''

``Aqui todo mundo fica dizendo `vamos!'\,'', pensou Alice, conforme seguia
lentamente atrás dele. ``Nunca recebi tantas ordens na minha vida,
jamais!''

Eles não haviam ido muito longe quando avistaram a Falsa Tartaruga à
distância, sentada, triste e solitária, em um lajedo de pedra, e,
conforme se aproximaram dela, Alice pôde ouvi-la suspirar, como se seu
coração fosse se partir. Ela teve muita pena dela. ``Qual é o motivo de
tanta tristeza?'', ela perguntou ao Grifo, e o Grifo respondeu,
praticamente com as mesmas palavras de antes: ``É tudo fantasia dela,
isso aí: ela não está triste coisa nenhuma, você sabe\ldots{} Vamos!''.

Então eles foram até a Falsa Tartaruga, que olhou para eles com seus
olhos grandes cheios de lágrimas, mas não disse nada.

``Esta jovem dama aqui'', disse o Grifo, ``ela quer conhecer a sua
história, quer mesmo.''

``Vou contar para ela'', disse a Falsa Tartaruga em tom grave,
retumbante. ``Sentem-se, vocês dois, e não digam nada enquanto eu não
terminar.''

De modo que eles se sentaram, e ninguém falou nada por alguns minutos.
Alice pensou consigo mesma: ``Não sei como ela vai terminar, se nunca
começa\ldots{}''. Mas esperou pacientemente.

``Um dia'', disse a Falsa Tartaruga por fim, com um longo suspiro, ``já
fui uma verdadeira Tartaruga.''

Essas palavras foram seguidas por um silêncio muito longo, interrompido
apenas por uma ocasional exclamação de ``Hjckrrh!'' da parte do Grifo e
pelos constantes soluços da Falsa Tartaruga. Alice estava muito perto de
se levantar e dizer ``Obrigada, senhora, por sua interessante história'',
mas não conseguia evitar de pensar que devia haver mais coisas pela
frente, de modo que ela se sentou imóvel e ficou calada.

``Quando éramos pequenos'', a Falsa Tartaruga prosseguiu por fim, mais
serenamente, embora ainda soluçando um pouco de quando em quando,
``nossa escola era no mar. O professor era uma velha Tartaruga\ldots{}
costumávamos chamá-lo de senhor Jabuti\ldots{}''

``Por que vocês o chamavam de Jabuti, se ele era uma Tartaruga?'', Alice
perguntou.

``Chamávamos de Jabuti porque foi o que ele nos disse'', disse a Falsa
Tartaruga irritadamente. ``Realmente você é muito burra!''

``Você devia ter vergonha de fazer uma pergunta ingênua assim'',
acrescentou o Grifo, e então os dois ficaram sentados em silêncio
olhando para a pobre Alice, que se sentiu querendo afundar na terra.

Por fim, o Grifo disse à Falsa Tartaruga: ``Continue, minha velha. Não
temos o dia inteiro!'', e ela prosseguiu com as seguintes palavras:

``Sim, nossa escola era no mar, embora você talvez não acredite\ldots{}''

``Eu não disse nada que não acreditava!'', interrompeu Alice.

``Mas você não acredita'', disse a Falsa Tartaruga.

``Controle sua língua!'', acrescentou o Grifo, antes que Alice falasse
de novo. A Falsa Tartaruga prosseguiu:

``Nós tivemos a melhor educação possível\ldots{} na verdade, íamos todos os
dias à escola\ldots{}''

``Eu também vou à escola todo dia'', disse Alice, ``não há nenhum motivo
de orgulho nisso.''

``Com matérias optativas?'', perguntou a Falsa Tartaruga um tanto
ansiosamente.

``Sim'', disse Alice, ``aprendemos francês e música.''

``E banho?'', disse a Falsa Tartaruga.

``Certamente, não!'', disse Alice, indignada.

``Ah! Então a sua não era realmente uma boa escola'', disse a Falsa
Tartaruga em tom aliviado. ``Agora, na nossa, eles tinham, no final do
curso, `Francês, música e banho\ldots{} como optativas'.''

``Não poderia ser diferente'', disse Alice, ``vivendo no fundo do mar.''

``Eu não podia pagar optativas'', disse a Falsa Tartaruga com um
suspiro. ``Eu só fiz o curso normal.''

``Quais eram as matérias?'', indagou Alice.

``Balançar e Contorcer, evidentemente, para começo de conversa'', a
Falsa Tartaruga respondeu; ``e depois os diferentes ramos da Aritmética:
Ambição, Distração, Enfeiamento e Zombaria.''

``Nunca ouvi falar em `Enfeiamento'\,'', Alice arriscou dizer. ``O que
é?''

O Grifo levantou duas patas de tanta surpresa. ``Nunca ouviu falar em
enfeiamento?!'', ele exclamou. ``Você sabe o que é embelezar, não sabe?''

``Sei'', disse Alice, desconfiada. ``Quer dizer\ldots{} deixar\ldots{} tudo\ldots{} mais
bonito.''

``Pois então'', o Grifo prosseguiu, ``se você não souber o que é
enfeiar, você é uma ignorante.''

Alice não se sentiu encorajada a fazer mais perguntas sobre aquilo, de
modo que se virou para a Falsa Tartaruga e disse: ``O que mais foi teve
que aprender?''.

``Bem, tínhamos aula de Mistério'', a Falsa Tartaruga respondeu,
contando as matérias em suas nadadeiras. ``\ldots{} Mistério, antigo e
moderno, com Marografia: e depois Fala-arrastada --- o professor de
Fala-arrastada era uma velha enguia, que vinha uma vez por semana; ele
nos deu aula de Fala-arrastada, Alongamento e Tontura em Espiral.''

``Como era isso?'', disse Alice.

``Bem, eu não serei capaz de mostrar'', a Falsa Tartaruga disse. ``Sou
muito dura. E o Grifo nunca aprendeu direito.''

``Não deu tempo'', disse o Grifo. ``Mas eu tive aula com o professor do
Clássico. Era um velho caranguejo, isso sim.''

``Nunca fui aluna dele'', a Falsa Tartaruga disse com um suspiro. ``Ele
dava aula de Latido e Grito, como se dizia.''

``Ele era assim mesmo, era mesmo'', disse o Grifo, suspirando por sua
vez; e ambos esconderam os rostos nas patas.

``E qual era a carga horária diária das aulas?'', disse Alice, querendo
mudar logo de assunto.

``Dez horas no primeiro dia'', disse a Falsa Tartaruga. ``Nove no
segundo, e assim por diante.''

``Que programa curioso!'', exclamou Alice.

``Era por isso que chamavam de carga'', o Grifo comentou, ``porque ia
descarregando dia após dia.''

Essa foi uma ideia bastante nova para Alice, e ela pensou nisso um
pouquinho antes de fazer o próximo comentário. ``Então no
décimo primeiro dia devia ser feriado.''

``Claro que era'', disse a Falsa Tartaruga.

``E como vocês faziam no décimo segundo dia?'', Alice prosseguiu
avidamente.

``Já chega de falar de escola'', o Grifo interrompeu em tom muito
decidido. ``Conte para ela alguma coisa sobre brincadeiras agora.''

%Capítulo X
\quebra\chapter{A quadrilha da Lagosta}

A Falsa Tartaruga suspirou profundamente e passou uma de suas
nadadeiras sobre os olhos. Ela olhou para Alice e tentou falar, mas,
por um ou dois minutos, soluços sufocaram sua voz: ``Como se estivesse
com um osso entalado na garganta'', disse o Grifo; e se pôs a sacudi-lo
e dar tapas em suas costas. Por fim, a Falsa Tartaruga recuperou a voz
e, com lágrimas escorrendo pelas faces, prosseguiu:

``Talvez você não tenha vivido muito tempo no fundo do mar\ldots{}'' (``Não
mesmo'', disse Alice.) ``E talvez não tenha sido jamais apresentada a uma
lagosta\ldots{}'' (Alice começou a dizer ``Um dia eu provei\ldots{}'', mas se
deteve rapidamente, e disse: ``Não, nunca''.) ``\ldots{} Então você não faz
ideia da delícia que é uma Quadrilha de Lagosta!''

``Não, de fato'', disse Alice. ``Que tipo de dança é?''

``Ora'', disse o Grifo, ``primeiro você forma uma fila no litoral\ldots{}''

``Duas filas!'', gritou a Falsa Tartaruga. ``Focas, tartarugas, e assim
por diante; aí, depois que você tirou a água-viva do caminho\ldots{}''

``Isso geralmente leva algum tempo'', interrompeu o Grifo.

``\ldots{} você dá dois passos para a frente\ldots{}''

``Cada um tendo uma lagosta como par!'', exclamou Grifo.

``Claro'', a Falsa Tartaruga disse: ``dois passos para a frente, atenção
aos pares\ldots{}''

``\ldots{} troque de lagosta e recue na mesma ordem'', continuou o Grifo.

``Então, você sabe'', a Falsa Tartaruga prosseguiu, ``você joga\ldots{}''

``As lagostas!'', berrou o Grifo, dando um pulo no ar.

``\ldots{} no mar o mais longe que conseguir\ldots{}''

``Nade atrás delas!'', gritou o Grifo.

``Dê uma cambalhota no mar!'', exclamou a Falsa Tartaruga, saltitando
loucamente.

``Troque de lagosta outra vez!'', berrou o Grifo.

``Volte de novo para a praia, e\ldots{} essa é a primeira figura'', disse a
Falsa Tartaruga, de repente baixando a voz; e as duas criaturas, que
haviam ficado pulando feito loucas todo esse tempo, sentaram-se
novamente, muito tristes e caladas, e olharam para Alice.

``Deve ser uma dança muito bonita'', disse Alice, timidamente.

``Você gostaria de ver um pouco?'', disse a Falsa Tartaruga.

``Gostaria muitíssimo'', disse Alice.

``Venha, vamos tentar a primeira figura!'', disse a Falsa Tartaruga ao
Grifo. ``Podemos fazer sem lagostas, sabe como é\ldots{} Quem vai cantar?''

``Oh, você canta'', disse o Grifo. ``Esqueci a letra.''

Então eles começaram solenemente a dançar dando voltas em torno de
Alice, de quando em quando pisando no seu pé quando passavam muito
perto, e acenando com as patas dianteiras para marcar o tempo, enquanto
a Falsa Tartaruga cantava o seguinte, muito lenta e tristemente:

\begin{quote}
``Você pode ir mais depressa?'', disse o badejo ao caramujo.\\
``Há uma toninha logo atrás, que avança enquanto fujo.\\
Veja as lagostas e as tartarugas, veja só que maravilha!\\
Estão esperando no cascalho --- você vem para a quadrilha?\\
Você vem, não vem? Você vem, não vem? Junte-se à quadrilha!\\
Você vem, não vem? Você vem, não vem? Junte-se à quadrilha!
\medskip
``Você não faz ideia da delícia que será,\\
Quando nos atirarem, com as lagostas, no mar!''\\
Mas o caramujo disse: ``Muito longe!'' e olhou de soslaio ---\\
Disse obrigado ao badejo, mas não se juntou ao baile.\\
Não iria, não poderia, não iria, não poderia se juntar ao baile.\\
Não iria, não poderia, não iria, não poderia se juntar ao baile.
\medskip
``E daí se é longe?'', o amigo escamoso retrucou.\\
``Há outra praia, você sabe, do outro lado.\\
Quanto mais longe da Inglaterra, mais perto da França.\\
Então não se avexe, amado caramujo, mas junte-se à dança.\\
Você vai, não vai? Você vai, não vai? Não vai se juntar à dança?
\end{quote}

``Obrigada, é uma dança muito interessante de assistir'', disse Alice,
sentindo-se muito contente por ter terminado finalmente. ``E gostei
muito da canção sobre o badejo!''

``Oh, quanto ao badejo'', disse a Falsa Tartaruga, ``eles\ldots{} você já viu
badejo, não é mesmo?''

``Já'', disse Alice. ``Vi muitos badejos no jan\ldots{}'', ela se interrompeu
bruscamente.

``Não sei onde é Jan'', disse a Falsa Tartaruga, ``mas, se você já os viu
muitas vezes, deve saber como são os badejos.''

``Acredito que sim'', Alice respondeu, pensativa. ``Eles têm os rabos nas
bocas\ldots{} e ficam cobertos de migalhas.''

``Você se engana sobre as migalhas'', disse a Falsa Tartaruga. ``As
migalhas saem na água do mar. Mas de fato eles têm os rabos nas bocas; e
o motivo disso é que\ldots{}'', aqui a Falsa Tartaruga bocejou e fechou os
olhos. ``Diga você os motivos e tudo o mais'', ela disse ao Grifo.

``O motivo é que'', disse o Grifo, ``eles vão com as lagostas ao baile.
De modo que são atirados ao mar. De modo que caem bem no fundo. De modo
que põem o rabo na boca. De modo que não conseguem mais tirar o rabo da
boca. Só isso.''

``Obrigada'', disse Alice. ``Muito interessante. Nunca aprendi tanto
sobre badejos.''

``Posso contar muito mais, se você quiser'', disse o Grifo. ``Sabe por
que se chamam badejos?''

``Nunca tinha pensado nisso'', disse Alice. ``Por quê?''

``Porque eles engraxam botas e sapatos'', o Grifo respondeu muito solene.
%Não funcionou muito bem em português

Alice ficou totalmente confusa. ``Engraxam botas e sapatos!'', ela
repetiu em tom maravilhado.

``Ora, como você engraxa os seus sapatos?'', disse o Grifo. ``Quero
dizer, o que faz para que eles brilhem?''

Alice olhou para os próprios pés e considerou um pouco antes de dar uma
resposta. ``Acho que são engraxados com graxa.''

``As botas e os sapatos no mar'', o Grifo prosseguiu com voz grave,
``são engraxados com badejos. Agora você sabe.''

``Do que é feita essa graxa?'', Alice perguntou em tom de grande
curiosidade.

``Solhas e moreias, é claro'', o Grifo respondeu um tanto
impacientemente. ``Qualquer camarão sabe disso.''

``Se eu fosse o badejo'', disse Alice, cujos pensamentos ainda entoavam
a canção, ``eu teria dito à toninha `Afaste-se, por favor: não queremos
nada com você!'.''

``Eles são obrigados a ficar com elas'', a Falsa Tartaruga disse.
``Nenhum peixe sábio vai a lugar nenhum sem uma toninha.''

``É mesmo?'', disse Alice em tom de grande surpresa.

``Claro que não'', disse a Falsa Tartaruga. ``Ora, se um peixe viesse
até mim e me dissesse que ia viajar, eu diria: `Com qual toninha?'.''

``Você quis dizer `sentido'?'', disse Alice.
%Em português não funcionou toninha/sentido

``Quis dizer o que eu disse'', a Falsa Tartaruga respondeu em tom
ofendido. E o Grifo acrescentou: ``Venha, vamos ouvir agora alguma
aventura das suas''.

``Eu poderia contar minhas aventuras\ldots{} a começar por esta manhã'',
disse Alice um pouco tímida, ``mas não adianta começar por ontem, porque
ontem eu era uma pessoa diferente.''

``Explique melhor'', disse a Falsa Tartaruga.

``Não, não! Primeiro as aventuras'', disse o Grifo em tom impaciente.
``Explicar demora demais.''

Então Alice começou a contar a eles suas aventuras a partir do momento em
que viu pela primeira vez o Coelho Branco. Ela ficou um pouco nervosa a
princípio, pois as duas criaturas estavam perto demais, uma de cada lado,
e abriam muito os olhos e as bocas, mas ela foi ganhando coragem
conforme prosseguia. Seus ouvintes ficaram totalmente calados até a
parte em que ela recitou ``Você está velho, pai William'' para a
Lagarta, e as palavras saíram todas diferentes, e então a Falsa
Tartaruga respirou fundo e disse: ``Isso é muito curioso''.

``É praticamente a coisa mais curiosa que existe'', disse o Grifo.

``Saiu tudo diferente!'', a Falsa Tartaruga repetiu pensativamente. ``Eu
gostaria que ela recitasse de novo. Diga para ela recitar.'' Ele olhou
para o Grifo como se achasse que ele tinha algum tipo de autoridade
sobre Alice.

``Fique de pé e recite `Eis a voz do preguiçoso'\,'', disse o Grifo.

``Como essas criaturas gostam de dar ordens e mandar os outros recitar
lições!'', pensou Alice. ``É como se eu estivesse na escola.'' No
entanto, ela se levantou e começou a recitar, mas sua cabeça estava tão
cheia da Quadrilha da Lagosta que ela mal se deu conta do que estava
dizendo, e as palavras de fato saíram muito esquisitas:

\begin{quote}
Eis a voz do Lagosta; escutei ele afirmar,\\
``Você me deixou marrom, meu pelo devo açucarar.''\\
Como o pato com as pálpebras, ele com seu nariz\\
Aperta cinto e botões, e de tudo o mais desiste.\\
Quando a areia está seca, ele é alegre como cotovia.\\
E, com desdém, com o Tubarão até conversaria:\\
Mas quando sobe a maré e vêm os tubarões,\\
Sua voz tem tímidos e trêmulos tons.
\end{quote}

``Está diferente do que eu costumava recitar quando criança'', disse o
Grifo.

``Bem, eu nunca tinha ouvido isso antes'', disse a Falsa Tartaruga,
``mas me soa como um contrassenso incomum.''

Alice não disse nada; ela havia sentado segurando o rosto nas mãos,
imaginando se alguma coisa \emph{algum dia} aconteceria naturalmente
outra vez.

``Eu gostaria que você explicasse'', disse a Falsa Tartaruga.

``Ela não poderá explicar'', apressadamente disse o Grifo. ``Vá logo para a
próxima estrofe.''

``Mas por que ele desiste de tudo?'', a Falsa Tartaruga insistiu. ``Como
ele fez essas coisas com o nariz?''

``Essa é a primeira posição da dança'', Alice disse; mas estava
pavorosamente intrigada com tudo aquilo, e ansiosa para mudar de
assunto.

``Mas passe logo à estrofe seguinte'', o Grifo repetiu. ``Começa assim:
`Passei por seu jardim\ldots{}'.''

Alice não ousou desobedecer, embora tivesse certeza de que sairia tudo
errado, e ela prosseguiu com voz trêmula:

\begin{quote}
Passei por seu jardim, e vi, de canto de olho,\\
Como a Coruja e a Pantera dividiam um bolo:\\
A Pantera comeu a casca, o molho, a carne,\\
Enquanto a Coruja engoliu o prato.\\
Quando o bolo acabou, a Coruja, como prêmio,\\
Embolsou a colher, generosamente:\\
Ao passo que a Pantera recebeu faca e garfo,\\
E concluiu o banquete com um\ldots{}
\end{quote}

``De que adianta repetir tudo isso'', a Falsa Tartaruga interrompeu,
``se você não explica as coisas que diz? É de longe a coisa mais confusa
que eu já ouvi nada na vida!''

``Sim, acho melhor você ir embora'', disse o Grifo; e Alice ficou
contente por fazê-lo.

``Que tal fazermos outra figura da Quadrilha da Lagosta?'', o Grifo
prosseguiu. ``Ou você preferiria que a Falsa Tartaruga cantasse outra
canção?''

``Oh, uma canção, por favor, se a Falsa Tartaruga fizesse essa
gentileza'', Alice respondeu, tão avidamente que o Grifo disse, em tom
um tanto ofendido: ``Humpf! Bem, gosto não se discute! Cante para ela
`Sopa de Tartaruga', pode ser, minha velha?''.

A Falsa Tartaruga soltou um suspiro profundo e começou, com a voz
sufocada de soluços, a cantar o seguinte:

\begin{quote}
Bela Sopa, cheia de verdura e vitamina,\\
Esperando na escaldante terrina!\\
Quem a tal delícia não diz ``oba''?\\
Sopa noturna, bela Sopa!\\
Sopa noturna, bela Sopa!\\
Be\ldots{}laaa So\ldots{}opa!\\
Be\ldots{}laaa So\ldots{}opa!\\
So\ldots{}opa no\ldots{}tur\ldots{}na,\\
Bela delícia de Sopa!
\medskip
Bela Sopa! Quem quer saber de peixe,\\
Caça ou outro prato que seja?\\
Quem não daria tudo por um pouco,\\
Dois tostões que fosse de uma bela Sopa?\\
Dois tostões que fosse de uma bela Sopa?\\
Be\ldots{}laaa So\ldots{}opa!\\
Be\ldots{}laaa So\ldots{}opa!\\
So\ldots{}opa no\ldots{}tur\ldots{}na,\\
Bela delí\ldots{}cia de \textsc{sopa}!
\end{quote}

``Refrão!'', exclamou o Grifo, e a Falsa Tartaruga havia começado a
repetir o refrão, quando um grito de ``Vai começar o julgamento!'' foi
ouvido ao longe.

``Vamos!'', exclamou o Grifo, e, pegando Alice pela mão, ele se
apressou, sem esperar o final da canção.

``Que julgamento é esse?'', Alice disse, ofegante, enquanto corria; mas o
Grifo só respondeu ``Vamos!'', e correu mais depressa, enquanto cada
vez mais fracamente, trazidas pela brisa que os acompanhava, ouviam-se as
melancólicas palavras:

\begin{quote}
So\ldots{}opa no\ldots{}tur\ldots{}na,\\
Bela delícia de Sopa!
\end{quote}

%Capítulo XI
\quebra\chapter[Quem roubou as tortas?]{Quem roubou as\break tortas?}

O Rei e a Rainha de Copas estavam sentados em seus tronos quando eles
chegaram, com uma grande multidão em volta --- todo tipo de passarinhos e
animais, assim como o baralho inteiro: o Valete estava de pé na frente
deles, acorrentado, com um soldado de cada lado a vigiá-lo; e ao lado do
Rei estava o Coelho Branco, com um clarim em uma mão e um rolo de
pergaminho na outra. Exatamente no meio da corte, havia uma mesa, com
uma bandeja enorme cheia de tortas: pareciam tão gostosas que Alice
olhou com água na boca para elas --- ``Quem dera esse julgamento acabasse
logo!'' Mas parecia não haver nenhuma possibilidade de isso ocorrer, de
modo que ela começou a olhar para os lados para matar o tempo.

Alice nunca estivera em um tribunal de justiça antes, mas havia lido a
respeito em livros, e ficou muito contente ao perceber que sabia o nome
de praticamente todos ali presentes. ``Aquele é o juiz'', ela disse
consigo mesma, ``por causa da enorme peruca.''

O juiz, aliás, era o Rei; e, como ele usava a coroa por cima da peruca,
não parecia nada confortável, e certamente não lhe caía nada bem.

``E ali fica o júri'', pensou Alice, ``e essas doze criaturas'' (ela foi
obrigada a dizer ``criaturas'', você sabe, porque alguns eram animais, e
alguns eram aves), ``imagino que sejam os jurados''. Ela disse essa
última palavra duas ou três vezes consigo mesma, sentindo um certo
orgulho de si mesma, pois, pensou, e com justiça, que pouquíssimas
garotinhas da sua idade sabiam o que eram jurados. No entanto, ``membros
do júri'' teria dado na mesma.\looseness=-1

Os doze jurados estavam escrevendo concentradamente em pequenas
lousinhas. ``O que será que estão todos fazendo?'', Alice sussurrou para
o Grifo. ``Eles não deveriam escrever nada enquanto o julgamento não
começa.''

``Eles estão escrevendo os próprios nomes'', o Grifo sussurrou em
resposta, ``pois receiam se esquecer até o final do julgamento.''

``Criaturas estúpidas!'', Alice começou a dizer em voz alta e indignada,
mas logo parou, pois o Coelho Branco exclamara ``Silêncio no
tribunal!'', e o Rei pôs os óculos e olhou aflito para os lados para
ver quem tinha falado.

Alice conseguiu ver, como se estivesse olhando sobre seus ombros, que
todos os jurados estavam escrevendo ``criaturas estúpidas!'' em suas
lousinhas, e ela conseguiu ver até que um deles não sabia soletrar
``estúpidas'' e precisara perguntar ao vizinho. ``Essas lousinhas
ficarão ilegíveis até o final do julgamento!'', pensou Alice.

Um dos jurados tinha um giz que rangia. Isso, é claro, Alice não
conseguiu suportar, e ela deu a volta por trás dele e logo encontrou
uma oportunidade de tirar o giz do jurado. Ela fez isso tão depressa que
o pobrezinho do jurado (era Bill, o Lagarto) não entendeu o que
acontecera; de modo que, depois de procurar o giz perdido, foi obrigado a
escrever com o dedo pelo resto do dia; e isso não adiantava quase nada,
pois não deixava marcas na lousinha.

``Arauto, leia a acusação!'', disse o Rei.

Nesse momento, o Coelho Branco soprou três vezes seu clarim, então
desenrolou o pergaminho e leu o seguinte:

\begin{quote}
A Rainha de Copas fez as tortas,\\
No verão, um belo dia:\\
O Valete de Copas, roubou as tortas,\\
E levou-as consigo!
\end{quote}

``Já chegaram a um veredito?'', o Rei disse ao júri.

``Ainda não, ainda não!'', o Coelho apressado interrompeu. ``Ainda falta
muito antes disso!''

``Chame a primeira testemunha'', disse o Rei; e o Coelho soprou três
vezes seu clarim e chamou: ``A primeira testemunha!''.

A primeira testemunha era o Chapeleiro. Ele chegou com uma xícara de chá
em uma mão e um pedaço de pão com manteiga na outra. ``Sinto muito, sua
Majestade'', ele começou, ``por trazer isso comigo, mas eu não havia
terminado meu chá quando me chamaram.''

``Você devia ter terminado antes'', disse o Rei. ``Que horas você
começou?''

O Chapeleiro olhou para a Lebre de Março, que o acompanhara até o
tribunal, de braço dado com o Arganaz. ``Catorze de março, acho que
foi'', ele disse.

``Quinze'', disse a Lebre de Março.

``Dezesseis'', disse o Arganaz.

``Escrevam isso'', o Rei disse ao júri, e o júri avidamente escreveu as
três datas nas lousinhas, e depois somaram as três e reduziram a
resposta a xelins e pence.

``Tire sua cartola'', o Rei disse ao Chapeleiro.

``Não é minha'', disse o Chapeleiro.

``Roubada!'', o Rei exclamou, virando-se para o júri, que
instantaneamente redigiu um memorando do fato.

``Eu vendo cartolas'', o Chapeleiro agregou para explicar. ``Nenhuma das
cartolas é minha. Sou só o chapeleiro.''

Aqui a Rainha pôs os óculos e começou a olhar fixamente para o
Chapeleiro, que ficou pálido e agitado.

``Apresente uma prova'', disse o Rei; ``e não fique nervoso, ou mando
executá-lo aqui mesmo.''

Isso não pareceu encorajar muito a testemunha: ele ficava alternando o
peso de um pé para o outro, parecendo constrangido diante da Rainha, e
nessa confusão mordeu a xícara em vez de morder o pão.

Nesse exato momento, Alice teve uma sensação muito curiosa, que a
intrigou um bocado, até que percebesse o que era: ela estava começando a
crescer de novo e pensou a princípio em se levantar e sair do tribunal;
mas pensou melhor e resolveu ficar onde estava enquanto houvesse espaço
suficiente.

``Eu adoraria que você não me espremesse tanto'', disse o Arganaz,
sentado ao lado dela. ``Mal consigo respirar assim.''

``Não é minha culpa'', disse Alice muito delicadamente. ``Estou
crescendo.''

``Você não tem direito de crescer aqui'', disse o Arganaz.

``Não diga absurdos'', disse Alice com mais ousadia. ``Você sabe que
você também está crescendo.''

``Sim, mas eu cresço em um ritmo razoável'', disse o Arganaz; ``não
dessa maneira ridícula.'' E ele se levantou contrariado e atravessou o
recinto até o outro lado do tribunal.

Todo esse tempo a Rainha havia continuado olhando fixamente para o
Chapeleiro, e, assim que o Arganaz atravessou o tribunal, ela disse a um
dos mensageiros. ``Traga-me a lista dos cantores do último recital!'',
aquele em que o infeliz Chapeleiro havia tremido tanto, que se sacudira
a ponto de ficar sem os sapatos.

``Apresente uma prova'', o Rei repetiu irritado, ``ou mando executá-lo
aqui mesmo, nervoso ou não.''

``Sou um homem pobre, Majestade'', o Chapeleiro começou, com voz
trêmula, ``\ldots{} e eu mal havia começado o meu chá\ldots{} não mais de uma
semana e pouco atrás\ldots{} e o pão com manteiga foi ficando tão escasso\ldots{} e
o pisca-pisca do chá\ldots{}''

``O pisca-pisca do quê?'', disse o Rei.

``Começou com ch\ldots{}'', o Chapeleiro respondeu.

``Claro que começou!'', o Rei disse rispidamente. ``Você acha que eu sou
um burro? Continue!''

\textls[-15]{``Sou um homem pobre'', o Chapeleiro prosseguiu, ``e quase tudo ficou
piscando depois disso\ldots{} Só que a Lebre de Março disse\ldots{}''}\looseness=-1

``Não disse nada!'', a Lebre de Março interrompeu com muita pressa.

``Disse, sim!'', disse o Chapeleiro.

``Eu nego que tenha dito!'', disse a Lebre de Março.

``A lebre nega'', disse o Rei. ``Esqueçam essa parte.''

\textls[-15]{``Bem, seja como for, o Arganaz disse que\ldots{}'', o Chapeleiro prosseguiu,
olhando aflito para os lados para ver se o Arganaz também negaria: mas o
Arganaz não negou nada, pois havia adormecido.}\looseness=-1

``Depois disso'', continuou o Chapeleiro, ``cortei mais pão com
manteiga\ldots{}''

``Mas o que disse o Arganaz afinal?'', um jurado perguntou.

``Isso eu não me lembro'', disse o Chapeleiro.

``Você tem que se lembrar'', observou o Rei, ``ou vou mandar
executá-lo.''

O angustiado Chapeleiro deixou cair a xícara e o pão com manteiga e se
apoiou em um joelho. ``Sou um homem pobre, Majestade'', ele começou.

``Você é um orador paupérrimo de fato'', disse o Rei.

\textls[-20]{Nisso, um porquinho-da-índia aplaudiu e foi imediatamente reprimido
pelos mensageiros. (Como esse é um termo pesado, vou explicar como
fizeram. Eles tinham uma sacola grande de lona, amarrada na boca com
cordões: nessa sacola, puseram o porquinho-da-índia, primeiro a cabeça,
e depois sentaram em cima.)}\looseness=-1

``Fico contente por ter visto isso'', pensou Alice. ``Muitas vezes li no
jornal, ao final de um julgamento, `Houve algumas tentativas de aplauso,
que foram imediatamente reprimidas pelos mensageiros', e nunca tinha
entendido o que significava até agora.''

``Se é só isso que você sabe, pode descer'', continuou o Rei.

``Não posso descer mais do que isso'', disse o Chapeleiro. ``Já estou no
chão.''

``Então você pode se sentar'', o Rei respondeu.

Aqui o outro porquinho-da-índia aplaudiu e foi reprimido.

``Bem, assim ficamos livres dos porquinhos-da-índia!'', pensou Alice.
``Agora tudo prosseguirá sem delongas''.

``Eu preferiria ter terminado o meu chá'', disse o Chapeleiro, olhando
aflito para a Rainha, que estava lendo a lista dos cantores.

``Você pode ir agora'', disse o Rei; e o Chapeleiro saiu apressadamente
do tribunal, sem sequer esperar para calçar os sapatos.

``\ldots{} e cortem-lhe a cabeça lá fora'', a Rainha acrescentou para um dos
mensageiros; mas o Chapeleiro já havia desaparecido antes que o
mensageiro chegasse à porta.

``Chamem a próxima testemunha!'', disse o Rei.

A próxima testemunha era a cozinheira da Duquesa. Ela chegou com o
pimenteiro na mão, e Alice logo adivinhou quem era, mesmo antes de ela
entrar no tribunal, pelo modo como as pessoas perto da porta começaram a
espirrar imediatamente.

``Apresente sua prova'', disse o Rei.

``Não dá'', disse a cozinheira.

O Rei olhou aflito para o Coelho Branco, que disse em voz baixa: ``Sua
Majestade também precisa interrogar essa testemunha''.

``Bem, se eu preciso, então, eu preciso'', o Rei disse com ar
melancólico e, depois de cruzar os braços e franzir a testa diante da
cozinheira até seus olhos ficarem quase invisíveis, ele disse com voz
grave: ``Do que são feitas as tortas?''.

``Basicamente, pimenta'', disse a cozinheira.

``Melaço'', disse uma voz sonolenta atrás dela.

``Enforquem esse Arganaz'', a Rainha berrou. ``Cortem a cabeça do
Arganaz! Tirem esse Arganaz daqui! Reprimam esse Arganaz! Belisquem!
Cortem suas suíças!''

Por alguns minutos, o tribunal ficou uma confusão total; nesse ínterim,
entre tirarem o Arganaz do recinto, até voltarem e tudo se acalmar, a
cozinheira havia desaparecido.

``Não faz mal'', disse o Rei, com ar de grande alívio. ``Chamem a
próxima testemunha.'' E ele acrescentou discretamente para a Rainha:
``De fato, querida, você precisa interrogar a próxima testemunha. Minha
testa já está doendo!''.

\textls[-15]{Alice observou o Coelho Branco repassando a lista, sentindo-se muito
curiosa para ver como seria a próxima testemunha, ``\ldots{} pois ninguém
apresentou prova nenhuma até agora'', ela disse consigo mesma. Imagine a
surpresa dela quando o Coelho Branco leu, com sua vozinha estridente, a
plenos pulmões, o nome ``Alice!''.}\looseness=-1

%Capítulo XII
\quebra\chapter{A prova de Alice}

``Aqui!'', exclamou Alice, quase esquecendo no calor do momento o tanto
que havia crescido nos últimos minutos, e ela se levantou tão depressa
que derrubou os jurados com a ponta da saia, espalhando-os no meio da
multidão que assistia, e ali ficaram os jurados espalhados,
lembrando-lhe muito o aquário do peixe dourado que acidentalmente ela
derrubara uma semana antes.

``Oh, eu sinto muito!'', ela exclamou em tom de grande desolação, e
começou a recolher os jurados, o mais depressa que conseguiu, pois o
acidente do peixe dourado continuava em seus pensamentos, e lhe ocorreu
uma espécie de ideia vaga, de que o júri devia ser recolhido logo e ser
devolvido no lugar, do contrário os jurados morreriam.

``O julgamento não pode continuar'', disse o Rei com voz muito soturna,
``enquanto todos os jurados não estiverem em sua posição correta\ldots{}
todos'', ele repetiu com grande ênfase, olhando duramente para Alice ao
dizê-lo.

Alice olhou para o júri e notou que, na pressa, pusera o Lagarto de
ponta-cabeça, e a pobre criaturazinha acenava o rabo de modo
melancólico, quase incapaz de se mover. Ela logo o tirou dali e o
deixou na posição correta; ``não que isso altere muita coisa'', ela
disse consigo mesma; ``eu diria que ele será tão útil no julgamento
assim sentado como antes virado''.

Assim que os jurados se recuperaram um pouco do choque de terem sido
virados, e suas lousinhas e seus gizes foram encontrados e devolvidos
para eles, puseram-se a trabalhar diligentemente, escrevendo uma
história do acidente, todos menos o Lagarto, que parecia abalado demais
para fazer qualquer outra coisa além de contemplar boquiaberto o teto da
corte.

``O que você sabe sobre isso?'', o Rei disse a Alice.

``Nada'', disse Alice.

``Nada mesmo?'', insistiu o Rei.

``Nada mesmo'', disse Alice.

``Isso é muito relevante'', o Rei disse, virando-se para o júri. Eles
estavam começando a escrever isso em suas lousinhas, quando o Coelho
interrompeu: ``\emph{I}rrelevante, Majestade, evidentemente'', ele disse
em tom muito respeitoso, mas franzindo a testa e fazendo caretas para
ele enquanto falava.

``\emph{I}rrelevante, claro, foi o que eu quis dizer'', o Rei
rapidamente respondeu, e prosseguiu em tom discreto, ``relevante\ldots{}
irrelevante\ldots{} irrelevante\ldots{} relevante\ldots{}'', como se estivesse
experimentando para ver qual palavra soava melhor.

Alguns jurados escreveram ``relevante'' e alguns, ``irrelevante''. Alice
conseguiu enxergar isso, pois estava perto o suficiente para ver suas
lousinhas por cima; ``mas não faz nenhuma diferença'', ela pensou
consigo mesma.

Nesse momento, o Rei, que ficara algum tempo escrevendo em seu caderno,
gritou ``Silêncio!'' e leu o que havia redigido: ``Regra Quarenta e
dois. \emph{Toda pessoa com mais de uma milha de altura deve deixar a
corte}''.

Todo mundo olhou para Alice.

``Eu não tenho uma milha de altura'', disse Alice.

``Tem, sim'', disse o Rei.

``Quase duas milhas'', agregou a Rainha.

``Bem, mas eu não vou embora'', disse Alice. ``Além do mais, essa regra
não existia: você inventou agorinha.''

``É a regra mais antiga que temos'', disse o Rei.

``Então devia ser a Número Um'', disse Alice.

O Rei ficou pálido e fechou depressa o caderno. ``Chegaram a um
veredito?'', ele disse aos jurados, em voz grave e trêmula.

``Ainda há mais provas para serem apresentadas, Majestade'', disse o
Coelho Branco, saltando, muito apressado. ``Este envelope acabou de
chegar''.

``O que está escrito?'', disse a Rainha.

``Ainda não abri'', disse o Coelho Branco, ``mas parece ser uma carta,
escrita pelo prisioneiro\ldots{} para alguém.''

``Deve ser isso'', disse o Rei, ``a não ser que tenha sido escrita para
ninguém, o que não é comum, você sabe\ldots{}''

``Está destinada a quem?'', disse um jurado.

``Não está destinada a ninguém'', disse o Coelho Branco; ``na verdade,
não há nada escrito do lado de fora.'' Ele abriu o envelope enquanto
falava e acrescentou: ``Pelo visto, não é uma carta afinal, mas um
apanhado de versos''.

``Está escrito com a caligrafia do prisioneiro?'', perguntou outro
jurado.

``Não, não está'', disse o Coelho Branco, ``e isso é a coisa mais
esquisita.'' (Os jurados olharam com expressão intrigada.)

``Ele deve ter imitado a caligrafia de alguém'', disse o Rei. (Os
jurados se entusiasmaram de novo.)

``Majestade'', disse o Valete, ``eu não escrevi isso, e ninguém pode
provar que escrevi: não há nenhum nome assinado embaixo.''

``Se você não assinou'', disse o Rei, ``isso só piora ainda mais as
coisas. Você devia ter alguma má intenção, do contrário teria assinado
com seu nome, como um homem honesto.''

Diante disso, todos os presentes aplaudiram: foi a primeira coisa
realmente inteligente que o Rei disse aquele dia.

``Isso prova que ele é culpado, evidentemente'', disse a Rainha.
``Portanto, cortem\ldots{}''

``Isso não prova coisa nenhuma!'', disse Alice. ``Ora, vocês nem sabem o
que está escrito!''

``Leia'', disse o Rei.

O Coelho Branco pôs os óculos. ``De onde eu começo, Majestade?'', ele
perguntou.

``Comece do começo'', o Rei disse gravemente, ``e siga em frente até
chegar no fim; então pare.''

Fez-se um silêncio mortal na corte, enquanto o Coelho Branco leu os
seguintes versos:

\begin{quote}
Disseram que você esteve com ela,\\
E mencionou meu nome a ele:\\
Ela me deu um bom papel,\\
Mas disse que eu não sabia nadar.
\medskip
Ele os avisou de que eu não tinha ido,\\
(Sabemos que isso é verdade):\\
Se ela insistir com esse assunto,\\
O que será de você?
\medskip
Dei a ela uma, eles a ele deram duas,\\
Você nos deu três ou mais;\\
Todas voltaram dele para você,\\
Embora antes fossem minhas.
\medskip
Se eu ou ela corremos risco\\
De envolvimento nesse caso,\\
Ele espera que você as livre,\\
Exatamente como éramos.
\medskip
Minha opinião era que você havia sido\\
(Antes que ela tivesse esse surto)\\
Um obstáculo interposto entre\\
Ele, nós, e isto.
\medskip
Não o deixe saber que ela gostava mais delas,\\
Pois isso deverá sempre ser\\
Um segredo, ignorado por todo o resto,\\
Entre mim e você.
\end{quote}

``Essa é a prova mais relevante que ouvimos até agora'', disse o Rei,
esfregando as mãos; ``então agora deixemos o júri\ldots{}''

``Se algum jurado puder explicar isso'', disse Alice (ela ficara tão
grande nos últimos minutos que não teve medo nenhum de interrompê-lo),
``eu lhe darei uma moeda. Não acredito que haja um pingo de sentido
nisso.''

Os jurados escreveram em suas lousinhas ``Ela não acredita que haja um
pingo de sentido nisso'', mas nenhum deles tentou explicar o que estava
escrito no papel.

``Se não há sentido nisso'', disse o Rei, ``isso nos poupa um grande
trabalho, você sabe, pois não precisamos nem tentar procurar nenhum
sentido nisso. E no entanto eu não sei'', ele prosseguiu, abrindo o
papel com os versos sobre seu joelho, e olhando para eles com um olho
só; ``Acho que enxergo algum sentido nisso afinal. `Disse que eu não
sabia nadar\ldots{}', você não sabe nadar, não é?'', ele agregou, virando-se
para o Valete.

O Valete balançou tristonho a cabeça. ``O que você acha?'', ele disse.
(Certamente ele não nadava, sendo feito de papel-cartão.)

``Muito bem, até aqui'', disse o Rei, prosseguindo, resmungando para si
mesmo os versos: ```Sabemos que isso é verdade'\ldots{} isso se refere ao
júri, evidentemente\ldots{} `Se ela insistir com esse assunto'\ldots{} isso deve
se referir à Rainha\ldots{} `O que será de você?'\ldots{} Ora, de fato! `Dei a ela
uma, eles a ele deram duas', ora, isso deve ser o que ele fez com as
tortas, você sabe\ldots{}''

``Mas depois diz `todas voltaram dele para você','' disse Alice.

``Ora, elas estão aí!'', disse o Rei triunfantemente, apontando para as
tortas na mesa. ``Nada pode ser mais claro do que isso. Mas,
novamente\ldots{} `antes que ela tivesse esse surto'\ldots{} você nunca teve
surtos, querida, não é?'', ele disse à Rainha.

``Jamais!'', disse a Rainha furiosamente, atirando um tinteiro no
Lagarto ao falar. (O infeliz do pequeno Bill parara de escrever com o
dedo na lousinha, pois descobrira que não deixava marcas; mas agora
rapidamente começou de novo, usando a tinta, que escorria em seu rosto,
até acabar a tinta.)

``Então as palavras não \emph{surtiram} efeito em vocês?'', disse o Rei,
olhando para os presentes, com um sorriso. Fez-se um silêncio mortal.

``É um trocadilho!'', o Rei acrescentou em tom irritado, e todos deram
risada.

``Que o júri chegue ao veredito logo'', o Rei disse, talvez pela
vigésima vez naquele dia.

``Não, não!'', disse a Rainha. ``Primeiro a sentença, depois o
veredito.''

``Mas que absurdo!'', disse Alice em voz alta. ``Onde já se viu
`primeiro a sentença'?!''

``Controle sua língua!'', disse a Rainha, ficando roxa.

``Não vou!'', disse Alice.

``Cortem-lhe a cabeça!'', a Rainha berrou a plenos pulmões. Ninguém se
mexeu.

``Quem tem medo de vocês?'', disse Alice (a essa altura ela atingira sua
altura máxima). ``Vocês não passam de cartas de baralho!''

Nesse momento, o baralho inteiro se ergueu no ar e veio voando para
cima dela: ela soltou um gritinho, um tanto assustado, um tanto furioso,
e tentou espantá-los aos tapas, e se viu deitada na margem do rio, com a
cabeça no colo da irmã, que delicadamente tirava algumas folhas secas
que haviam caído das árvores sobre seu rosto.

``Acorde, Alice querida!'', disse a irmã. ``Ora, você dormiu bastante!''

``Oh, tive um sonho tão curioso!'', disse Alice, e ela contou à irmã, da
melhor forma de que se lembrava, todas essas estranhas Aventuras que você
acabou de ler; e, quando terminou, a irmã lhe deu um beijo e disse:
``Certamente, querida, foi um sonho curioso, mas agora vá correndo lá
para dentro para tomar o seu chá; está ficando tarde''. Então Alice se
levantou e foi embora correndo, pensando enquanto corria, da melhor
forma que podia, que sonho maravilhoso havia sido.

Mas a irmã continuou ali sentada, desde o momento em que ela saiu
correndo, com a cabeça inclinada para o lado, assistindo ao pôr do sol
e pensando na pequena Alice e em todas as suas maravilhosas Aventuras,
até que ela também começou a sonhar de certa forma, e eis o sonho que
ela teve:

Primeiro, ela sonhou com a própria pequena Alice, e mais uma vez as
mãozinhas minúsculas se agarraram ao seu joelho, e aqueles olhos grandes
e brilhantes estavam olhando para ela\ldots{} ela podia ouvir até o tom da
voz dela, e aquele gesto engraçado de tirar seus cabelos soltos que
sempre caíam sobre os olhos\ldots{} e, mesmo imóvel como ela estava, ouvindo,
ou parecendo ouvir, o lugar inteiro à sua volta se tornou vivo com as
estranhas criaturas do sonho de sua irmãzinha.

A grama alta farfalhou aos seus pés quando o Coelho Branco passou
apressado\ldots{} o assustadiço Camundongo chapinhando atravessava a piscina
vizinha com estardalhaço\ldots{} ela podia ouvir o chocalhar das xícaras de
chá, conforme a Lebre de Março e seus amigos faziam sua interminável
refeição, e a voz estridente da Rainha mandando executar os infelizes
convidados\ldots{} mais uma vez o bebê-porco estava espirrando no colo da
Duquesa, enquanto pratos e bandejas se estilhaçavam com estrondo\ldots{} mais
uma vez o grito do Grifo, o rangido do giz na lousinha do Lagarto, e os
engasgos dos porquinhos-da-índia reprimidos encheram o ar, mesclados
aos soluços distantes da angustiada Falsa Tartaruga.

%País das Maravilhas?
De modo que ela ficou sentada de olhos fechados, acreditando em parte
também no Reino da Maravilha, embora sabendo que precisaria abri-los em
algum momento, e que tudo se transformaria em uma realidade banal\ldots{} a
grama só farfalhando por causa do vento, e a piscina só agitada por
causa do balanço dos juncos\ldots{} o chocalhar das xícaras de chá se
transformando no som dos badalos das ovelhas, e os gritos estridentes da
Rainha nos chamados do pastorzinho\ldots{} e os espirros do bebê, o grito do
Grifo, e todos os outros barulhos esquisitos se transformariam (ela
sabia) no clamor confuso dos trabalhadores no sítio --- enquanto os
mugidos do gado ao longe tomariam o lugar dos pesados soluços da Falsa
Tartaruga.

Por fim, ela imaginou como sua irmãzinha ficaria, depois de algum tempo,
quando se tornasse uma mulher adulta; e como ela conservaria, em seus
anos mais maduros, o coração simples e amoroso de sua infância; e como
ela reuniria ao seu redor suas próprias criancinhas e deixaria os olhos
delas arregalados e brilhantes com muitas histórias estranhas, talvez
até mesmo com o sonho do Reino da Maravilha de muito tempo atrás; e como
ela sentiria suas tristezas singelas e encontraria prazer em suas
singelas alegrias, lembrando-se de sua própria vida de criança e
daqueles dias felizes de verão.
%País das Maravilhas?

% \begin{center}
% \textsc{fim}
% \end{center}