\hyphenation{Todorov}

\blankpage
\chapter*{Apresentação\smallskip\subtitulo{Alice acha feio o que não é espelho}}
\addcontentsline{toc}{chapter}{Apresentação, \textit{por Ricardo Ramos Filho}}

\vspace*{-.5\baselineskip}

\begin{flushright}
\textsc{ricardo ramos filho}
\end{flushright}

\noindent A Alice de hoje, menina criada em contato com as redes sociais e,
portanto, totalmente entregue ao campo narcísico, muito mais do que ao
das ideias, talvez sentisse falta de registrar alguma das imagens de
suas aventuras às voltas com espelhos, adivinhações, criaturas
peculiares e antropomórficas, lógicas do absurdo, sonhos. Afinal, o
passeio de barco pelo rio Tâmisa em 4 de julho de 1862, que proporcionou
a Lewis Carroll (pseudônimo de Charles Lutwidge Dodgson) a oportunidade
de contar a história a ser publicada em livro mais tarde, em 1865,
mereceria algumas \emph{selfies}. As irmãs Lorine Charlotte, Edith Mary
e Alice Pleasance Liddell quietinhas, ouvindo. O barco, o arco, meu
coração\ldots{}

\begin{quote}
Alice estava começando a ficar muito cansada de ficar sentada ao
lado da irmã na margem do rio, e de não ter nada para fazer: uma ou duas
vezes, ela havia espiado no livro que a irmã estava lendo, mas não tinha
nenhuma figura, nem diálogos, ``e para que serve um livro'', pensou
Alice, ``sem figuras, nem diálogos?''\footnote{Ver p.\,\pageref{ref1} desta edição.}
\end{quote}

\textls[-10]{Assim começa \emph{Alice no país das maravilhas}. De imediato uma
observação importante do autor. A criança busca nos textos infantis
principalmente ilustrações e diálogos. Carroll obedeceu. Trouxe para o
universo maravilhoso uma prosa ágil. Evidente intenção de fisgar o
leitor mirim com maior espontaneidade. Desde a queda na toca do coelho,
e a decisão de segui-lo, o texto verbal, e as magníficas ilustrações de
John Tenniel, em texto visual paralelo, constroem um universo capaz de
atiçar a curiosidade. Ninguém permaneceria imune ao convite de mergulhar
nas páginas da obra. Palavras e estampas seguindo juntas, construindo a
fabulação. Aquele texto repleto de enigmas e situações alucinantes,
viagem pelo imaginativo, ganhando força também no traço do artista. O~absurdo em cada situação criada. \emph{Nonsense}. Em certo sentido, a
obra é moderna desde o começo. Meu coração não aguenta\ldots{}}\looseness=-1

Tenniel notabilizou-se pelos cartuns produzidos para a revista
\emph{Punch}, ou \emph{The London Charivari}, semanário britânico de
humor e sátira, que começou a circular em 1841, e teve seu auge na
década de 1940. Bem de acordo com o significado da denominação da
publicação, \textit{socava} com suas imagens também mordazes e radicais,
colocando em foco as importantes mudanças políticas e sociais vividas
pelo país, e refletindo as tensões acaloradas da época. Lewis Carroll
era leitor regular da publicação e resolveu apostar no talento de John
Tenniel. Assim, deu oportunidade ao artista de retratar, na época
vitoriana, a imaginação da garota mais conhecida da literatura mundial,
a descobridora do universo onírico do Coelho Branco.

Exatamente três anos após ser contada, em 4 de julho de 1865, aquela
história nascida da tradição oral seria publicada na forma como é
conhecida hoje. E aí somos colocados em contato com um acontecimento
singular, revelador do papel protagonista assumido por Tenniel na
realização da obra. A tiragem inicial de dois mil exemplares seria
removida das prateleiras devido à não aprovação do ilustrador. Ele havia
ficado insatisfeito com a qualidade da impressão. Navegar é preciso\ldots{}

Peter Hunt, uma das autoridades em matéria do estudo da literatura
infantil, ensina que uma das razões para a crítica dos livros com
ilustrações ser econômica na observação do texto visual, tendendo,
muitas vezes, a recorrer a chavões figurativos, não seria por considerar
o trabalho deficiente. A má vontade surgiria não como fruto de uma
avaliação séria, e sim pelo fato de os críticos considerarem imagens a
parte menos importante do universo do livro. Dentro deste cenário
\textit{intolerante} desde sempre, ganha força o fato de um ilustrador
conseguir ter tanta influência e adiar a publicação de uma obra.
\emph{Alice no país das maravilhas}, história escrita para crianças, já
em seu berço revelava a disposição de aproximar escritor e ilustrador. O
porto, não\ldots{}

Apenas em dezembro de 1865, embora conste como publicada em 1866, a
segunda edição chegaria às livrarias. Sabe-se ter sido lida por Oscar
Wilde e pela rainha Vitória. Se, para Todorov, o fantástico é a
hesitação experimentada por alguém face a um acontecimento aparentemente
sobrenatural, podemos tentar imaginar a leitura da soberana. Ela ante a
\textit{irrealidade da realidade}. Ao observar as rosas brancas sendo pintadas
de vermelho, talvez considerasse ser a melhor forma de resolver os
problemas do país. Os soldados cartas de um baralho. Marcadas. Marcados.
Ou cortar cabeças. O argumento da Rainha. Iria mandar cortar a cabeça de
todo mundo. Wilde certamente gostou. Do livro.

\begin{quote}
{[}\ldots{}{]} mas quando o Coelho
realmente \emph{tirou um relógio do bolso do colete}, e olhou para as
horas, e então se apressou, Alice se pôs de pé, pois lhe passou pela
cabeça que nunca tinha visto nem coelho de colete, nem relógio guardado
no bolso, e ardendo de curiosidade, ela correu pelo campo atrás dele, e
chegou a tempo de vê-lo pular dentro de uma grande toca embaixo da
cerca-viva.\footnote{Ver p.\,\pageref{ref2} desta edição.}
\end{quote}

\textls[-5]{E a curiosidade moveu, catalisadora, a história. Se até aquele momento
Alice bocejava aborrecida, estendida no jardim, criticando a leitura
tediosa da irmã, o livro sem ilustrações e figuras, havia agora
alternativa. Ela poderia substituir o ramerrão insosso daquele cotidiano
pelo desafio da aventura, a maravilha do mundo de cabeça para baixo, o
sorriso do gato de Cheshire. Quem sabe a Monalisa de Da Vinci seria uma
gata Cheshire? Surpresa!}\looseness=-1

Mas \emph{Alice no país das maravilhas} é também um livro para adultos.
Como pode ser fruída por diversos públicos leitores, a obra coloca em
perspectiva a possibilidade de trazer \textit{infinitas} indagações a ela
própria. Abre-se para um leque variado de possibilidades, vai além do
que deve ter imaginado Lewis Carroll, até por ser muito difícil
descobrir quais as intenções originais do autor. Podemos, e não cansamos
de fazê-lo, conjecturar, investigar as pistas que o texto fornece, mas
estaremos sempre tangenciando certezas. Por maior que seja nossa
liberdade, estaremos invariavelmente presos, já que os processos de
leitura e interpretação não podem pressupor uma análise pré-definida e
estruturada do texto.

Como leitores, e receptores, damos vazão à nossa
tarefa pessoal de extração de significados que a leitura fornece. A
noção de \emph{obra aberta}, termo criado por Umberto Eco, advém
da necessidade, cada vez mais patente, de se compreender e valorizar a
capacidade criativa e interpretativa que conduz, sempre que necessário,
a uma reestruturação do pensamento. Embora estejamos diante de um
clássico da literatura universal, ele, como produção literária, não se
encontra de todo acabado em si mesmo. Inexiste plenamente definido
enquanto estrutura finita.
Ao contrário, possibilita diversas interpretações e reformulações.
Assume-se, então, uma nova dialética entre a obra e o intérprete, já que
a primeira, \emph{fechada}, no sentido de \emph{concluída},
acaba por ser igualmente uma obra \emph{aberta}, ou seja, passível de
sugerir interpretações bastante diversificadas.

De todo este intrincado
processo, surge o que nos parece a grande beleza da leitura de
\emph{Alice no país das maravilhas}. Derrida, embora conceituando de
forma genérica, e não especificamente pensando no clássico inglês, bem
formulou a questão ao afirmar que o sentido de um texto surge ao mesmo
tempo em que vamos lendo as palavras, havendo complementaridade entre
criação e recepção. O intérprete, o leitor, vai \textit{descobrindo}
dinamicamente a obra de acordo com a sua própria personalidade,
interesses, experiências pessoais cotidianas. Tal entendimento está
intrinsecamente relacionado, não podemos esquecer, à cultura dele.
Assim, a inglesinha \textit{Álice} transforma-se na brasileiríssima
\textit{Alíce}. Tanta tormenta, alegria\ldots{}

O \emph{nonsense} de Lewis Carroll é muitas vezes categorizado como a
forma de maior relevância de um gênero literário inaugurado por Edward
Lear (1812--1888), seu contemporâneo. Este ilustrador e poeta
notabilizou-se por seus \emph{limericks}, denominados assim por terem
origem na cidade irlandesa de Limerick. São poemas de quatro a cinco
versos, acompanhados por uma ilustração, caracterizados por um desfecho
mais forte chamado \emph{punch line}. A~aproximação de Carroll com Lear
torna-se quase que automática quando nos lembramos que \emph{Alice no
país das maravilhas} é uma história \emph{nonsense}, ilustrada por John
Tenniel, artista que atuava em uma revista chamada \emph{Punch}. Muito
provavelmente Carroll tinha tudo isso em conta na hora de escolher
aquele que deu as primeiras feições à Alice. O trilho solto, o
barulho\ldots{}

\begin{quote}
Logo seus olhos depararam com uma caixinha de vidro que estava embaixo \label{ref3}
da mesa: ela abriu, e encontrou um bolinho muito pequeno, no qual estava
escrito, lindamente composto com uvas passas, ``coma-me''. ``Bem coma'',
disse Alice, ``e se isso me fizer crescer, posso alcançar a chave; e se
me fizer diminuir, poderei me arrastar por baixo da porta; de modo que
seja como for chegarei no jardim, e não me importa o que vai
acontecer!''\footnote{Ver p.\,\pageref{ref3} desta edição.}
\end{quote}

Ainda tecendo considerações sobre a falta de sentido, \emph{nonsense},
de \emph{Alice no país das maravilhas}, não podemos nos distanciar da
noção segundo a qual o \emph{nonsense carrolliano} não é avesso ao
significado, já que nasce justamente do exercício da produção de novos
sentidos --- possível graças aos seus criativos universos de ficção. A
argumentação encontrada no excerto acima é logicamente correta.
Evidencia toda a coerência racional existente no pensamento de Alice,
mesmo quando aplicado às suas interações absurdas com objetos do
universo \emph{nonsense}. E é justamente a partir da possibilidade de
novos sentidos, respectivos a outros sistemas referenciais, que somos
apresentados às situações mais engraçadas da literatura \emph{nonsense}
\textit{carrolliana}: muitas vezes fruto da ambiguidade de um termo, como a
intenção do Rato (Camundongo) de secar a todos com uma conversa seca, ou
árida.

\begin{quote}
Por fim, o Camundongo, que parecia ter alguma autoridade entre eles,
exclamou: ``Sentem-se, todos vocês, e me escutem! Deixem comigo, vocês
rapidamente ficarão secos''. Todos eles se sentaram ao mesmo tempo, em
um grande círculo, com o Camundongo no meio. Alice ficou olhando fixa e
ansiosamente para ele, pois tinha certeza de que pegaria um resfriado se
não se secasse logo.

``Aham!'', disse o Camundongo com um ar importante. ``Vocês estão
prontos? Essa é a coisa mais árida que eu conheço. Silêncio, todos
vocês, por favor! (\ldots)''.\footnote{Ver p.\,\pageref{ref4} desta edição.}
\end{quote}

Crianças sentem-se bem em locais seguros. Não era o caso de nossa
heroína. Ao contrário da personagem Coelho Branco, aparentemente
atrasado e às voltas com seu relógio, tomado o tempo todo pelo medo, a
menina era corajosa. Se, para a pesquisadora Nelly Novaes Coelho, é
importante haver nas histórias infantis a presença de um \emph{locus
amoenus}, lugar \emph{ideal} onde não há desarmonia nem desequilíbrios
em parte alguma, espaço gratificante, de natureza acolhedora, o mundo
aqui criado por Carroll foge a esta regra. Difere de um \emph{Sítio do
Picapau Amarelo} de Monteiro Lobato, Tatipirun, de \emph{A terra dos
meninos pelados}, de Graciliano Ramos, do \emph{Sítio de Taquara-Póca}
de Francisco Marins, ou mesmo de Hogwarts, a escola onde o bruxo Harry
Potter estudou, na série de livros escritos por J.\,K.\,Rowling,
conterrânea do pai de Alice.\looseness=-1

Mas como seria em termos de ambiente o universo criado por Lewis
Carroll, o \emph{locus} onde literalmente despenca Alice? Certamente bem
distante do termo \emph{amoenus}. Nada \emph{cozy}, para usarmos um
termo tão caro aos ingleses. Não aconchegante, ou acolhedor. Quem
entraria no país das maravilhas atrás do Coelho Branco? A curiosidade
pode fazer uma história começar. E coragem. Ora seríamos enormes, ora
muito pequenos. Loucos. O que é ser normal? O medo não deve sobrepor-se
aos sonhos. Quanto dura a eternidade? Apenas um segundo. \emph{Alice no
país das maravilhas}. Porque também sou \emph{nonsense}, e desejo ser
\emph{nonsense}. Estico o braço. Sorrio como um gato de Cheshire.
\emph{Selfie}. Navegar é preciso, viver não é preciso.

%\emph{15/06/2022}


\pagebreak
\thispagestyle{empty}
\movetooddpage
\chapter*{}
\addcontentsline{toc}{part}{Alice no país das maravilhas}
\begin{center}
\begin{vplace}[0.3]
\thispagestyle{empty}
\huge\formular\bfseries
Alice no país das maravilhas\\\bigskip
Alice's Adventures in Wonderland
\end{vplace}
\end{center}
\thispagestyle{empty}
