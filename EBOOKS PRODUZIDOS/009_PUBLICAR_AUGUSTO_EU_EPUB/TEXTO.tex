
%%% 1912

\clearpage

\vspace*{.6\textheight}

{\raggedleft\itshape
\textsc{À memória de meu pai}

\medskip

À minha Mãe --- Córdula C. R. dos Anjos

À minha Mulher --- Ester Fialho R. dos Anjos

À minha filhinha --- Glória

\medskip

Aos meus irmãos
\par}

\cleardoublepage


\oneside

\makeatletter
\renewcommand\section{\@startsection%
   {section}{1}%
   {\z@}% indent
   {3.5ex \@plus1ex \@minus.5ex}% beforeskip (se < 0, sem indentação na sequência)
   {.2ex}% afterskip (se < 0, run-in head)
   {\linespread{0.8}\centering\normalfont\Large\scshape\MakeTextLowercase}% estilo
}
\makeatother

\chapter{Monólogo de uma sombra}


“Sou uma Sombra! Venho de outras eras,&
Do cosmopolitismo das moneras\ldots{}&
Pólipo de recônditas reentrâncias,&
Larva de caos telúrico, procedo&
Da escuridão do cósmico segredo,&
Da substância de todas as substâncias!

A simbiose das coisas me equilibra.&
Em minha ignota mônada, ampla, vibra&
A alma dos movimentos rotatórios\ldots{}&
E é de mim que decorrem, simultâneas,&
A saúde das forças subterrâneas&
E a morbidez dos seres ilusórios!

Pairando acima dos mundanos tetos,&
Não conheço o acidente da \textit{Senectus}&
--- Esta universitária sanguessuga&
Que produz, sem dispêndio algum de vírus,&
O amarelecimento do papírus&
E a miséria anatômica da ruga!

Na existência social, possuo uma arma&
--- O metafisicismo de Abidharma ---&
E trago, sem bramânicas tesouras,&
Como um dorso de azêmola passiva,&
A solidariedade subjetiva&
De todas as espécies sofredoras.

Com um pouco de saliva quotidiana&
Mostro meu nojo à Natureza Humana.&
A podridão me serve de Evangelho\ldots{}&
Amo o esterco, os resíduos ruins dos quiosques&
E o animal inferior que urra nos bosques&
É com certeza meu irmão mais velho!

Tal qual quem para o próprio túmulo olha,&
Amarguradamente se me antolha,&
À luz do americano plenilúnio,&
Na alma crepuscular de minha raça&
Como uma vocação para a Desgraça&
E um tropismo ancestral para o Infortúnio.

Aí vem sujo, a coçar chagas plebeias,&
Trazendo no deserto das ideias&
O desespero endêmico do inferno,&
Com a cara hirta, tatuada de fuligens&
Esse mineiro doido das origens,&
Que se chama o Filósofo Moderno!

Quis compreender, quebrando estéreis normas,&
A vida fenomênica das Formas,&
Que, iguais a fogos passageiros, luzem\ldots{}&
E apenas encontrou na ideia gasta,&
O horror dessa mecânica nefasta,&
A que todas as coisas se reduzem!

E hão de achá-lo, amanhã, bestas agrestes,&
Sobre a esteira sarcófaga das pestes&
A mostrar, já nos últimos momentos,&
Como quem se submete a uma charqueada,&
Ao clarão tropical da luz danada,&
O espólio dos seus dedos peçonhentos.

Tal a finalidade dos estames!&
Mas ele viverá, rotos os liames&
Dessa estranguladora lei que aperta&
Todos os agregados perecíveis,&
Nas eterizações indefiníveis&
Da energia intra-atômica liberta!

Será calor, causa ubíqua de gozo,&
Raio \textsc{x}, magnetismo misterioso,&
Quimiotaxia, ondulação aérea,&
Fonte de repulsões e de prazeres,&
Sonoridade potencial dos seres,&
Estrangulada dentro da matéria!

E o que ele foi: clavículas, abdômen,&
O coração, a boca, em síntese, o Homem,&
--- Engrenagem de vísceras vulgares ---&
Os dedos carregados de peçonha,&
Tudo coube na lógica medonha&
Dos apodrecimentos musculares.

A desarrumação dos intestinos&
Assombra! Vede-a! Os vermes assassinos&
Dentro daquela massa que o húmus come,&
Numa glutoneria hedionda, brincam,&
Como as cadelas que as dentuças trincam&
No espasmo fisiológico da fome.

É uma trágica festa emocionante!&
A bacteriologia inventariante&
Toma conta do corpo que apodrece\ldots{}&
E até os membros da família engulham,&
Vendo as larvas malignas que se embrulham&
No cadáver malsão, fazendo um \textit{s}.

E foi então para isto que esse doudo&
Estragou o vibrátil plasma todo,&
À guisa de um faquir, pelos cenóbios?!\ldots{}&
Num suicídio graduado, consumir-se,&
E após tantas vigílias, reduzir-se&
À herança miserável de micróbios!

Estoutro agora é o sátiro peralta&
Que o sensualismo sodomita exalta,&
Nutrindo sua infâmia a leite e a trigo\ldots{}&
Como que, em suas células vilíssimas,&
Há estratificações requintadíssimas&
De uma animalidade sem castigo.

Brancas bacantes bêbedas o beijam.&
Suas artérias hírcicas latejam,&
Sentindo o odor das carnações abstêmias,&
E à noite, vai gozar, ébrio de vício,&
No sombrio bazar do meretrício,&
O cuspo afrodisíaco das fêmeas.

No horror de sua anômala nevrose,&
Toda a sensualidade da simbiose,&
Uivando, à noite, em lúbricos arroubos,&
Como no babilônico \textit{sansara},&
Lembra a fome incoercível que escancara&
A mucosa carnívora dos lobos.

Sôfrego, o monstro as vítimas aguarda.&
Negra paixão congênita, bastarda,&
Do seu zooplasma ofídico resulta\ldots{}&
E explode, igual à luz que o ar acomete,&
Com a veemência mavórtica do ariete&
E os arremessos de uma catapulta.

Mas muitas vezes, quando a noite avança,&
Hirto, observa através a tênue trança&
Dos filamentos fluídicos de um halo&
A destra descarnada de um duende,&
Que, tateando nas tênebras, se estende&
Dentro da noite má, para agarrá-lo!

Cresce-lhe a intracefálica tortura,&
E de su’alma na caverna escura,&
Fazendo ultraepiléticos esforços,&
Acorda, com os candeeiros apagados,&
Numa coreografia de danados,&
A família alarmada dos remorsos.

É o despertar de um povo subterrâneo!&
É a fauna cavernícola do crânio&
--- Macbeths da patológica vigília,&
Mostrando, em rembrandtescas telas várias,&
As incestuosidades sanguinárias&
Que ele tem praticado na família.

As alucinações táteis pululam.&
Sente que megatérios o estrangulam\ldots{}&
A asa negra das moscas o horroriza;&
E autopsiando a amaríssima existência&
Encontra um cancro assíduo na consciência&
E três manchas de sangue na camisa!

Mingua-se o combustível da lanterna&
E a consciência do sátiro se inferna,&
Reconhecendo, bêbedo de sono,&
Na própria ânsia dionísica do gozo,&
Essa necessidade de \textit{horroroso},&
Que é talvez propriedade do carbono!

Ah! Dentro de toda a alma existe a prova&
De que a dor como um dartro se renova,&
Quando o prazer barbaramente a ataca\ldots{}&
Assim também, observa a ciência crua,&
Dentro da elipse ignívoma da lua&
A realidade de uma esfera opaca.

Somente a Arte, esculpindo a humana mágoa,&
Abranda as rochas rígidas, torna água&
Todo o fogo telúrico profundo&
E reduz, sem que, entanto, a desintegre,&
À condição de uma planície alegre,&
A aspereza orográfica do mundo!

Provo desta maneira ao mundo odiento&
Pelas grandes razões do sentimento,&
Sem os métodos da abstrusa ciência fria&
E os trovões gritadores da dialética,&
Que a mais alta expressão da dor estética&
Consiste essencialmente na alegria.

Continua o martírio das criaturas:&
--- O homicídio nas vielas mais escuras,&
--- O ferido que a hostil gleba atra escarva,&
--- O último solilóquio dos suicidas ---&
E eu sinto a dor de todas essas vidas&
Em minha vida anônima de larva!”

Disse isto a Sombra. E, ouvindo estes vocábulos,&
Da luz da lua aos pálidos venábulos,&
Na ânsia de um nervosíssimo entusiasmo,&
Julgava ouvir monótonas corujas,&
Executando, entre caveiras sujas,&
A orquestra arrepiadora do sarcasmo!

Era a elegia panteísta do Universo,&
Na podridão do sangue humano imerso,&
Prostituído talvez, em suas bases\ldots{}&
Era a canção da Natureza exausta,&
Chorando e rindo na ironia infausta&
Da incoerência infernal daquelas frases.

E o turbilhão de tais fonemas acres&
Trovejando grandíloquos massacres,&
Há de ferir-me as auditivas portas,&
Até que minha efêmera cabeça&
Reverta à quietação da treva espessa&
E à palidez das fotosferas mortas!



\chapter{Agonia de um filósofo}


Consulto o Phtah-Hotep. Leio o obsoleto&
Rig-Veda. E, ante obras tais, me não consolo\ldots{}&
O Inconsciente me assombra e eu nele rolo&
Com a eólica fúria do harmatã inquieto!

Assisto agora à morte de um inseto!\ldots{}&
Ah! todos os fenômenos do solo&
Parecem realizar de polo a polo&
O ideal de Anaximandro de Mileto!

No hierático areópago heterogêneo&
Das ideias, percorro como um gênio&
Desde a alma de Haeckel à alma cenobial!\ldots{}

Rasgo dos mundos o velário espesso;&
E em tudo, igual a Gœthe, reconheço&
O império da \textit{substância universal}!



\chapter{O morcego}


Meia-noite. Ao meu quarto me recolho.&
Meu Deus! E este morcego! E, agora, vede:&
Na bruta ardência orgânica da sede,&
Morde-me a goela ígneo e escaldante molho.

“Vou mandar levantar outra parede\ldots{}”&
--- Digo. Ergo-me a tremer. Fecho o ferrolho&
E olho o teto. E vejo-o ainda, igual a um olho,&
Circularmente sobre a minha rede!

Pego de um pau. Esforços faço. Chego&
A tocá-lo. Minh’alma se concentra.&
Que ventre produziu tão feio parto?!

A Consciência Humana é este morcego!&
Por mais que a gente faça, à noite, ele entra&
Imperceptivelmente em nosso quarto!



\chapter{Psicologia de um vencido}


Eu, filho do carbono e do amoníaco,&
Monstro de escuridão e rutilância,&
Sofro, desde a epigênesis da infância,&
A influência má dos signos do zodíaco.

Profundissimamente hipocondríaco,&
Este ambiente me causa repugnância\ldots{}&
Sobe-me à boca uma ânsia análoga à ânsia&
Que se escapa da boca de um cardíaco.

Já o verme --- este operário das ruínas ---&
Que o sangue podre das carnificinas&
Come, e à vida em geral declara guerra,

Anda a espreitar meus olhos para roê-los,&
E há de deixar-me apenas os cabelos,&
Na frialdade inorgânica da terra!



\chapter{A ideia}


De onde ela vem?! De que matéria bruta&
Vem essa luz que sobre as nebulosas&
Cai de incógnitas criptas misteriosas&
Como as estalactites duma gruta?!

Vem da psicogenética e alta luta&
Do feixe de moléculas nervosas,&
Que, em desintegrações maravilhosas,&
Delibera, e depois, quer e executa!

Vem do encéfalo absconso que a constringe,&
Chega em seguida às cordas do laringe,&
Tísica, tênue, mínima, raquítica\ldots{}

Quebra a força centrípeta que a amarra,&
Mas, de repente, e quase morta, esbarra&
No molambo da língua paralítica!



\chapter{O lázaro da pátria}


Filho podre de antigos Goitacases,&
Em qualquer parte onde a cabeça ponha,&
Deixa circunferências de peçonha,&
Marcas oriundas de úlceras e antrazes.

Todos os cinocéfalos vorazes&
Cheiram seu corpo. À noite, quando sonha,&
Sente no tórax a pressão medonha&
Do bruto embate férreo das tenazes.

Mostra aos montes e aos rígidos rochedos&
A hedionda elefantíase dos dedos\ldots{}&
Há um cansaço no Cosmos\ldots{} Anoitece.

Riem as meretrizes no Cassino,&
E o Lázaro caminha em seu destino&
Para um fim que ele mesmo desconhece!



\chapter[Idealização da humanidade futura]{Idealização da\break humanidade futura}


Rugia nos meus centros cerebrais&
A multidão dos séculos futuros&
--- Homens que a herança de ímpetos impuros&
Tornara etnicamente irracionais! ---

Não sei que livro, em letras garrafais,&
Meus olhos liam! No húmus dos monturos,&
Realizavam-se os partos mais obscuros,&
Dentre as genealogias animais!

Como quem esmigalha protozoários&
Meti todos os dedos mercenários&
Na consciência daquela multidão\ldots{}

E, em vez de achar a luz que os Céus inflama,&
Somente achei moléculas de lama&
E a mosca alegre da putrefação!

\chapter{Soneto}

Ao meu primeiro filho nascido&
morto com 7 meses incompletos.&
2 fevereiro 1911.

Agregado infeliz de sangue e cal,&
Fruto rubro de carne agonizante,&
Filho da grande força fecundante&
De minha brônzea trama neuronial,

Que poder embriológico fatal&
Destruiu, com a sinergia de um gigante,&
Em tua \textit{morfogênese} de infante&
A minha \textit{morfogênese} ancestral?!

Porção de minha plásmica substância,&
Em que lugar irás passar a infância,&
Tragicamente anônimo, a feder?!\ldots{}

Ah! Possas tu dormir, feto esquecido,&
Panteisticamente dissolvido&
Na \textit{noumenalidade} do \textsc{não ser}!



\chapter{Versos a um cão}


Que força pôde, adstrita a embriões informes,&
Tua garganta estúpida arrancar&
Do segredo da célula ovular&
Para latir nas solidões enormes?!

Esta obnóxia inconsciência, em que tu dormes,&
Suficientíssima é, para provar&
A incógnita alma, avoenga e elementar&
Dos teus antepassados vermiformes.

Cão! --- Alma de inferior rapsodo errante!&
Resigna-a, ampara-a, arrima-a, afaga-a, acode-a&
A escala dos latidos ancestrais\ldots{}

E irá assim, pelos séculos, adiante,&
Latindo a esquisitíssima prosódia&
Da angústia hereditária dos teus pais!



\chapter{O deus-verme}


Fator universal do transformismo.&
Filho da teleológica matéria,&
Na superabundância ou na miséria,&
\textit{Verme} --- é o seu nome obscuro de batismo.

Jamais emprega o acérrimo exorcismo&
Em sua diária ocupação funérea,&
E vive em contubérnio com a bactéria,&
Livre das roupas do antropomorfismo.

Almoça a podridão das drupas agras,&
Janta hidrópicos, rói vísceras magras&
E dos defuntos novos incha a mão\ldots{}

Ah! Para ele é que a carne podre fica,&
E no inventário da matéria rica&
Cabe aos seus filhos a maior porção!



\chapter{Debaixo do tamarindo}


No tempo de meu Pai, sob estes galhos,&
Como uma vela fúnebre de cera,&
Chorei bilhões de vezes com a canseira&
De inexorabilíssimos trabalhos!

Hoje, esta árvore, de amplos agasalhos,&
Guarda, como uma caixa derradeira,&
O passado da Flora Brasileira&
E a paleontologia dos Carvalhos!

Quando pararem todos os relógios&
De minha vida, e a voz dos necrológios&
Gritar nos noticiários que eu morri,

Voltando à pátria da homogeneidade,&
Abraçada com a própria Eternidade&
A minha sombra há de ficar aqui!



\chapter{As cismas do destino}

### I


Recife, Ponte Buarque de Macedo.&
Eu, indo em direção à casa do Agra,&
Assombrado com a minha sombra magra,&
Pensava no Destino, e tinha medo!

Na austera abóbada alta o fósforo alvo&
Das estrelas luzia\ldots{} O calçamento&
Sáxeo, de asfalto rijo, atro e vidrento,&
Copiava a polidez de um crânio calvo.

Lembro-me bem. A ponte era comprida,&
E a minha sombra enorme enchia a ponte,&
Como uma pele de rinoceronte&
Estendida por toda a minha vida!

A noite fecundava o ovo dos vícios&
Animais. Do carvão da treva imensa&
Caía um ar danado de doença&
Sobre a cara geral dos edifícios!

Tal uma horda feroz de cães famintos,&
Atravessando uma estação deserta,&
Uivava dentro do \textit{eu}, com a boca aberta,&
A matilha espantada dos instintos!

Era como se, na alma da cidade,&
Profundamente lúbrica e revolta,&
Mostrando as carnes, uma besta solta&
Soltasse o berro da animalidade.

E aprofundando o raciocínio obscuro,&
Eu vi, então, à luz de áureos reflexos,&
O trabalho genésico dos sexos,&
Fazendo à noite os homens do Futuro.

Livres de microscópios e escalpelos,&
Dançavam, parodiando saraus cínicos,&
Bilhões de \textit{centrossomas} apolínicos&
Na câmara promíscua do \textit{vitellus}.

Mas, a irritar-me os globos oculares,&
Apregoando e alardeando a cor nojenta,&
Fetos magros, ainda na placenta,&
Estendiam-me as mãos rudimentares!

Mostravam-me o apriorismo \mbox{incognoscível}&
Dessa fatalidade igualitária,&
Que fez minha família originária&
Do antro daquela fábrica terrível!

A corrente atmosférica mais forte&
Zunia. E, na ígnea crosta do Cruzeiro,&
Julgava eu ver o fúnebre candeeiro&
Que há de me alumiar na hora da morte.

Ninguém compreendia o meu soluço,&
Nem mesmo Deus! Da roupa pelas brechas,&
O vento bravo me atirava flechas&
E aplicações hiemais de gelo russo.

A vingança dos mundos astronômicos&
Enviava à terra extraordinária faca,&
Posta em rija adesão de goma laca&
Sobre os meus elementos anatômicos.

Ah! Com certeza, Deus me castigava!&
Por toda a parte, como um réu confesso,&
Havia um juiz que lia o meu processo&
E uma forca especial que me esperava!

Mas o vento cessara por instantes&
Ou, pelo menos, o \textit{ignis sapiens} do Orco&
Abafava-me o peito arqueado e porco&
Num núcleo de substâncias abrasantes.

É bem possível que eu um dia cegue.&
No ardor desta letal tórrida zona,&
A cor do sangue é a cor que me impressiona&
E a que mais neste mundo me persegue!

Essa obsessão cromática me abate.&
Não sei por que me vêm sempre à lembrança&
O estômago esfaqueado de uma criança&
E um pedaço de víscera escarlate.

Quisera qualquer coisa provisória&
Que a minha cerebral caverna entrasse,&
E até o fim, cortasse e recortasse&
A faculdade aziaga da memória.

Na ascensão barométrica da calma,&
Eu bem sabia, ansiado e contrafeito,&
Que uma população doente do peito&
Tossia sem remédio na minh’alma!

E o cuspo que essa hereditária tosse&
Golfava, à guisa de ácido resíduo,&
Não era o cuspo só de um indivíduo&
Minado pela tísica precoce.

Não! Não era o meu cuspo, com certeza&
Era a expectoração pútrida e crassa&
Dos brônquios pulmonares de uma raça&
Que violou as leis da Natureza!

Era antes uma tosse ubíqua, estranha,&
Igual ao ruído de um calhau redondo&
Arremessado no apogeu do estrondo,&
Pelos fundibulários da montanha!

E a saliva daqueles infelizes&
Inchava, em minha boca, de tal arte,&
Que eu, para não cuspir por toda a parte,&
Ia engolindo, aos poucos, a \textit{hemoptísis}!

Na alta alucinação de minhas cismas&
O microcosmos líquido da gota&
Tinha a abundância de uma artéria rota,&
Arrebentada pelos aneurismas.

Chegou-me o estado máximo da mágoa!&
Duas, três, quatro, cinco, seis e sete&
Vezes que eu me furei com um canivete,&
A hemoglobina vinha cheia de água!

Cuspo, cujas caudais meus beiços regam,&
Sob a forma de mínimas camândulas,&
Benditas sejam todas essas glândulas,&
Que, quotidianamente, te segregam!

Escarrar de um abismo noutro abismo,&
Mandando ao Céu o fumo de um cigarro,&
Há mais filosofia neste escarro&
Do que em toda a moral do Cristianismo!

Porque, se no orbe oval que os meus pés tocam&
Eu não deixasse o meu cuspo carrasco,&
Jamais exprimiria o acérrimo asco&
Que os canalhas do mundo me provocam!


### II


Foi no horror dessa noite tão funérea&
Que eu descobri, maior talvez que Vinci,&
Com a força visualística do lince,&
A falta de unidade na matéria!

Os esqueletos desarticulados,&
Livres do acre fedor das carnes mortas,&
Rodopiavam, com as brancas tíbias tortas,&
Numa dança de números quebrados!

Todas as divindades malfazejas,&
Siva e Arimã, os duendes, o In e os trasgos,&
Imitando o barulho dos engasgos,&
Davam pancadas no adro das igrejas.

Nessa hora de monólogos sublimes,&
A companhia dos ladrões da noite,&
Buscando uma taverna que os acoite,&
Vai pela escuridão pensando crimes.

Perpetravam-se os atos mais funestos,&
E o luar, da cor de um doente de icterícia,&
Iluminava, a rir, sem pudicícia,&
A camisa vermelha dos incestos.

Ninguém, de certo, estava ali, a espiar-me,&
Mas um lampião, lembrava ante o meu rosto,&
Um sugestionador olho, ali posto&
De propósito, para hipnotizar-me!

Em tudo, então, meus olhos distinguiram&
Da miniatura singular de uma aspa,&
À anatomia mínima da caspa,&
Embriões de mundos que não progrediram!

Pois quem não vê aí, em qualquer rua,&
Com a fina nitidez de um claro jorro,&
Na paciência budista do cachorro&
A alma embrionária que não continua?!

Ser cachorro! Ganir incompreendidos&
Verbos! Querer dizer-nos que não finge,&
E a palavra embrulhar-se na laringe,&
Escapando-se apenas em latidos!

Despir a putrescível forma tosca,&
Na atra dissolução que tudo inverte,&
Deixar cair sobre a barriga inerte&
O apetite necrófago da mosca!

A alma dos animais! Pego-a, distingo-a,&
Acho-a nesse interior duelo secreto&
Entre a ânsia de um vocábulo completo&
E uma expressão que não chegou à língua!

Surpreendo-a em quatrilhões de corpos vivos,&
Nos antiperistálticos abalos&
Que produzem nos bois e nos cavalos&
A contração dos gritos instintivos!

Tempo viria, em que, daquele horrendo&
Caos de corpos orgânicos disformes&
Rebentariam cérebros enormes,&
Como bolhas febris de água, fervendo!

Nessa época que os sábios não ensinam,&
A pedra dura, os montes argilosos&
Criariam feixes de cordões nervosos&
E o neuroplasma dos que raciocinam!

Almas pigmeias! Deus subjuga-as, cinge-as&
À imperfeição! Mas vem o Tempo, e vence-O,&
E o meu sonho crescia no silêncio,&
Maior que as epopeias carolíngias!

Era a revolta trágica dos tipos&
Ontogênicos mais elementares,&
Desde os foraminíferos dos mares&
À grei liliputiana dos pólipos.

Todos os personagens da tragédia,&
Cansados de viver na paz de Buda,&
Pareciam pedir com a boca muda&
A ganglionária célula intermédia.

A planta que a canícula ígnea torra,&
E as coisas inorgânicas mais nulas&
Apregoavam encéfalos, medulas&
Na alegria guerreira da desforra!

Os protistas e o obscuro acervo rijo&
Dos espongiários e dos infusórios&
Recebiam com os seus órgãos sensórios&
O triunfo emocional do regozijo!

E apesar de já não ser assim tão tarde,&
Aquela humanidade parasita,&
Como um bicho inferior, berrava, aflita,&
No meu temperamento de covarde!

Mas, refletindo, a sós, sobre o meu caso,&
Vi que, igual a um amniota subterrâneo,&
Jazia atravessada no meu crânio&
A intercessão fatídica do atraso!

A hipótese genial do \textit{microzima}&
Me estrangulava o pensamento guapo,&
E eu me encolhia todo como um sapo&
Que tem um peso incômodo por cima!

Nas agonias do \textit{delirium tremens},&
Os bêbedos alvares que me olhavam,&
Com os copos cheios esterilizavam&
A substância prolífica dos semens!

Enterravam as mãos dentro das goelas,&
E sacudidos de um tremor indômito&
Expeliam, na dor forte do vômito,&
Um conjunto de gosmas amarelas.

Iam depois dormir nos lupanares&
Onde, na glória da concupiscência,&
Depositavam quase sem consciência&
As derradeiras forças musculares.

Fabricavam destarte os blastodermas,&
Em cujo repugnante receptáculo&
Minha perscrutação via o espetáculo&
De uma progênie idiota de palermas.

Prostituição ou outro qualquer nome,&
Por tua causa, embora o homem te aceite,&
É que as mulheres ruins ficam sem leite&
E os meninos sem pai morrem de fome!

Por que há de haver aqui tantos enterros?!&
Lá no “Engenho” também, a morte é ingrata\ldots{}&
Há o malvado carbúnculo que mata&
A sociedade infante dos bezerros!

Quantas moças que o túmulo reclama!&
E após a podridão de tantas moças,&
Os porcos espojando-se nas poças&
Da virgindade reduzida à lama!

Morte, ponto final da última cena,&
Forma difusa da matéria imbele,&
Minha filosofia te repele,&
Meu raciocínio enorme te condena!

Diante de ti, nas catedrais mais ricas,&
Rolam sem eficácia os amuletos,&
Oh! Senhora dos nossos esqueletos&
E das caveiras diárias que fabricas!

E eu desejava ter, numa ânsia rara,&
Ao pensar nas pessoas que perdera,&
A inconsciência das máscaras de cera&
Que a gente prega, como um cordão, na cara!

Era um sonho ladrão de submergir-me&
Na vida universal, e, em tudo imerso,&
Fazer da parte abstrata do Universo,&
Minha morada equilibrada e firme!

Nisto, pior que o remorso do assassino,&
Reboou, tal qual, num fundo de caverna,&
Numa impressionadora voz interna,&
O eco particular do meu Destino:


### III


“Homem! por mais que a Ideia desintegres,&
Nessas perquisições que não têm pausa,&
Jamais, magro homem, saberás a causa&
De todos os fenômenos alegres!

Em vão, com a bronca enxada árdega, sondas&
A estéril terra, e a hialina lâmpada oca,&
Trazes, por perscrutar (oh! ciência louca!)&
O conteúdo das lágrimas hediondas.

Negro e sem fim é esse em que te mergulhas&
Lugar do Cosmos, onde a dor infrene&
É feita como é feito o querosene&
Nos recôncavos úmidos das hulhas!

Porque, para que a Dor perscrutes, fora&
Mister que, não como és, em síntese, antes&
Fosses, a refletir teus semelhantes,&
A própria humanidade sofredora!

A universal complexidade é que Ela&
Compreende. E se, por vezes, se divide,&
Mesmo ainda assim, seu todo não reside&
No quociente isolado da parcela!

Ah! Como o ar imortal a Dor não finda!&
Das papilas nervosas que há nos tatos&
Veio e vai desde os tempos mais transatos&
Para outros tempos que hão de vir ainda!

Como o machucamento das insônias&
Te estraga, quando toda a estuada Ideia&
Dás ao sôfrego estudo da ninfeia&
E de outras plantas dicotiledôneas!

A diáfana água alvíssima e a hórrida áscua&
Que da ígnea flama bruta, estriada, espirra;&
A formação molecular da mirra,&
O cordeiro simbólico da Páscoa;

As rebeladas cóleras que rugem&
No homem civilizado, e a ele se prendem&
Como às pulseiras que os mascates vendem&
A aderência teimosa da ferrugem;

O orbe feraz que bastos tojos acres&
Produz; a rebelião que, na batalha,&
Deixa os homens deitados, sem mortalha,&
Na sangueira concreta dos massacres;

Os sanguinolentíssimos chicotes&
Da hemorragia; as nódoas mais espessas,&
O achatamento ignóbil das cabeças,&
Que ainda degrada os povos hotentotes;

O Amor e a Fome, a fera ultriz que o fojo&
Entra, à espera que a mansa vítima o entre,&
--- Tudo que gera no materno ventre&
A causa fisiológica do nojo;

As pálpebras inchadas na vigília,&
As aves moças que perderam a asa,&
O fogão apagado de uma casa,&
Onde morreu o chefe da família;

O trem particular que um corpo arrasta&
Sinistramente pela via férrea,&
A cristalização da massa térrea,&
O tecido da roupa que se gasta;

A água arbitrária que hiulcos caules grossos&
Carrega e come; as negras formas feias&
Dos aracnídeos e das centopeias,&
O fogo-fátuo que ilumina os ossos;

As projeções flamívomas que ofuscam,&
Como uma pincelada rembrandtesca,&
A sensação que uma coalhada fresca&
Transmite às mãos nervosas dos que a buscam;

O antagonismo de Tifon e Osíris,&
O homem grande oprimindo o homem pequeno,&
A lua falsa de um parasseleno,&
A mentira meteórica do arco-íris;

Os terremotos que, abalando os solos,&
Lembram paióis de pólvora explodindo,&
A rotação dos fluidos produzindo&
A depressão geológica dos polos;

O instinto de procriar, a ânsia legítima&
Da alma, afrontando ovante aziagos riscos,&
O juramento dos guerreiros priscos&
Metendo as mãos nas glândulas da vítima;

As diferenciações que o psicoplasma&
Humano sofre na mania mística,&
A pesada opressão característica&
Dos 10 minutos de um acesso de asma;

E, (conquanto contra isto ódios regougues)&
A utilidade fúnebre da corda&
Que arrasta a rês, depois que a rês engorda,&
À morte desgraçada dos açougues\ldots{}

Tudo isto que o terráqueo abismo encerra&
Forma a complicação desse barulho&
Travado entre o dragão do humano orgulho&
E as forças inorgânicas da terra!

Por descobrir tudo isso, embalde cansas!&
Ignoto é o gérmen dessa força ativa&
Que engendra, em cada célula passiva,&
A heterogeneidade das mudanças!

Poeta, feto malsão, criado com os sucos&
De um leite mau, carnívoro asqueroso,&
Gerado no atavismo monstruoso&
Da alma desordenada dos malucos;

Última das criaturas inferiores&
Governada por átomos mesquinhos,&
Teu pé mata a uberdade dos caminhos&
E esteriliza os ventres geradores!

O áspero mal que a tudo, em torno, trazes,&
Análogo é ao que, negro e a seu turno,&
Traz o ávido filóstomo noturno&
Ao sangue dos mamíferos vorazes!

Ah! Por mais que, com o espírito, trabalhes&
A perfeição dos seres existentes,&
Hás de mostrar a cárie dos teus dentes&
Na anatomia horrenda dos detalhes!

O Espaço --- esta abstração spencereana&
Que abrange as relações de coexistência&
É só! Não tem nenhuma dependência&
Com as vértebras mortais da espécie humana!

As radiantes elipses que as estrelas&
Traçam, e ao espectador falsas se antolham&
São verdades de luz que os homens olham&
Sem poder, no entretanto, compreendê-las.

Em vão, com a mão corrupta, outro éter pedes&
Que essa mão, de esqueléticas falanges,&
Dentro dessa água que com a vista abranges,&
Também prova o princípio de Arquimedes!

A fadiga feroz que te esbordoa&
Há de deixar-te essa medonha marca,&
Que, nos corpos inchados de anasarca,&
Deixam os dedos de qualquer pessoa!

Nem terás no trabalho que tiveste&
A misericordiosa toalha amiga,&
Que afaga os homens doentes de bexiga&
E enxuga, à noite, as pústulas da peste!

Quando chegar depois a hora tranquila,&
Tu serás arrastado, na carreira,&
Como um cepo inconsciente de madeira&
Na evolução orgânica da argila!

Um dia comparado com um milênio&
Seja, pois, o teu último Evangelho\ldots{}&
É a evolução do novo para o velho&
E do homogêneo para o heterogêneo!

Adeus! Fica-te aí, com o abdômen largo&
A apodrecer!\ldots{} És poeira, e embalde vibras!&
O corvo que comer as tuas fibras&
Há de achar nelas um sabor amargo!”


### IV


Calou-se a voz. A noite era funesta.&
E os queixos, a exibir trismos danados,&
Eu puxava os cabelos desgrenhados&
Como o rei Lear, no meio da floresta!

Maldizia, com apóstrofes veementes,&
No estentor de mil línguas insurretas,&
O convencionalismo das Pandetas&
E os textos maus dos códigos recentes!

Minha imaginação atormentada&
Paria absurdos\ldots{} Como diabos juntos,&
Perseguiam-me os olhos dos defuntos&
Com a carne da esclerótica esverdeada.

Secara a clorofila das lavouras.&
Igual aos sustenidos de uma endecha&
Vinha-me às cordas glóticas a queixa&
Das coletividades sofredoras.

O mundo resignava-se invertido&
Nas forças principais do seu trabalho\ldots{}&
A gravidade era um princípio falho,&
A análise espectral tinha mentido!

O Estado, a Associação, os Municípios&
Eram mortos. De todo aquele mundo&
Restava um mecanismo moribundo&
E uma teleologia sem princípios.

Eu queria correr, ir para o inferno,&
Para que, da psique no oculto jogo,&
Morressem sufocadas pelo fogo&
Todas as impressões do mundo externo!

Mas a Terra negava-me o equilíbrio\ldots{}&
Na Natureza, uma mulher de luto&
Cantava, espiando as árvores sem fruto.&
A canção prostituta do ludíbrio!



\chapter{Budismo moderno}


Tome, Dr., esta tesoura, e\ldots{} corte&
Minha singularíssima pessoa.&
Que importa a mim que a bicharia roa&
Todo o meu coração, depois da morte?!

Ah! Um urubu pousou na minha sorte!&
Também, das diatomáceas da lagoa&
A criptógama cápsula se esbroa&
Ao contato de bronca destra forte!

Dissolva-se, portanto, minha vida&
Igualmente a uma célula caída&
Na aberração de um óvulo infecundo;

Mas o agregado abstrato das saudades&
Fique batendo nas perpétuas grades&
Do último verso que eu fizer no mundo!



\chapter{Sonho de um monista}


Eu e o esqueleto esquálido de Esquilo&
Viajávamos, com uma ânsia sibarita,&
Por toda a pró-dinâmica infinita,&
Na inconsciência de um zoófito tranquilo.

A verdade espantosa do \textit{Protilo}&
Me aterrava, mas dentro da alma aflita&
Via Deus --- essa mônada esquisita ---&
Coordenando e animando tudo aquilo!

E eu bendizia, com o esqueleto ao lado,&
Na guturalidade do meu brado,&
Alheio ao velho cálculo dos dias,

Como um pagão no altar de Proserpina,&
A energia intracósmica divina&
Que é o pai e é a mãe das outras energias!



\chapter{Solitário}


Como um fantasma que se refugia&
Na solidão da natureza morta,&
Por trás dos ermos túmulos, um dia,&
Eu fui refugiar-me à tua porta!

Fazia frio e o frio que fazia&
Não era esse que a carne nos conforta\ldots{}&
Cortava assim como em carniçaria&
O aço das facas incisivas corta!

Mas tu não vieste ver minha Desgraça!&
E eu saí, como quem tudo repele,&
--- Velho caixão a carregar destroços ---

Levando apenas na tumbal carcaça&
O pergaminho singular da pele&
E o chocalho fatídico dos ossos!



\chapter{\textit{Mater originalis}}


Forma vermicular desconhecida&
Que estacionaste, mísera e mofina,&
Como quase impalpável gelatina,&
Nos estados prodrômicos da vida;

O hierofante que leu a minha sina&
Ignorante é de que és, talvez, nascida&
Dessa homogeneidade indefinida&
Que o insigne Herbert Spencer nos ensina.

Nenhuma ignota união ou nenhum nexo&
À contingência orgânica do sexo&
A tua estacionária alma prendeu\ldots{}

Ah! De ti foi que, autônoma e sem normas,&
Oh! Mãe original das outras formas,&
A minha forma lúgubre nasceu!



\chapter{O lupanar}


Ah! Por que monstruosíssimo motivo&
Prenderam para sempre, nesta rede,&
Dentro do ângulo diedro da parede,&
A alma do homem polígamo e lascivo?!

Este lugar, moços do mundo, vede:&
É o grande bebedouro coletivo,&
Onde os bandalhos, como um gado vivo,&
Todas as noites, vêm matar a sede!

É o afrodístico leito do hetairismo,&
A antecâmara lúbrica do abismo,&
Em que é mister que o gênero humano entre,

Quando a promiscuidade aterradora&
Matar a última força geradora&
E comer o último óvulo do ventre!



\chapter{Idealismo}


Falas de amor, e eu ouço tudo e calo!&
O amor na Humanidade é uma mentira.&
É. E é por isso que na minha lira&
De amores fúteis poucas vezes falo.

O amor! Quando virei por fim a amá-lo?!&
Quando, se o amor que a Humanidade inspira&
É o amor do sibarita e da hetaíra,&
De Messalina e de Sardanapalo?!

Pois é mister que, para o amor sagrado,&
O mundo fique imaterializado&
--- Alavanca desviada do seu fulcro ---

E haja só amizade verdadeira&
Duma caveira para outra caveira,&
Do meu sepulcro para o teu sepulcro?!



\chapter{Último credo}


Como ama o homem adúltero o adultério&
E o ébrio a garrafa tóxica de rum,&
Amo o coveiro --- este ladrão comum&
Que arrasta a gente para o cemitério!

É o transcendentalíssimo mistério!&
É o \textit{nous}, é o \textit{pneuma}, é o \textit{ego sum qui sum},&
É a morte, é esse danado número \textit{Um}&
Que matou Cristo e que matou Tibério!

Creio, como o filósofo mais crente,&
Na generalidade decrescente&
Com que a substância cósmica evolui\ldots{}

Creio, perante a evolução imensa,&
Que o homem universal de amanhã vença&
O homem particular que eu ontem fui!



\chapter{O caixão fantástico}


Célere ia o caixão, e, nele, inclusas,&
Cinzas, caixas cranianas, cartilagens&
Oriundas, como os sonhos dos selvagens,&
De aberratórias abstrações abstrusas!

Nesse caixão iam talvez as Musas,&
Talvez meu Pai! Hoffmânnicas visagens&
Enchiam meu encéfalo de imagens&
As mais contraditórias e confusas!

A energia monística do Mundo,&
À meia-noite, penetrava fundo&
No meu fenomenal cérebro cheio\ldots{}

Era tarde! Fazia muito frio.&
Na rua apenas o caixão sombrio&
Ia continuando o seu passeio!



\chapter{Solilóquio de um visionário}


Para desvirginar o labirinto&
Do velho e metafísico Mistério,&
Comi meus olhos crus no cemitério,&
Numa antropofagia de faminto!

A digestão desse manjar funéreo&
Tornado sangue transformou-me o instinto&
De humanas impressões visuais que eu sinto,&
Nas divinas visões do íncola etéreo!

Vestido de hidrogênio incandescente,&
Vaguei um século, improficuamente,&
Pelas monotonias siderais\ldots{}

Subi talvez às máximas alturas,&
Mas, se hoje volto assim, com a alma às escuras,&
É necessário que inda eu suba mais!



\chapter{A um carneiro morto}


Misericordiosíssimo carneiro&
Esquartejado, a maldição de Pio&
Décimo caia em teu algoz sombrio&
E em todo aquele que for seu herdeiro!

Maldito seja o mercador vadio&
Que te vender as carnes por dinheiro,&
Pois tua lã aquece o mundo inteiro&
E guarda as carnes dos que estão com frio!

Quando a faca rangeu no teu pescoço,&
Ao monstro que espremeu teu sangue grosso&
Teus olhos --- fontes de perdão --- perdoaram!

Oh! tu que no Perdão eu simbolizo,&
Se fosses Deus, no Dia do Juízo,&
Talvez perdoasses os que te mataram!



\chapter{Vozes da morte}


Agora, sim! Vamos morrer, reunidos,&
Tamarindo de minha desventura,&
Tu, com o envelhecimento da nervura,&
Eu, com o envelhecimento dos tecidos!

Ah! Esta noite é a noite dos Vencidos!&
E a podridão, meu velho! E essa futura&
Ultrafatalidade de ossatura,&
A que nos acharemos reduzidos!

Não morrerão, porém, tuas sementes!&
E assim, para o Futuro, em diferentes&
Florestas, vales, selvas, glebas, trilhos,

Na multiplicidade dos teus ramos,&
Pelo muito que em vida nos amamos,&
Depois da morte, inda teremos filhos!



\chapter{Insânia de um simples}


Em cismas patológicas insanas,&
É-me grato adstringir-me, na hierarquia&
Das formas vivas, à categoria&
Das organizações liliputianas;

Ser semelhante aos zoófitos e às lianas,&
Ter o destino de uma larva fria,&
Deixar enfim na cloaca mais sombria&
Este feixe de células humanas!

E enquanto arremedando Éolo iracundo,&
Na orgia heliogabálica do mundo,&
Ganem todos os vícios de uma vez,

Apraz-me, adstrito ao triângulo mesquinho&
De um delta humilde, apodrecer sozinho&
No silêncio de minha pequenez!



\chapter{Os doentes}

### I


Como uma cascavel que se enroscava,&
A cidade dos lázaros dormia\ldots{}&
Somente, na metrópole vazia,&
Minha cabeça autônoma pensava!

Mordia-me a obsessão má de que havia,&
Sob os meus pés, na terra onde eu pisava,&
Um fígado doente que sangrava&
E uma garganta órfã que gemia!

Tentava compreender com as conceptivas&
Funções do encéfalo as substâncias vivas&
Que nem Spencer, nem Haeckel compreenderam\ldots{}

E via em mim, coberto de desgraças,&
O resultado de bilhões de raças&
Que há muitos anos desapareceram!


### II


Minha angústia feroz não tinha nome.&
Ali, na urbe natal do Desconsolo,&
Eu tinha de comer o último bolo&
Que Deus fazia para a minha fome!

Convulso, o vento entoava um pseudosalmo.&
Contrastando, entretanto, com o ar convulso&
A noite funcionava como um pulso&
Fisiologicamente muito calmo.

Caíam sobre os meus centros nervosos,&
Como os pingos ardentes de cem velas,&
O uivo desenganado das cadelas&
E o gemido dos homens bexigosos.

Pensava! E em que eu pensava, não perguntes!&
Mas, em cima de um túmulo, um cachorro&
Pedia para mim água e socorro&
À comiseração dos transeuntes!

Bruto, de errante rio, alto e hórrido, o urro&
Reboava. Além jazia aos pés da serra,&
Criando as superstições de minha terra,&
A queixada específica de um burro!

Gordo adubo de agreste urtiga brava,&
Benigna água, magnânima e magnífica,&
Em cuja álgida unção, branda e beatífica,&
A Paraíba indígena se lava!

A manga, a ameixa, a amêndoa, a abóbora, o álamo&
E a câmara odorífera dos sumos&
Absorvem diariamente o ubérrimo húmus&
Que Deus espalha à beira do seu tálamo!

Nos de teu curso desobstruídos trilhos,&
Apenas eu compreendo, em quaisquer horas,&
O hidrogênio e o oxigênio que tu choras&
Pelo falecimento dos teus filhos!

Ah! Somente eu compreendo, satisfeito,&
A incógnita psiquê das massas mortas&
Que dormem, como as ervas, sobre as hortas,&
Na esteira igualitária do teu leito!

O vento continuava sem cansaço&
E enchia com a fluidez do eólico hissope&
Em seu fantasmagórico galope&
A abundância geométrica do espaço.

Meu ser estacionava, olhando os campos&
Circunjacentes. No Alto, os astros miúdos&
Reduziam os Céus sérios e rudos&
A uma epiderme cheia de sarampos!


### III


Dormia embaixo, com a promíscua véstia&
No embotamento crasso dos sentidos,&
A comunhão dos homens reunidos&
Pela camaradagem da moléstia.

Feriam-me o nervo óptico e a retina&
Aponevroses e tendões de Aquiles,&
Restos repugnantíssimos de bílis,&
Vômitos impregnados de ptialina.

Da degenerescência étnica do Ária&
Se escapava, entre estrépitos e estouros,&
Reboando pelos séculos vindouros,&
O ruído de uma tosse hereditária.

Oh! desespero das pessoas tísicas,&
Adivinhando o frio que há nas lousas,&
Maior felicidade é a destas cousas&
Submetidas apenas às leis físicas!

Estas, por mais que os cardos grandes rocem&
Seus corpos brutos, dores não recebem;&
Estas dos bacalhaus o óleo não bebem,&
Estas não cospem sangue, estas não tossem!

Descender dos macacos catarríneos,&
Cair doente e passar a vida inteira&
Com a boca junto de uma escarradeira,&
Pintando o chão de coágulos sanguíneos!

Sentir, adstritos ao quimiotropismo&
Erótico, os micróbios assanhados&
Passearem, como inúmeros soldados,&
Nas cancerosidades do organismo!

Falar somente uma linguagem rouca.&
Um português cansado e incompreensível,&
Vomitar o pulmão na noite horrível&
Em que se deita sangue pela boca!

Expulsar, aos bocados, a existência&
Numa bacia autômata de barro,&
Alucinado, vendo em cada escarro&
O retrato da própria consciência!

Querer dizer a angústia de que é pábulo,&
E com a respiração já muito fraca&
Sentir como que a ponta de uma faca,&
Cortando as raízes do último vocábulo.

Não haver terapêutica que arranque&
Tanta opressão como se, com efeito,&
Lhe houvessem sacudido sobre o peito&
A máquina pneumática de Bianchi!

E o ar fugindo e a Morte a arca da tumba&
A erguer, como um cronômetro gigante,&
Marcando a transição emocionante&
Do lar materno para a catacumba!

Mas vos não lamenteis, magras mulheres,&
Nos ardores danados da febre hética,&
Consagrando vossa última fonética&
A uma recitação de misereres.

Antes levardes ainda uma quimera&
Para a garganta onívora das lajes&
Do que morrerdes, hoje, urrando ultrajes&
Contra a dissolução que vos espera!

Porque a morte, resfriando-vos o rosto,&
Consoante a minha concepção vesânica,&
É a alfândega, onde toda a vida orgânica&
Há de pagar um dia o último imposto!


### IV


Começara a chover. Pelas algentes&
Ruas, a água, em cachoeiras desobstruídas,&
Encharcava os buracos das feridas,&
Alagava a medula dos Doentes!

Do fundo do meu trágico destino,&
Onde a Resignação os braços cruza,&
Saía, com o vexame de uma fusa,&
A mágoa gaguejada de um cretino.

Aquele ruído obscuro de gagueira&
Que à noite, em sonhos mórbidos, me acorda,&
Vinha da vibração bruta da corda&
Mais recôndita da alma brasileira!

Aturdia-me a tétrica miragem&
De que, naquele instante, no Amazonas,&
Fedia, entregue a vísceras glutonas,&
A carcaça esquecida de um selvagem.

A civilização entrou na taba&
Em que ele estava. O gênio de Colombo&
Manchou de opróbrios a alma do \textit{mazombo},&
Cuspiu na cova do \textit{morubixaba}!

E o índio, por fim, adstrito à étnica escória,&
Recebeu, tendo o horror no rosto impresso,&
Esse achincalhamento do progresso&
Que o anulava na crítica da História!

Como quem analisa um apostema,&
De repente, acordando na desgraça,&
Viu toda a podridão de sua raça&
\quad\quad Na tumba de Iracema!\ldots{}

Ah! Tudo, como um lúgubre ciclone,&
Exercia sobre ele ação funesta&
Desde o desbravamento da floresta&
À ultrajante invenção do telefone.

E sentia-se pior que um vagabundo&
Microcéfalo vil que a espécie encerra&
Desterrado na sua própria terra,&
Diminuído na crônica do mundo!

A hereditariedade dessa pecha&
Seguiria seus filhos. Dora em diante&
Seu povo tombaria agonizante&
Na luta da espingarda com a flecha!

Veio-lhe então como à fêmea vêm antojos,&
Uma desesperada ânsia improfícua&
De estrangular aquela gente iníqua&
Que progredia sobre os seus despojos!

Mas, diante a xantocroide raça loura,&
Jazem, caladas, todas as inúbias,&
E agora, sem difíceis nuanças dúbias,&
Com uma clarividência aterradora,

Em vez da prisca tribo e indiana tropa&
A gente deste século, espantada,&
Vê somente a caveira abandonada&
De uma raça esmagada pela Europa!


### V


Era a hora em que arrastados pelos ventos,&
Os fantasmas hamléticos dispersos&
Atiram na consciência dos perversos&
A sombra dos remorsos famulentos.

As mães sem coração rogavam pragas&
Aos filhos bons. E eu, roído pelos medos,&
Batia com o pentágono dos dedos&
Sobre um fundo hipotético de chagas!

Diabólica dinâmica daninha&
Oprimia meu cérebro indefeso&
Com a força onerosíssima de um peso&
Que eu não sabia mesmo de onde vinha.

Perfurava-me o peito a áspera pua&
Do desânimo negro que me prostra,&
E quase a todos os momentos mostra&
Minha caveira aos bêbedos da rua.

Hereditariedades politípicas&
Punham na minha boca putrescível&
Interjeições de abracadabra horrível&
E os verbos indignados das Filípicas.

Todos os vocativos dos blasfemos,&
No horror daquela noite monstruosa,&
Maldiziam, com voz estentorosa,&
A peçonha inicial de onde nascemos.

Como que havia na ânsia de conforto&
De cada ser, ex.: o homem e o ofídio,&
Uma necessidade de suicídio&
E um desejo incoercível de ser morto!

Naquela angústia absurda e tragicômica&
Eu chorava, rolando sobre o lixo,&
Com a contorção neurótica de um bicho&
Que ingeriu 30 gramas de nux-vômica.

E, como um homem doido que se enforca,&
Tentava, na terráquea superfície,&
Consubstanciar-me todo com a imundície,&
Confundir-me com aquela coisa porca!

Vinha, às vezes, porém, o anelo instável&
De, com o auxílio especial do osso masseter&
Mastigando homeomerias neutras de éter&
Nutrir-me da matéria imponderável.

Anelava ficar um dia, em suma,&
Menor que o anfióxus e inferior à tênia,&
Reduzido à plastídula homogênea,&
Sem diferenciação de espécie alguma.

Era (nem sei em síntese o que diga)&
Um velhíssimo instinto atávico, era&
A saudade inconsciente da monera&
Que havia sido minha mãe antiga!

Com o horror tradicional da raiva corsa&
Minha vontade era, perante a cova,&
Arrancar do meu próprio corpo a prova&
Da persistência trágica da força.

A pragmática má de humanos usos&
Não compreende que a Morte que não dorme&
É a absorção do movimento enorme&
Na dispersão dos átomos difusos.

Não me incomoda esse último abandono.&
Se a carne individual hoje apodrece,&
Amanhã, como Cristo, reaparece&
Na universalidade do carbono!

A vida vem do éter que se condensa,&
Mas o que mais no Cosmos me entusiasma&
É a esfera microscópica do plasma&
Fazer a luz do cérebro que pensa.

Eu voltarei, cansado da árdua liça,&
À substância inorgânica primeva,&
De onde, por epigênese, veio Eva&		%epigênesis na edição que consultei
E a \textit{stirpe radiolar} chamada \textit{Actissa}!

Quando eu for misturar-me com as violetas&
Minha lira, maior que a \textit{Bíblia} e a \textit{Fedra},&
Reviverá, dando emoção à pedra,&
Na acústica de todos os planetas!


### VI


À álgida agulha, agora, alva, a saraiva&
Caindo, análoga era\ldots{} Um cão agora&
Punha a atra língua hidrófoba de fora&
Em contrações miológicas de raiva.

Mas, para além, entre oscilantes chamas,&
Acordavam os bairros da luxúria\ldots{}&
As prostitutas, doentes de hematúria,&
\quad\quad Se extenuavam nas camas.

Uma, ignóbil, derreada de cansaço,&
Quase que escangalhada pelo vício,&
Cheirava com prazer no sacrifício&
A lepra má que lhe roía o braço!

E ensanguentava os dedos da mão nívea&
Com o sentimento gasto e a emoção pobre,&
Nessa alegria bárbara que cobre&
Os saracoteamentos da lascívia\ldots{}

De certo, a perversão de que era presa&
O \textit{sensorium} daquela prostituta&
Vinha da adaptação quase absoluta&
À ambiência microbiana da baixeza!

Entanto, virgem fostes, e, quando o éreis,&
Não tínheis ainda essa erupção cutânea,&
Nem tínheis, vítima última da insânia,&
Duas mamárias glândulas estéreis!

Ah! Certamente, não havia ainda&
Rompido, com violência, no horizonte,&
O sol malvado que secou a fonte&
De vossa castidade agora finda!

Talvez tivésseis fome, e as mãos, embalde,&
Estendestes ao mundo, até que, à toa,&
Fostes vender a virginal coroa&
Ao primeiro bandido do arrabalde.

E estais velha! --- De vós o mundo é farto,&
E hoje, que a sociedade vos enxota,&
Somente as \textit{bruxas} negras da derrota&
Frequentam diariamente vosso quarto!

Prometem-vos (quem sabe?!) entre os ciprestes&
Longe da mancebia dos alcouces,&
Nas quietudes nirvânicas mais doces,&
O noivado que em vida não tivestes!


### VII


Quase todos os lutos conjugados,&
Como uma associação de monopólio,&
Lançavam pinceladas pretas de óleo&
Na arquitetura arcaica dos sobrados.

Dentro da noite funda um braço humano&
Parecia cavar ao longe um poço&
Para enterrar minha ilusão de moço,&
Como a boca de um poço artesiano!

Atabalhoadamente pelos becos,&
Eu pensava nas coisas que perecem,&
Desde as musculaturas que apodrecem&
À ruína vegetal dos lírios secos.

Cismava no propósito funéreo&
Da mosca debochada que fareja&
O defunto, no chão frio da igreja,&
E vai depois levá-lo ao cemitério!

E esfregando as mãos magras, eu, inquieto,&
Sentia, na craniana caixa tosca,&
A racionalidade dessa mosca,&
A consciência terrível desse inseto!

Regougando, porém, \textit{argots} e aljâmias,&
Como quem nada encontra que o perturbe,&
A energúmena grei dos ébrios da urbe&
Festejava seu sábado de infâmias.

A estática fatal das paixões cegas,&
Rugindo fundamente nos neurônios,&
Puxava aquele povo de demônios&
Para a promiscuidade das adegas.

E a ébria turba que escaras sujas masca,&
À falta idiossincrásica de escrúpulo,&
Absorvia com gáudio absinto, lúpulo&
E outras substâncias tóxicas da tasca.

O ar ambiente cheirava a ácido acético,&
Mas, de repente, com o ar de quem empesta,&
Apareceu, escorraçando a festa,&
A mandíbula inchada de um morfético!

Saliências polimórficas vermelhas,&
Em cujo aspecto o olhar perspícuo prendo,&
Punham-lhe num destaque horrendo o horrendo&
Tamanho aberratório das orelhas.

O fácies do morfético assombrava!&
--- Aquilo era uma negra eucaristia,&
Onde minh’alma inteira surpreendia&
A Humanidade que se lamentava!

Era todo o meu sonho, assim, inchado,&
Já podre, que a morfeia miserável&
Tornava às impressões táteis, palpável,&
Como se fosse um corpo organizado!


### VIII


Em torno a mim, nesta hora, estriges voam,&
E o cemitério, em que eu entrei adrede,&
Dá-me a impressão de um \textit{boulevard} que fede,&
Pela degradação dos que o povoam.

Quanta gente, roubada à humana coorte,&
Morre de fome, sobre a palha espessa,&
Sem ter, como Ugolino, uma cabeça&
Que possa mastigar na hora da morte;

E nua, após baixar ao caos budista,&
Vem para aqui, nos braços de um canalha,&
Porque o madapolão para a mortalha&
Custa 1\$200 ao lojista!

Que resta das cabeças que pensaram?!&
E afundado nos sonhos mais nefastos,&
Ao pegar num milhão de miolos gastos,&
Todos os meus cabelos se arrepiaram.

Os evolucionismos benfeitores&
Que por entre os cadáveres caminham,&
Iguais a irmãs de caridade, vinham&
Com a podridão dar de comer às flores!

Os defuntos então me ofereciam&
Com as articulações das mãos inermes,&
Num prato de hospital, cheio de vermes,&
Todos os animais que apodreciam!

É possível que o estômago se afoite&
(Muito embora contra isto a alma se irrite)&
A cevar o antropófago apetite,&
Comendo carne humana, à meia-noite!

Com uma ilimitadíssima tristeza,&
Na impaciência do estômago vazio,&
Eu devorava aquele bolo frio&
Feito das podridões da Natureza!

E hirto, a camisa suada, a alma aos arrancos,&
Vendo passar com as túnicas obscuras,&
As escaveiradíssimas figuras&
Das negras desonradas pelos brancos;

Pisando, como quem salta, entre fardos,&
Nos corpos nus das moças hotentotes&
Entregues, ao clarão de alguns archotes,&
À sodomia indigna dos moscardos;

Eu maldizia o deus de mãos nefandas&
Que, transgredindo a igualitária regra&
Da Natureza, atira a raça negra&
Ao contubérnio diário das quitandas!

Na evolução de minha dor grotesca,&
Eu mendigava aos vermes insubmissos&
Como indenização dos meus serviços,&
O benefício de uma cova fresca.

Manhã. E eis-me a absorver a luz de fora,&
Como o íncola do polo ártico, às vezes,&
Absorve, após a noite de seis meses,&
Os raios caloríficos da aurora.

Nunca mais as goteiras cairiam&
Como propositais setas malvadas,&
No frio matador das madrugadas,&
Por sobre o coração dos que sofriam!

Do meu cérebro à absconsa tábua rasa&
Vinha a luz restituir o antigo crédito,&
Proporcionando-me o prazer inédito,&
De quem possui um sol dentro de casa.

Era a volúpia fúnebre que os ossos&
Me inspiravam, trazendo-me ao sol claro,&
À apreensão fisiológica do faro&
O odor cadaveroso dos destroços!


### IX


O inventário do que eu já tinha sido&
Espantava. Restavam só de Augusto&
A forma de um mamífero vetusto&
E a cerebralidade de um vencido!

O gênio procriador da espécie eterna&
Que me fizera, em vez de hiena ou lagarta,&
Uma sobrevivência de Sidarta,&
Dentro da filogênese moderna;

E arrancara milhares de existências&
Do ovário ignóbil de uma fauna imunda,&
Ia arrastando agora a alma infecunda&
Na mais triste de todas as falências.

Um céu calamitoso de vingança&
Desagregava, déspota e sem normas,&
O adesionismo biôntico das formas&
Multiplicadas pela lei da herança!

A ruína vinha horrenda e deletéria&
Do subsolo infeliz, vinha de dentro&
Da matéria em fusão que ainda há no centro,&
Para alcançar depois a periféria!

Contra a Arte, oh! Morte, em vão teu ódio exerces!&
Mas, a meu ver, os sáxeos prédios tortos&
Tinham aspectos de edifícios mortos&
Decompondo-se desde os alicerces!

A doença era geral, tudo a extenuar-se&
Estava. O Espaço abstrato que não morre&
Cansara\ldots{} O ar que, em colônias fluidas, corre,&
Parecia também desagregar-se!

O pródromos de um tétano medonho&
Repuxavam-me o rosto\ldots{} Hirto de espanto,&
Eu sentia nascer-me n’alma, entanto,&
O começo magnífico de um sonho!

Entre as formas decrépitas do povo,&
Já batiam por cima dos estragos&
A sensação e os movimentos vagos&
Da célula inicial de um Cosmos novo!

O letargo larvário da cidade&
Crescia. Igual a um parto, numa furna,&
Vinha da original treva noturna&
O vagido de uma outra Humanidade!

E eu, com os pés atolados no Nirvana,&
Acompanhava, com um prazer secreto,&
A gestação daquele grande feto,&
Que vinha substituir a Espécie Humana!



\chapter{Asa de corvo}


Asa de corvos carniceiros, asa&
De mau agouro que, nos doze meses,&
Cobre às vezes o espaço e cobre às vezes&
O telhado de nossa própria casa\ldots{}

Perseguido por todos os reveses,&
É meu destino viver junto a essa asa,&
Como a cinza que vive junto à brasa,&
Como os Goncourts, como os irmãos siameses!

É com essa asa que eu faço este soneto&
E a indústria humana faz o pano preto&
Que as famílias de luto martiriza\ldots{}

É ainda com essa asa extraordinária&
Que a Morte --- a costureira funerária ---&
Cose para o homem a última camisa!



\chapter{Uma noite no Cairo}


Noite no Egito. O céu claro e profundo&
Fulgura. A rua é triste. A lua cheia&
Está sinistra, e sobre a paz do mundo&
A alma dos Faraós anda e vagueia.

Os mastins negros vão ladrando à lua\ldots{}&
O Cairo é de uma formosura arcaica.&
No ângulo mais recôndito da rua&
Passa cantando uma mulher hebraica.

O Egito é sempre assim quando anoitece!&
Às vezes, das pirâmides o quedo&
E atro perfil, exposto ao luar, parece&
Uma sombria interjeição de medo!

Como um contraste àqueles misereres,&
Num quiosque em festa alegre turba grita,&
E dentro dançam homens e mulheres&
Numa aglomeração cosmopolita.

Tonto do vinho, um saltimbanco da Ásia,&
Convulso e roto, no apogeu da fúria,&
Executando evoluções de \textit{razzia}&
Solta um brado epilético de injúria!

Em derredor duma ampla mesa preta&
--- Última nota do conúbio infando ---&
Veem-se dez jogadores de roleta&
Fumando, discutindo, conversando.

Resplandece a celeste superfície.&
Dorme soturna a natureza sábia\ldots{}&
Embaixo, na mais próxima planície,&
Pasta um cavalo esplêndido da Arábia.

Vaga no espaço um silfo solitário.&
Troam \textit{kinnors}! Depois tudo é tranquilo\ldots{}&
Apenas como um velho stradivário,&
Soluça toda a noite a água do Nilo!



\chapter{O martírio do artista}


Arte ingrata! E conquanto, em desalento,&
A órbita elipsoidal dos olhos lhe arda,&
Busca exteriorizar o pensamento&
Que em suas fronetais células guarda!

Tarda-lhe a Ideia! A inspiração lhe tarda!&
E ei-lo a tremer, rasga o papel, violento,&
Como o soldado que rasgou a farda&
No desespero do último momento!

Tenta chorar e os olhos sente enxutos!\ldots{}&
É como o paralítico que, à míngua&
Da própria voz e na que ardente o lavra

Febre de em vão falar, com os dedos brutos&
Para falar, puxa e repuxa a língua,&
E não lhe vem à boca uma palavra!



\chapter{Duas estrofes}

{\raggedleft\itshape
(À memória de João de Deus)
\par}

\epigraph{\itshape
Ahi! ciechi! il tanto affaticar che giova?&
Tutti torniamo alla gran madre antica&
E il nostro nome appena si ritrova.}{Petrarca}


A queda do teu lírico arrabil&
De um sentimento português ignoto&
Lembra Lisboa, bela como um brinco,&
Que um dia no ano trágico de mil&
E setecentos e cinquenta e cinco,&
Foi abalada por um terremoto!

A água quieta do Tejo te abençoa.&
Tu representas toda essa Lisboa&
De glórias quase sobrenaturais,&
Apenas com uma diferença triste,&
Com a diferença que Lisboa existe&
E tu, amigo, não existes mais!



\chapter{O mar, a escada e o homem}


“Olha agora, mamífero inferior,&
À luz da epicurista \textit{ataraxia},&
O fracasso de tua geografia&
E do teu escafandro esmiuçador!

Ah! Jamais saberás ser superior,&
Homem, a mim, conquanto ainda hoje em dia,&
Com a ampla hélice auxiliar com que outrora ia&
Voando ao vento o vastíssimo vapor,

Rasgue a água hórrida a nau árdega e singre-me!”&
E a verticalidade da Escada íngreme:&
“Homem, já transpuseste os meus degraus?!”

E Augusto, o Hércules, o Homem, aos soluços,&
Ouvindo a Escada e o Mar, caiu de bruços&
No pandemônio aterrador do Caos!



\chapter{Decadência}


Iguais às linhas perpendiculares&
Caíram, como cruéis e hórridas hastas,&
Nas suas 33 vértebras gastas&
Quase todas as pedras tumulares!

A frialdade dos círculos polares,&
Em sucessivas atuações nefastas,&
Penetrara-lhe os próprios neuroplastas,&
Estragara-lhe os centros medulares!

Como quem quebra o objeto mais querido&
E começa a apanhar piedosamente&
Todas as microscópicas partículas,

Ele hoje vê que, após tudo perdido,&
Só lhe restam agora o último dente&
E a armação funerária das clavículas!



\chapter{\textit{Ricordanza della mia gioventù}}


A minha ama de leite Guilhermina&
Furtava as moedas que o Doutor me dava.&
Sinhá-Mocinha, minha Mãe, ralhava\ldots{}&
Via naquilo a minha própria ruína!

Minha ama, então, hipócrita, afetava&
Susceptibilidades de menina:&
“--- Não, não fora ela! ---” E maldizia a sina,&
Que ela absolutamente não furtava.

Vejo, entretanto, agora, em minha cama,&
Que a mim somente cabe o furto feito\ldots{}&
Tu só furtaste a moeda, o ouro que brilha\ldots{}

Furtaste a moeda só, mas eu, minha ama,&
Eu furtei mais, porque furtei o peito&
Que dava leite para a tua filha!



\chapter{A um mascarado}


Rasga essa máscara ótima de seda&
E atira-a à arca ancestral dos palimpsestos\ldots{}&
É noite, e, à noite, a escândalos e incestos&
É natural que o instinto humano aceda!

Sem que te arranquem da garganta queda&
A interjeição danada dos protestos,&
Hás de engolir, igual a um porco, os restos&
Duma comida horrivelmente azeda!

A sucessão de hebdômadas medonhas&
Reduzirá os mundos que tu sonhas&
Ao microcosmos do ovo primitivo\ldots{}

E tu mesmo, após a árdua e atra refrega,&
Terás somente uma vontade cega&
E uma tendência obscura de ser vivo!



\chapter{Vozes de um túmulo}


Morri! E a Terra --- a mãe comum --- o brilho&
Destes meus olhos apagou!\ldots{} Assim&
Tântalo, aos reais convivas, num festim,&
Serviu as carnes do seu próprio filho!

Por que para este cemitério vim?!&
Por quê?! Antes da vida o angusto trilho&
Palmilhasse, do que este que palmilho&
E que me assombra, porque não tem fim!

No ardor do sonho que o fronema exalta&		%fronema
Construí de orgulho ênea pirâmide alta\ldots{}&
Hoje, porém, que se desmoronou

A pirâmide real do meu orgulho,&
Hoje que apenas sou matéria e entulho&
Tenho consciência de que nada sou!



\chapter{Contrastes}


A antítese do novo e do obsoleto,&
O Amor e a Paz, o Ódio e a Carnificina,&
O que o homem ama e o que o homem abomina,&
Tudo convém para o homem ser completo!

O ângulo obtuso, pois, e o ângulo reto,&
Uma feição humana e outra divina&
São como a eximenina e a endimenina&
Que servem ambas para o mesmo feto!

Eu sei tudo isto mais do que o Eclesiastes!&
Por justaposição destes contrastes,&
Junta-se um hemisfério a outro hemisfério,

Às alegrias juntam-se as tristezas,&
E o carpinteiro que fabrica as mesas&
Faz também os caixões do cemitério!\ldots{}



\chapter{Gemidos de arte}

### I


Esta desilusão que me acabrunha&
É mais traidora do que o foi Pilatos!\ldots{}&
Por causa disto, eu vivo pelos matos,&
Magro, roendo a substância córnea da unha.

Tenho estremecimentos indecisos&
E sinto, haurindo o tépido ar sereno,&
O mesmo assombro que sentiu Parfeno&
Quando arrancou os olhos de Dionisos!

Em giro e em redemoinho em mim caminham&
Ríspidas mágoas estranguladoras,&
Tais quais, nos fortes fulcros, as tesouras&
Brônzeas, também giram e redemoinham.

Os pães --- filhos legítimos dos trigos ---&
Nutrem a geração do Ódio e da Guerra\ldots{}&
Os cachorros anônimos da terra&
São talvez os meus únicos amigos!

Ah! Por que desgraçada contingência&
À híspida aresta sáxea áspera e abrupta&
Da rocha brava, numa ininterrupta&
Adesão, não prendi minha existência?!

Por que Jeová, maior do que Laplace,&
Não fez cair o túmulo de Plínio&
Por sobre todo o meu raciocínio&
Para que eu nunca mais raciocinasse?!

Pois minha Mãe tão cheia assim daqueles&
Carinhos, com que guarda meus sapatos,&
Por que me deu consciência dos meus atos&
Para eu me arrepender de todos eles?!

Quisera, antes, mordendo glabros talos,&
Nabucodonosor ser no Pau d’Arco,&
Beber a acre e estagnada água do charco,&
Dormir na manjedoura com os cavalos!

Mas a carne é que é humana! A alma é divina.&
Dorme num leito de feridas, goza&
O lodo, apalpa a úlcera cancerosa,&
Beija a peçonha, e não se contamina!

Ser homem! escapar de ser aborto!&
Sair de um ventre inchado que se anoja,&
Comprar vestidos pretos numa loja&
E andar de luto pelo pai que é morto!

E por trezentos e sessenta dias&
Trabalhar e comer! Martírios juntos!&
Alimentar-se dos irmãos defuntos,&
Chupar os ossos das alimarias!

Barulho de mandíbulas e abdomens!&
E vem-me com um desprezo por tudo isto&
Uma vontade absurda de ser Cristo&
Para sacrificar-me pelos homens!

Soberano desejo! Soberana&
Ambição de construir para o homem uma&
Região, onde não cuspa língua alguma&
O óleo rançoso da saliva humana!

Uma região sem nódoas e sem lixos,&
Subtraída à hediondez de ínfimo casco,&
Onde a forca feroz coma o carrasco&
E o olho do estuprador se encha de bichos!

Outras constelações e outros espaços&
Em que, no agudo grau da última crise,&
O braço do ladrão se paralise&
E a mão da meretriz caia aos pedaços!


### II


O sol agora é de um fulgor compacto,&
E eu vou andando, cheio de chamusco,&
Com a flexibilidade de um molusco,&
Úmido, pegajoso e untuoso ao tacto!

Reúnam-se em rebelião ardente e acesa&
Todas as minhas forças emotivas&
E armem ciladas como cobras vivas&
Para despedaçar minha tristeza!

O sol de cima espiando a flora moça&
Arda, fustigue, queime, corte, morda!\ldots{}&
Deleito a vista na verdura gorda&
Que nas hastes delgadas se balouça!

Avisto o vulto das sombrias granjas&
Perdidas no alto\ldots{} Nos terrenos baixos,&
Das laranjeiras eu admiro os cachos&
E a ampla circunferência das laranjas.

Ladra furiosa a tribo dos podengos.&
Olhando para as pútridas charnecas&
Grita o exército avulso das marrecas&
Na úmida copa dos bambus verdoengos.

Um pássaro alvo artífice da teia&
De um ninho, salta, no árdego trabalho,&
De árvore em árvore e de galho em galho,&
Com a rapidez duma semicolcheia.

Em grandes semicírculos aduncos,&
Entrançados, pelo ar, largando pelos,&
Voam à semelhança de cabelos&
Os chicotes finíssimos dos juncos.

Os ventos vagabundos batem, bolem&
Nas árvores. O ar cheira. A terra cheira\ldots{}&
E a alma dos vegetais rebenta inteira&
De todos os corpúsculos do pólen.

A câmara nupcial de cada ovário&
Se abre. No chão coleia a lagartixa.&
Por toda a parte a seiva bruta esguicha&
Num extravasamento involuntário.

Eu, depois de morrer, depois de tanta&
Tristeza, quero, em vez do nome --- \textit{Augusto},&
Possuir aí o nome dum arbusto&
Qualquer ou de qualquer obscura planta!


### III


Pelo acidentadíssimo caminho&
Faísca o sol. Nédios, batendo a cauda,&
Urram os bois. O céu lembra uma lauda&
Do mais incorruptível pergaminho.

Uma atmosfera má de incômoda hulha&
Abafa o ambiente. O aziago ar morto à morte&
Fede. O ardente calor da areia forte&
Racha-me os pés como se fosse agulha.

Não sei que subterrânea e atra voz rouca,&
Por saibros e por cem côncavos vales,&
Como pela avenida das Mappales,&		%nota para Mappales
Me arrasta à casa do finado \textit{Tôca}!

Todas as tardes a esta casa venho.&
Aqui, outrora, sem conchego nobre,&
Viveu, sentiu e amou este homem pobre&
Que carregava canas para o engenho!

Nos outros tempos e nas outras eras,&
Quantas flores! Agora, em vez de flores,&
Os musgos, como exóticos pintores,&
Pintam caretas verdes nas taperas.

Na bruta dispersão de vítreos cacos,&
À dura luz do sol resplandecente,&
Trôpega e antiga, uma parede doente&
Mostra a cara medonha dos buracos.

O cupim negro broca o âmago fino&
Do teto. E traça trombas de elefantes&
Com as circunvoluções extravagantes&
Do seu complicadíssimo intestino.

O lodo obscuro trepa-se nas portas.&
Amontoadas em grossos feixes rijos,&
As lagartixas dos esconderijos&
Estão olhando aquelas coisas mortas!

Fico a pensar no Espírito disperso&
Que unindo a pedra ao \textit{gneiss} e a árvore à criança,&	%nota para gneiss
Como um anel enorme de aliança,&
Une todas as coisas do Universo!

E assim pensando, com a cabeça em brasas&
Ante a fatalidade que me oprime,&
Julgo ver este Espírito sublime,&
Chamando-me do sol com as suas asas!

Gosto do sol ignívomo e iracundo&
Como o réptil gosta quando se molha&
E na atra escuridão dos ares, olha&
Melancolicamente para o mundo!

Essa alegria imaterializada,&
Que por vezes me absorve, é o óbolo obscuro,&
É o pedaço já podre de pão duro&
Que o miserável recebeu na estrada!

Não são os cinco mil milhões de francos&
Que a Alemanha pediu a Jules Favre\ldots{}&		%nota para Jules Favre
É o dinheiro coberto de azinhavre&
Que o escravo ganha, trabalhando aos brancos!

Seja este sol meu último consolo;&
E o espírito infeliz que em mim se encarna&
Se alegre ao sol, como quem raspa a sarna,&
Só, com a misericórdia de um tijolo!\ldots{}

Tudo enfim a mesma órbita percorre&
E as bocas vão beber o mesmo leite\ldots{}&
A lamparina quando falta o azeite&
Morre, da mesma forma que o homem morre.

Súbito, arrebentando a horrenda calma,&
Grito, e se grito é para que meu grito&
Seja a revelação deste Infinito&
Que eu trago encarcerado na minh’alma!

Sol brasileiro! Queima-me os destroços!&
Quero assistir, aqui, sem pai que me ame,&
De pé, à luz da consciência infame,&
À carbonização dos próprios ossos!


{\raggedleft\itshape
Pau d’Arco, 4--5--1907
\par}


\chapter{Versos de amor}

{\raggedleft\itshape
A um poeta erótico
\par}


Parece muito doce aquela cana.&
Descasco-a, provo-a, chupo-a\ldots{} ilusão treda!&
O amor, poeta, é como a cana azeda,&
A toda a boca que o não prova engana.

Quis saber que era o amor, por experiência,&
E hoje que, enfim, conheço o seu conteúdo,&
Pudera eu ter, eu que idolatro o estudo,&
Todas as ciências menos esta ciência!

Certo, este o amor não é que, em ânsias, amo&
Mas certo, o egoísta amor este é que acinte&
Amas, oposto a mim. Por conseguinte&
Chamas amor aquilo que eu não chamo.

Oposto ideal ao meu ideal conservas.&
Diverso é, pois, o ponto outro de vista&
Consoante o qual observo o amor, do egoísta&
Modo de ver, consoante o qual o observas.

Porque o amor, tal como eu o estou amando,&
É Espírito, é éter, é substância fluida,&
É assim como o ar que a gente pega e cuida,&
Cuida, entretanto, não o estar pegando!

É a transubstanciação de instintos rudes,&
Imponderabilíssima e impalpável,&
Que anda acima da carne miserável&
Como anda a garça acima dos açudes!

Para reproduzir tal sentimento&
Daqui por diante, atenta a orelha cauta,&
Como Mársias --- o inventor da flauta ---&
Vou inventar também outro instrumento!

Mas de tal arte e espécie tal fazê-lo&
Ambiciono, que o idioma em que te eu falo&
Possam todas as línguas decliná-lo&
Possam todos os homens compreendê-lo!

Para que, enfim, chegando à última calma&
Meu podre coração roto não role,&
Integralmente desfibrado e mole,&
Como um saco vazio dentro d’alma!


{\raggedleft\itshape
Pau d’Arco, agosto, 1907
\par}


\chapter{Sonetos}

### I

{\raggedleft\itshape
A meu pai doente
\par}


Para onde fores, pai, para onde fores,&
Irei também, trilhando as mesmas ruas\ldots{}&
Tu, para amenizar as dores tuas,&
Eu, para amenizar as minhas dores!

Que coisa triste! O campo tão sem flores,&
E eu tão sem crença e as árvores tão nuas&
E tu, gemendo, e o horror de nossas duas&
Mágoas crescendo e se fazendo horrores!

Magoaram-te, meu pai?! Que mão sombria,&
Indiferente aos mil tormentos teus&
De assim magoar-te sem pesar havia?!

--- Seria a mão de Deus?! Mas Deus enfim&
É bom, é justo, e sendo justo, Deus,&
Deus não havia de magoar-te assim!


### II

{\raggedleft\itshape
A meu pai morto
\par}


Madrugada de Treze de Janeiro.&
Rezo, sonhando, o ofício da agonia.&
Meu pai nessa hora junto a mim morria&
Sem um gemido, assim como um cordeiro!

E eu nem lhe ouvi o alento derradeiro!&
Quando acordei, cuidei que ele dormia,&
E disse à minha mãe que me dizia:&
“Acorda-o”! deixa-o, mãe, dormir primeiro!

E saí para ver a Natureza!&
Em tudo o mesmo abismo de beleza,&
Nem uma névoa no estrelado véu\ldots{}

Mas pareceu-me, entre as estrelas flóreas,&
Como Elias, num carro azul de glórias,&
Ver a alma de meu pai subindo ao céu!


### III


Podre meu pai! A morte o olhar lhe vidra.&
Em seus lábios que os meus lábios osculam&
Micro-organismos fúnebres pululam&
Numa fermentação gorda de cidra.

Duras leis as que os homens e a hórrida hidra&
A uma só lei biológica vinculam,&
E a marcha das moléculas regulam,&
Com a invariabilidade da clepsidra!\ldots{}

Podre meu pai! E a mão que enchi de beijos&
Roída toda de bichos, como os queijos&
Sobre a mesa de orgíacos festins!\ldots{}

Amo meu pai na atômica desordem&
Entre as bocas necrófagas que o mordem&
E a terra infecta que lhe cobre os rins!



\chapter{Depois da orgia}


O prazer que na orgia a hetaíra goza&
Produz no meu \textit{sensorium} de bacante&
O efeito de uma túnica brilhante&
Cobrindo ampla apostema escrofulosa!

Troveja! E anelo ter, sôfrega e ansiosa,&
O sistema nervoso de um gigante&
Para sofrer na minha carne estuante&
A dor da força cósmica furiosa.

Apraz-me, enfim, despindo a última alfaia&
Que ao comércio dos homens me traz presa,&
Livre deste cadeado de peçonha,

Semelhante a um cachorro de atalaia&
Às decomposições da Natureza,&
Ficar latindo minha dor medonha!



\chapter{A árvore da serra}


--- As árvores, meu filho, não têm alma!&
E esta árvore me serve de empecilho\ldots{}&
É preciso cortá-la, pois, meu filho,&
Para que eu tenha uma velhice calma!

--- Meu pai, por que sua ira não se acalma?!&
Não vê que em tudo existe o mesmo brilho?!&
Deus pôs almas nos cedros\ldots{} no junquilho\ldots{}&
Esta árvore, meu pai, possui minh’alma!\ldots{}

--- Disse --- e ajoelhou-se, numa rogativa:&
“Não mate a árvore, pai, para que eu viva!”&
E quando a árvore, olhando a pátria serra,

Caiu aos golpes do machado bronco,&
O moço triste se abraçou com o tronco&
E nunca mais se levantou da terra!



\chapter{Vencido}


No auge de atordoadora e ávida sanha&
Leu tudo, desde o mais prístino mito,&
Por exemplo: o do boi Ápis do Egito&
Ao velho Niebelungen da Alemanha.

Acometido de uma febre estranha&
Sem o escândalo fônico de um grito,&
Mergulhou a cabeça no Infinito,&
Arrancou os cabelos na montanha!

Desceu depois à gleba mais bastarda,&
Pondo a áurea insígnia heráldica da farda&
À vontade do vômito plebeu\ldots{}

E ao vir-lhe o cuspo diário à boca fria&
O vencido pensava que cuspia&
Na célula infeliz de onde nasceu.


{\raggedleft\itshape
Paraíba, 1909
\par}


\chapter{O corrupião}


Escaveirado corrupião idiota,&
Olha a atmosfera livre, o amplo éter belo,&
E a alga criptógama e a úsnea e o cogumelo,&
Que do fundo do chão todo o ano brota!

Mas a ânsia de alto voar, de à antiga rota&
Voar, não tens mais! E pois, preto e amarelo,&
Pões-te a assobiar, bruto, sem cerebelo&
A gargalhada da última derrota!

A gaiola aboliu tua vontade.&
Tu nunca mais verás a liberdade!\ldots{}&
Ah! Tu somente ainda és igual a mim.

Continua a comer teu milho alpiste.&
Foi este mundo que me fez tão triste,&
Foi a gaiola que te pôs assim!



\chapter{Noite de um visionário}


Número cento e três. Rua Direita.&
Eu tinha a sensação de quem se esfola&
E inopinadamente o corpo atola&
Numa poça de carne liquefeita!

--- “Que esta alucinação tátil não cresça!”&
--- Dizia; e erguia, oh! céu, alto, por ver-vos,&
Com a rebeldia acérrima dos nervos&
Minha atormentadíssima cabeça.

É a potencialidade que me eleva&
Ao grande Deus, e absorve em cada viagem&
Minh’alma --- este sombrio personagem&
Do drama panteístico da treva!

Depois de dezesseis anos de estudo&
Generalizações grandes e ousadas&
Traziam minhas forças concentradas&
Na compreensão monística de tudo.

Mas a aguadilha pútrida o ombro inerme&
Me aspergia, banhava minhas tíbias,&
E a ela se aliava o ardor das sirtes líbias,&
Cortando o melanismo da epiderme.

Arimânico gênio destrutivo&
Desconjuntava minha autônoma alma&
Esbandalhando essa unidade calma,&
Que forma a coerência do ser vivo.

E eu saí a tremer com a língua grossa&
E a volição no cúmulo do exício,&
Como quem é levado para o hospício&
Aos trambolhões, num canto de carroça!

Perante o inexorável céu aceso&
Agregações abióticas espúrias,&
Como uma cara, recebendo injúrias,&
Recebiam os cuspos do desprezo.

A essa hora, nas telúricas reservas,&
O reino mineral americano&
Dormia, sob os pés do orgulho humano,&
E a cimalha minúscula das ervas.

E não haver quem, íntegra, lhe entregue,&
Com os ligamentos glóticos precisos,&
A liberdade de vingar em risos&
A angústia milenária que o persegue!

Bulia nos obscuros labirintos&
Da fértil terra gorda, úmida e fresca,&
A ínfima fauna abscôndita e grotesca&
Da família bastarda dos helmintos.

As vegetalidades subalternas&
Que os serenos noturnos orvalhavam,&
Pela alta frieza intrínseca, lembravam&
Toalhas molhadas sobre as minhas pernas.

E no estrume fresquíssimo da gleba&
Formigavam, com a símplice sarcode,&
O vibrião, o ancilóstomo, o colpode&
E outros irmãos legítimos da ameba!

E todas essas formas que Deus lança&
No Cosmos, me pediam, com o ar horrível,&
Um pedaço de língua disponível&
Para a filogenética vingança!

A cidade exalava um podre báfio:&
Os anúncios das casas de comércio,&
Mais tristes que as elegias de Propércio,&
Pareciam talvez meu epitáfio.

O motor teleológico da Vida&
Parara! Agora, em diástoles de guerra,&
Vinha do coração quente da terra&
Um rumor de matéria dissolvida.

A química feroz do cemitério&
Transformava porções de átomos juntos&
No óleo malsão que escorre dos defuntos,&
Com a abundância de um geiser deletério.

Dedos denunciadores escreviam&
Na lúgubre extensão da rua preta&
Todo o destino negro do planeta,&
Onde minhas moléculas sofriam.

Um necrófilo mau forçava as lousas&
E eu --- coetâneo do horrendo \mbox{cataclismo ---}&
Era puxado para aquele abismo&
No redemoinho universal das cousas!



\chapter{Alucinação à beira-mar}


Um medo de morrer meus pés esfriava.&
Noite alta. Ante o telúrico recorte,&
Na diuturna discórdia, a equórea coorte&
Atordoadoramente ribombava!

Eu, ególatra cético, cismava&
Em meu destino!\ldots{} O vento estava forte&
E aquela matemática da Morte&
Com os seus números negros, me assombrava!

Mas a alga usufrutuária dos oceanos&
E os malacopterígios sub-raquianos& 
Que um castigo de espécie emudeceu,

No eterno horror das convulsões marítimas&
Pareciam também corpos de vítimas&
Condenadas à Morte, assim como eu!



\chapter{Vandalismo}


Meu coração tem catedrais imensas,&
Templos de priscas e longínquas datas,&
Onde um nume de amor, em serenatas,&
Canta a aleluia virginal das crenças.

Na ogiva fúlgida e nas colunatas&
Vertem lustrais irradiações intensas&
Cintilações de lâmpadas suspensas&
E as ametistas e os florões e as pratas.

Como os velhos Templários medievais&
Entrei um dia nessas catedrais&
E nesses templos claros e risonhos\ldots{}

E erguendo os gládios e brandindo as hastas,&
No desespero dos iconoclastas&
Quebrei a imagem dos meus próprios sonhos!


{\raggedleft\itshape
Pau d’Arco, 1904
\par}


\chapter{Versos íntimos}


Vês! Ninguém assistiu ao formidável&
Enterro de tua última quimera.&
Somente a Ingratidão --- esta pantera ---&
Foi tua companheira inseparável!

Acostuma-te à lama que te espera!&
O Homem, que, nesta terra miserável,&
Mora, entre feras, sente inevitável&
Necessidade de também ser fera.

Toma um fósforo. Acende teu cigarro!&
O beijo, amigo, é a véspera do escarro,&
A mão que afaga é a mesma que apedreja.

Se a alguém causa inda pena a tua chaga,&
Apedreja essa mão vil que te afaga,&
Escarra nessa boca que te beija!


{\raggedleft\itshape
Pau d’Arco, 1901
\par}


\chapter{Vencedor}


Toma as espadas rútilas, guerreiro,&
E à rutilância das espadas, toma&
A adaga de aço, o gládio de aço, e doma&
Meu coração --- estranho carniceiro!

Não podes?! Chama então presto o primeiro&
E o mais possante gladiador de Roma.&
E qual mais pronto, e qual mais presto assoma&
Nenhum pôde domar o prisioneiro.

Meu coração triunfava nas arenas.&
Veio depois um domador de hienas&
E outro mais, e, por fim, veio um atleta,

Vieram todos, por fim; ao todo, uns cem\ldots{}&
E não pôde domá-lo enfim ninguém,&
Que ninguém doma um coração de poeta!


{\raggedleft\itshape
Pau d’Arco, 1902
\par}


\chapter{A ilha de Cipango}


Estou sozinho! A estrada se desdobra&
Como uma imensa e rutilante cobra&
De epiderme finíssima de areia\ldots{}&
E por essa finíssima epiderme&
Eis-me passeando como um grande verme&
Que, ao sol, em plena podridão, passeia!

A agonia do sol vai ter começo!&
Caio de joelhos, trêmulo\ldots{} Ofereço&
Preces a Deus de amor e de respeito&
E o Ocaso que nas águas se retrata&
Nitidamente reproduz, exata,&
A saudade interior que há no meu peito\ldots{}

Tenho alucinações de toda a sorte\ldots{}&
Impressionado sem cessar com a Morte&
E sentindo o que um lázaro não sente,&
Em negras nuanças lúgubres e aziagas&
Vejo terribilíssimas adagas,&
Atravessando os ares bruscamente.

Os olhos volvo para o céu divino&
E observo-me pigmeu e pequenino&
Através de minúsculos espelhos.&
Assim, quem diante duma cordilheira,&
Para, entre assombros, pela vez primeira,&
Sente vontade de cair de joelhos!

Soa o rumor fatídico dos ventos,&
Anunciando desmoronamentos&
De mil lajedos sobre mil lajedos\ldots{}&
E ao longe soam trágicos fracassos&
De heróis, partindo e fraturando os braços&
Nas pontas escarpadas dos rochedos!

Mas de repente, num enleio doce,&
Qual se num sonho arrebatado fosse,&
Na ilha encantada de Cipango tombo,&
Da qual, no meio, em luz perpétua, brilha&
A árvore da perpétua maravilha,&
À cuja sombra descansou Colombo!

Foi nessa ilha encantada de Cipango,&
Verde, afetando a forma de um losango,&
Rica, ostentando amplo floral risonho,&
Que Toscanelli viu seu sonho extinto&
E como sucedeu a Afonso Quinto&
Foi sobre essa ilha que extingui meu sonho!

Lembro-me bem. Nesse maldito dia&
O gênio singular da Fantasia&
Convidou-me a sorrir para um passeio\ldots{}&
Iríamos a um país de eternas pazes&
Onde em cada deserto há mil oásis&
E em cada rocha um cristalino veio.

Gozei numa hora séculos de afagos,&
Banhei-me na água de risonhos lagos,&
E finalmente me cobri de flores\ldots{}&
Mas veio o vento que a Desgraça espalha&
E cobriu-me com o pano da mortalha,&
Que estou cosendo para os meus amores!

Desde então para cá fiquei sombrio!&
Um penetrante e corrosivo frio&
Anestesiou-me a sensibilidade&
E a grandes golpes arrancou as raízes&
Que prendiam meus dias infelizes&
A um sonho antigo de felicidade!

Invoco os Deuses salvadores do erro.&
A tarde morre. Passa o seu enterro!\ldots{}&
A luz descreve ziguezagues tortos&
Enviando à terra os derradeiros beijos.&
Pela estrada feral dois realejos&
Estão chorando meus amores mortos!

E a treva ocupa toda a estrada longa\ldots{}&
O Firmamento é uma caverna oblonga&
Em cujo fundo a Via-Láctea existe.&
E como agora a lua cheia brilha!&
Ilha maldita vinte vezes a ilha&
Que para todo o sempre me fez triste!


{\raggedleft\itshape
Pau d’Arco, 1904
\par}


\chapter{\textit{Mater}}


Como a crisálida emergindo do ovo&
Para que o campo flórido a concentre,&
Assim, oh! Mãe, sujo de sangue, um novo&
Ser, entre dores, te emergiu do ventre!

E puseste-lhe, haurindo amplo deleite,&
No lábio róseo a grande teta farta&
--- Fecunda fonte desse mesmo leite&
Que amamentou os efebos de Esparta. ---

Com que avidez ele essa fonte suga!&
Ninguém mais com a Beleza está de acordo,&
Do que essa pequenina sanguessuga,&
Bebendo a vida no teu seio gordo!

Pois, quanto a mim, sem pretensões, comparo,&
Essas humanas coisas pequeninas&
A um \textit{biscuit} de quilate muito raro&
Exposto aí, à amostra, nas vitrinas.

Mas o ramo fragílimo e venusto&
Que hoje nas débeis gêmulas se esboça,&
Há de crescer, há de tornar-se arbusto&
E álamo altivo de ramagem grossa.

Clara, a atmosfera se encherá de aromas,&
O Sol virá das épocas sadias\ldots{}&
E o antigo leão, que te esgotou as pomas,&
Há de beijar-te as mãos todos os dias!

Quando chegar depois tua velhice&
Batida pelos bárbaros invernos,&
Relembrarás chorando o que eu te disse,&
À sombra dos sicômoros eternos!


{\raggedleft\itshape
Pau d’Arco, 1905
\par}


\chapter{Poema negro}

{\raggedleft\itshape
A Santos Neto
\par}


Para iludir minha desgraça, estudo.&
Intimamente sei que não me iludo.&
Para onde vou (o mundo inteiro o nota)&
Nos meus olhares fúnebres, carrego&
A indiferença estúpida de um cego&
E o ar indolente de um chinês idiota!

A passagem dos séculos me assombra.&
Para onde irá correndo minha sombra&
Nesse cavalo de eletricidade?!&
Caminho, e a mim pergunto, na vertigem:&
--- Quem sou? Para onde vou? Qual minha origem?&
E parece-me um sonho a realidade.

Em vão com o grito do meu peito impreco!&
Dos brados meus ouvindo apenas o eco,&
Eu torço os braços numa angústia doida&
E muita vez, à meia-noite, rio&
Sinistramente, vendo o verme frio&
Que há de comer a minha carne toda!

É a Morte --- esta carnívora assanhada ---&
Serpente má de língua envenenada&
Que tudo que acha no caminho, come\ldots{}&
--- Faminta e atra mulher que, a 1 de Janeiro,&
Sai para assassinar o mundo inteiro,&
E o mundo inteiro não lhe mata a fome!

Nesta sombria análise das cousas,&
Corro. Arranco os cadáveres das lousas&
E as suas partes podres examino\ldots{}&
Mas de repente, ouvindo um grande estrondo,&
Na podridão daquele embrulho hediondo&
Reconheço assombrado o meu Destino!

Surpreendo-me, sozinho, numa cova.&
Então meu desvario se renova\ldots{}&
Como que, abrindo todos os jazigos,&
A Morte, em trajos pretos e amarelos,&
Levanta contra mim grandes cutelos&
E as baionetas dos dragões antigos!

E quando vi que aquilo vinha vindo&
Eu fui caindo como um sol caindo&
De declínio em declínio; e de declínio&
Em declínio, com a gula de uma fera,&
Quis ver o que era, e quando vi o que era,&
Vi que era pó, vi que era esterquilínio!

Chegou a tua vez, oh! Natureza!&
Eu desafio agora essa grandeza,&
Perante a qual meus olhos se extasiam\ldots{}&
Eu desafio, desta cova escura,&
No histerismo danado da tortura&
Todos os monstros que os teus peitos criam.

Tu não és minha mãe, velha nefasta!&
Com o teu chicote frio de madrasta&
Tu me açoitaste vinte e duas vezes\ldots{}&
Por tua causa apodreci nas cruzes,&
Em que pregas os filhos que produzes&
Durante os desgraçados nove meses!

Semeadora terrível de defuntos,&
Contra a agressão dos teus contrastes juntos&
A besta, que em mim dorme, acorda em berros;&
Acorda, e após gritar a última injúria,&
Chocalha os dentes com medonha fúria&
Como se fosse o atrito de dois ferros!

Pois bem! Chegou minha hora de vingança.&
Tu mataste o meu tempo de criança&
E de segunda-feira até domingo,&
Amarrado no horror de tua rede,&
Deste-me fogo quando eu tinha sede\ldots{}&
Deixa-te estar, canalha, que eu me vingo!

Súbito outra visão negra me espanta!&
Estou em Roma. É Sexta-feira Santa.&
A treva invade o obscuro orbe terrestre.&
No Vaticano, em grupos prosternados,&
Com as longas fardas rubras, os soldados&
Guardam o corpo do Divino Mestre.

Como as estalactites da caverna,&
Cai no silêncio da Cidade Eterna&
A água da chuva em largos fios grossos\ldots{}&
De Jesus Cristo resta unicamente&
Um esqueleto; e a gente, vendo-o, a gente&
Sente vontade de abraçar-lhe os ossos!

Não há ninguém na estrada da Ripetta.&
Dentro da igreja de São Pedro, quieta,&
As luzes funerais arquejam fracas\ldots{}&
O vento entoa cânticos de morte.&
Roma estremece! Além, num rumor forte,&
Recomeça o barulho das matracas.

A desagregação da minha Ideia&
Aumenta. Como as chagas da morfeia,&
O medo, o desalento e o desconforto&
Paralisam-me os círculos motores.&
Na Eternidade, os ventos gemedores&
Estão dizendo que Jesus é morto!

Não! Jesus não morreu! Vive na serra&
Da Borborema, no ar de minha terra,&
Na molécula e no átomo\ldots{} Resume&
A espiritualidade da matéria&
E ele é que embala o corpo da miséria&
E faz da cloaca uma urna de perfume.

Na agonia de tantos pesadelos&
Uma dor bruta puxa-me os cabelos.&
Desperto. É tão vazia a minha vida!&
No pensamento desconexo e falho&
Trago as cartas confusas de um baralho&
E um pedaço de cera derretida!

Dorme a casa. O céu dorme. A árvore dorme,&
Eu, somente eu, com a minha dor enorme&
Os olhos ensanguento na vigília!&
E observo, enquanto o horror me corta a fala,&
O aspecto sepulcral da austera sala&
E a impassibilidade da mobília.

Meu coração, como um cristal, se quebre;&
O termômetro negue minha febre,&
Torne-se gelo o sangue que me abrasa,&
E eu me converta na cegonha triste&
Que das ruínas duma casa assiste&
Ao desmoronamento de outra casa!

Ao terminar este sentido poema&
Onde vazei a minha dor suprema&
Tenho os olhos em lágrimas imersos\ldots{}&
Rola-me na cabeça o cérebro oco.&
Porventura, meu Deus, estarei louco?!&
Daqui por diante não farei mais versos.


{\raggedleft\itshape
Paraíba, 1906
\par}


\chapter{Eterna mágoa}


O homem por sobre quem caiu a praga&
Da tristeza do Mundo, o homem que é triste&
Para todos os séculos existe&
E nunca mais o seu pesar se apaga!

Não crê em nada, pois nada há que traga&
Consolo à Mágoa, a que só ele assiste.&
Quer resistir, e quanto mais resiste&
Mais se lhe aumenta e se lhe afunda a chaga.

Sabe que sofre, mas o que não sabe&
É que essa mágoa infinda assim, não cabe&
Na sua vida, é que essa mágoa infinda

Transpõe a vida do seu corpo inerme;&
E quando esse homem se transforma em verme&
É essa mágoa que o acompanha ainda!


{\raggedleft\itshape
Pau d’Arco, 1904
\par}


\chapter{Queixas noturnas}


Quem foi que viu a minha Dor chorando?!&
Saio. Minh’alma sai agoniada.&
Andam monstros sombrios pela estrada&
E pela estrada, entre estes monstros, ando!

Não trago sobre a túnica fingida&
As insígnias medonhas do infeliz&
Como os falsos mendigos de Paris&
Na atra rua de Santa Margarida.

O quadro de aflições que me consomem&
O próprio Pedro Américo não pinta\ldots{}&
Para pintá-lo, era preciso a tinta&
Feita de todos os tormentos do homem!

Como um ladrão sentado numa ponte&
Espera alguém, armado de arcabuz,&
Na ânsia incoercível de roubar a luz,&
Estou à espera de que o Sol desponte!

Bati nas pedras dum tormento rude&
E a minha mágoa de hoje é tão intensa&
Que eu penso que a Alegria é uma doença&
E a Tristeza é minha única saúde.

As minhas roupas, quero até rompê-las!&
Quero, arrancado das prisões carnais,&
Viver na luz dos astros imortais,&
Abraçado com todas as estrelas!

A Noite vai crescendo apavorante&
E dentro do meu peito, no combate,&
A Eternidade esmagadora bate&
Numa dilatação exorbitante!

E eu luto contra a universal grandeza&
Na mais terrível desesperação\ldots{}&
É a luta, é o prélio enorme, é a rebelião&
Da criatura contra a natureza!

Para essas lutas uma vida é pouca&
Inda mesmo que os músculos se esforcem;&
Os pobres braços do mortal se torcem&
E o sangue jorra, em coalhos, pela boca.

E muitas vezes a agonia é tanta&
Que, rolando dos últimos degraus,&
O Hércules treme e vai tombar no caos&
De onde seu corpo nunca mais levanta!

É natural que esse Hércules se estorça,&
E tombe para sempre nessas lutas,&
Estrangulado pelas rodas brutas&
Do mecanismo que tiver mais força.

Ah! Por todos os séculos vindouros&
Há de travar-se essa batalha vã&
Do dia de hoje contra o de amanhã,&
Igual à luta dos cristãos e mouros!

Sobre histórias de amor o interrogar-me&
É vão, é inútil, é improfícuo, em suma;&
Não sou capaz de amar mulher alguma&
Nem há mulher talvez capaz de amar-me.

O amor tem favos e tem caldos quentes&
E ao mesmo tempo que faz bem, faz mal;&
O coração do Poeta é um hospital&
Onde morreram todos os doentes.

Hoje é amargo tudo quanto eu gosto:&
A bênção matutina que recebo\ldots{}&
E é tudo: o pão que como, a água que bebo,&
O velho tamarindo a que me encosto!

Vou enterrar agora a harpa boêmia&
Na atra e assombrosa solidão feroz&
Onde não cheguem o eco duma voz&
E o grito desvairado da blasfêmia!

Que dentro de minh’alma americana&
Não mais palpite o coração --- esta arca,&
Este relógio trágico que marca&
Todos os atos da tragédia humana! ---

Seja esta minha queixa derradeira&
Cantada sobre o túmulo de Orfeu;&
Seja este, enfim, o último canto meu&
Por esta grande noite brasileira!

Melancolia! Estende-me tu’asa!&
És a árvore em que devo reclinar-me\ldots{}&
Se algum dia o Prazer vier procurar-me&
Dize a este monstro que fugi de casa!


{\raggedleft\itshape
Pau d’Arco, 1906
\par}


\chapter{Insônia}


Noite. Da Mágoa o espírito noctâmbulo&
Passou de certo por aqui chorando!&
Assim, em mágoa, eu também vou passando&
Sonâmbulo\ldots{} sonâmbulo\ldots{} sonâmbulo\ldots{}

Que voz é esta que a gemer concentro&
No meu ouvido e que do meu ouvido&
Como um bemol e como um sustenido&
Rola impetuosa por meu peito adentro?!

--- Por que é que este gemido me acompanha?!&
Mas dos meus olhos no sombrio palco&
Súbito surge como um catafalco&
Uma cidade ao mapa-múndi estranha.

A dispersão dos sonhos vagos reúno.&
Desta cidade pelas ruas erra&
A procissão dos Mártires da Terra&
Desde os Cristãos até Giordano Bruno!

Vejo diante de mim Santa Francisca&
Que com o cilício as tentações suplanta,&
E invejo o sofrimento desta Santa,&
Em cujo olhar o Vício não faísca!

Se eu pudesse ser puro! Se eu pudesse,&
Depois de embebedado deste vinho,&
Sair da vida puro como o arminho&
Que os cabelos dos velhos embranquece!

Por que cumpri o universal ditame?!&
Pois se eu sabia onde morava o Vício,&
Por que não evitei o precipício&
Estrangulando minha carne infame?!

Até que dia o intoxicado aroma&
Das paixões torpes sorverei contente?&
E os dias correrão eternamente?!&
E eu nunca sairei desta Sodoma?!

À proporção que a minha insônia aumenta&
Hieróglifos e esfinges interrogo\ldots{}&
Mas, triunfalmente, nos céus altos, logo&
Toda a alvorada esplêndida se ostenta.

Vagueio pela Noite decaída\ldots{}&
No espaço a luz de Aldebarã e de Argos&
Vai projetando sobre os campos largos&
O derradeiro fósforo da Vida.

O Sol, equilibrando-se na esfera,&
Restitui-me a pureza da hematose&
E então uma interior metamorfose&
Nas minhas arcas cerebrais se opera.

O odor da margarida e da begônia&
Subitamente me penetra o olfato\ldots{}&
Aqui, neste silêncio e neste mato,&
Respira com vontade a alma campônia!

Grita a satisfação na alma dos bichos.&
Incensa o ambiente o fumo dos cachimbos.&
As árvores, as flores, os corimbos,&
Recordam santos nos seus próprios nichos.

Com o olhar a verde periferia abarco.&
Estou alegre. Agora, por exemplo,&
Cercado destas árvores, contemplo&
As maravilhas reais do meu Pau d’Arco!

Cedo virá, porém, o funerário,&
Atro dragão da escura noite, hedionda,&
Em que o Tédio, batendo na alma, estronda&
Como um grande trovão extraordinário.

Outra vez serei pábulo do susto&
E terei outra vez de, em mágoa imerso,&
Sacrificar-me por amor do Verso&
No meu eterno leito de Procusto!


{\raggedleft\itshape
Pau d’Arco, 1905
\par}


\chapter{Barcarola}


Cantam nautas, choram flautas&
Pelo mar e pelo mar&
Uma sereia a cantar&
Vela o Destino dos nautas.

Espelham-se os esplendores&
Do céu, em reflexos, nas&
Águas, fingindo cristais&
Das mais deslumbrantes cores.

Em fulvos filões dourados&
Cai a luz dos astros por&
Sobre o marítimo horror&
Como globos estrelados.

Lá onde as rochas se assentam&
Fulguram como outros sóis&
Os flamívomos faróis&
Que os navegantes orientam.

Vai uma onda, vem outra onda&
E nesse eterno vaivém&
Coitadas! não acham quem,&
Quem as esconda, as esconda\ldots{}

Alegoria tristonha&
Do que pelo Mundo vai!&
Se um sonha e se ergue, outro cai;&
Se um cai, outro se ergue e sonha.

Mas desgraçado do pobre&
Que em meio da Vida cai!&
Esse não volta, esse vai&
Para o túmulo que o cobre.

Vagueia um poeta num barco.&
O Céu, de cima, a luzir&
Como um diamante de Ofir&
Imita a curva de um arco.

A Lua --- globo de louça ---&
Surgiu, em lúcido véu.&
Cantam! Os astros do Céu&
Ouçam e a Lua Cheia ouça!

Ouço do alto a Lua Cheia&
Que a sereia vai falar\ldots{}&
Haja silêncio no mar&
Para se ouvir a sereia.

Que é que ela diz?! Será uma&
História de amor feliz?&
Não! O que a sereia diz&
Não é história nenhuma.

É como um réquiem profundo&
De tristíssimos bemóis\ldots{}&
Sua voz é igual à voz&
Das dores todas do mundo!

“Fecha-te nesse medonho&
Reduto de Maldição,&
Viajeiro da Extrema-unção,&
Sonhador do último sonho!

Numa redoma ilusória&
Cercou-te a glória falaz,&
Mas nunca mais, nunca mais&
Há de cercar-te essa glória!

Nunca mais! Sê, porém, forte.&
O poeta é como Jesus!&
Abraça-te à tua Cruz&
E morre, poeta da Morte!”

--- E disse e porque isto disse&
O luar no Céu se apagou\ldots{}&
Súbito o barco tombou&
Sem que o poeta o pressentisse!

Vista de luto o Universo&
E Deus se enlute no Céu!&
Mais um poeta que morreu,&
Mais um coveiro do Verso!

Cantam nautas, choram flautas&
Pelo mar e pelo mar&
Uma sereia a cantar&
Vela o Destino dos nautas!



\chapter[Tristezas de um quarto minguante]{Tristezas de um\break quarto minguante}


Quarto minguante! E, embora a lua o aclare,&
Este \textit{Engenho Pau d’Arco} é muito triste\ldots{}&
Nos engenhos da \textit{várzea} não existe&
Talvez um outro que se lhe equipare!

Do observatório em que eu estou situado&
A lua magra, quando a noite cresce,&
Vista, através do vidro azul, parece&
Um paralelepípedo quebrado!

O sono esmaga o encéfalo do povo.&
Tenho 300 quilos no epigastro\ldots{}&
Dói-me a cabeça. Agora a cara do astro&
Lembra a metade de uma casca de ovo.

Diabo! não ser mais tempo de milagre!&
Para que esta opressão desapareça&
Vou amarrar um pano na cabeça,&
Molhar a minha fronte com vinagre.

Aumentam-se-me então os grandes medos.&
O hemisfério lunar se ergue e se abaixa&
Num desenvolvimento de borracha,&
Variando à ação mecânica dos dedos!

Vai-me crescendo a aberração do sonho.&
Morde-me os nervos o desejo doido&
De dissolver-me, de enterrar-me todo&
Naquele semicírculo medonho!

Mas tudo isto é ilusão de minha parte!&
Quem sabe se não é porque não saio&
Desde que, 6ª feira, 3 de maio,&
Eu escrevi os meus Gemidos de Arte?!

A lâmpada a estirar línguas vermelhas&
Lambe o ar. No bruto horror que me arrebata,&
Como um degenerado psicopata&
Eis-me a contar o número das telhas!

--- Uma, duas, três, quatro\ldots{} E aos tombos, tonta&
Sinto a cabeça e a conta perco; e, em suma,&
A conta recomeço, em ânsias: --- Uma\ldots{}&
Mas novamente eis-me a perder a conta!

Sucede a uma tontura outra tontura.&
--- Estarei morto?! E a esta pergunta estranha&
Responde a Vida --- aquela grande aranha&
Que anda tecendo a minha desventura! ---

A luz do quarto diminuindo o brilho&
Segue todas as fases de um eclipse\ldots{}&
Começo a ver coisas de Apocalipse&
No triângulo escaleno do ladrilho!

Deito-me enfim. Ponho o chapéu num gancho.&
Cinco lençóis balançam numa corda,&
Mas aquilo mortalhas me recorda,&
E o amontoamento dos lençóis desmancho.

Vêm-me à imaginação sonhos dementes.&
Acho-me, por exemplo, numa festa\ldots{}&
Tomba uma torre sobre a minha testa,&
Caem-me de uma só vez todos os dentes!

Então dois ossos roídos me assombraram\ldots{}&
--- “Porventura haverá quem queira roer-nos?!&
Os vermes já não querem mais comer-nos&
E os formigueiros já nos desprezaram”.

Figuras espectrais de bocas tronchas&
Tornam-me o pesadelo duradouro\ldots{}&
Choro e quero beber a água do choro&
Com as mãos dispostas à feição de conchas.

Tal uma planta aquática submersa,&
Antegozando as últimas delícias&
Mergulho as mãos --- vis raízes adventícias ---&
No algodão quente de um tapete persa.

Por muito tempo rolo no tapete.&
Súbito me ergo. A lua é morta. Um frio&
Cai sobre o meu estômago vazio&
Como se fosse um copo de sorvete!

A alta frialdade me insensibiliza;&
O suor me ensopa. Meu tormento é infindo\ldots{}&
Minha família ainda está dormindo&
E eu não posso pedir outra camisa!

Abro a janela. Elevam-se fumaças&
Do engenho enorme. A luz fulge abundante&
E em vez do sepulcral Quarto minguante&
Vi que era o sol batendo nas vidraças.

Pelos respiratórios tênues tubos&
Dos poros vegetais, no ato da entrega&
Do mato verde, a terra resfolega&
Estrumada, feliz, cheia de adubos.

Côncavo, o céu, radiante e estriado, observa&
A universal criação. Broncos e feios,&
Vários répteis cortam os campos, cheios&
Dos tenros tinhorões e da úmida erva.

Babujada por baixos beiços brutos,&
No húmus feraz, hierática, se ostenta&
A monarquia da árvore opulenta&
Que dá aos homens o óbolo dos frutos.

De mim diverso, rígido e de rastos&
Com a solidez do tegumento sujo&
Sulca, em diâmetro, o solo um caramujo&
Naturalmente pelos mata-pastos.

Entretanto, passei o dia inquieto,&
A ouvir, nestes bucólicos retiros,&
Toda a salva fatal de 21 tiros&
Que festejou os funerais de Hamleto!

Ah! Minha ruína é pior do que a de Tebas!&
Quisera ser, numa última cobiça,&
A fatia esponjosa de carniça&
Que os corvos comem sobre as jurubebas!

Porque, longe do pão com que me nutres&
Nesta hora, oh! Vida, em que a sofrer me exortas&
Eu estaria como as bestas mortas&
Pendurado no bico dos abutres!


{\raggedleft\itshape
Pau d’Arco, maio 1907
\par}


\chapter{Mistérios de um fósforo}


Pego de um fósforo. Olho-o. Olho-o ainda. \mbox{Risco-o}&
Depois. E o que depois fica e depois&
Resta é um ou, por outra, é mais de um, são dois&
Túmulos dentro de um carvão promíscuo.

Dois são, porque um, certo, é do sonho assíduo&
Que a individual psique humana tece e&
O outro é o do sonho altruístico da espécie&
Que é o \textit{substractum} dos sonhos do indivíduo!

E exclamo, ébrio, a esvaziar báquicos odres:&
--- “Cinza, síntese má da podridão,&
Miniatura alegórica do chão,&
Onde os ventres maternos ficam podres;

Na tua clandestina e erma alma vasta,&
Onde nenhuma lâmpada se acende,&
Meu raciocínio sôfrego surpreende&
Todas as formas da matéria gasta!”

Raciocinar! Aziaga contingência!&
Ser quadrúpede! Andar de quatro pés&
É mais do que ser Cristo e ser Moisés&
Porque é ser animal sem ter consciência!

Bêbedo, os beiços na ânfora ínfima, harto,&
Mergulho, e na ínfima ânfora, harto, sinto&
O amargor específico do absinto&
E o cheiro animalíssimo do parto!

E afogo mentalmente os olhos fundos&
Na amorfia da cítula inicial,&
De onde, por epigênese geral,&
Todos os organismos são oriundos.

Presto, irrupto, através ovoide e hialino&
Vidro, aparece, amorfo e lúrido, ante&
Minha massa encefálica minguante&
Todo o gênero humano intrauterino!

É o caos da avita víscera avarenta&
--- Mucosa nojentíssima de pus,&
A nutrir diariamente os fetos nus&
Pelas vilosidades da placenta! ---

Certo, o arquitetural e íntegro aspecto&
Do mundo o mesmo inda é, que, ora, o que nele&
Morre, sou eu, sois vós, é todo aquele&
Que vem de um ventre inchado, ínfimo e infecto!

É a flor dos genealógicos abismos&
--- Zooplasma pequeníssimo e plebeu,&
De onde o desprotegido homem nasceu&
Para a fatalidade dos tropismos. ---

Depois, é o céu abscôndito do Nada,&
É este ato extraordinário de morrer&
Que há de, na última hebdômada, atender&
Ao pedido da célula cansada!

Um dia restará, na terra instável,&
De minha antropocêntrica matéria&
Numa côncava xícara funérea&
Uma colher de cinza miserável!

Abro na treva os olhos quase cegos.&
Que mão sinistra e desgraçada encheu&
Os olhos tristes que meu pai me deu&
De alfinetes, de agulhas e de pregos?!

Pesam sobre o meu corpo oitenta arráteis!&
Dentro um dínamo déspota, sozinho,&
Sob a morfologia de um moinho,&
Move todos os meus nervos vibráteis.

Então, do meu espírito, em segredo,&
Se escapa, dentre as tênebras, muito alto,&
Na síntese acrobática de um salto,&
O espectro angulosíssimo do Medo!

Em cismas filosóficas me perco&
E vejo, como nunca outro homem viu,&
Na anfigonia que me produziu&
Nonilhões de moléculas de esterco.

Vida, mônada vil, cósmico zero,&
Migalha de albumina semifluida,&
Que fez a boca mística do druida&
E a língua revoltada de Lutero;

Teus gineceus prolíficos envolvem&
Cinza fetal!\ldots{} Basta um fósforo só&
Para mostrar a incógnita de pó,&
Em que todos os seres se resolvem!

Ah! Maldito o conúbio incestuoso&
Dessas afinidades eletivas,&
De onde quimicamente tu derivas,&
Na aclamação simbiótica do gozo!

O enterro de minha última neurona&
Desfila\ldots{} E eis-me outro fósforo a riscar,& %cf. riscar.
E esse acidente químico vulgar&
Extraordinariamente me impressiona!

Mas minha crise artrítica não tarda.&
Adeus! Que eu vejo enfim, com a alma vencida,&
Na abjeção embriológica da vida&
O futuro de cinza que me aguarda!


{\raggedleft\itshape
Paraíba, 1910
\par}

\twoside

