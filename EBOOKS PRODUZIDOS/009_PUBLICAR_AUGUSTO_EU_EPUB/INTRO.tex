\SVN $Id: INTRO.tex 9559 2011-08-15 16:37:29Z iuri $
\chapter[Introdução, por Fabiano Calixto]{introdução}
\hedramarkboth{introdução}{fabiano calixto}

\section[Augusto dos demônios ou Apocalipsis litteris]{Augusto dos demônios
ou\break Apocalipsis litteris}


\epigraph{\itshape
--- Et pourtant vous serez semblable à cette ordure,\\
A cette horrible infection}{Charles Baudelaire}

\epigraph{\itshape
Maggots and grubs bore into the mouldy remains}{Carcass}{}

\section[Foi este mundo que me fez tão triste ou Uma anedota]{Foi este mundo que me fez tão triste\break ou Uma anedota}

Contam os historiadores da literatura brasileira que, num dia
perdido do ano de 1914, Órris Soares e Heitor Lima caminhavam pela
avenida Central (atual avenida Rio Branco), no Rio de Janeiro. Passando
pela porta da Casa Lopes Fernandes, viram o “Príncipe dos Poetas
Brasileiros”, vulgar e humanamente conhecido como Olavo Bilac, e
resolveram parar para dois dedos de prosa. Ao notá"-los com ar de
tristeza, o poeta parnasiano perguntou o que havia acontecido. Os dois
sujeitos então noticiaram a morte recentíssima de um grande poeta.

--- Quem? --- perguntou o membro da corte do sorriso da sociedade.

--- Augusto dos Anjos --- responderam os dois camaradas.

--- Grande poeta? Nunca ouvi falar! --- retrucou Bilac, com o tático
desprezo dos medíocres ilustrados.

Os dois amigos ficaram espantados, e Bilac, então, indagou: 

--- Sabem algum poema dele?

Heitor Lima, então, recitou o soneto “Versos a um coveiro”:

\begin{verse}
Numerar sepulturas e carneiros,\\
Reduzir carnes podres a algarismos,\\
Tal é, sem complicados silogismos,\\
A aritmética hedionda dos coveiros!

Um, dois, três, quatro, cinco\ldots{} Esoterismos\\
Da morte! E eu vejo, em fúlgidos letreiros,\\
Na progressão dos números inteiros\\
A gênese de todos os abismos!

Oh! Pitágoras da última aritmética,\\
Continua a contar na paz ascética\\
Dos tábidos carneiros sepulcrais

Tíbias, cérebros, crânios, rádios e úmeros,\\
Porque, infinita como os próprios números,\\
A tua conta não acaba mais!\footnote{
Esse poema consta da segunda edição do \textit{Eu}, publicada pela
Imprensa Oficial da Paraíba em 1920, graças ao esforço de Órris
Soares, amigo do poeta. Nesta segunda edição foram acrescidos mais de
quarenta poemas escritos entre os anos de 1912 e 1914.}
\end{verse}

--- Era este o poeta? --- questionou o parnasiano, com o desdém doirado
de quem ouve estrelas, e completou: --- Ah, então, fez bem em morrer. Não
se perdeu grande coisa\ldots{} E saiu andando, todo pimpão, com seu bom
gosto sorridente --- talvez bolando alguma engenhosa trova publicitária
para o xarope Bromil ---, desfilando seu indefectível e engomado bigode.
Jamais imaginaria que aquele jovem poeta morto legaria um dos conjuntos
de poesia mais fortes de toda a história da literatura em língua
portuguesa. 


\section{O horror no rosto impresso\break ou O Criador}

No dia 20 de abril de 1884, perto da Vila do Espírito Santo, ou,
mais precisamente, no Engenho do Pau d’Arco, no interior da Paraíba,
nasce Augusto Carvalho Rodrigues dos Anjos, o sujeito que irá
infernizar a poesia brasileira ali no começo do século seguinte.
Augusto é filho do bacharel Alexandre Rodrigues dos Anjos e de Dona
Córdula Carvalho Rodrigues dos Anjos. O casal teve outros vários
filhos, mas apenas um ao gosto dos anjos e demônios, o Augusto. Quando
chega o ano de 1900, o jovem Augusto entra no Liceu Paraibano e lá
estuda humanidades --- seu aprendizado será complementado pelas 
orientações de leitura de seu pai. É considerado pelos
professores do Liceu um aluno brilhante. Não deixa por menos e, no ano
seguinte, com apenas dezessete anos de idade, entre uma soneca e outra à
sombra do tamarindo, escreve seus primeiros poemas e começa a
publicá"-los no jornal \textit{O Comércio}. Continua compondo poemas
como em toda grande adolescência. Em 1903, o jovem poeta se matricula
na Faculdade de Direito do Recife. E em suas perambulações pela agitada
faculdade conhece Órris Soares e Gilberto Amado e toma contato
com as ideias que espocavam por ali. Sua formação se dá sob o influxo
das teorias positivistas que circulavam na Escola do Recife --- movimento
cultural surgido na capital pernambucana na segunda metade do século
\textsc{xix}, cujas inovações intelectuais viriam, principalmente, das
ideias de Tobias Barreto.\footnote{ Tobias Barreto (1839--1889) foi um
intelectual sergipano de destaque na segunda metade do século \textsc{xix}.
Escreveu poemas, textos filosóficos e jurídicos e crítica literária.
Integrou a Escola do Recife, onde era considerado um mestre pelos
discípulos (entre eles, Sílvio Romero e Graça Aranha). É patrono da
cadeira 38 da Academia Brasileira de Letras.} Neste ambiente, Augusto dos Anjos toma
contato com a obra materialista e evolucionista de filósofos que fariam
parte do seu repertório pessoal --- entre eles, Auguste Comte, Ernst
Haeckel, Charles Darwin e Herbert Spencer. As sabichonas ideias
desses filósofos serviam de explicação para Deus e tudo no mundo e,
lustradas por um deslumbramento ingênuo dos pensadores locais,
eram aplicadas à solução de impasses da vida social, entendidos como efeitos de “causas
biológicas fatais e incontornáveis, até os problemas sociais, jurídicos
e psicológicos”,\footnote{Sérgio Alcides, “Augusto dos Anjos e o \textit{Eu} universal”. In:
Augusto dos Anjos. \textit{Eu e Outras poesias}. São Paulo:
Ática, 2005, p.~16.} redundando num repulsivo racismo
cientificista. O poeta, claro, não cai nessa e usa todo o circo
cientificista a seu favor, valendo"-se dos termos duros como vírus em
sua obra, como instrumentos de questionamento tanto da vida social do
país quanto do cientificismo mesmo. Dois anos antes de se formar, uma
desgraça acontece em sua vida: em 13 de janeiro de 1905, seu pai 
morre. Com o luto ainda lancinante, escreve uma trinca de
sonetos que a posteridade faria célebre e a publica no jornal \textit{O
Comércio}.\footnote{O poeta jamais imaginara que muito depois de sua morte, no
ano da graça de 1973, um compositor capixaba, portador da mesma
corrente sanguínea, comporia uma belíssima canção que dialoga
diretamente com o terceiro soneto de sua trinca. O compositor é Sérgio
Sampaio e a canção é “Pobre meu pai”.} Ainda em 1905, o poeta começa a
publicar regularmente, e no mesmo jornal, textos em prosa na sua coluna
“Crônica Paudarquense”. Os tempos são de dureza e em 1907, ano em que
se torna bacharel em Direito, vendo"-se na pindaíba, coloca anúncios nos
jornais do Estado para dar aulas de humanidades. No ano seguinte,
escrevendo poemas e mais poemas, tentando sobreviver às agruras deste
país, é nomeado professor de Literatura interino no Liceu Paraibano. A
vida vai, entre notícias de troca"-troca presidencial e os sóis sem
sombra do país. Casa"-se em 1910 com Ester Fialho. Após desentendimentos
com o então governador João Machado, desliga"-se do Liceu. Por conta de
problemas financeiros, a família se vê obrigada a vender o Engenho do
Pau d’Arco. Sai da Paraíba rumo ao Rio de Janeiro, onde chega com a
esposa em outubro de 1910. Aos 26 anos, fora de seu lugar, batalha
muito para conseguir emprego e alguma pecúnia para sustentar a família.
Em 1911, outra tragédia desaba sobre o poeta: sua esposa
perde o bebê que gestava. Começa a dar aulas de Geografia na Escola Normal e também
no Colégio Pedro \textsc{ii}, onde substitui o professor e político, também
paraibano, João Coelho Lisboa, que muito o ajuda nos tempos de penúria.
Em novembro daquele ano nasce sua primogênita Glória. Chega 1912 e com
ele o primeiro livro de Augusto dos Anjos é publicado no Rio de
Janeiro, no mês de junho. O livro chama"-se \textit{Eu}. A crítica, como
sempre enferma de seu anacronismo crônico e espantada com a força
estranha do livro, lhe dá quase nenhuma atenção. O ano de 1913 continua
a germinar agruras monetárias na família Dos Anjos. Sem dinheiro e sem
emprego fixo, sua situação financeira se mantém deplorável --- desde a
chegada da família ao Rio, mudam mais de dez vezes de endereço, morando
aqui e ali, nas piores espeluncas da praça.\footnote{ No momento em que
publica \textit{Eu} a situação do poeta era a seguinte: “28 anos de
idade, forasteiro, tentando impor"-se, mas sem muito sucesso, e passando
apertos com a família. No agitado meio literário do Rio de Janeiro, era
um peixe fora d’água”. Cf. Alcides, op. cit., p.~10.} No mês de junho, um ano
após a publicação de seu poemário de estreia, nasce seu filho
Guilherme. Em julho de 1914, com auxílio de um familiar, o poeta é
nomeado diretor do Grupo Escolar Ribeiro Junqueira, no município de
Leopoldina, interior do Estado de Minas Gerais. Poucos meses depois,
tendo encontrado alguma paz de espírito e renda estável para cuidar dos
seus, o destino destila"-lhe uma ironia e o poeta, cansado, adoece
gravemente no final de outubro e, em 12 de novembro, Augusto dos Anjos
morre em consequência de pneumonia, aos 31 anos de idade. Segundo
consta, já muito debilitado, mas juntando as derradeiras heroicas
forças, dita, no dia anterior à morte, “O último soneto”, que é
publicado na \textit{Gazeta de Leopoldina} no dia 13 de novembro. 

\begin{verse}
Hora da minha morte. Hirta, ao meu lado,\\
A Ideia estertorava"-se\ldots{} No fundo\\
Do meu entendimento moribundo\\
Jazia o Último Número cansado.

Era de vê"-lo, imóvel, resignado,\\
Tragicamente de si mesmo oriundo,\\
Fora da sucessão, estranho ao mundo,\\
Com o reflexo fúnebre do Incriado:

Bradei: --- Que fazes ainda no meu crânio?\\
E o Último Número, atro e subterrâneo,\\
Parecia dizer"-me: `É tarde, amigo!

Pois que a minha autogênica Grandeza\\
Nunca vibrou em tua língua presa,\\
Não te abandono mais! Morro contigo!'
\end{verse}

Pouco depois, eclode a Primeira Guerra Mundial. Seis anos após a
morte do poeta, em 1920, seu amigo Órris Soares, ocupando o cargo de
Secretário de Estado, publica, através da Imprensa Oficial do Estado da
Paraíba, a segunda edição do \textit{Eu} --- acrescida de mais de quarenta
poemas inéditos. Em 1928 é lançada uma terceira edição de sua poesia
pela Livraria Castilho, do Rio de Janeiro. Nesta edição já está
estampado o título que percorreria todas as outras edições do livro:
\textit{Eu e Outras poesias}. Dali em diante, sua obra terá uma enorme
recepção popular (fato raríssimo nas letras nacionais), gerando
sucessivas edições. O respeito por parte da crítica viria depois do
sucesso popular --- Ferreira Gullar afirma espirituosamente que “não foi
a crítica que descobriu Augusto; foi Augusto que ‘descobriu’ a crítica”.\footnote{Ferreira 
Gullar, “Augusto dos Anjos ou Vida e morte
nordestina”. In: Augusto dos Anjos. \textit{Toda a poesia}.
Rio de Janeiro: Paz e Terra, 1978, p.~29.}  O sanguíneo Augusto operou uma sangria na poesia
de seu tempo. Destoou, por determinar uma escrita da coragem para si, e
por isso permanece entre os poetas mais importantes de todos os tempos
--- e não só da língua portuguesa.\footnote{João Cabral de Melo Neto, em seu belíssimo poema “O sim contra o
sim” (em que fala de poetas do quilate de Marianne Moore, Francis Ponge
e Cesário Verde), escreve uma incisiva homenagem ao autor do
\textit{Eu}.} 

%\begin{verse}
%Augusto dos Anjos não tinha\\
%\hspace*{2em} dessa tinta água clara.\\
%\hspace*{2em} Se água, do Paraíba\\
%nordestino, que ignora a fábula.
%
%Tais águas não são lavadeiras,\\
%\hspace*{2em} deixam tudo encardido:\\
%\hspace*{2em} o vermelho das chitas\\
%ou o reluzente dos estilos.
%
%E quando usadas como tinta\\
%\hspace*{2em} escrevem negro tudo:\\
%\hspace*{2em} dão um mundo velado\\
%por véus de lama, véus de luto.
%
%Donde decerto o timbre fúnebre,\\
%\hspace*{2em} dureza da pisada,\\
%\hspace*{2em} geometria de enterro\\
%de sua poesia enfileirada.
%\end{verse}

\section[Cantar de preferência o Horrível! ou A criatura]{Cantar de preferência o Horrível!\break ou A criatura}

Perto de completar o centenário de publicação (em 2012), \textit{Eu}
continua sendo uma das obras"-primas máximas da poesia em língua
portuguesa. É um daqueles livros cuja leitura, como diria o leitor
Carlos Drummond de Andrade, é “um soco na cara”. Impossível sentir"-se
indiferente aos poemas desse sinistro e \textit{gore} poeta
paraibano.

O livro foi publicado em 1912, em edição independente financiada
pelo próprio autor e por seu irmão, Odilon dos Anjos. Em suas páginas
figuram 58 poemas --- 58 peças que desconcertariam a crítica de seu
tempo. 

Augusto residia, quando da publicação do \textit{Eu}, no Rio de
Janeiro, mas o livro passou despercebido pelo ambiente literário
carioca (e brasileiro) à época --- que fruía a literatura como simples
entretenimento, como um “sorriso da sociedade” (numa expressão de
Afrânio Peixoto), e, obviamente, não teria como entender a voz gutural
que desentranhava de tal poesia. No poema “Os doentes”,
Augusto parecia projetar essa decadência: “A doença era geral, tudo a
extenuar"-se / Estava”.

Octavio Brandão, no periódico libertário \textit{A Plebe}, escreve
que os poetas (partículas integrantes da entidade parasita que Brandão
chamava de “intelectual indígena”) apenas nutriam seu acomodado mundo
burguês, blindando"-se com o próprio conforto, e se aglomeravam nos
cafés, “se babavam, enlevados, em discussões intermináveis sobre as
futilidades da Forma, sobre as torturas do Metro, enquanto lá embaixo,
na cansada Europa, as revoluções rebentavam furiosas”. As preocupações
históricas passavam ao largo dos poetas que, ainda segundo Brandão,
sonhavam nas “torres de luar, burilando frases preciosas, sonoros
períodos”, que cantavam “sofrimentos inventados, amores inexistentes,
dores imaginárias” e que viviam “no mundo vazio das ilusões”.\footnote{
Cf. Antonio Arnoni Prado e Francisco Foot
Hardman, Introdução, em \textit{Contos anarquistas}, edição organizada pelos
mesmos (São Paulo: Brasiliense, 1985).}

Mesmo depois, na fase do modernismo heroico, a poesia de Augusto dos
Anjos seria desprezada. Otto Maria Carpeaux aponta que “até durante a			%fonte
fase modernista da literatura brasileira, os versos de Augusto dos
Anjos passaram por exemplos de mau gosto de uma época superada”. 

Quer dizer, em tal circunstância histórica, a poesia do paraibano
não era compreendida nem pelos adeptos da estética parnasiana (a
“arte"-arteriosclerose”, como foi chamada pelo grupo libertário
\textit{Renovação}), nem pela vanguarda estética que se estava
germinando naquelas primeiras décadas do século passado.\footnote{
Curioso notar, entretanto, que intelectuais anarquistas (a vanguarda
política) o admiravam. José Oiticica, por exemplo, escreveu, em 1920,
que: “Poucos o compreenderão hoje. No futuro será, sem possível dúvida,
o mais assinalado poeta brasileiro de seu tempo!”.}

\textit{Eu} surge, portanto, numa ambiência literária e política
difícil, tomada pela futilidade e pelo baile de máscaras de uma
sociedade que macaqueava valores europeus e fechava os olhos a tudo que
não fosse do mais alto “bom gosto”.

O livro abre"-se com o extraordinário “Monólogo de uma sombra”, poema
que, com seus 186 versos divididos em 31 sextilhas, desponta como uma
espécie de súmula\footnote{ Como já foi observado por José Paulo Paes em seu
importante ensaio “Augusto dos Anjos ou O evolucionismo às avessas”.} de
todo o conjunto, apresentando um denso niilismo crítico, de base
schopenhaueriana:\footnote{ Anatol Rosenfeld afirma 
que a influência de Schopenhauer é mais profunda na obra de
Augusto dos Anjos que a de Haeckel ou Spencer.}

\begin{verse}
Sou uma Sombra! Venho de outras eras,\\
Do cosmopolitismo das moneras\ldots{}\\
Pólipo de recônditas reentrâncias,\\
Larva de caos telúrico, procedo\\
Da escuridão do cósmico segredo,\\
Da substância de todas as substâncias!
\end{verse}

Como aponta Sérgio Alcides, na edição que preparou em 2005, num
“livro cujo título é um pronome pessoal (\textit{Eu}), a primeira
palavra do primeiro poema é um verbo conjugado na pessoa
correspondente: ‘sou’\,”.\footnote{Alcides, op. cit., p. 49} Quer dizer, dentre os
múltiplos epítetos encontráveis no volume, o enunciador primeiro nos
mostra aquele que provavelmente o definirá melhor: \textit{eu sou uma
sombra}. Esta sombra, conhecedora das tragédias metafísicas da existência cósmica, 
vem das origens longínquas da eternidade, destila a
dúvida e aposta “na simbiose das coisas” para a possibilidade de
equilíbrio e no “metafisicismo de Abidharma” (o budismo), como
ferramenta de compreensão da dor. Enunciadora da “podridão” que lhe
“serve de Evangelho”, logo afirma sua solidariedade por tudo aquilo que
sofre:

\begin{verse}
E trago, sem bramânicas tesouras,\\
Como um dorso de azêmola passiva,\\
A solidariedade subjetiva\\
De todas as espécies sofredoras.
\end{verse}

Com um humor proparoxítono que opera a autópsia da desventura
existencial e nela encontra “um cancro assíduo na consciência” e “três
manchas de sangue” na camisa que lhe veste, paira acima dos “mundanos
tetos” e mostra seu “nojo à Natureza Humana”. Esse humor, negro, é um
riso amargo, que tenta dar consciência à hipotética “propriedade do
carbono” que é o horror, e através dele dar voz ao outro, ao que sofre.
E este “outro” é um complexo que abraça o micro e o macro\footnote{
José Paulo Paes, em já referido ensaio, diz que “a animização do
universo, desde a microscopia da monera à telescopia das forças
cósmicas, é característica do \textit{Eu}”.} --- já que desintegrada a
subjetividade, cortada a “singularíssima pessoa”, funda"-se essa
metafísica lírica que integra o Eu e o Cosmos.\footnote{Paes, op. cit., pp.~9--10.} E o
sofrimento não é lateral, mas universal, a um e outro que, na verdade,
são \textit{uno}. O todo"-total que sofre. Daí a solidariedade como uma
instância de resistência, uma cavidade utópica no apocalíptico caminhar
dessa “canção da Natureza exausta”. 

Ao mesmo tempo que reconhece a misteriosa mônada (e, por
conseguinte, toda a parafernália do evolucionismo biológico pregado por
Haeckel), a sombra enunciadora destila seu desdém pela filosofia
moderna, representada pelo Filósofo Moderno:

\begin{verse}
Aí vem sujo, a coçar chagas plebeias,\\
Trazendo no deserto das ideias\\
O desespero endêmico do inferno,\\
Com a cara hirta, tatuada de fuligens\\
Esse mineiro doido das origens,\\
Que se chama o Filósofo Moderno!
\end{verse}

Então, mesmo coagulada de termos científico"-filosóficos, estes são
uma endemia no “deserto das ideias” do “mineiro doido das origens” e
funcionam, no conjunto da obra, como virais, anulando o discurso
científico"-filosófico através do escárnio. E, como virais espalhados
numa “orquestra arrepiadora do sarcasmo” imersa na “podridão do sangue
humano”, geram o riso e o espanto que guiam essa escritura repleta
de uma macabra mística da putrefação --- termos \textit{trash} dentro das
composições com os quais a sombra (em sua “vida anônima de larva”)
pontua um desbocado misticismo em constante mutação. 

\vfil\pagebreak

Essa característica demonstra a falta de lugar dessa poesia naquele
ambiente literário, o que levaria, inclusive, o livro “a ter sido
incorporado à biblioteca da Faculdade de Medicina do Rio de Janeiro”, 
\footnote{Antonio Arnoni Prado, “Um fantasma na noite dos vencidos”. In:
\textsc{Anjos}, Augusto dos. \textit{Eu e Outras poesias}. São Paulo:
Martins Fontes, 2000, p.~43.} mostrando a dificuldade de classificação dessa
furiosa poética. 

Anatol Rosenfeld, num dos mais lúcidos ensaios sobre o “poeta do
hediondo”, destaca a sedução erótica que os termos científicos
exercem sobre o autor, e relata a ligação desta característica com
ocorrências semelhantes no expressionismo alemão.\footnote{ Dentre as
tentativas de classificação da obra de Augusto dos Anjos, creio que a
aproximação ao expressionismo alemão seja a mais instigante --- mesmo com
todo o abismo de fatores que diferencia as poéticas em questão, como,
aliás, Rosenfeld também aponta.} Aponta, principalmente, a
ligação entre a poética de Augusto dos Anjos e a de Gottfried Benn:

\begin{hedraquote}
Em conexão com a terminologia clínico"-científica --- que, sem ser
monopólio desses dois poetas, é por eles usada com insistência
excepcional --- surge em ambos os casos o que se poderia chamar uma
poesia de necrotério na qual se disseca e desmonta `a glória da
criação, o porco, o homem' (Benn), o `filho do carbono e do amoníaco'
(Augusto).\footnote{Anatol Rosenfeld, “A costela de prata de A. dos Anjos”. In:
\textsc{Rosenfeld}, Anatol. \textit{Texto/Contexto \textsc{i}}. São Paulo:
Perspectiva, 1996, pp.~264--265.}
\end{hedraquote}

A poesia de ambos os poetas destrói as esperadas leituras de seu
tempo na tentativa desesperada de criar outro ambiente habitável e
outra roupagem sonora e semântica, exigindo outros modos de travessia
de leitura, questionando a todo momento seu chão e relendo o mundo pela
chave da putrefação.

Assim, como exemplo da aproximação feita por Rosenfeld, o poema
“Homem e mulher passeiam no Pavilhão do Câncer”, de Benn:

\begin{verse}
Nesta fila aqui estão os ventres apodrecidos\\
e nesta está o peito apodrecido.\\
Lado a lado camas malcheirosas.\footnote{ Tradução Claudia Cavalcanti, \textit{Poesia
expressionista alemã --- Uma antologia}. São Paulo: Estação Liberdade,
2000.}
\end{verse}

Um fragmento do “Monólogo de uma sombra”, do poeta
paraibano:\footnote{ É bom salientar que não teria sido possível
qualquer contato de Augusto dos Anjos com a obra dos autores do
expressionismo alemão.}

\begin{verse}
É uma trágica festa emocionante!\\
A bacteriologia inventariante\\
Toma conta do corpo que apodrece\ldots{}\\
E até os membros da família engulham, \\
Vendo as larvas malignas que se embrulham \\
No cadáver malsão, fazendo um \textit{s}.
\end{verse}

A podridão e a consciência da mesma (numa esfera existencial e
política) é que fornece energia crítica a estes poemas. É uma leitura
de mundo pela base, onde mais fede. Bate"-se onde mais dói.

Há a instância sem remédio dos conflitos básicos (a contradição
sujeito/mundo), que proporciona à poesia de Augusto dos Anjos uma
intensa tensão, sem a menor chance de cura ou mesmo alívio, já que,
alimentando a melancolia, os “anúncios das casas de comércio” nada mais
são que o epitáfio do poeta. A civilização é hostil e a pena por nela
viver é ter a morte escancarada e já feita produto comercial, que
transforma toda a experiência histórica em cadáveres novos, em
\textit{fast"-food} para vermes. O homem é um desamparado na irreparável
decomposição universal. Alfredo Bosi assinala que:

\begin{hedraquote}
[\ldots{}] a postura existencial do poeta lembra o inverso do cientismo: uma
angústia funda, letal, ante a fatalidade que arrasta toda carne para a
decomposição. E já não será lícito falar em Spencer ou em Haeckel para
definir a sua cosmovisão, mas no alto pessimismo de Arthur
Schopenhauer, que identifica na vontade"-de"-viver a raiz de todas as
dores. Fundem"-se a visão cósmica e o desespero radical produzindo esta
poesia violenta e nova em língua portuguesa.\footnote{Alfredo Bosi, \textit{História concisa da literatura
brasileira}. São Paulo: Cultrix, 1974, pp.~322--323.}
\end{hedraquote}

A obra de Augusto dos Anjos é paradoxal e daí um dos muitos
elementos de sua vitalidade. Uma visão de mundo científica articulada,
ao mesmo tempo, a um delírio imaginativo como ferramenta de compreensão
da realidade em putrefação na qual o poeta habita: corpos em
decomposição, mutilados, recheados de vermes, que emitem o odor
desagradável do fracasso histórico da civilização ocidental. 

Já classificada de muitas maneiras\footnote{ Muitas vezes é chamada
de poesia cientificista, mas não o vejo assim. Dentre os poetas
cientificistas que as histórias literárias cobrem, não vejo nenhum que
mantenha o arco teso e arque com as consequências estéticas dentro do
parâmetro ético --- aquilo que Haroldo de Campos chama de “a plena
consciência de seu \textit{métier} e de sua peripécia histórica”.
Acredito que toda a tara científica em Augusto dos Anjos era uma
maneira de desestabilizar dogmas e ideias feitas, funcionando como uma
feroz crítica a um e a outro lado da questão. Só para constar, Silvio
Romero, Martins Júnior e Rocha Lima foram alguns dos defensores da
poesia científica no Brasil.} --- sina das obras que desconcertam os
parâmetros de recepção ---, esta poesia desbanca leituras superficiais,
pois se mostra, no âmago, independente. Então, uma hora parnasiano,
noutra simbolista, com resquícios de um tardo"-romantismo aqui, mais um
pré"-modernista ali, e por aí vai --- num caldeirão de bobagens, pois
vistas de perto nenhuma dessas classificações se ajusta à poesia em
questão. O poeta, caminhante de seu tempo, mas sempre de olhos abertos
a outras perspectivas poéticas, lança mão, sim, de artifícios tanto
parnasianos quanto simbolistas, sem, porém, se render a nenhuma das
duas escolas. Ao contrário, infecta"-os com a música terrificante de
seu niilismo \textit{gore}. 

Carpeaux fala, creio que acertadamente, da influência de Charles
Baudelaire e Cesário Verde. Sérgio Alcides aponta para uma mistura nova
da “objetividade dos parnasianos com o subjetivismo dos simbolistas”.
\footnote{Sérgio Alcides, “Augusto dos Anjos e o \textit{Eu} universal”, 
op. cit., p.~13.} Acrescentaria ainda, para engrossar o caldo e dar
mais sabor à soberba sopa, a moderna gargalhada \textit{gore} de sua
escrita sobre as pitadas parnasianas ou simbolistas. A poesia do
\textit{Eu} nasce a partir de uma necessidade de reflexão sobre o
sujeito e o mundo e traz para si instrumentos à mão e com os quais fará
a autópsia do seu momento histórico. Não sem duvidar o tempo todo
destes instrumentos, desfigurando"-os e questionando"-os, fornecendo
assim tensão constante à sua poética. O lugar da poesia de Augusto dos
Anjos, na cena brasileira, é o impossível.

O que, de fato, podemos sinalizar nesta poesia é sua violência, sua
novidade e sua consciência de modernidade. Uma poesia que, desde a
primeira leitura, mostra"-se impactante. É uma poética que se mostra
como deboche à oficialidade dos donos do vernáculo e do “sorriso da
sociedade”, tornando"-se, deste modo, uma contrapoética. Sua estrutura é
similar,\footnote{ Trata"-se, afinal de contas, de poemas, e, em sua
maioria, sonetos decassílabos --- carregados, portanto, de uma gigantesca
tradição.} porém paródica em relação à literatura. Há uma hesitação
formal nas peças do \textit{Eu}: à rigidez do decassílabo e à temática
metafísica (da vida e da morte), características que compunham aquilo
conhecido como “poesia lírica” ou “poesia poética” (ou, ainda,
parnasianismo), se contrapõem as violentas e antipoéticas imagens
(visuais e sonoras), subvertendo completamente as expectativas de
leitura. A forma convencional é, portanto, o troco irônico que sua
contrapoesia despeja sobre os despojos da poesia fútil e fru"-fru do seu
tempo. 

A constituição de sua linguagem busca, apesar da tara científica
(que, lembremos, funciona como contraponto), uma espécie de retorno à
essência, à pura animalidade, que coloque questões truncadas à
existência humana e sua relação com a realidade --- pelo viés crítico do
asco e da desestabilização da ideia de “bom gosto”.\footnote{ A ideia de
“bom gosto” guarda em sua gaveta semântica outras ideias sinistras,
tais como: preconceito, redundância, tradicionalismo, autoritarismo,
intolerância, conformismo etc.} Se a poesia era de “áureos relevos”,
“taça amiga”, “mãozinha cruel”, “versos de ouro”, “cabeleiras
líquidas”, “melancólica e linda estrela d’alva”, “valvas de nácar”, “a
força e a graça na simplicidade”, “turbilhão de lava”, “ouvir estrelas”
etc., depois da necrópsia operada por Augusto dos Anjos, a poesia se
torna: 

\begin{verse}
Os defuntos então me ofereciam \\
Com as articulações das mãos inermes,\\
Num prato de hospital, cheio de vermes,\\
Todos os animais que apodreciam!

É possível que o estômago se afoite \\
(Muito embora contra isto a alma se irrite)\\
A cevar o antropófago apetite,\\
Comendo carne humana, à meia"-noite!

Com uma ilimitadíssima tristeza,\\
Na impaciência do estômago vazio,\\
Eu devorava aquele bolo frio\\
Feito das podridões da Natureza!
\end{verse}

A busca por problematizar a condição humana, através de uma espécie
de jogo em que há a desarticulação das categorias de normatização
(políticas do “bom gosto” que constroem as posições de poder), é que
permite, pela profanação, à linguagem (como de resto, outros espaços de
vida dentro da própria vida) se abrir e, assim, se renovar. É uma
oficina do mau gosto como instrumento de resistência, que se aproxima da
paródia, e, pensando"-a com Alfredo Bosi, a “paródia tem em comum com a
sátira o espírito do contraste e leva a dissonância até o coração da
ideologia literária, que se chama \textit{gosto}”.\footnote{Alfredo Bosi, \textit{O ser 
e o tempo da poesia}. São Paulo: Companhia das Letras, 2004, p.~201.}
Quer dizer, trazendo para o contexto augustiano, sua estética do
macabro problematiza toda uma concepção poética (e, por abertura
metonímica, política) de seu tempo, derrubando a redundância estéril,
demolindo o maquinário do eixo da tradição e expondo os ossos de um corpo
(“bom gosto”) que, enfermo, recusa a morte --- mas precisa morrer e
justamente por obra do “mau gosto”. A poesia de Augusto não abre mão,
portanto, dos elementos da tradição, mas os distorce, os mutila, os
sangra, insere o veneno da violência existencial no vaso grego de
flores coloridas e perfumadas da poesia que se fazia até então.  

Às vésperas da Primeira Guerra Mundial, que outra poesia, no Brasil,
pôde projetar, para usar aqui as palavras de Anatol Rosenfeld, “o desafio do
radicalmente feio à face do pacato burguês, desmascarando, pela
deformação hedionda, a superfície harmônica e açucarada de um mundo
intimamente podre”,\footnote{Anatol Rosenfeld, “A costela de prata de A. dos Anjos”. In:
\textsc{Rosenfeld}, Anatol. \textit{Texto/Contexto \textsc{i}}, op. cit., 
p.~265.} funcionando, deste modo, como
termômetro de seu momento histórico? Esta pergunta se autorresponde
após a leitura do \textit{Eu}.

O discurso vigente da estúpida \textit{belle époque} tupiniquim era
de uma malas"-arte euforia modernizante. O Rio de Janeiro (capital do
país) aspirava a ser Paris. Ali se operam reformas urbanas, serviço de
transporte público, os automóveis começam a buzinar pela cidade, os
filmes chegam ao cinema etc. Tudo às mil maravilhas para uma sociedade
que só sorri, com enormes dentes de ouro, sendo aclamada pelo coro dos
contentes formado pela fauna fútil e endinheirada dos clubes e dos
salões. Por outro lado, essa “intelectualidade indígena” pouco se
importava com as contradições históricas do período e a essa
transformação violenta não havia quem pudesse fazer oposição. Nicolau
Sevcenko aponta que quatro princípios fundamentais conduziram essas
transformações: 

\begin{hedraquote}
a condenação dos hábitos e costumes ligados pela memória à sociedade
tradicional; a negação de todo e qualquer elemento da cultura popular
que pudesse macular a imagem civilizada da sociedade dominante; uma
política rigorosa de expulsão dos grupos populares da área central da
cidade, que será praticamente isolada para o desfrute exclusivo das
camadas aburguesadas; e um cosmopolitismo agressivo, profundamente
identificado com a vida parisiense.\footnote{Nicolau Sevcenko, \textit{Literatura como missão: tensões
sociais e criação cultural na primeira república.} São Paulo: Brasiliense, 1995, p.~30.}
\end{hedraquote}

A exclusão e a violência foram, portanto, os itens que marcaram a
transformação do Rio de Janeiro sob a égide da burguesia, do
preconceito (em todas as suas formas) e da europeização. Esse era o lugar
de atuação (e também de escrita) de Augusto dos Anjos\footnote{
Acrescentemos ainda, como situação de escrita, a derrocada da sua
situação familiar, levando à venda do Engenho do Pau d’Arco e, em
consequência, ao empobrecimento da família. As dificuldades financeiras o
acompanhariam durante toda a vida. Numa carta endereçada à mãe, datada
de 18 de fevereiro de 1911, o poeta escreve: “Pelo menos é o que
deseja, de toda a alma, um bacharel depenado, antigo professor de
província, e possuidor de outros títulos congêneres de
desmoralização”.} e, como aponta Alcides, “não há enquadramento
onde ele não figure como \textit{outsider}”.\footnote{Sérgio Alcides, 
“Augusto dos Anjos e o mito do ‘Eu’\,”. In: \textit{Travessias do pós"-trágico: 
os dilemas de uma leitura do Brasil}. Orgs. Ettore Finazzi"-Agro, Roberto Vecchi e Maria Betânia
Amoroso. São Paulo: Unimarco, 2006, p.~128.}

A partir de tão hostil e horroroso ambiente, uma pergunta se
formula: se a sociedade, imunda, fedia, que outro tipo de poesia daria
conta de abordar essa degenerescência social senão a que lemos no livro
de estreia deste poeta paraibano? Para uma sociedade putrefata, uma
poesia putrefata, oras! 

Seguindo a leitura, em “Os doentes”, um dos mais importantes poemas
do livro e de toda a literatura brasileira, há amostras da aspiração
por mudança que tal poética propunha como contraponto ao parasitismo
alienado daquela sociedade. Como já observaram outros críticos, este
poema é uma espécie de cosmogonia. Pelo contraste, o enunciador mostra
que é através do “fígado doente” sangrando, da “epiderme cheia de
sarampos”, da “camaradagem da moléstia”, da doença que faz a vida
passar"-se junto a uma escarradeira “pintando o chão de coágulos de
sangue”, que se pode vislumbrar o nascimento de outra época. A morte
então surge para redimir e tudo deve morrer para que se chegue ao Nada,
configurando, assim, a única possibilidade viável ao sujeito individual
e ao Cosmos, desintegrando"-os, numa “glutoneria hedionda”, para,
finalmente, integrá"-los num Nirvana. Um projeto que prevê que a
composição só é possível através da recomposição pela decomposição.

\begin{verse}
Como uma cascavel que se enroscava,\\
A cidade dos lázaros dormia\ldots{}\\
Somente, na metrópole vazia,\\
Minha cabeça autônoma pensava!

Mordia"-me a obsessão má de que havia,\\
Sob os meus pés, na terra onde eu pisava,\\
Um fígado doente que sangrava\\
E uma garganta órfã que gemia!

Tentava compreender com as conceptivas\\
Funções do encéfalo as substâncias vivas\\
Que nem Spencer, nem Haeckel compreenderam\ldots{}

E via em mim, coberto de desgraças,\\
O resultado de bilhões de raças\\
Que há muito desapareceram!
\end{verse}

Aqui, mais uma vez, o enunciado coloca em dúvida a sabença dos
filósofos, mostrando o quão incapazes são de compreender a complexidade
macabra da existência (o trajeto inevitável de nascer, viver, adoecer e
morrer), e que somente ao enunciador cabe tal conhecimento
(“Ah! Somente eu compreendo, satisfeito, / A incógnita psique das
massas mortas”). Portanto, toma a si mesmo como síntese de toda a
existência anterior, projetando, em seus mais de quatrocentos versos,
uma monstruosa saga que se fecha com a chave da tragédia. 

Colocando à face do alienado burguês as tragédias históricas do povo
que cultivou o Brasil, neste poema um feixe de assuntos se
entreatritam, na tentativa exasperada (e fatalista) de dar conta do
complexo panorama da experiência, apontando, pela aberração que sempre
constitui os planos de construção de lugares políticos, que a
insurgência contra tal estado de coisas (destruindo"-as, devorando"-as) é
a única saída. Antonio Arnoni Prado observa que:

\begin{hedraquote}
No amplo painel do poema “Os doentes” Augusto dos Anjos acompanha a
dinâmica de que resultou a sociedade moderna: a destruição dos
indígenas, a “raça esmagada pela Europa”; a “raça negra” atirada ao
“contubérnio diário das quitandas” por um “deus nefando” que
transgrediu a “igualitária regra da Natureza”; a marginalização dos
doentes, dos ébrios, dos mendigos, das prostitutas [\ldots{}].\footnote{Antonio 
Arnoni Prado, “Um fantasma na noite dos vencidos”. In: \textsc{Anjos}, Augusto dos. 
\textit{Eu e Outras poesias}. São Paulo: Martins Fontes, 2000, p.~40.}
\end{hedraquote}

Uma causa que projeta para o futuro, através da desintegração do
viciado e apodrecido mundo antigo, a criação de um novo projeto de
organização humana --- “Essa necessidade de \textit{horroroso}”. É
possível ler essa contrapoesia através da chave do anarquismo
visionário kropotkiniano, que possui um forte sentido de negação (do
Estado, da acumulação obscena de capital, de qualquer espécie de
autoridade ou opressão, do egoísmo). Mas, ao mesmo tempo, essa negação
possui “um profundo sentido afirmativo”,\footnote{Piotr Kropotkin. \textit{O princípio anarquista e outros
ensaios}. São Paulo: Hedra, 2007, p.~33.} pois
visa à formação de um mundo igualitário cuja organização social se
baseie na liberdade, no respeito e no apoio mútuo --- num símile com a
metamorfose da matéria e das ciências através da dialética
destruição/construção. Por uma linha discursiva parecida, a
contrapoética augustiana, através do discurso do horror, que se diz
contra, que nega, afirma a todo instante a possibilidade de restauração
pela “célula inicial de um Cosmos novo” que gera o “grande feto” que
substituirá a espécie que reina, os humanos: 

\begin{verse}
Entre as formas decrépitas do povo,\\
Já batiam por cima dos estragos\\
A sensação e os movimentos vagos\\
Da célula inicial de um Cosmos novo!

O letargo larvário da cidade\\
Crescia. Igual a um parto, numa furna,\\
Vinha da original treva noturna\\
O vagido de uma outra Humanidade!

E eu, com os pés atolados no Nirvana,\\
Acompanhava, com um prazer secreto,\\
A gestação daquele grande feto,\\
Que vinha substituir a Espécie Humana!
\end{verse}

\asterisc

Ainda dentro de muitas possibilidades de leitura que essa obra
suscita, insisto no tom hilário, no humor negro que se faz presente em
muitos momentos. O uso cientificista dos vocábulos aponta para isso,
assim como certas construções formais. Por exemplo, o extremo inchaço
vocabular da seguinte quadra de “As cismas do destino”:

\begin{verse}
A vingança dos mundos astronômicos\\
Enviava à terra extraordinária faca,\\
Posta em rija adesão de goma laca\\
Sobre os meus elementos anatômicos.
\end{verse}

Toda esta quadra, encavalada em decassílabos e frases que beiram o
hilário, para apenas falar do vento frio que o assola! A imagem da faca
(para \textit{frio cortante}) é, apesar de clichê, bonita, mas o verso
três, com sua “rija adesão de goma laca”, e o verso quatro, com seus
“elementos anatômicos”, dentro da ideia do poema (que é, diga"-se, um
ótimo poema), parecem exprimir um sentido jocoso.\footnote{ Lembrando,
inclusive, o Augusto dos Anjos do conhecido (e pedante) discurso
realizado em comemoração à data de abolição da escravidão, no Teatro
Santa Rosa, em João Pessoa, em 13 de maio de 1909.} Assim como esta
outra, do poema “Tristezas de um quarto minguante”:

\begin{verse}
Por muito tempo rolo no tapete.\\
Súbito me ergo. A lua é morta. Um frio\\
Cai sobre o meu estômago vazio\\
Como se fosse um copo de sorvete!
\end{verse}

A terminologia científica, como já apontado, funciona como elemento
irônico, como virais que infeccionam a estrutura linguística
tradicional, corrompendo e desestabilizando a enunciação. Esse humor
virótico e grotesco afirma uma revolta contra o
\textit{establishment}, criando, inevitavelmente, espaços de
resistência onde surge sarro cadavérico, a paródia \textit{gore}.
Alfredo Bosi, ao comentar a paródia, percebe nela a ambiguidade que
“repete os modos e metros convencionados ao mesmo tempo em que os dissocia dos
valores para os quais esses modos e metros são habitualmente acionados”.\footnote{Alfredo Bosi.
\textit{O ser e o tempo da poesia}, op. cit., p.~198.} 
Quando infecta, com seus virais macabros, o soneto
(trono da poesia conformada e conformista de seu tempo) e o faz
espirrar sangue e horror, o poeta não faz outra coisa senão insurgir"-se
contra a poesia esquemática e de retórica anêmica, maquinaria frouxa só
de perfumaria, feita por seus contemporâneos. Crítica feroz à
velhacaria linguística praticada pelos poetas engabinetados, a
contrapoesia do \textit{Eu} se firma, portanto, como um duro projeto de
resistência, afinal:

\begin{hedraquote}
O ataque às maneiras de dizer se identifica ao ataque às maneiras de ver
(ser, conhecer) de uma época; se é na (e pela) linguagem que os homens
externam sua visão de mundo (justificando, explicitando, desvelando,
simbolizando ou encobrindo suas relações reais com a natureza e a
sociedade) investir contra o falar de um tempo será investir contra o
ser desse tempo.\footnote{João Luiz Lafetá, \textit{1930: a crítica e o modernismo}. São
Paulo: Duas Cidades / Editora 34, 2001, p.~20.}
\end{hedraquote}

Sob essa gargalhada crítica aparece também a seriedade e a densidade
com a qual essa visão de mundo se forma. Não é apenas rir para
desmontar um círculo vicioso, mas constatar, ainda, a tragédia do
sujeito moderno, no interior de um projeto excludente, sem tensão e
mergulhado no mais puro deleite egoísta. Sérgio Alcides observa, na
comédia augustiana, que:

\begin{hedraquote}
Em todos os casos [\ldots{}] a ironia, a comédia e o humor só podem
despertar o riso amargo de quem reconhece, logo abaixo do efeito
engraçado, um substrato sério e doloroso, difícil de aceitar. É a
ironia da decadência da sociedade, a comédia da vaidade dos esforços
humanos diante da morte, o humor com o qual o indivíduo moderno
reconhece a sua impotência e sua angústia, o seu sentimento de
desamparo.\footnote{S. Alcides, “Augusto dos Anjos e o \textit{Eu} 
universal”, op. cit., p.~20.}
\end{hedraquote}

O reconhecimento desse “substrato sério e doloroso” é que configura
e dá potência ao núcleo duro dessa poesia --- que “É a luta, é o prélio
enorme, é a rebelião/ Da criatura contra a natureza!”. Ignorá"-lo é o
mesmo que subestimar a obra augustiana. E se “Andam monstros sombrios
pela estrada/ E pela estrada, entre estes monstros, ando!”, não há um
lamento sequer que tenha sabor doce. Esse constante atrito de opostos
em sua obra organiza sua consciência mesma da tragédia:

\begin{verse}
Bati nas pedras dum tormento rude\\
E a minha mágoa de hoje é tão intensa\\
Que eu penso que a Alegria é uma doença\\
E a Tristeza é minha única saúde.
\end{verse}

\asterisc

Esta contrapoética regula sua revolta \textit{gore} e seu
humor negro (como um George Romero \textit{bellepoquista}, câmera em
mãos destilando seu ácido sulfúrico apocalíptico mediante as gosmas
fantasmagóricas de seus zumbis) através de uma finíssima arquitetura
escritural. Formalmente subverte a fôrma fixa do soneto (e/ou do
decassílabo), libertando as amarras silábicas, deixando que as sílabas
fortes e fracas conduzam a música triturante e triturada --- aspectos que
transformam o verso clássico numa máquina irritada que, não poucas
vezes, subjuga e ri de si mesma, com elegância. Como nessa estrofe de
“Gemidos da arte”:

\begin{verse}
Ah! Por que desgraçada contingência\\
À híspida aresta sáxea áspera e abrupta\\
Da rocha brava, numa ininterrupta\\
Adesão, não prendi minha existência?! 
\end{verse}

Ou deixando transitar em seu verso a palavra rara, polida, com a
palavra coloquial, arruaceira, modulando"-as de maneira que tudo soe com
extrema naturalidade. Como nos mostra o célebre soneto “Psicologia de
um vencido”: 

\begin{verse}
Eu, filho do carbono e do amoníaco,\\
Monstro de escuridão e rutilância,\\
Sofro, desde a epigênesis da infância,\\
A influência má dos signos do zodíaco.

Profundissimamente hipocondríaco,\\
Este ambiente me causa repugnância\ldots{}\\
Sobe"-me à boca uma ânsia análoga à ânsia\\
Que se escapa da boca de um cardíaco.

Já o verme --- este operário das ruínas ---\\
Que o sangue podre das carnificinas\\
Come, e à vida em geral declara guerra,

Anda a espreitar meus olhos para roê"-los,\\
E há de deixar"-me apenas os cabelos,\\
Na frialdade inorgânica da terra!
\end{verse}

Construção elaborada, em que a ânsia, o “conjunto de fenômenos
mórbidos que antecedem a morte”, numa das acepções que o
\textit{Dicionário Houaiss da Língua Portuguesa} fornece ao vocábulo,
perpassa todo o poema, repetindo"-se vertiginosamente como as infinitas
contrações do epigástrio de quem está passando muito mal. Então, ao
fechar o segundo quarteto, o vocábulo aparece cinco vezes, seja ele
próprio (no verso 7, em ocorrência dupla), seja como sufixo, embutido
nas palavras “rutilância”, “infância” e “repugnância”. Pode"-se ainda
escutá"-la, levemente distorcida, no verso 4: “influência”. E, perdida e
estilhaçada, na observação do verme, no verso 6: “Que o s\textbf{an}gue
podre das carnifi\textbf{ci}n\textbf{as}”; assim como o anagrama no
corpo da palavra “inorg\textbf{ânica}” (ânica / ância), no verso final
do soneto. É um mal"-estar dilacerante que exige uma vontade enérgica de
terminar com tudo; por isso, a mesma ânsia, substantivo, se desdobra
depois no verbo “ansiar”, daquele que “deseja com veemência”, que
“almeja”, segundo informação do mesmo dicionário. O enunciado almeja
que o verme que espreita seus olhos com imenso apetite deixe"-lhe
“apenas os cabelos/ Na frialdade orgânica da terra!”. Poderosa imagem
da ânsia pela morte como salvação: o clamor pelo golpe de misericórdia
do “operário das ruínas” como remédio que acabará com toda a ciência e
consciência que lhe maltratam a vida, arruinada, de vencido.

Há de se notar, ainda, como artefato construtivo, que com apenas
duas palavras o poeta consegue formar um decassílabo (no primeiro verso
do segundo quarteto), uma característica curiosa que parece exercer um
grande fascínio ao poeta, visto que usa de construções parecidas em
outros momentos do livro.

Outra iguaria rítmica de grande engenhosidade está na bateria
anafórica do magnífico “Poema negro”, em que a repetição linguística
causa vertigem e orquestra perfeitamente a situação de espasmo do
enunciado:

\begin{verse}
E quando vi que aquilo vinha vindo\\
Eu fui caindo como um sol caindo\\
De declínio em declínio; e de declínio \\
Em declínio, com a gula de uma fera,\\
Quis ver o que era, e quando vi o que era,\\
Vi que era pó, vi que era esterquilínio!
\end{verse}

Podemos notar também outras construções sonoras em todo este
conjunto de poemas. Belas aliterações, como:

\begin{verse}
Voando ao vento o vastíssimo vapor

Brancas bacantes bêbadas o beijam
\end{verse}

Outras engraçadas, como:

\begin{verse}
Babujada por baixos beiços brutos.
\end{verse}

Ferreira Gullar, no ensaio “Augusto dos Anjos ou Vida e morte
nordestina”, aponta que a morte atinge a vida orgânica e o universo
numa movimentação micro/macro que é, em suma, o mecanismo de
funcionamento de toda a matéria existente, e também a dialética operada
nestes poemas. Acredito, indo um pouco mais adiante (dentro do reino da
linguagem poética), que, além da vida orgânica e do universo, a morte
se incrusta na própria estrutura do verso, dando"-lhe, por paradoxo,
vida.\footnote{ Interessante perceber que a morte em Augusto dos Anjos
não é oposição à vida, mas sim parte da estrutura geral da movimentação
cósmica, do ciclo infinito que azeita a evolução. José Paulo Paes dirá
que “a morte é nela vista menos como antípoda da vida do que parte
integrante do seu perene ciclo evolutivo, porta do acesso panteísta ao
Grande Todo que a religiosidade laica do monismo de Haeckel
identificava como Deus, mas que no \textit{Eu} se identifica ao nirvana
búdico”. Cf. José Paulo Paes, “Augusto dos Anjos ou O evolucionismo às
avessas”, op. cit., p.~23.} Como no primeiro terceto do soneto “O
Deus"-Verme”, em que o enunciado é repleto de atritos consonantais em
\textit{dr} e \textit{gr}, e seus \textit{r} roedores parecem fornecer
o ambiente sonoro da mastigação vermicular: 

\begin{verse}
Almoça a podridão das drupas agras,\\
Janta hidrópicos, rói vísceras magras\\
E dos defuntos novos incha a mão\ldots{}
\end{verse}

Outros exemplos, dentre muitos, recolhidos deste \textit{Eu}:

\begin{verse}
A criptógama cápsula se esbroa\\
Ao contato de bronca destra forte!

Atro dragão da escura noite, hedionda,\\
Em que o Tédio, batendo na alma, estronda\\
Como um grande trovão extraordinário

De aberratórias abstrações abstrusas!

Bruto, de errante rio, alto e hórrido, o urro

Rasgue a água hórrida a nau árdega e singre"-me!
\end{verse}

O bardo Augusto foi também portador de um rimário amplo, estranho e
forte. Entrerrimou palavras portuguesas com estrangeiras
(escalpelos/\textit{vitellus}), nomes próprios com substantivos
(Ripetta/quieta, reúno/Giordano Bruno), termos científicos com
adjetivos ou substantivos (blastodermas/palermas), verbos com letra do
alfabeto (apodrece/\textit{s}), palavras eruditas com palavras vulgares
(cenobial/universal), deus e o diabo! 

Também sangrou algumas palavras em busca da sonoridade ideal,
movimentando os acentos tônicos das palavras, inserindo seu bisturi no
âmago da prosódia, usando da sístole e da diástole. 

No caso da primeira, a sístole, em que o acento é transferido para a
sílaba posterior, temos a mudança de acento na palavra “periferia”,
transformando"-a em proparoxítona (“periféria”) para rimar com
“deletéria”, em “Os doentes”:

\begin{verse}
A ruína vinha horrenda e deletéria\\
Do subsolo infeliz, vinha de dentro\\
Da matéria em fusão que ainda há no centro,\\
Para alcançar depois a periféria!
\end{verse}

A diástole ocorre no poema “Gemidos da arte”, em que o acento tônico
da palavra “alimária” desaparece, tonificando assim a sílaba seguinte,
formando uma paroxítona para rimar com “dias”:

\begin{verse}
E por trezentos e sessenta dias\\
Trabalhar e comer! Martírios juntos!\\
Alimentar"-se dos irmãos defuntos,\\
Chupar os ossos das alimarias!
\end{verse}

Ou, em “Monólogo de uma sombra”, extirpando o acento tônico da
palavra “aríete”, paroxitonando"-a para formar rima com
“acomete”:

\begin{verse}
E explode, igual à luz que o ar acomete,\\
Com a veemência mavórtica do ariete
\end{verse}

Outro traço importante desta contrapoética vertiginosa é o uso
superabundante dos superlativos (alguns raros e que beiram o absurdo e
que jogam com certo tom jocoso, como “sanguinolentíssimos” ou
“escaveiradíssimas”), hipérboles e pontos de exclamação. Elementos que
parecem reforçar os indícios de desespero e urgência, como acentos de
violência e gritos de horror. Os versos do poema “Gemidos da arte”
parecem explicar o desespero na eterna e hedionda noite da vida pela
qual vagava o poeta:

\begin{verse}
Súbito, arrebentando a horrenda calma,\\
Grito, e se grito é para que meu grito \\
Seja a revelação deste Infinito\\
Que eu trago encarcerado na minh’alma!
\end{verse}

O grito é intenso, desopilador, e o infinito interior reitera a
abolição de fronteiras entre o Eu e o Cosmos, é a solidariedade total
na qual se identifica a dor própria com a dor alheia de todo o mundo:

\begin{verse}
Uivava dentro do \textit{eu}, com a boca aberta,\\
A matilha espantada dos instintos!
\end{verse}

Criando uma certa cenografia expressionista (para usar, ainda uma
vez, a boa leitura de Rosenfeld), em alguns poemas há um fluxo
narrativo cinematográfico, como neste exemplo extraído novamente do
belíssimo “As cismas do destino”:

\begin{verse}
Recife, Ponte Buarque de Macedo.\\
Eu, indo em direção à casa do Agra,\\
Assombrado com a minha sombra magra, \\
Pensava no Destino, e tinha medo!

Na austera abóbada alta o fósforo alvo\\
Das estrelas luzia\ldots{} O calçamento\\
Sáxeo, de asfalto rijo, atro e vidrento,\\
Copiava a polidez de um crânio alvo.

Lembro"-me bem. A ponte era comprida,\\
E a minha sombra enorme enchia a ponte,\\
Como uma pele de rinoceronte\\
Estendida por toda a minha vida!

A noite fecundava o ovo dos vícios\\
Animais. Do carvão da treva imensa\\
Caía um ar danado de doença\\
Sobre a cara geral dos edifícios!
\end{verse}

Aqui, sob o ambiente noturno, o enunciador aponta o local onde se
encontra, o destino e o caminho para se chegar a tal lugar. Com
poderosos \textit{zooms}, da cidade à ponte, da sombra ao corpo e deste
à infinitude do pensamento (horrorizado), subtraindo da sintaxe imagens
monstruosas e surreais, os versos imprimem a sensação de desespero,
“aprofundando o raciocínio obscuro”, em que não se vê outra saída, além
da hecatombe, ainda que se veja “O trabalho genésico dos sexos,/
Fazendo à noite os homens do Futuro”, o que poderia soar como uma
espécie de alívio, mas não: esses homens desse Futuro (com inicial em
maiúscula, portanto não qualquer futuro) apenas irritam o poeta, pois
se mostram inacabados na raiz, afirmando o fracasso daquela “fábrica
terrível”: 

\begin{verse}
Mas, a irritar"-me os globos oculares,\\
Apregoando e alardeando a cor nojenta,\\
Fetos magros, ainda na placenta,\\
Estendiam"-me as mãos rudimentares!
\end{verse}

O desamparo é tamanho que nem Deus é capaz de compreender tal
dilacerante dor: “Ninguém compreendia o meu soluço,/ Nem mesmo Deus!”.
E o castigava em sua cruzada até o cemitério, como se cruzasse, épico e
fantasmagórico, o Hades.

O local da enunciação não é nítido (se a ponte no interior da cidade
ou o pensamento dentro do crânio, o que cria a realidade? --- se é que há
mesmo alguma realidade) e desenha uma atmosfera de expressão forte,
\textit{noir}, repleta de vertiginosas distorções que apontam para uma
desordem que, de alguma maneira, tenta dar conta da situação de caos. 

A obsessão pela putrefação, como possibilidade redentora, é
reafirmada neste fragmento francamente \textit{splatter}:		%nota para splatter

\begin{verse}
Essa obsessão cromática me abate.\\
Não sei por que me vêm sempre à lembrança\\
O estômago esfaqueado de uma criança\\
E um pedaço de víscera escarlate.
\end{verse}

A potência desses versos, dentro da estrutura semântica do poema
como um todo, é gigante --- as imagens são violentas, macabras, causando
asco e susto no receptor. O enunciado coloca à disposição do caríssimo
leitor seu “apetite necrófago”, na época monstruosa em que “os sábios
não ensinam”, em que a morte, sua gritaria e seu odor (como num filme
de Tobe Hooper\footnote{ Cineasta norte"-americano que dirigiu \textit{O
massacre da serra elétrica} (1974), um clássico do terror.}), põe para
fora todas as vísceras, gosmas e postas. O poeta opera com extremo
rigor compositivo neste poema de longo fôlego, em que destrincha, como
uma espécie de açougueiro do verso, as patologias sociais, fisiológicas
e psicológicas que roíam o corpo apodrecido da história. José Paulo
Paes observa, com muita propriedade, que:

\begin{hedraquote}
A medida da competência dessa arte está no vigor com que logra exprimir
sensações de horror e náusea que só nos contos de Poe, não na sua
poesia, iremos encontrar tão vigorosamente expressas.\footnote{Paes, op. cit., p.~18.}
\end{hedraquote}

E durante a longa meditação, a revolta invade o enunciador que, numa
violenta crítica à sociedade (“Aquela humanidade parasita”), nomeia os
cadáveres e põe à vista as “rebeladas cóleras que rugem/ No homem
civilizado”:

\begin{verse}
O Estado, a Associação, os Municípios\\
Eram mortos. De todo aquele mundo\\
Restava um mecanismo moribundo\\
E uma teleologia sem princípios.
\end{verse}

A violência com que o enunciador constrói seu lúgubre percurso
pretende não deixar nada de pé. A revolta ruge, homem civilizado que é,
dentro de si. E é no oceano de escarros onde grassam as desgraças e
mentiras, a “teleologia sem princípios” e “os canalhas do mundo”, que
o furioso enunciador enterra o ideário cristão:

\begin{verse}
Escarrar de um abismo noutro abismo,\\
Mandando ao Céu o fumo de um cigarro,\\
Há mais filosofia neste escarro\\
Do que em toda a moral do Cristianismo!
\end{verse}

E “a revolta do poeta se comunica às criaturas, e ele --- tão
fluentemente articulado --- torna"-se o porta"-voz de todos os seres que a
evolução deixará para trás”,\footnote{Alcides, “Augusto dos Anjos e o \textit{Eu} 
universal”, op. cit., p.~82.} evidenciando,
novamente, uma extrema solidariedade com os destituídos de esperança no
mundo que “resignava"-se invertido”, bizarro e sem piedade, fazendo com
que, sob monumental melancolia, o poeta clamasse que

\begin{verse}
Morressem sufocadas pelo fogo\\
Todas as impressões do mundo externo!
\end{verse}

Com uma belíssima e melancólica imagem, o enunciador lamenta que a
Terra, onde caminha o destino de todos os humanos e onde se prefigura o
real (ou a ideia que se tem dele), não lhe permite equilíbrio:

\begin{verse}
Mas a Terra negava"-me o equilíbrio\ldots{}\\
Na Natureza, uma mulher de luto\\
Cantava, espiando as árvores sem fruto.\\
A canção prostituta do ludíbrio!
\end{verse}

É grande a beleza plástica da imagem da mulher de luto que canta o
ludíbrio fitando uma árvore sem frutos, como num triste e miserável
exercício de exorcismo.

Em outros poemas notam"-se pontuações espaço"-temporais como se usaria muito, 
dali a pouco, no cinema e, depois, nas histórias em quadrinhos.
Como no exemplo tirado do segundo soneto da série “Sonetos”, dedicados
ao pai do poeta: 

\begin{verse}
Madrugada de Treze de Janeiro.
\end{verse}

Da “Noite de um visionário”, numa pavorosa cena de decomposição:

\begin{verse}
Número cento e três. Rua Direita.\\
Eu tinha a sensação de quem se esfola\\
E inopinadamente o corpo atola\\
Numa poça de carne liquefeita!
\end{verse}

De “O caixão fantástico”:

\begin{verse}
Era tarde! Fazia muito frio.\\
Na rua apenas o caixão sombrio\\
Ia continuando o seu passeio!
\end{verse}

Ou, do “Poema negro”:

\begin{verse}
Súbito outra visão negra me espanta!\\
Estou em Roma. É Sexta"-feira Santa.

[\ldots{}]

Dorme a casa. O céu dorme. A árvore dorme.
\end{verse}

\asterisc

Cavalcanti Proença aponta para a recorrência da palavra “sangue” e
suas cognatas que, segundo o crítico, configuram uma ambiência de
aniquilamento total. Esse apocalipsismo \textit{gore} parece apontar
para uma tentativa de ressemantizar a essência humana através de sua
metáfora ao mesmo tempo mais natural e violenta, dando às lentes
críticas uma potência maior e fazendo"-as focar com nitidez o que antes
era apenas vulto. O sangue visto como expressão máxima da mutilação e
da reconstrução da vida, numa dialética constante. A contrapoética
augustiana, como vimos até agora, quer pôr tudo abaixo\footnote{
Curioso notar que no manifesto expressionista chamado “Revolução”, de
1913, de autoria do poeta anarquista alemão Erich Mühsam (um dos nomes
fortes da poesia anarquista no século \textsc{xx}), pode"-se ler: “As forças
motrizes da revolução são tédio e nostalgia, sua expressão é destruição
e criação./ Destruição e criação são idênticas na revolução. Todo
desejo destruidor é um desejo criador (Bakunin)” (cf. Cavalcanti, \textit{Poesia
expressionista alemã}, op. cit., p.~21.). Essas palavras do poeta anarquista, tão sugestivas se aplicadas à
contrapoética de Augusto dos Anjos, corroboram a ideia de Rosenfeld.}
(Eros e Thanatos se entreacariciando) para poder criar novos espaços de
habitação em sua poesia, quadro que porta aquele caráter destrutivo
apontado por Walter Benjamin em clássico texto: 

\begin{hedraquote}
O caráter destrutivo não vê nada de duradouro. Mas eis precisamente por
que vê caminhos por toda parte. Onde outros esbarram em muros ou
montanhas, também aí ele vê um caminho. Já que o vê por toda parte, tem
de desobstruí"-lo também por toda parte. Nem sempre com brutalidade, às
vezes com refinamento. Já que vê caminhos por toda parte, está sempre
na encruzilhada. Nenhum momento é capaz de saber o que o próximo traz.
O que existe ele converte em ruínas, não por causa das ruínas, mas por
causa do caminho que passa através delas.\footnote{Walter Benjamin, “O caráter destrutivo”. 
In \textit{Rua de mão única}. São Paulo: Brasiliense, 2000, p.~237.}
\end{hedraquote}

Essa forte pulsão \textit{gore} (“A cor do sangue é a cor que me
impressiona”) enuncia a destruição, mas prevê também a reconstrução.
Onde todos veem civilização, o poeta vê decomposição, mas dentro da
“carne podre” habitam os vermes que, monstruosamente, inauguram outra
organização social --- cujo progresso será operar o ritual de passagem da
carne podre (humana) ao Nada (nirvana búdico). Então é uma afirmação
negativa que, por poderoso oxímoro, se faz positiva, pois busca
alternativa de redenção e equilíbrio. Se tudo sangra é porque há
enfrentamento. O mundo é violento e hostil e as mudanças só se fazem
pela violência, como está escrito em “Queixas noturnas”:

\begin{verse}
Para essas lutas uma vida é pouca\\
Inda mesmo que os músculos se esforcem; \\
Os pobres braços do mortal se torcem \\
E o sangue jorra, em coalhos, pela boca.
\end{verse}

A linguagem se torna corpórea, pois exige ação direta, mesmo que
fadada à derrota. A linguagem é o homem. A linguagem agora é o sangue,
os ossos, as tripas, a garganta. Mais que a linguagem, a língua: o
“órgão muscular recoberto de mucosa, situado na boca e na faringe,
responsável pelo paladar e auxiliar na mastigação e na deglutição, e
também na produção de sons”.\footnote{ Segundo acepção do
\textit{Dicionário Houaiss da Língua Portuguesa}.}

\begin{verse}
Febre de em vão falar, com os dedos brutos\\
Para falar, puxa e repuxa a língua,\\
E não lhe vem à boca uma palavra!
\end{verse}

Rompe"-se com a visão acadêmica e burguesa da poesia como um artefato
meramente literário, quebra"-se a superstição da “arte pela
arte”.\footnote{ “Arte pela arte”, “\textit{business is business}” e
“cada um com seus problemas” são, afinal, galhos da mesma árvore.}
Esta contrapoesia do \textit{Eu} estava fincada na vida, no contexto
histórico,\footnote{ Num texto de jornal, de 3 de fevereiro de 1908,
Augusto dos Anjos escreverá: “Nesta cidade a política e o carnaval, num
sentido degradante, ocupam a atenção do público, insuficientemente
culto para a verdadeira compreensão dos fins humanos”.} alimentando,
pelo viés do humor negro e do grotesco, uma visão de mundo, uma
violenta crítica à sociedade \textit{bellepoquista} --- “esta grande
noite brasileira” em que o “sono esmaga o encéfalo do povo”. Como diria
Carpeaux, “o grande poeta que sabe dizer como este povo sofre e lhe
prever uma nova aurora”. Como, aliás, escreverá o poeta no poema
“Barcarola”:

\begin{verse}
Mas desgraçado do pobre\\
Que em meio da Vida cai!\\
Esse não volta, esse vai \\
Para o túmulo que o cobre.
\end{verse}

Daí se instaura, na poesia de Augusto dos Anjos, uma poética da vida
de todos os dias transubstanciada pela melancolia imanente do Cosmos,
ou, melhor dizendo, para reiterar, uma contrapoética da sociedade com
poderosas lentes de aumento. Sua obra é uma crítica da linguagem e da
sociedade, e explora muitas camadas de significação, o que a torna
forte e atemporal. Essas camadas formam seu complexo painel. A
recusa de Augusto figura em sua linguagem nas mais variadas frentes,
ele faz a poesia descer do pedestal, sair da torre de marfim e caminhar
no \textit{wild side}, elaborando uma crítica incisiva e violenta. Sua
poesia desceu “ao mais sórdido da miséria humana para, aí, iluminá"-la”.
\footnote{Gullar, “Augusto dos Anjos ou Vida e morte
nordestina”, op. cit., p.~37.} Essa iluminação faz parte do projeto destrutivo que
carrega em si a reconstrução (Benjamin). Essa iluminação é, portanto,
“um gesto crítico da inteligência que resiste”.\footnote{Prado, “Um fantasma 
na noite dos vencidos”, op. cit., p.~25.}

Em “Gemidos da arte”, a ruína do mundo todo expressa nos cacos
quebrados de uma janela, na parede abandonada, do patamar comum das
coisas do mundo. Uma das sequências mais belas da poesia brasileira. A
morte dentro da morte, autodevorando"-se como num caleidoscópio
aterrorizante: 

\begin{verse}
Na bruta dispersão de vítreos cacos,\\
À dura luz do sol resplandecente,\\
Trôpega e antiga, uma parede doente\\
Mostra a cara medonha dos buracos.

O cupim negro broca o âmago fino\\
Do teto. E traça trombas de elefantes\\
Com as circunvoluções extravagantes\\
Do seu complicadíssimo intestino. 

O lodo obscuro trepa"-se nas portas.\\
Amontoadas em grossos feixes rijos,\\\pagebreak
As lagartixas dos esconderijos\\
Estão olhando aquelas coisas mortas!
\end{verse}

Como no correr de sua obra, o homem nu, aberto ao mundo, no meio da
escuridão de sua pobre existência, está exposto ao avesso, por dentro,
visto pelo ponto de vista do verme, do fantástico monstro da
decadência. O homem está para o objeto como este para o homem --- a
realidade é cosmo do cuspo, o universo é o número 103 da rua Direita, o
infinito é a insônia iluminada por um candeeiro. O muro, das estrofes
anteriores, é então parte desse organismo que também decai, e a cal e a
carne:

\begin{verse}
A lamparina quando falta o azeite\\
Morre, da mesma forma que o homem morre.
\end{verse}

No poema “As cismas do destino”, a imagem do apagamento retorna e
novamente funde sujeito e objeto, numa comovente e bela imagem:

\begin{verse}
As pálpebras inchadas na vigília,\\
As aves moças que perderam a asa,\\
O fogão apagado de uma casa,\\
Onde morreu o chefe da família;
\end{verse}

A ausência do fogo no fogão é a presença da fome no vácuo deixado
pela morte do chefe da família. O apagamento é também uma espécie de
verme que elimina do tempo os traços da vida, abrindo espaço para o
Nada, assim como verme que elimina do corpo a carne que outrora o
animara. A produção de ruínas no relógio da existência é a concepção
própria e primordial de seu mecanismo. A ruína é, portanto, condição
\textit{sine qua non} de tudo que existe. 

Uma concepção política da vibração estética, claro. A arte se faz
como contraparte da vida, uma reverberando a outra, numa complexa
dialética. Daí a impossibilidade de se expulsar o cotidiano da
geografia árdua do poema. Uma arte que baste a si mesma não tem o menor
sentido. Alfonso Berardinelli, ao comentar a lírica defendida por Hugo
Friedrich em seu \textit{Estrutura da lírica moderna}, escreve que

\begin{hedraquote}
a lírica de que nos fala Friedrich em seu livro basta a si mesma. Não
necessita mais do mundo, evita qualquer vínculo com a realidade.
Nega"-lhe até a existência. Fecha"-se numa dimensão absolutamente
autônoma. Fantasia ditatorial, transcendência vazia, puro movimento da
linguagem, ausência de fins comunicativos, fuga da realidade empírica,
fundação de um espaço tempo sem relações causais e dissociado da
psicologia e da história.\footnote{Alfonso Berardinelli, \textit{Da poesia à prosa}. São Paulo:
CosacNaify, 2007, p.~21.}
\end{hedraquote}

A poesia que Friedrich defende é, fazendo um paralelo com o Brasil
do início do século \textsc{xx}, antípoda à que defendiam tanto a vanguarda
política (o anarquismo e o movimento operário) quanto a vanguarda
estética (os nomes de ponta do modernismo heroico). Se a arte fechada
numa “dimensão absolutamente autônoma” não era, portanto, opção de
escritores como Lima Barreto, Fábio Luz, José Oiticica ou Domingos
Ribeiro Filho, nem para Manuel Bandeira, Mário de Andrade, Luís Aranha,
Oswald de Andrade ou Alcântara Machado, tampouco era para Augusto dos
Anjos. O poeta paraibano não sofre deste esnobismo burguês de pantufas
e luva de pelica. A poesia, como se mostra neste espetacular livro
chamado \textit{Eu}, exige bem mais que simples versos. Exige uma
revolução onde o ético organize o estético de alguma maneira para,
assim, operar uma transformação nos modos de ver e pensar o mundo, num
gesto de enfrentamento que configura um programa de resistência.

\asterisc

Augusto dos Anjos viria a influenciar, de uma maneira ou de outra,
inúmeros poetas posteriores. Podemos notar, por vários aspectos, a voz
do autor do \textit{Eu} em poetas tão diferentes como Carlos Drummond
de Andrade, Jorge de Lima, João Cabral de Melo Neto ou Ferreira Gullar,
ou, ainda, mais contemporaneamente, Glauco Mattoso, Sebastião Nunes ou
Helio Neri.

E essa influência não é por acaso. Augusto dos Anjos possuía 
aguda consciência da modernidade,\footnote{ Álvaro Lins chegaria a
afirmá"-lo como o único “realmente moderno, com uma poesia que pode ser
compreendida e sentida como a de um contemporâneo”. No ensaio “Augusto
dos Anjos, poeta moderno”, constante do volume \textit{Augusto dos
Anjos. Textos críticos.} (Brasília: \textsc{inl}, 1973), organizado por Afrânio
Coutinho e Sônia Brayner.} fato que nem os modernistas heroicos
conseguiram perceber, pois se perderam na superfície e não conseguiram
captar, no osso da obra augustiana, as imensas questões que esta
propunha. 

Sua poesia não deu moleza para as letras de seu tempo. Recusou o
fácil e o confortável e pagou o preço por isso. Mas, justiça feita, sua
obra hoje é reconhecida e apreciada tanto pelo público quanto pela
crítica. O domínio completo de sua arte, de sua expressão, paga a
fiança da cegueira histórica. E agora, dezenas e dezenas de edições
depois, \textit{Eu} está preparado para a festa de seu centenário.
Chegar aos cem anos com tanta força e vitalidade é para poucos,
pouquíssimos! 

Augusto dos Anjos, o “espião do apocalipse”, numa expressão feliz de
José Paulo Paes, possuía, para usar um termo de outro poeta anárquico e
demolidor (Torquato Neto), “consciência da tragédia em plena tragédia”.


Sua dicção sinuosa, entalhada por biopalavras, não cedeu concessões
ao seu tempo. Pelo contrário, numa escrita da coragem, optou por
instaurar, na sala de leitura do mundo, novas possibilidades poéticas.
Um sublime desconcerto, presente e pra sempre, Augusto dos Anjos deixou
de ser um poeta do início do século \textsc{xx} para se tornar, com o passar das
décadas, um poeta do tempo presente. O poeta que produziu uma distorção
fatal na bem comportada e engroselhada poesia de seu tempo não merecia
lugar melhor na história. 


\section{Nota sobre a edição}

Esta edição contém o texto integral de \textit{Eu} publicado em 1912.

O estabelecimento de texto foi feito a partir da primeira edição de
\textit{Eu}\footnote{ A edição consultada contém a seguinte
dedicatória: “Ao grande espírito de Américo Facó homenagem affectuosa
de (impresso: Augusto dos Anjos). Rio, 15--6--1912”.} (1912)
integrante do acervo do Instituto de Estudos Brasileiros da
Universidade de São Paulo --- \textsc{ieb"-usp}. Foram usadas, como
apoio ao trabalho, as seguintes edições recentes:

\textit{Eu e Outras poesias}. (Org. Antonio Arnoni Prado). São Paulo:
Martins Fontes, 2000.

\textit{Eu e Outras poesias}. (Org. Sérgio Alcides). São Paulo: Ática,
2005.

Nesta edição, respeitou"-se o novo Acordo Ortográfico da Língua
Portuguesa. Entretanto, optou"-se por manter as variações autorais ---
dois/dous, coisas/cousas etc.

O organizador agradece a colaboração de Juliana Marks, Roberto Zular,
Sérgio Alcides e Tiago Pinheiro.




\begin{bibliohedra}
Alcides, Sérgio. “O gosto péssimo”. In \textit{Bravo!} n. 86,
2004.

idem. “Augusto dos Anjos e o \textit{Eu} universal”. In
\textsc{Anjos}, Augusto dos. \textit{Eu e Outras poesias}. São Paulo:
Ática, 2005.

idem. “Augusto dos Anjos e o mito do ‘Eu’\,”. In
\textit{Travessias do pós"-trágico: os dilemas de uma leitura do
Brasil}. (Orgs. Ettore Finazzi"-Agro, Roberto Vecchi e Maria Betânia
Amoroso). São Paulo: Unimarco, 2006.

Anjos, Augusto dos. \textit{Eu}. Rio de Janeiro: \textsc{s.c.p.}, 1912.

idem. \textit{Eu e Outras poesias}. (Org. Antonio Arnoni Prado).
São Paulo: Martins Fontes, 2000. 

idem. \textit{Eu e Outras poesias}. (Org. Sérgio Alcides). São
Paulo: Ática, 2005.

Benjamin, Walter. “O caráter destrutivo”. In
\textit{Rua de mão única}. São Paulo: Brasiliense, 2000.

Berardinelli, Alfonso. \textit{Da poesia à prosa}. São Paulo:
CosacNaify, 2007.

Bosi, Alfredo. \textit{História concisa da literatura
brasileira}. São Paulo: Cultrix, 1974.

idem. “Poesia"-resistência”. In 
\textit{O ser e o tempo da poesia}. São Paulo: Companhia das Letras, 2004.

Cavalcanti, Claudia. (Org. e trad.) \textit{Poesia
expressionista alemã --- Uma antologia}. São Paulo: Estação Liberdade,
2000.

Gullar, Ferreira. “Augusto dos Anjos ou Vida e morte
nordestina”. In \textsc{Anjos}, Augusto dos. \textit{Toda a poesia}.
Rio de Janeiro: Paz e Terra, 1978.

Kropotkin, Piotr. \textit{O princípio anarquista e outros
ensaios}. São Paulo: Hedra, 2007.

Lafetá, João Luiz. \textit{1930: a crítica e o modernismo}. São
Paulo: Duas Cidades / Editora 34, 2001.

Micheli, Mario de. \textit{As vanguardas artísticas}. São
Paulo: Martins Fontes, 2004.

Paes, José Paulo. “Augusto dos Anjos ou O evolucionismo às
avessas”. In \textsc{Anjos}, Augusto dos. \textit{Os melhores poemas
de\ldots{}} São Paulo: Global, 1997.

Prado, Antonio Arnoni. “Um fantasma na noite dos vencidos”. In
\textsc{Anjos}, Augusto dos. \textit{Eu e Outras poesias}. São Paulo:
Martins Fontes, 2000.

Rosenfeld, Anatol. “A costela de prata de A. dos Anjos”. In
\textsc{Rosenfeld}, Anatol. \textit{Texto/Contexto \textsc{i}}. São Paulo:
Perspectiva, 1996.

Sevcenko, Nicolau. \textit{Literatura como missão: tensões
sociais e criação cultural na primeira república.} São Paulo:
Brasiliense, 1995.
\end{bibliohedra}


